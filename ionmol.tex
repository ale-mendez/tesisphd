\chapter{Ionización de moléculas biológicas}

\section{Introducción}

El interés de la ionización de moléculas biológicas por el impacto de 
iones de carga múltiple ha crecido en el último tiempo debido a sus 
aplicaciones médicas y ambientales~\cite{Liamsuwan:13}, incluyendo 
tratamientos médicos~\cite{Mohamad:17,Baskar:12,Denifl:11,Solov:09} y 
reconocimiento de contaminantes en materiales biológicos~\cite{Gafur:18,
FerrazDias:13}. Particularmente, el estudio del daño causado por el 
impacto de projectiles pesados cargados en blancos biológicos es de gran
interés debido a su aplicación en la terapia contra el cáncer, que 
implementa haces de iones~\cite{Baskar:12}. La ionización de moléculas 
biológicas por iones de carga múltiple constituye el principal mecanismo 
de daño celular. Así, la efectividad de la radiación depende de la 
elección de los iones a implementar. En particular, estudios teóricos y 
experimentales con diferentes projectiles han concluido que los iones 
cargados de carbón podrían ser los iones más apropiados para dicha 
implementación~\cite{Mohamad:17}. Sin embargo, el estudio de tales 
sistemas representa un gran desafío desde el punto de vista teórico. 

A lo largo de las últimas décadas, se ha propuesto una amplia variedad de 
métodos teóricos con el fin de predecir la ionización de estos sistemas 
debido al impacto de iones cargados. Por ejemplo, se ha estudiado la 
ionización de agua, bases del ADN y ARN debido al impacto de protones y 
partículas $\alpha$ implementando el método de trayectorias clásicas 
Monte Carlo (\acs{ctmc}) en combinación con el criterio de sobrebarrera 
clásica (\acs{cob}) \cite{Abbas:08,Lekadir:09}. Los primeros cálculos 
mecanico-cuánticos de este proceso en moléculas biológicas fueron 
realizados bajo el formalismo de la primera aproximación de Born 
(\acs{fba}) \cite{DalCappello:08,Champion:10}. A altas energías, este 
método perturbativo garantiza las leyes de $Z^2$, donde $Z$ es la carga 
del projectil incidente. Sin embargo, el daño causado por la ionización 
está concentrado en los alrededores del pico de Bragg, esto es, a 
energías de unos cientos de keV/amu. Sin embargo, es precisamente en esta 
región donde la aproximación de Born empieza a fallar. 

Una de las grandes dificultades del modelado de la ionización de estos 
sistemas está dada por está la descripción de la estructura del blanco
mediante métodos de primeros principios. Los primeros cálculos de las 
funciones de onda moleculares implementaron el método de Hartree--Fock
(\acs{hf}) con geometría optimizada, mediante la expansión de un centro 
(\acs{sce}) \cite{DalCappello:08} y el método de omisión completa de 
superposición diferencial (\acs{cndo}) \cite{Champion:10}. En este último 
trabajo, la hipótesis principal se basa en el modelo de átomo 
independiente~(\acs{iam}); así, las secciones eficaces moleculares de 
ionización se obtienen a partir de la combinación lineal de secciones 
eficaces atómicas pesadas, donde los factores de peso son obtenidos 
mediante el análisis de la población de los orbitales moleculares. 

Las limitaciones de los métodos perturbativos de primer orden son 
superadas implementando aproximaciones con correcciones de mayor orden. 
Por ejemplo, el trabajo de Galassi \textit{et al.} \cite{Galassi:00} 
predice con éxito la ionización de moléculas simples por impacto de 
protones mediante el método de onda continua distorsionada con estado 
inicial de Eikonal (\acs{cdw-eis}) \cite{Fainstein:88,Miraglia:08,
Miraglia:09}. Esta metodología también ha sido utilizada para modelar la 
ionización de nucleobases debido al impacto de protones~\cite{Galassi:12}.
%%%% Acá va la descripción de trabajo de Ludde et al. %%%%
Más recientemente, y también siguiendo la línea del \acs{iam}, Lu\"udde 
y colaboradores~\cite{Ludde:16,Ludde:18,Ludde:19,Ludde:20} han propuesto 
la combinación de secciones eficaces atómicas con correcciones 
geométricas de apantallamiento. En este caso, los autores obtienen las 
secciones eficaces atómicas a partir de la teoría del functional densidad 
dependiente del tiempo~(\acs{tddft}). 

En este capítulo trataremos los dos aspectos principales de la ionización 
de moléculas biológicas debido a iones de carga múltiple: el orden de 
aproximación del proceso colisional ion--molécula y el método que permite
describir el blanco. Nuestro trabajo implementa el IAM, que se desprende 
de la regla aditiva de Bragg~(\acs{bar}), para describir la ionización de 
blancos moleculares por iones de carga múltiple. De manera que, 
primeramente, se implementará el método CDW-EIS para obtener una 
descripción apropiada del mecanismo de daño principal causado por los 
átomos que constituyen las moléculas a estudiar. Detalles sobre el método 
y los cálculos realizados se presentan en la Sección~\ref{sec:atoms}. 
Nuestro trabajo se desarrolla bajo la premisa de que el proceso de 
ionización es el mecanismo que deposita la mayor cantidad de energía 
primaria en el sistema. Sin embargo, se conoce que los electrones 
residuales de la ionización son una fuente significativa de daño 
biológico local~\cite{Denifl:11}. En efecto, los electrones secundarios 
son incluidos en simulaciones de Monte Carlo~\cite{Champion:16,Quinto:17,
Acocer-Avila:19}, y por lo tanto su comportamiento requiere especial 
atención. En las Secciones \ref{subsec:meanener} y \ref{subsec:meanang}, 
estudiamos las distribuciones energéticas y angulares medias de los 
electrones ejectados. Contrariamente a lo predicho por la FBA, 
encontramos una dependencia sustancial de estos valores con la carga del 
projectil. 
%En segundo lugar, examinamos la implementación del IAM, que se desprende
%de la regla aditiva de Bragg~(\acs{bar}), para describir la ionización de 
%blancos moleculares por iones de carga múltiple. 
En la Sección~\ref{sec:SSM}, tratamos la complejidad de la ionización 
molecular implementando el modelo estequiométrico simple (\acs{ssm}), el 
cual consiste en asumir que las moleculas están compuestas por átomos 
aislados e independientes, y que la sección eficaz total se expresa como 
una combinación lineal de cálculos atómicos pesados según la 
estequiometría de la molécula. Así, implementando en conjunto el método 
CDW y SSM, obtenemos secciones eficaces de ionización de diversas 
moléculas de interés biológico, incluyendo las cinco nucleobases 
--adenina, citosina, guanina, timina, uracilo--, tetrahidrofurano 
(\acs{thf}), pirimidina y agua, debido al impacto de antiprotones, 
H$^{+}$, He$^{2+}$, Be$^{4+}$, C$^{6+}$, y O$^{8+}$. 

En la Sección~\ref{sec:scaling}, estudiamos diversas reglas de escala; 
por ejemplo, examinamos la regla de escala de Toburen~\cite{Toburen:75,
Toburen:76}, que establece que la razón entre la sección eficaz de 
ionización y el número de electrones débilmente ligados se puede ubicar 
sobre una delgada banda universal en términos de la velocidad del 
projectil. Aplicamos esta regla a un número significativo de sistemas 
colisionales, incluyendo --además de los blancos ya mencionados-- un 
número significativo de hidrocarburos y moléculas CHNO. A partir de esta
escala, encontramos que el ancho de las bandas resultantes puede ser 
reducida significativamente modificando los números de Toburen por un 
nuevo conjunto de números de electrones activos. A partir del estudio de 
las secciones eficaces totales obtenidas por el método CDW, se proponen
nuevos números de electrones activos. La regla de escala resultante es 
implementada a nuestros valores teóricos y comparada con datos 
experimentales disponibles. Por otro lado, siguiendo la escala propuesta 
por Montenegro y colaboradores~\cite{Dubois:13,Montenegro:13}, la ley de 
escala $Z^2$ se reescribe en términos de un parámetro $\alpha$. 
Combinando el escaleo de las secciones eficaces totales con el número de 
electrones activos de los blancos y la carga del ion incidente, obtenemos 
una regla de escala única e independiente del sistema colisional. La 
generalidad de nuestra regla es puesta a prueba con datos experimentales 
de otros sistemas colisionales, no considerados previamente en esta 
investigación.

Por último, la aproximación SSM propuesta considera los átomos en la 
molécula como si fueran neutrales, lo cual no es correcto. En la 
Sección~\ref{sec:molcalculations}, consideramos los cálculos de 
estructura molecular realizados con el código {\sc gamess}~\cite{gamess} 
para computar el exceso o defecto de densidad electrónica en los átomos 
que componen las moléculas. Así, modificamos la fórmula estequimétrica 
para tener en cuenta el alejamiento de la neutralidad de los átomos. 
Encontramos que modificación a la aproximación SSM para las moléculas de 
ADN no introduce cambios sustanciales en las secciones eficaces totales 
de ionización.

%%%%%%%%%%%%%%%%%%%%%%%%%%%%%%%%%%%%%%%%%%%%%%%%%%%%%%%%%%%%%%%%%%%%%%%%
\section{Ionización de átomos constituyentes}
\label{sec:atoms}
%%%%%%%%%%%%%%%%%%%%%%%%%%%%%%%%%%%%%%%%%%%%%%%%%%%%%%%%%%%%%%%%%%%%%%%%

La sección de ionización total $\sigma_{\alpha}$ del átomo $\alpha$, que 
será luego implementada en el modelo molecular, se obtiene a partir de la 
aproximación del método CDW-EIS (ver Apéndice~\ref{app:CDW}). Las 
funciones de onda radiales de los estados inicial ligado y final continuo 
se calculan usando el código~\textsc{radialf}, desarrollado por Salvat y 
colaboradores~\cite{salvat1995}, e implementando potenciales efectivos. 
Los potenciales utilizados se obtuvieron a partir de la implementación 
del método de Inversión Depurada (\acs{dim})~\cite{Mendez:16,Mendez:18} 
desarrollado en el capítulo anterior. Usamos un par de miles de puntos 
como pivotes para resolver la ecuación de Schr\"{o}dinger, dependiendo 
del número de oscilaciones del estado del contínuo. La integración radial 
fue realizada usando la técnica de interpolación cúbica. Las funciones de 
onda del estado final en el continuo fueron expandidas como
\begin{equation}
\psi_{\mbox{\scriptsize$\mathbf{k}$}}^-(\mathbf{r})=\sum_{l=0}^{l_{\max
}}\sum_{m=-l}^l R_{kl}^-(r)\,Y_l^m(\hat{r})\,Y_l^{m^*}
(\hat{k})\,.
\label{eq:contwave}
\end{equation}
El número de momentos angulares considerados variaron desde 8, para 
electrones expulsados a muy bajas energías, hasta $l_{\max}\sim 30$, para 
las energías más altas consideradas. Se requirieron el mismo número de 
ángulos azimutales para obtener las secciones eficaces diferenciales 
cuádruples. 
%El cálculo realizado no muestra discrepancias en las versiones 
%posteriores y previas del método. 
Cada sección eficaz atómica total fue calculada usando entre 35 y 100 
valores de transferencia de momento, 28 ángulos electrónicos fijos, y 
alrededor de 45 valores de energía electrónica, dependiendo de la energía 
de impacto del proyectil. Para más detalle, se puede consultar el trabajo
de la Ref.~\cite{montanari2017}. 

\begin{figure}
\centering
\includegraphics[width=0.9\textwidth]{ionmol/atomicscaling.eps}
\caption[Sección eficaz total de ionización atómica CDW reducida.]
{Sección eficaz total de ionización CDW reducida $\sigma_{\alpha}/Z^2$ 
de cuatro blancos atómicos relevantes. Curvas: cálculos teóricos CDW. 
Experimentos: ionización de H por impacto de H$^+$~\cite{Shah:81}; 
ionización por impacto de $e^-$ en H~\cite{Shah:87}, C~\cite{Brook:78}, 
N~\cite{Brook:78} y O~\cite{Thompson:95}.}
\label{fig:atomscaling}
\end{figure} 

La mayor parte de las moléculas orgánicas contienen átomos de hidrógeno, 
carbono, nitrógeno, oxígeno, fósforo y azufre. Los sistemas colisionales
que estudiamos a continuación están compuestos de cuatro blancos 
atómicos, $\alpha=$ H, C, N y O, mientras que los proyectiles incidentes 
serán seis: antiprotones $\bar{p}$, H$^{+}$, He$^{2+}$, Be$^{4+}$, 
C$^{6+}$, y O$^{8+}$. En la Fig.~\ref{fig:atomscaling} se muestran las 
secciones eficaces totales de ionización implementando el método de 
CDW-EIS de estos blancos atómicos $\alpha$ debido al impacto de los seis 
iones cargados. Para comparar los 24 sistemas blanco--proyectil 
resultantes en una única figura, consideramos el hecho que en la primera 
aproximación de Born, la sección eficaz de ionización escala con el 
cuadrado de la carga del ion incidente, es decir $Z^{2}$. 
Las energías de impacto consideradas van de 0.1 a 10 MeV/amu, donde el 
método CDW-EIS tiene validez. Particularmente, para los proyectiles de 
carga más altas, el valor de energía mínimo donde se espera que la 
CDW-EIS tenga validez aumenta hasta aproximadamente 400 keV. Nuestros 
resultados se comparan con secciones eficaces experimentales para el caso 
de ionización de H por impacto de H$^+$~\cite{Shah:81}. Se incluyen 
mediciones de ionización por impacto de electrones en H~\cite{Shah:87}, 
C~\cite{Brook:78}, N~\cite{Brook:78} y O~\cite{Thompson:95}, con la 
correspondiente conversión de equivelocidad, para energías incidentes 
superiores a 300~eV. Es esperable que en esta región de energía la 
ionización por impacto de H$^+$ y $e^-$ convergen. También realizamos 
cálculos similares con la FBA (no se muestran aquí), y corroboramos que 
ésta provee resultados confiables para valores de energía mayores a unos 
cuantos MeV/amu. Usamos el mismo color de línea para indicar la carga del 
proyectil en todas las figuras que se muestran a lo largo de este 
Capítulo: discontinua--roja, sólida--roja, azul, magenta, oliva y naranja 
para antiprotones, H$^{+}$, He$^{2+}$, Be$^{4+}$, C$^{6+}$, y O$^{8+}$, 
respectivamente. 

%%%%%%%%%%%%%%%%%%%%%%%%%%%%%%%%%%%%%%%%%%%%%%%%%%%%%%%%%%%%%%%%%%%%%%%%
\subsection{Distribución energética de electrones}
\label{subsec:meanener}
%%%%%%%%%%%%%%%%%%%%%%%%%%%%%%%%%%%%%%%%%%%%%%%%%%%%%%%%%%%%%%%%%%%%%%%%

En un medio biológico dado, la ionización directa debido al impacto de un 
ion representa solo una fracción del daño total. Los electrones 
secundarios, así como el retroceso de los iones del blanco, también 
contribuyen sustancialmente al daño total~\cite{Denifl:11}. Podemos 
considerar que la sección eficaz de ionización diferencial en función de 
la energía del electrón eyectado $E$ de la capa $nl$ del átomo $\alpha$,
$d\sigma_{\alpha nl}/dE$, es una función de distribución 
simple~\cite{surdutovic2018}. Así, podemos definir un valor medio 
$\overline{E}_{\alpha}$, como proponen Abril y 
coloboradores~\cite{abril2015},
\begin{eqnarray}
\overline{E}_{\alpha} &=&\frac{\langle E_{\alpha}\rangle}{\langle
1\rangle}=\frac{1}{\sigma_{\alpha}}\sum\limits_{nl}\int dE\,E
\frac{d\sigma_{\alpha,nl}}{dE}\,,  
\label{eq:meanener} \\
\langle 1\rangle &=&\sigma_{\alpha}=\sum\limits_{nl}\int dE\,
\frac{d\sigma_{\alpha,nl}}{dE}\,. 
\label{eq:normener}
\end{eqnarray}
donde $\Sigma_{nl}$ tiene en cuenta la suma de las diferentes 
contribuciones de cada capa del elemento $\alpha$.

\begin{figure}
\centering
\includegraphics[width=0.9\textwidth]{ionmol/ener_mean.eps}
\caption[Distribución energética media de electrones emitidos.]
{Distribución energética media de electrones emitidos por la ionización 
debido al impacto de iones cargados dada por la Ec.~(\ref{eq:meanener}). 
Curvas: cálculos teóricos FBA con $Z=1$ (punteada) y CDW (sólidas y 
discontinua).}
\label{fig:emittedener}
\end{figure} 

Las energías medias de los electrones emitidos $\overline{E}_{\alpha}$ 
por H, C, N y O se muestran en la Fig.~\ref{fig:emittedener}. El rango 
de velocidad de impacto fue reducido a $v=10$~a.u. debido a las 
limitaciones numéricas en la expansión de esféricos armónicos dados por 
la Ec.~(\ref{eq:contwave}). A medida que la velocidad de impacto $v$ 
aumenta, también aumenta $\langle E_{\alpha}\rangle$ y $l_{\max}$, lo que 
resulta en la inclusión de funciones con muchas oscilaciones en el 
integrando. Más aún, el integrando de $\langle E_{\alpha}\rangle$ incluye 
la energía cinética $E$, que reduce su valor en la región de energías 
pequeñas y refuerza los valores grandes, haciendo que el resultado sea 
más sensible a los momentos angulares mayores. Independientemente, para 
$v>10$ a.u., la primera aproximación de Born es válida. 

En la Fig.~\ref{fig:emittedener} estimamos $\overline{E}_{\alpha}$ de los 
electrones emitidos en el rango de energía de 10 a 70 eV, para todos los 
blancos atómicos. Nuestros resultados concuerdan con los hallazgos 
experimentales~\cite{surdutovic2018}. Como se puede observar en la 
figura, el valor de energía media es sorprendentemente sensible a la 
carga del proyectil $Z$, que puede duplicar los resultados de protón en 
la región intermedia, i.e., 100--400 keV/amu. El efecto observado puede 
atribuirse al repulsión electrónica causada por los iones de carga 
múltiple en los electrones de baja energía. Este comportamiento no se 
puede encontrar en la primera aproximación de Born, donde la ley de 
escala $Z^2$ cancela la dependencia con $Z$ de la Ec.~(\ref{eq:meanener}).
A altas energías, $\overline{E}_{\alpha}$ tiende a un valor universal 
para todos los iones, como puede verse en la Fig.~\ref{fig:emittedener}.

%%%%%%%%%%%%%%%%%%%%%%%%%%%%%%%%%%%%%%%%%%%%%%%%%%%%%%%%%%%%%%%%%%%%%%%%
\subsection{Distribución angular de electrones}
\label{subsec:meanang}
%%%%%%%%%%%%%%%%%%%%%%%%%%%%%%%%%%%%%%%%%%%%%%%%%%%%%%%%%%%%%%%%%%%%%%%%

\begin{figure}
\centering
\includegraphics[width=0.9\textwidth]{ionmol/ang_mean.eps}
\caption[Distribución angular media de electrones emitidos.]
{Distribución angular media de electrones emitidos por la ionización 
debido al impacto de iones cargados dada por Ec.~(\ref{eq:meanang}). 
Curvas: cálculos teóricos FBA con $Z=1$ (punteada) y CDW (sólidas y 
discontinua).}
\label{fig:emittedang}
\end{figure} 

Como mencionamos anteriormente, la emisión de electrones secundarios 
contribuye al daño total. Entonces, no sólo es esencial conocer la 
distribución de energía de los electrones eyectados, sino también la 
dirección a la que éstos son emitidos. Una vez más, podemos considerar 
que la sección eficaz diferencial de ionización en función del ángulo 
sólido de eyección del electrón $\Omega$, $d\sigma_{\alpha,nl}/d\Omega$, 
puede expresarse como una función de distribución. Así, el ángulo medio 
de emisión $\overline{\theta}_{\alpha}$ se define como
\begin{eqnarray}
\overline{\theta}_{\alpha}&=&\frac{\langle\theta_{\alpha}\rangle}
{\langle 1\rangle}=\frac{1}{\sigma_{\alpha}}\sum\limits_{nl}
\int d\Omega\,\theta\,\frac{d\sigma_{\alpha,nl}}{d\Omega} 
\label{eq:meanang} \\
\langle 1\rangle &=&\sigma_{\alpha}=\sum\limits_{nl}\int d\Omega\,
\frac{d\sigma_{\alpha,nl}}{d\Omega}\,.
\end{eqnarray}

Los ángulos medios de emisión electrónica $\overline{\theta}_{\alpha}$ 
de los cuatro átomos y seis iones estudiados aquí se muestran en la 
figura~\ref{fig:emittedang}. Se puede observar una dependencia 
significativa de $\overline{\theta}_{\alpha}$ con $Z$ para todos los 
sistemas colisionales. Una vez más, este efecto no puede ser observado en 
la implementación del FBA (línea punteada).

En la emisión de electrones de baja energía, la dispersión angular es 
casi isotrópica~\cite{Rudd1992}. Un valor típico para el ángulo de 
eyección considerado en la literatura es 
$\overline{\theta}_{\alpha}\sim$~70\textdegree~\cite{surdutovic2018}, el 
cual resulta bastante certero en el rango de validez de la FBA para 
cualquier objetivo. Sin embargo, cuando se usa una aproximación de onda 
distorsionada, $\overline{\theta}_{\alpha}$ disminuye sustancialmente con 
$Z$ en la región de energía intermedia, como se observa en la 
Fig.~\ref{fig:emittedang}. Cuanto mayor sea la carga $Z$, menor será 
$\overline{\theta}$. Por supuesto, este efecto solo es válido en energías 
intermedias y no en el rango de altas energías.

Para ilustrar esta característica, consideramos el impacto de C$^{6+}$ 
con una energía de 500~keV sobre oxígeno. La primera aproximación de Born 
predice electrones emitidos con energías medias de $46.7$ eV y ángulos 
medios de 78\textdegree, mientras que la aproximación CDW establece 
energías medias de $62.5$~eV y un ángulo de emisión igual a 60\textdegree. 
Estos resultados implican una penetración más profunda de los electrones 
secundarios con una orientación más cercana a la dirección del ion. 
Podemos atribuir esta corrección de la dirección de avance al efecto de 
captura del continuo.

Además, la Fig.~\ref{fig:emittedang} proporciona una descripción 
ilustrativa del comportamiento de los antiprotones: el proyectil repele 
los electrones, siendo $\overline{\theta}_{\alpha}\sim$~90\textdegree. 
Nótese el efecto opuesto de protones y antiprotones con respecto a la 
primera aproximación de Born; este fenómeno constituye un efecto Barkas 
angular.

%%%%%%%%%%%%%%%%%%%%%%%%%%%%%%%%%%%%%%%%%%%%%%%%%%%%%%%%%%%%%%%%%%%%%%%%
\section{El modelo estequiométrico}
\label{sec:SSM}
%%%%%%%%%%%%%%%%%%%%%%%%%%%%%%%%%%%%%%%%%%%%%%%%%%%%%%%%%%%%%%%%%%%%%%%%

El modelo estequiométrico simple (\acs{ssm}) que proponemos aquí para 
predecir secciones eficaces de ionización totales moleculares está basado 
en la aproximación de átomo independiente, también llamada regla aditiva 
de Bragg. Este modelo supone que los átomos que componen una molécula $M$ 
sólo interactúan con el proyectil incidente pero no entre sí. Así, si 
suponemos que una molécula compuesta por $n_{\alpha}$ átomos del elemento 
$\alpha$, el modelo estequiométrico aproxima la sección eficaz de 
ionización total de la molécula $\sigma_M$ como la suma de secciones 
eficaces de ionización de los átomos aislados $\sigma_{\alpha}$ ponderada 
por $n_{\alpha}$, 
\begin{equation}
\sigma_{M}=\sum\limits_{\alpha}n_{\alpha}\sigma_{\alpha}\,.  
\label{eq:sumion}
\end{equation}
Los blancos moleculares examinados a lo largo de este capítulo se 
clasifican en tres familias: CH, CHNO, y ADN, como se muestra en la 
Tabla~\ref{tab:families}.

\begin{table}
\begin{center}
\begin{tabular}{|p{0.08\textwidth}|p{0.8\textwidth}|}
\hline
\multirow{2}{*}{CH} & CH$_4$ (metano), C$_2$H$_2$ (acetileno), 
C$_2$H$_4$ (eteno), C$_2$H$_6$ (etano), \\ & C$_6$H$_6$ (benceno) \\
\hline
\multirow{2}{*}{CHNO} & C$_5$H$_5$N (piridina), C$_4$H$_4$N$_2$ (pirimidina), 
C$_2$H$_7$N (dimetilamina), \\ & CH$_5$N (monometilamina), 
C$_4$H$_8$O (THF) \\[0.2em]
\hline
\multirow{2}{*}{DNA} & C$_5$H$_5$N$_5$ (adenina), C$_4$H$_5$N$_3$O (citosina), 
C$_5$H$_5$N$_5$O (guanina), \\ & C$_5$H$_6$N$_2$O$_2$ (timina),
C$_4$H$_4$N$_2$O$_2$ (uracilo), H$_2$O (agua) \\
\hline
\end{tabular}
\caption[Blancos moleculares examinados y clasificados en tres familias.]
{Blancos moleculares de interés examinados en el presente trabajo y 
clasificados en tres familias.}
\label{tab:families}
\end{center}
\end{table}

En la Fig.~\ref{fig:crossDNA_1} reportamos las secciones eficaces de 
ionización totales reducidas con la carga del ión incidente, 
$\sigma_M/Z^2$, de las cuatro nucleobases del ADN --adenina, citosina, 
guanina y timina-- debido al impacto de iones de carga múltiple, que se 
obtienen a partir del modelo estequiométrico y el método CDW. Para 
adenina, los datos experimentales disponibles para el impacto de 
protones~\cite{iriki2011} tienen un excelente acuerdo con nuestras 
predicciones. La única medición disponible de C$^{4+}$~\cite{Sens:20} 
sobre adenina coincide con el modelo SSM--CDW, dentro del margen de 
error. Por otro lado, los resultados teóricos de ionización de adenina
debido a C$^{6+}$ discrepan con el valor experimental 
disponible~\cite{Bhattacharjee:19} en un factor dos. Esta discrepancia es 
intrigante ya que la medición se encuentra en el rango de altas energías, 
donde generalmente la teoría predice muy bien los experimentos.

No hemos encontrado en la bibliografía datos experimentales de ionización 
por impacto de iones cargados para el resto de las nucleobases de ADN.
Incluimos en la Fig.~\ref{fig:crossDNA_1} mediciones de ionización por 
impacto de electrones~\cite{rahman2016}, con la correspondiente 
conversión de equivelocidad, para energías incidentes superiores a 
300~eV. En esta región, la sección eficaz de ionización por impacto de 
protones y electrones debería coincidir. Aunque las mediciones de impacto 
de electrones están por encima de nuestros hallazgos para todos los 
objetivos moleculares, vale la pena señalar que los resultados SSM--CDW
concuerdan muy bien con otras predicciones teóricas de ionización por 
impacto de electrones~\cite{mozejko2003,tan2018}.

\begin{figure}
\centering
\includegraphics[width=0.9\textwidth]{ionmol/adn1.eps}
\caption[Sección eficaz total de ionización reducida por $Z$ (Parte I).]
{Sección eficaz total de ionización CDW reducida $\sigma_{M}/Z^2$ como 
una función de la energía de impacto del ion. 
Experimentos: datos experimentales para impacto de protón~\cite{iriki2011}, 
C$^{4+}$~\cite{Sens:20}, C$^{6+}$~\cite{Bhattacharjee:19}, y 
e$^-$~\cite{rahman2016} con conversión de equivelocidad.}
\label{fig:crossDNA_1}
\end{figure} 

\begin{figure}
\centering
\includegraphics[width=0.9\textwidth]{ionmol/adn2.eps}
\caption[Sección eficaz total de ionización reducida por $Z$ (Parte II).]
{Sección eficaz total de ionización CDW reducida $\sigma_{M}/Z^2$ como 
una función de la energía de impacto del ion. Experimentos: impacto de 
protón en uracilo~\cite{itoh2013}, pirimidina~\cite{wolff2014}, 
THF~\cite{wang2016} y agua~\cite{Luna2007,Bolorizadeh86,H_Rudd85,toburen80}. 
Impacto de C$^{4+}$~\cite{Sens:20} y C$^{4+}$, C$^{6+}$, O$^{6+}$, F$^{6+}$, 
O$^{8+}$, y F$^{8+}$ en uracilo~\cite{agnihotri2012,agnihotri2013}. 
Ionización de agua por impacto de He$^{2+}$~\cite{Ohsawa05,He_Rudd85,
toburen80}, C$^{6+}$~\cite{DalCappello2009,Bhattacharjee:17} y 
O$^{8+}$~\cite{Bhattacharjee:16}. Impacto de e$^-$~\cite{bug2017,wolf2019,
fuss2009} con conversión de equivelocidad.}
\label{fig:crossDNA_2}
\end{figure} 

Las secciones eficaces de ionización total reducidas $\sigma_M/Z^2$ 
para uracilo, pirimidina, THF y agua se muestran en la 
Fig.~\ref{fig:crossDNA_2}. En uracilo, las mediciones experimentales de 
ionización por impacto de protones de Itoh~\textit{et al.}~\cite{itoh2013} 
tienen un buen acuerdo con nuestras predicciones. Sin embargo, para el 
mismo blanco, nuestra teoría predice secciones eficaces con un factor 
de dos por encima de las mediciones experimentales de ionización de 
Agnihotri \textit{et al.}~\cite{agnihotri2012,agnihotri2013} para el 
impacto de iones de carga múltiple. No obstante, cabe señalar que 
nuestros resultados teóricos coinciden con los cálculos de Champion, 
Rivarola y colaboradores~\cite{agnihotri2012,champion2012}. También 
incluimos una reciente medición experimental de ionización en uracilo
debido al impacto de C$^{4+}$~\cite{Sens:20} que coincide con la teoría, 
lo podría indicar algún problema con los valores experimentales de 
Agnihotri y colaboradores.

Para pirimidina, mostramos una comparación de nuestros resultados con 
los datos experimentales de ionización por impacto de protones de 
Wolff \textit{et al.}~\cite{wolff2014} y también para la ionización por 
impacto de electrones~\cite{bug2017} a altas energías. Las mediciones de 
impacto de electrones concuerdan con las predicciones SSM--CDW para 
energías superiores a 500~keV. Inesperadamente, las secciones eficaces de 
ionización por impacto de protones son significativamente más bajas que 
nuestros resultados. Una mayor cantidad de experimentos se encuentran 
disponibles para la ionización de la molécula de THF por impacto de 
H$^+$~\cite{wang2016} y de e$^-$~\cite{bug2017,wolf2019,fuss2009}. Los 
resultados que obtenemos de la combinación del SSM y las secciones 
eficaces atómicas CDW muestran un buen acuerdo general con esta data.

En el caso de la molécula de agua, los resultados experimentales son más 
abundantes que para el resto de los sistemas moleculares. Los resultados 
experimentales~\cite{Luna2007,Bolorizadeh86,H_Rudd85,toburen80,Ohsawa05,
He_Rudd85,toburen80,Bhattacharjee:16} coinciden con 
nuestras predicciones para todos los iones de carga múltiple examinados 
aquí, excepto para C$^{6+}$~\cite{DalCapello:09,Bhattacharjee:17}, para 
el que la teoría sobrestima la ionización en un factor dos en el rango de 
altas energías. A diferencia de las nucleobases, pirimidina y THF, la 
estequiometría del agua es más simple; sin embargo, observamos que 
nuestro modelo responde bien incluso para predecir la ionización de 
moléculas pequeñas debido a la incidencia de iones pesados, como 
O$^{6+}$.

%%%%%%%%%%%%%%%%%%%%%%%%%%%%%%%%%%%%%%%%%%%%%%%%%%%%%%%%%%%%%%%%%%%%%%%%
\section{Reglas de escala}
\label{sec:scaling}

%%%%%%%%%%%%%%%%%%%%%%%%%%%%%%%%%%%%%%%%%%%%%%%%%%%%%%%%%%%%%%%%%%%%%%%%
\subsection{Escala con electrones activos del blanco}
\label{subsec:ne_scaling}

%%%%%%%%%%%%%%%%%%%%%%%%%%%%%%%%%%%%%%%%%%%%%%%%%%%%%%%%%%%%%%%%%%%%%%%%
\subsubsection{Regla de Toburen}
\label{subsec:toburen}
%%%%%%%%%%%%%%%%%%%%%%%%%%%%%%%%%%%%%%%%%%%%%%%%%%%%%%%%%%%%%%%%%%%%%%%%

Toburen y colaboradores~\cite{Toburen:75,Toburen:76} propusieron el 
primer intento de desarrollar un modelo fenomenológico completo pero 
sencillo para la eyección de electrones de moléculas complejas. Los 
autores encontraron conveniente escalar la sección eficaz de ionización 
experimental en términos del número de electrones ligados débilmente, $n_e$.
Por ejemplo, para C, N y O, este número es el número total de 
electrones menos la capa K. Siguiendo a Toburen, la sección eficaz de 
ionización escalada por electrón débilmente ligado $\sigma_{e}^T$ es
\begin{equation}
\sigma_{e}^T=\frac{\sigma_{M}}{n_e}\,, 
\label{27} 
\end{equation}
donde $n_e=\sum_{\alpha}n_{\alpha}\nu_{\alpha}^T$, y $\nu_{\alpha}^T$ 
son los números de Toburen dados por
\begin{equation}
\nu_{\alpha}^T=\left\{ 
\begin{array}{ll}
1, & \text{para H,} \\
4, & \text{para C,} \\ 
5, & \text{para N,} \\ 
6, & \text{para O}\,.
\end{array}\right.
\label{eq:nelec} 
\end{equation} 
La regla de Toburen se puede enunciar diciendo que $\sigma_{e}$ es un 
parámetro universal independiente de la molécula, que depende únicamente 
de la velocidad del impacto y que se aplica a energías de alto impacto 
(es decir, 0,25 a 5 MeV/amu). Estos números $\nu_{\alpha}^T$ pueden 
interpretarse como el número de electrones activos en la colisión. A 
energías muy altas, los electrones de la capa K también se ionizarán y 
estos números serán diferentes. Una dependencia similar con el número de 
electrones ligados débilmente fue hallada por Itoh y 
coloboradores~\cite{itoh2013} para el impacto de protones sobre uracilo 
y adenina.

Siguiendo la ley de escala de Toburen, calculamos las secciones eficaces 
de CDW escaladas $\sigma_{e}^T$ para los blancos moleculares dados en la
tabla~\ref{tab:families}. Nuestros resultados se muestran en la 
figura~\ref{fig:newscaling}(a) en función de la energía de impacto para 
diferentes proyectiles de carga múltiple. Aunque la escala de Toburen 
se mantiene para altas energías, su desempeño no es muy satisfactorio: 
como se puede observar en esta figura, la banda universal es bastante 
ancha.

\begin{figure}
\centering
\includegraphics[width=0.9\textwidth]{ionmol/CDWscaling.eps}
\caption[Sección eficaz de ionización escalada por $n_e$.]
{Sección eficaz de ionización escala por electrón débilmente ligado 
usando (a)~los números de Toburen $\nu_{\alpha}^T$, y (b) los
números propuestos según los resultados CDW $\nu_{\alpha}^{\text{CDW}}$ 
para las moléculas enlistadas en la tabla~\ref{tab:families}. 
Experimentos: iguales a figuras~\ref{fig:crossDNA_1} y 
\ref{fig:crossDNA_2}.}
\label{fig:newscaling}
\end{figure}

%%%%%%%%%%%%%%%%%%%%%%%%%%%%%%%%%%%%%%%%%%%%%%%%%%%%%%%%%%%%%%%%%%%%%%%%
\subsubsection{Números CDW}
\label{subsec:CDW}
%%%%%%%%%%%%%%%%%%%%%%%%%%%%%%%%%%%%%%%%%%%%%%%%%%%%%%%%%%%%%%%%%%%%%%%%

La desviación de nuestros resultados teóricos de la regla de escala de 
Toburen puede entenderse fácilmente inspeccionando la
figura~\ref{fig:atomscaling}. Se puede observar que la regla 
$\sigma_{\alpha}/\nu_{\alpha}^T\sim \sigma_{e}^T$, aproximadamente 
constante, no es satisfecha de manera apropiada por los cálculos CDW. 
Por ejemplo, la figura~\ref{fig:atomscaling} muestra que las secciones
eficaces de O son muy similares a las secciones eficaces de C, lo que 
sugiere 4 electrones activos en O en lugar de 6. De la misma manera, el 
número de electrones activos para N que se obtienen a partir de los 
resultados teóricos CDW también son diferentes de los $\nu_{\alpha}^T$ 
dados por la ecuación~(\ref{eq:nelec}).

Basándonos en los resultados CDW, proponemos una nueva escala,
\begin{equation}
\sigma_{e}'=\frac{\sigma_M}{n_e'},
\label{32} 
\end{equation}
donde $n_e'=\sum_{\alpha}n_{\alpha}\nu_{\alpha}^{\text{CDW}}$, y
$\nu_{\alpha}^{\text{CDW}}$ son los números de electrones activos 
por átomo obtenidos de las secciones eficaces de ionización CDW para los 
diferentes iones en los blancos atómicos H, C, N y O. Estos valores están 
dados por
\begin{equation}
\nu_{\alpha }^{\text{CDW}} \sim\left\{ 
\begin{array}{ll}
1, & \text{para H,} \\
4, & \text{para C, N, y O}\,. \\ 
%4.5, & \text{para P y S}\,.
\end{array}
\right. 
\label{eq:scalingCDW}
\end{equation}

Las nuevas secciones eficaces escaladas $\sigma_{e}'$ se muestran en la
figura~\ref{fig:newscaling}(b). Los datos experimentales para la 
ionización de adenina~\cite{iriki2011}, uracilo~\cite{itoh2013}, 
pirimidina~\cite{wolff2014}, y THF~\cite{wang2016} por impacto de proton 
que se muestran en la figura~\ref{fig:newscaling}(b) corroboran nuestro
propuesta de escaleo. También incluimos mediciones de ionización por 
impacto de electrones con conversión de equivelocidad en 
pirimidina~\cite{bug2017} y THF~\cite{bug2017,wolf2019,fuss2009}. 
Será interesante verificar nuestras predicciones con experimentos 
futuros, principalmente para estados de carga de proyectiles más altos.

\begin{table}
\begin{center}
\begin{tabular}{|p{0.12\textwidth}p{0.15\textwidth}p{0.03\textwidth}
p{0.03\textwidth}|p{0.12\textwidth}p{0.15\textwidth}p{0.03\textwidth}
p{0.03\textwidth}|}
\hline
Molécula        & Nombre      & $n_e$ & $n_e'$ & 
Molécula        & Nombre      & $n_e$ & $n_e'$ \\
\hline
H$_2$           & Dihidrógeno & 2      & 2     & 
C$_4$H$_4$N$_2$ & Pirimidina  & 30     & 28    \\
H$_2$O          & Agua        & 8      & 6     & 
C$_6$H$_6$      & Etano       & 30     & 30    \\
NH$_3$          & Amoníaco    & 8      & 7     & 
C$_4$H$_4$N$_2$O$_2$ & Uracilo & 40    & 36    \\
CH$_4$          & Metano      & 8      & 8     & 
C$_4$H$_5$N$_3$O & Citosina   & 42     & 37    \\
CH$_5$N         & Metilamina  & 14     & 13    & 
C$_5$H$_6$N$_2$O$_2$ & Timina & 48     & 42    \\
C$_2$H$_7$N     & Etilamina   & 20     & 19    & 
C$_5$H$_5$N$_5$ & Adenina     & 50     & 45    \\
C$_4$H$_8$O     & THF         & 30     & 28    & 
C$_5$H$_5$N$_5$O & Guanina    & 56     & 49    \\
\hline
\end{tabular}
\caption[Números de escala de Toburen y CDW.]
{Números de Toburen $n_e$ y de escala obtenidos a partir de los cálculos 
CDW $n_e'$ para algunos blancos moleculares de interés biológico.}
\label{nn}
\end{center}
\end{table}

\begin{figure}
\centering
\includegraphics[width=0.9\textwidth]{ionmol/scale_ne.eps}
\caption[Sección eficaz de ionización por impacto de protón en términos 
de $n_e$.]
{Sección eficaz de ionización por impacto de protón a 0.5, 1,
y 2 MeV en términos del númbero de electrones activos dado por la 
tabla~\ref{nn}. Experimentos: 
\mbox{\Large$\circ$}~adenina~\cite{iriki2011}, 
$\triangle$ uracilo~\cite{itoh2013}, 
$\bigtriangledown$ pirimidina~\cite{wolff2014}, 
$\blacktriangle$ C$_2$H$_7$N, CH$_5$N, metano y amoníaco~\cite{Lynch:76},
\mbox{\scriptsize$\bigstar$} amoníaco y H$_2$~\cite{Rudd:85}, y 
\mbox{\Large$\bullet$} agua~\cite{Luna2007}.}
\label{fig:recta}
\end{figure}

Usando la ecuación~(\ref{eq:scalingCDW}), definimos nuevos números de 
electrones activos $n_e'$ para las moléculas consideradas. En la 
tabla~\ref{nn}, mostramos los nuevos valores $n_e'$ y los números de 
Toburen $n_e$, que resultan de aplicar la ecuación~(\ref{eq:nelec}). 
Nuestros resultados son diferentes a los propuestos por Toburen, los 
cuales son utilizados por otros autores~\cite{itoh2013}, principalmente 
debido a las diferencias en los números de electrones activos del 
oxígeno. Una forma alternativa de probar la escala propusta se puede
obtener dibujando las secciones eficaces de ionización de las moléculas 
en función de los valores dados para $n_e'$ según la tabla~\ref{nn}. 
Nuestros resultados se muestran en la figura~\ref{fig:recta} para 
energías de impacto de 0,5, 1 y 2 MeV. Como puede observarse, las 
secciones eficaces de ionización CDW calculadas para todas las moléculas 
muestran una dependencia lineal con el número de electrones $n_e'$ de 
la tabla~\ref{nn}. Obtenemos resultados similares, incluso para 
$E=10$~MeV. La comparación con los datos experimentales disponibles 
muestra una buena concordancia general, desde moléculas pequeñas, tales
como H$_2$, H$_2$O, and CH$_4$, hasta las más complejas, como la adenina. 
Para los datos de ionización por impacto de electrón, los datos 
experimentales se interpolaron entre vecinos cercanos. 
%It is worth mentioning that an equivalent plot using the Toburen numbers 
%$n_e$ does not exhibit the straight lines obtained with the present scaling. 

%While finishing the present work, we became aware of an accepted 
%manuscript by L\"udde~\textit{et al.}~\cite{ludde2019} on total 
%ionization of biological molecules by proton impact, using the
%independent--atom--model pixel counting method~\cite{ludde2016,ludde2018}.
%The authors also raised a scaling  with $\nu_{\alpha}=4$ for C, N, and O, 
%but $\nu_{\alpha}=6$ for P. The agreement with this independent method 
%for proton impact reinforces our multicharged--ion findings.

%%%%%%%%%%%%%%%%%%%%%%%%%%%%%%%%%%%%%%%%%%%%%%%%%%%%%%%%%%%%%%%%%%%%%%%%
\subsection{Escala con la carga del ion}
\label{sec:zscaling}
%%%%%%%%%%%%%%%%%%%%%%%%%%%%%%%%%%%%%%%%%%%%%%%%%%%%%%%%%%%%%%%%%%%%%%%%

A energías de impacto intermedias, la regla $Z^2$ ya no se cumple y se 
pueden considerar otras escalas en esta región. En la 
sección~\ref{sec:scaling} inspeccionaremos algunas de ellas.

%At intermediate impact energies, the $Z^2$ rule no longer holds, and 
%other scalings can be considered in this region. For example, the 
%molecular cross section and ion impact energy can be reduced with the 
%projectile charge $Z$, as suggested in in~\cite{janev1980,dubois2013}. 




En la literatura, encontramos dos tipos de leyes de escala con la carga 
$Z$ del ion incidente aplicables en el rango de energías de impacto
intermedias. La regla sugerida por Janev y Presnyakov~\cite{Janev:80} 
considera $\sigma/Z$ en función de $E/Z$ como la forma reducida 
\textit{natural} de la sección eficaz de ionización $\sigma$ y la energía
de ion incidente $E$. Más recientemente, Montenegro y 
colaboradores~\cite{Dubois:13,Montenegro:13} sugirieron una expresión
alternativa, la cual toma en cuenta que la sección eficaz es una función
de $Z^2/E$ a altas energías. La escala propuesta, dada por
\begin{equation}
 \sigma/Z^{\alpha}=f(E/Z^{2-\alpha}),
\label{eq:Montenegro}
\end{equation}
mantiene la relación $Z^2/E$ para cualquier valor del parámetro $\alpha$. 
Los autores propusieron el valor $\alpha=4/3$ para la ionización de He y
H$_2$ debido al impacto de diversos iones cargados~\cite{dubois2013}. 

Siguiendo el trabajo de Montenegro y colaboradores, encontramos que el
mejor valor del parámetro $\alpha$ que converge los resultados obtenidos
por la combinación del método CDW y SSM para los sistemas colisionales
estudiados aquí en el mayor rango de energías posible (de intermedias a 
altas) es $\alpha=1.2$. La validez de este particular escaleo es evidente
en la figura~\ref{fig:zreduced}, donde --para cada blanco-- las curvas 
CDW-SSM correspondientes a los diferentes iones se superponen. Es notable
como los resultados teóricos son válidos para energías de impacto incluso
por encima del máximo de las secciones eficaces, que corresponden a
rangos de energías incidentes desde 50 keV para H$^+$ hasta 250 keV/amu 
para O$^{+8}$.

\begin{figure}
\centering
\includegraphics[width=0.9\textwidth]{ionmol/adn1_zscale.eps}
\caption[Sección eficaz de ionización reducida por $Z$ y $\alpha$ 
(Parte I).]
{Sección eficaz de ionización reducida $\sigma/Z^{\alpha}$ como función
de la energía incidente del ion $E/Z^{2-\alpha}$ con $\alpha=1.2$. 
Curvas: resultados teóricos CDW-SSM. 
Símbolos: datos experimentales correspondientes a la 
figura~\ref{fig:crossDNA_1}.}
\label{fig:zreduced}
\end{figure} 

\begin{figure}
\centering
\includegraphics[width=0.9\textwidth]{ionmol/adn2_zscale.eps}
\caption[Sección eficaz de ionización reducida por $Z$ y $\alpha$ 
(Parte II).]
{Sección eficaz de ionización reducida $\sigma/Z^{\alpha}$ como función
de la energía incidente del ion $E/Z^{2-\alpha}$ con $\alpha=1.2$. 
Curvas: resultados teóricos CDW-SSM. 
Símbolos: datos experimentales correspondientes a la 
figura~\ref{fig:crossDNA_2}.}
\label{fig:zreduced}
\end{figure} 

También examinamos los datos experimentales disponibles para los sistemas
ion-blanco bajo estudio~\cite{itoh2013,iriki2011,wolff2014,wang2016,
Bhattacharjee2019,agnihotri2012,agnihotri2013,Luna2007,Bolorizadeh86,
H_Rudd85,toburen80,Ohsawa05,Bhattacharjee17,DalCappello2009,Tribedi_O_water} 
con la regla de escala $Z^\alpha$. 
En el caso de los blancos con pocos o ningún dato experimental incluimos 
secciones eficaces experimentales de ionización por impacto de electrón 
\cite{rahman2016,bug2017,wolf2019,fuss2009} a grandes velocidades con la
conversión correspondiente. Como se observa, la mayor parte de los datos
experimentales en la figura~\ref{fig:zreduced} confirma el escaleo 
sugerido aquí, incluso para O$^{+8}$ en agua~\cite{Tribedi_O_water}. 
Vale recalcar que la sección eficaz de ionización de uracilo por 
iones desnudos de C, O, y F correspondientes a las 
Refs.~\cite{agnihotri2012,agnihotri2013} están fuera de nuestras 
predicciones. Sin embargo, los cálculos recientes de 
Sarkadi~\cite{sarkadi2016} también se encuentran por arriba de los 
valores experimentales de Tribedi y 
colaboradores~\cite{agnihotri2012,agnihotri2013}.

%%%%%%%%%%%%%%%%%%%%%%%%%%%%%%%%%%%%%%%%%%%%%%%%%%%%%%%%%%%%%%%%%%%%%%%%
\subsection{Escala con electrones activos y carga del ion}
\label{sec:nez_scaling}
%%%%%%%%%%%%%%%%%%%%%%%%%%%%%%%%%%%%%%%%%%%%%%%%%%%%%%%%%%%%%%%%%%%%%%%%

Considerando la reducción con la carga del ión incidente $Z^\alpha$ y el 
escaleo con el número de electrones activos del blanco, introducimos 
la sección eficaz de ionización molecular escalada y reducida 
$\tilde{\sigma}$, que se expresa como función de $E/Z^{2-\alpha}$, y
está dada por
\begin{equation}
 \tilde{\sigma}=\frac{\sigma_e}{Z^{\alpha}}=\frac{\sigma_M/n_e}{Z^{\alpha}}\,,
\label{eq:u-scaling}
\end{equation}
donde $\sigma_M$ es la sección eficaz de ionización de un blanco 
molecular, $n_e$ es el número de electrones activos por molécula dado
en la tabla~\ref{nn}, y el parámetro es $\alpha=1.2$. La 
figura~\ref{fig:zalpha} muestra los valores teóricos y experimentales
de $\tilde{\sigma}$ (dados por la ecuación~\ref{eq:u-scaling}) para todos 
los sistemas presentados en las figuras~\ref{fig:crossDNA_1} y 
\ref{fig:crossDNA_2}. Como se observa, la regla de escala combinada 
funciona muy bien y es independiente tanto de la naturaleza del ion 
incidente como de la complejidad del blanco molecular. Nuestros 
resultados teóricos se ubican en una banda estrecha válida para cualquier
ion incidente (reducida con $Z^\alpha$) en cualquier molécula
(escalada con el número de electrones activos) con una dispersión de 
aproximadamente $\pm 20\%$. Si consideramos los valores experimentales
disponibles, la incertidumbre de nuestra escala independiente es 
crece a $\pm 30\%$, la cual es esquematizada en la 
figura~\ref{fig:zreduced} con un área gris. Notar que no hemos incluido 
en esta figura los resultados para uracilo de las 
referencias~\cite{agnihotri2012,agnihotri2013}. 

\begin{figure}[t]
\centering
\includegraphics[width=0.9\textwidth]{ionmol/Zne_scaling.eps}
\caption[Sección eficaz de ionización reducida por $Z$ y $n_e$.]
{Sección eficaz de ionización reducida con la carga $Z$ del ion incidente
y escalada con el número de electrones activo $n_e$ del blanco molecular,
dado por la ecuación~(\ref{eq:u-scaling}) con $\alpha=1.2$. 
Curvas: resultados teóricos CDW-SSM. 
Símbolos: datos experimentales correspondientes a las 
figuras~\ref{fig:crossDNA_1} y \ref{fig:crossDNA_2}.}
\label{fig:zalpha}
\end{figure} 

Consideramos que la regla de escala independiente propuesta es válida 
para cualquier combinación ion--molécula. Para evaluar la generalidad 
de nuestro modelo, incluimos en la figura~\ref{fig:zreduced} un grupo 
de datos experimentales correspondientes a blancos moleculares no 
considerados en el diseño de esta regla, tales como las mediciones de
Rudd~\textit{et al.}~\cite{Rudd:85,Rudd:83} para H$^{+}$ y He$^{+2}$ 
en N$_2$, O$_2$, CH$_4$, CO y CO$_2$, y los recientes experimentos de
Luna~\textit{et al.} \cite{Luna2019} de H$^{+}$ en CH$_4$. 

El buen acuerdo entre los resultados previstos por nuestro modelo y los
datos experimentales disponibles que se muestran en la 
figura~\ref{fig:zalpha} resume los principales resultados de este 
trabajo. Nuestro escaleo independiente muestra ser eficaz para predecir 
la ionización de los sistemas ion--blanco estudiados aquí sino también 
muestra potencial para reproducir una gran variedad de sistemas 
colisionales. Aunque los resultados teóricos CDW--SSM son válidos para
energías por debajo del máximo de la sección eficaz de ionización, se
puede observar en la figura~\ref{fig:zalpha} como el escaleo de los datos 
experimentales se extiene aún para valores de energía incidente menores.
Esperamos que mediciones experimentales para otros iones y moléculas 
refuercen el modelo y escaleo propuesto.

%%%%%%%%%%%%%%%%%%%%%%%%%%%%%%%%%%%%%%%%%%%%%%%%%%%%%%%%%%%%%%%%%%%%%%%%
\section{Estructura molecular de los blancos}
\label{sec:molcalculations}
%%%%%%%%%%%%%%%%%%%%%%%%%%%%%%%%%%%%%%%%%%%%%%%%%%%%%%%%%%%%%%%%%%%%%%%%

Finalmente, para probar el rango de validez del SSM, realizamos un 
cálculo molecular de primeros principios para cinco nucleobases 
empleando el código {\sc gamess}. La optimización de la geometría y los 
cálculos de energía de un centro se realizaron implementando el método
restringido de Hartree-Fock y el conjunto de bases 3-21G. 

\begin{figure}
\centering
\includegraphics[width=0.9\textwidth]{ionmol/levelsDNA.eps}
\caption[Energías de ligadura moleculares teóricas de ADN y ARN.]
{Energías de ligadura moleculares teóricas de adenina, citosina, guanina, 
timina, y uracilo, comparado con los valores correspondientes de los 
átomos que las constituyen.}
\label{fig:bindener}
\end{figure}

En la figura~\ref{fig:bindener} se muestran las energías de ligadura 
molecular de los electrones de valencia para las nucleobases: adenina, 
citosina, guanina, timina y uracilo. Las energías de ligadura del orbital 
molecular más alto (HOMO) obtenidos concuerda con los valores
experimentales~\cite{Hush,Verkin,Dougherty} en un 2\% para todas las 
moléculas consideradas. En el lado izquierdo de la figura~\ref{fig:bindener}, 
mostramos las energías atómicas de Hartree-Fock de los elementos 
constituyentes, lo que da una idea de la distribución de los electrones
débilmente ligados en las moléculas. Se traza una línea discontinua 
alrededor de $-26$~eV para separar la banda molecular en dos. Podemos 
considerar los niveles de energía atómica por encima de esta línea como 
los correspondientes a los electrones débilmente ligados de la 
ecuación~(\ref{eq:scalingCDW}). Por ejemplo, los electrones $2s$ y $2p$ 
del carbono se ubican por encima de la línea discontinua, que corresponde 
a los 4 electrones dados por la escala CDW. En el caso de O, solo 4 
electrones de los orbitales 2p se encuentran por encima de la línea 
divisoria propuesta, que se corresponde con el número de electrones
débilmente ligados dado por el escaleo CDW. El caso del átomo de N no 
es tan directo; el número $\nu_{N=4}^{\text{CDW}}$ sugiere que uno de 
los dos electrones de la capa $2s$ contribuye al esquema molecular.

%%%%%%%%%%%%%%%%%%%%%%%%%%%%%%%%%%%%%%%%%%%%%%%%%%%%%%%%%%%%%%%%%%%%%%%%
\subsection{Un modelo estequiométrico modificado}
%%%%%%%%%%%%%%%%%%%%%%%%%%%%%%%%%%%%%%%%%%%%%%%%%%%%%%%%%%%%%%%%%%%%%%%%

El modelo estequiometrico simple considera a la molécula como un 
conjunto de átomos neutrales aislados, lo cual es definitivamente irreal.
Una primera mejora se puede sugerir asumiendo que los átomos no son
efectivamente neutrales y que tienen una distribución dispar de los 
electrones dentro de la molécula, lo cual puede expresarse mediante una
carga efectiva $q_{\alpha}$ por átomo. La carga de Mulliken proporciona
un valor posible para $q_{\alpha}$; sin embargo, existe una gran variedad
de posibles distribuciones de carga~\cite{lee2003}.

\begin{table}
\begin{center}
\begin{tabular}{|p{0.12\textwidth}|p{0.08\textwidth}|p{0.08\textwidth}|p{
0.08\textwidth}|p{0.08\textwidth}|p{0.28\textwidth}|}
\hline
Molécula & C & H & N & O & Estequiometría modif. \\
\hline
Adenina & +0.32 & +0.23 & --0.55 &       & 
C$_{4.92}$H$_{4.77}$N$_{5.14}$ \\ 
\hline
Citosina & +0.28 & +0.21 & --0.56 & --0.53 & 
C$_{3.93}$H$_{4.79}$N$_{3.14}$O$_{1.13}$ \\ 
\hline
Guanina & +0.46 & +0.20 & --0.58 & --0.36 & 
C$_{4.89}$H$_{4.80}$N$_{5.15}$O$_{1.09}$ \\ 
\hline
Timina & +0.20 & +0.19 & --0.54 & --0.52 & 
C$_{4.95}$H$_{5.81}$N$_{2.13}$O$_{2.13}$ \\ 
\hline
Uracilo & +0.31 & +0.22 & --0.59 & --0.47 & 
C$_{3.92}$H$_{3.78}$N$_{2.15}$O$_{2.12}$ \\ 
\hline
\end{tabular}
\caption[Cargas efectivas medias de Mulliken por átomo]
{Cargas efectivas medias de Mulliken por átomo $q_{\alpha}$, y nueva formula
estequiométrica definida por la ecuación~(\ref{eq:newstoi}) para cinco
moléculas de ADN.}
\label{tab:newstoi}
\end{center}
\end{table}

Para tomar en cuenta este efecto, consideramos que el número total de 
electrones $Q_{\alpha }$ en el elemento $\alpha$ se distribuye de forma
dispar sobre todos los átomos $\alpha$. Por lo tanto, cada elemento  
$\alpha$ tendrá una carga adicional, $q_{\alpha}=Q_{\alpha}/n_{\alpha}$, 
que puede ser positiva o negativa. Esta valor dependerá de la 
electronegatividad relativa respecto a los otros átomos~\cite{rappe1991}. 
Siguiendo esta idea, podemos estimar el nuevo número de átomos por 
molécula $n_{\alpha }^{\prime }$, el cual está dado por
\begin{equation}
n_{\alpha }^{\prime }=n_{\alpha }-
\frac{q_{\alpha }}{\nu_{\alpha }^{\text{CDW}}}
\label{eq:newstoi}
\end{equation}
En el caso de átomos neutrales, $q_{\alpha}=0$ y 
$n_{\alpha}^{\prime}=n_{\alpha}$, como debería ser. En la 
tabla~\ref{tab:newstoi}, mostramos un valor promedio de carga efectiva por
átomo $q_{\alpha}$ de C, H, N, y O, para cinco moléculas del ADN,
las cuales se obtuvieron a partir de los cálculos moleculares descritos 
en la sección anterior.

Implementando la ecuación~(\ref{eq:newstoi}), es posible determinar
una nueva fórmula estequiométrica, la cual se da en la última columna de
la tabla~\ref{tab:newstoi}). Ahora, en vez de tener un número entero 
de átomos $n_{\alpha}$, tenemos un número fraccional dado por 
$n_{\alpha}^{\prime}$. Se pueden calcular nuevas secciones eficaces 
moleculares, $\sigma^{\prime}_{M}=\sum_{\alpha}n_{\alpha}'\sigma_{\alpha}$
considerando tales valores. Se calcularon errores relativos de las 
secciones eficaces de ionización para las bases de ADN de la 
tabla~\ref{tab:newstoi}. Las diferencias obtenidas fueron menores de
3\%, lo cual indica que el modelo estequiométrico simple 
es un modelo robusto con el que se pueden modelar este tipo de 
moléculas complejas dentro del margen de error esperado.

%%%%%%%%%%%%%%%%%%%%%%%%%%%%%%%%%%%%%%%%%%%%%%%%%%%%%%%%%%%%%%%%%%%%%%%%
\section{Conclusiones}
%%%%%%%%%%%%%%%%%%%%%%%%%%%%%%%%%%%%%%%%%%%%%%%%%%%%%%%%%%%%%%%%%%%%%%%%

En este capítulo, hemos estudiado la ionización de blancos moleculares 
de interés biológico debido al impacto de iones de carga múltple. Se han
calculado secciones eficaces de ionización de diesciseis blancos 
moleculares biological conteniendo H, C, N y O por el impacto de
antiprotones, H$^{+}$, He$^{2+}$, Be$^{4+}$, C$^{6+}$, y O$^{8+}$. 
Hemos combinado la implementación del método de onda distorcionada
CDW para blancos atómicos descritos con potenciales efectivos DIM y la 
estructura molecular fue modelada con un simple modelo estequiométrico.
También se estudiaron y calcularon los valores medios de energía y 
ángulo de emisión de los electrones ionizados de los blancos atómicos, 
que son de importancia en el daño secundario de las colisiones con iones.
Nuestros resultados muestran una clara dependencia de estos valores con 
la carga del ion $Z$. Para un blanco dado, a medida que $Z$ aumenta,
$\overline{E}_{\alpha}$ también aumentan. Por otro lado, 
$\overline{\theta}_{\alpha}$ decrece, lo que muestra una clara tendencia 
a la dirección del ion. A energía de impacto mayores a 2 MeV/amu, estos
valores convergen a la primera aproximación de Born, la cual establece 
la simple ley de escala $Z^{2}$. 

También se presentaron secciones eficaces de ionización total para 
adenina, citosina, timina, guanina, uracilo, THF y agua, siendo 
comparadas a su vez con datos experimentales disponibles. Proponemos 
tres reglas de escala para las secciones eficaces de ionización para 
estos blancos de interés biológico por cuatro iones desnudos. Primero,
exploramos la ampliamente usada regla de escala de Toburen, que escala 
la sección eficaz de ionización molecular con un determinado número de 
electrones débilmente ligados. Luego, a partir de los resultados CDW para 
los atómos H, C, N y O, encontramos que las secciones eficaces de 
ionización escalan mejor cuando se normalizan con diferentes números de 
electrones activos en la colisión. Así, definimos la primera regla de 
escala con el número de electrones activos CDW, la cual provee buenos 
resultados para los proyectiles y moléculas estudiadas aquí. La 
comparación con los datos experimentales refuerza nuestros hallazgos. 
Además, probamos la regla de escala CDW mediante la inclusión de datos 
experimentales de ionización de H$_2$, agua, metano y amoníaco por 
impacto de protones, los cuales muestran una buena concordancia en el 
rango de energías intermedias a altas.
La segunda regla de escala propuesta considera el ion incidente, 
reduciendo la naturaleza del proyectil incidente mediante el escaleo de 
la sección eficaz de ionización con la carga del ión, $Z^{\alpha}$, como
una función de la energía incidente reducida $E/Z^{2-\alpha}$, siendo 
$\alpha=1.2$. La última regla propuesta combina el escaleo de la sección 
eficaz de ionización total con el número de electrones activos CDW y la
reducción con la carga del ión, $Z^{\alpha}$, lo cual conduce a una ley
de escala independiente de la carga del ion y el blanco molecular. Las
tres reglas de escala obtenidas mediante nuestro modelo CDW--SSM para los 
64 sistemas colisionales examinados fueron comparadas con datos 
experimentales disponibles. La generalidad de nuestra regla de escala
independiente fue inspeccionada mediante la consideración de un número 
significativo de datos experimentales para otros sistemas colisionales 
(no considerados previamente), probando ésta ser válida incluso para 
energías fuera del rango de validez del método CDW--SSM.

Finalmente, realizamos cálculos moleculares completos para las 
nucleobases del ADN. Al inspeccionar las energías de ligadura 
moleculares obtenidas mediante cálculos mecanicocuánticos de primeros 
principios, pudimos comprender el número de electrones propuestos en 
nuestro nuevo escaleo basado en los cálculos CDW. Intentamos mejorar el 
modelo estequiométrico utilizando la carga de Mulliken para obtener 
proporciones fraccionarias en lugar de enteras. No encontramos una 
corrección sustancial, lo que indica que el SSM funciona bastante bien. 
En conclusión, los presentes resultados refuerzan la exactitud del modelo
estequiométrico simple para tratar con moléculas complejas en el rango 
de energía intermedia a alta. 

