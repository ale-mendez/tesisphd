\chapter{Ionización de moléculas biológicas: modelo estequiométrico}
\label{chap:ionmol}

\section{Introducción}

El interés sobre el estudio de la ionización de moléculas biológicas por 
el impacto de iones de carga múltiple ha crecido significativamente en
ciencias del ambiente~\cite{Gafur:18,FerrazDias:13} y 
medicina~\cite{Baskar:12,Solov:09}. En particular, debido a su 
aplicación en terapias contra el cáncer, y porque constituye el 
principal mecanismo de daño celular.

El estudio de blancos complejos, tales como las nucleobases, representa 
un gran desafío desde el punto de vista teórico. A lo largo de las 
últimas décadas, se ha propuesto una amplia variedad de aproximaciones 
teóricas con el fin de calcular la ionización de estos sistemas. Por 
ejemplo, el método de trayectorias clásicas 
Monte Carlo, en combinación con el criterio de sobrebarrera clásica, fue 
utilizado para estudiar la ionización por impacto de protones y 
partículas $\alpha$ en agua y bases del ADN y 
ARN~\cite{Abbas:08,Lekadir:09}. En el marco de la teoría cuántica, 
es común la utilización de la primera aproximación de Born~(\acs{fba}). 
Este método perturbativo proporciona excelentes resultados a altas 
energías. En esta región, las secciones eficaces se comportan con una 
dependencia de $Z^2$, donde $Z$ es la carga del proyectil incidente. El
problema de este método reside en que el daño causado por la ionización 
está concentrado en los alrededores del pico de Bragg --a energías de 
unos cientos de keV/amu--, donde la aproximación no es del todo 
apropiada. 

Una de las grandes dificultades del modelado de la ionización de 
moléculas biológicas está dada por la descripción de la estructura del 
blanco utilizando métodos de primeros principios. Las primeras 
aproximaciones a las funciones de onda de nucleobases de ADN y ARN se 
obtuvieron implementando los métodos de Hartree--Fock~(\acs{hf}), con 
geometría optimizada y expansión de un centro~\cite{DalCappello:08}, y 
de omisión completa de superposición diferencial~\cite{Champion:10}. En 
este último trabajo, la hipótesis principal se basa en el modelo de 
átomo independiente~(\acs{iam}), donde las secciones eficaces 
moleculares se obtienen mediante la combinación lineal de secciones 
eficaces atómicas con factores de peso moleculares. 

Las limitaciones de los métodos perturbativos de primer orden se superan 
considerando aproximaciones con correcciones de mayor orden. 
Por ejemplo, el trabajo de Galassi \textit{et al.} \cite{Galassi:00} 
modela con éxito la ionización de moléculas simples por impacto de 
protones utilizando el método de onda continua distorsionada con estado 
inicial de Eikonal (\acs{cdw-eis}) \cite{Fainstein:88,Miraglia:08,
Miraglia:09}. Esta aproximación constituye una de las teorías más 
exitosas para calcular procesos de ionización a energías intermedias y 
altas~\cite{Miraglia:08,Miraglia:09,Montanari:17-iongasesnobles}. 
La misma metodología fue usada para modelar la ionización de nucleobases 
de ADN debido al impacto de protones~\cite{Galassi:12} y de uracilo por 
iones de carbono, oxígeno y flúor~\cite{champion2012,agnihotri2012,
agnihotri2013}. Recientemente, L\"udde \textit{et al.}~\cite{Ludde:16,
Ludde:18,Ludde:19,Ludde:20} propusieron un modelo, también basado en el 
IAM, que combina de secciones eficaces atómicas con correcciones 
geométricas de apantallamiento. En este caso, las secciones eficaces 
atómicas se obtienen utilizando la teoría del functional densidad 
dependiente del tiempo. 

En este capítulo, el proceso colisional se calcula mediante el método 
CDW-EIS~\footnote{Por simplicidad, la aproximación CDW-EIS será referida 
como CDW}, mientras que para modelar los blancos se diseñan distintos 
modelos estequiométricos. Como resultados de estos cálculos, se postulan 
leyes de escala que permiten expresar las secciones eficaces de 
ionización molecular en forma independiente del 
(1) proyectil, del (2) blanco y del (3) sistema colisional, en un 
amplio rango de energías del proyectil incidente.

%%%%%%%%%%%%%%%%%%%%%%%%%%%%%%%%%%%%%%%%%%%%%%%%%%%%%%%%%%%%%%%%%%%%%%%%
\section{Ionización de átomos constituyentes}
\label{sec:atoms}
%%%%%%%%%%%%%%%%%%%%%%%%%%%%%%%%%%%%%%%%%%%%%%%%%%%%%%%%%%%%%%%%%%%%%%%%

En este trabajo, el modelo usado para describir la ionización de 
sistemas moleculares asume que las moléculas están compuestas por átomos 
aislados e independientes. Por ello, en primer lugar, se examina la 
ionización de los sistemas atómicos que componen los blancos biológicos 
de interés. La sección eficaz de ionización $\sigma_{\alpha}$ del átomo 
$\alpha$ se obtiene mediante la aproximación del método CDW en 
combinación con el método de inversión depurada (DIM). Como se explicó 
en el Capítulo~\ref{chap:iondim}, el DIM permite obtener potenciales 
efectivos. Con ellos se obtienen las funciones de onda radiales de los 
estados inicial (ligado) y final (continuo), resolviendo las 
ecuaciones de Schr\"odinger correspondientes. Por una cuestión de 
consistencia con los programas colisionales utilizados, los estados del 
blanco antes y después de la colisión se calculan introduciendo los 
potenciales DIM como entrada en el código~\textsc{radialf}, desarrollado 
por Salvat y colaboradores~\cite{salvat1995}.

\begin{figure}
\centering
\includegraphics[width=0.9\textwidth]{ionmol/atomicscaling.eps}
\caption[Sección eficaz total de ionización atómica CDW reducida.]
{Sección eficaz total de ionización reducida $\sigma_{\alpha}/Z^2$ 
de cuatro blancos atómicos relevantes. Curvas: cálculos teóricos CDW. 
Símbolos: datos experimentales de H$^+$ en H~\cite{Shah:81}, 
N~\cite{Rudd:85} y O~\cite{Rudd:85}; $e^-$ en 
H~\cite{Shah:87}, C~\cite{Brook:78}, N~\cite{Brook:78} y 
O~\cite{Thompson:95} con conversión de equivelocidad.}
\label{fig:atomscaling}
\end{figure} 

La mayor parte de las moléculas orgánicas contienen átomos de hidrógeno, 
carbono, nitrógeno y oxígeno. Así, las colisiones que se estudian en 
esta sección están compuestas por estos cuatro blancos 
atómicos y los proyectiles: antiprotones $\bar{p}$, H$^{+}$, He$^{2+}$, 
Be$^{4+}$, C$^{6+}$, y O$^{8+}$. Las secciones eficaces totales de 
ionización CDW-DIM de los 24 sistemas blanco--proyectil resultantes se 
muestran en la Fig.~\ref{fig:atomscaling}. 
A lo largo de este capítulo se usa el mismo color de línea para indicar 
la carga del proyectil en todas las figuras que muestran resultados de 
ionización por impacto de iones de carga múltiple: discontinua--roja, 
sólida--roja, azul, magenta, oliva y naranja para antiprotones, H$^{+}$, 
He$^{2+}$, Be$^{4+}$, C$^{6+}$, y O$^{8+}$, respectivamente. En el caso 
de los datos experimentales, se usan símbolos de los mismos colores para 
denotar la carga del ion incidente correspondiente y símbolos negros 
para electrones. En la primera aproximación de Born, la sección eficaz 
de ionización es proporcional al cuadrado de la carga del ion incidente, 
es decir $Z^{2}$. Como puede apreciarse en la figura, este 
comportamiento sólo es válido a altas energías (i.e. $E>5$~MeV/amu). La 
región energética de validez del método CDW se extiende desde $0.1$ 
hasta 10~MeV/amu. Para los proyectiles con cargas más altas, el mínimo 
valor de energía en el cual la aproximación tiene validez aumenta hasta 
400~keV. Los valores teóricos CDW se comparan con secciones eficaces 
experimentales disponibles en la literatura, tales como la ionización 
por impacto de H$^+$ en H~\cite{Shah:81}, N~\cite{Rudd:85} y 
O~\cite{Rudd:85}. Para ampliar la comparación, se incluyen mediciones de 
ionización por impacto de electrones en H~\cite{Shah:87}, 
C~\cite{Brook:78}, N~\cite{Brook:78} y O~\cite{Thompson:95}, para 
valores superiores a 300~eV. La teoría supone que las ionizaciones por 
impacto de H$^+$ y $e^-$ convergen para altas energías del proyectil. 
En la presente investigación se realizaron otros cálculos similares (no 
se muestran aquí), corroborando que la FBA provee resultados confiables 
para valores de energía mayores a unos cuantos MeV/amu. Los resultados 
de ionización CDW de cada capa de estos sistemas colisionales se pueden 
hallar en la Ref.~\cite{Miraglia:19}. Detalles sobre el cálculo se 
encuentran en la Ref.~\cite{Montanari:17-iongasesnobles}.


En un medio biológico dado, el proceso de ionización es el mecanismo que 
deposita la mayor cantidad de energía primaria en el sistema. Sin 
embargo, se conoce que los electrones residuales de la ionización son 
una fuente significativa de daño biológico local~\cite{Denifl:11}. En 
efecto, los electrones secundarios son incluidos en simulaciones de 
radiodosimetría~\cite{Champion:15,Quinto:17,Acocer-Avila:19}, y por lo 
tanto su comportamiento requiere especial atención. A continuación, se
examinan distribuciones energéticas y angulares medias de los electrones 
eyectados, calculadas con el método CDW.

%%%%%%%%%%%%%%%%%%%%%%%%%%%%%%%%%%%%%%%%%%%%%%%%%%%%%%%%%%%%%%%%%%%%%%%%
\subsection{Distribución energética de electrones}
\label{subsec:meanener}
%%%%%%%%%%%%%%%%%%%%%%%%%%%%%%%%%%%%%%%%%%%%%%%%%%%%%%%%%%%%%%%%%%%%%%%%

\begin{figure}
\centering
\includegraphics[width=0.9\textwidth]{ionmol/ener_mean.eps}
\caption[Distribución energética media de electrones emitidos.]
{Distribución energética media de electrones emitidos por la ionización 
debido al impacto de iones de carga múltiple dada por la 
Ec.~(\ref{eq:meanener}). Curvas: cálculos teóricos FBA (punteada) y CDW 
(sólidas y discontinua).}
\label{fig:emittedener}
\end{figure} 

La sección eficaz de ionización diferencial en función de la energía del 
electrón eyectado $E$ de la capa $nl$ del átomo $\alpha$ se puede 
considerar como una función de distribución simple~\cite{Surdutovic:18}. 
Así, siguiendo 
a Abril y colaboradores~\cite{Abril:15}, se define el valor medio de la 
energía de los electrones ionizados como
$\overline{E}_{\alpha}$, 
\begin{eqnarray}
\overline{E}_{\alpha} &=&\frac{\langle E_{\alpha}\rangle}{\sigma_{\alpha}}
=\frac{1}{\sigma_{\alpha}}\sum_{nl}\int\frac{\partial\sigma_{\alpha,nl}}{
\partial E}\,E\,dE\,, 
\label{eq:meanener} \\
\sigma_{\alpha}&=&\sum_{nl}\int\frac{\partial\sigma_{\alpha,nl}}{
\partial E}\,dE\,. 
\label{eq:normener}
\end{eqnarray}
donde $\Sigma_{nl}$ es la suma de las contribuciones de cada capa del 
elemento $\alpha$.

Las energías medias de los electrones emitidos $\overline{E}_{\alpha}$ 
de H, C, N y O por impacto de $\bar{p}$, H$^{+}$, He$^{2+}$, Be$^{4+}$, 
C$^{6+}$, y O$^{8+}$ se muestran en la Fig.~\ref{fig:emittedener}. 
Los valores CDW de $\overline{E}_{\alpha}$ de los electrones emitidos 
están en el rango de energía de 10 a 70 eV para todos los blancos 
atómicos. Estos resultados concuerdan con otros modelos 
teóricos~\cite{Surdutovic:18}. Como se puede observar en la figura, el 
valor de energía media es sensible a la carga del proyectil $Z$, que 
puede duplicar los resultados de protón en la región intermedia, i.e., 
100--400 keV/amu. El efecto observado puede atribuirse a la reducción en 
la emisión de electrones de baja energía por los iones de carga múltiple. 
Este comportamiento no aparece en la primera aproximación de 
Born, donde la ley de escala $Z^2$ cancela la dependencia con $Z$ de la 
Ec.~(\ref{eq:meanener}). A altas energías, $\overline{E}_{\alpha}$ 
tiende a un valor universal para todos los iones, como puede verse en la 
Fig.~\ref{fig:emittedener}.

%%%%%%%%%%%%%%%%%%%%%%%%%%%%%%%%%%%%%%%%%%%%%%%%%%%%%%%%%%%%%%%%%%%%%%%%
\subsection{Distribución angular de electrones}
\label{subsec:meanang}
%%%%%%%%%%%%%%%%%%%%%%%%%%%%%%%%%%%%%%%%%%%%%%%%%%%%%%%%%%%%%%%%%%%%%%%%

\begin{figure}
\centering
\includegraphics[width=0.9\textwidth]{ionmol/ang_mean.eps}
\caption[Distribución angular media de electrones emitidos.]
{Distribución angular media de electrones emitidos por la ionización 
debido al impacto de iones de carga múltiple dada por 
Ec.~(\ref{eq:meanang}). Curvas: cálculos teóricos FBA (punteada) y CDW 
(sólidas y discontinua).}
\label{fig:emittedang}
\end{figure} 

Como se estableció previamente, la emisión de electrones secundarios 
contribuye al daño total. Por lo tanto, no sólo es esencial conocer la 
distribución de energía de los electrones eyectados, sino también la 
dirección en la que éstos son emitidos. Una vez más, se puede considerar 
que la sección eficaz diferencial de ionización en función del ángulo 
sólido de eyección del electrón $\Omega$ puede expresarse como una 
función de distribución. Así, el ángulo medio de emisión 
$\overline{\theta}_{\alpha}$ se define como 
\begin{eqnarray}
\overline{\theta}_{\alpha}&=&\frac{\langle\theta_{\alpha}\rangle}
{\sigma_{\alpha}}=\frac{1}{\sigma_{\alpha}}\sum_{nl}
\int\frac{\partial\sigma_{\alpha,nl}}{\partial\Omega}\,\theta\,d\Omega\,,
\label{eq:meanang} \\
\sigma_{\alpha}&=&\sum_{nl}\int\frac{\partial\sigma_{\alpha,nl}}{
\partial\Omega}\,d\Omega\,.
\end{eqnarray}

Los ángulos medios de emisión electrónica $\overline{\theta}_{\alpha}$ 
de los cuatro átomos y seis iones estudiados en esta Sección se muestran 
en la Fig.~\ref{fig:emittedang}. Se puede observar una dependencia 
significativa de $\overline{\theta}_{\alpha}$ con $Z$ para todos los 
sistemas colisionales. Una vez más, este efecto no se observa en la 
implementación de la FBA (línea punteada), ya que sus resultados no 
dependen del valor de $Z$.

En la emisión de electrones a bajas energías, la dispersión angular es 
casi isotrópica~\cite{Rudd:92}. Un valor típico para el ángulo de 
eyección considerado en la literatura es 
$\overline{\theta}_{\alpha}\sim$~70\textdegree~\cite{Surdutovic:18}, el 
cual resulta bastante certero en el rango de validez de la FBA para 
cualquier blanco. Sin embargo, en el marco de la aproximación de onda 
distorsionada, $\overline{\theta}_{\alpha}$ disminuye sustancialmente 
con $Z$ en la región de energía intermedia. Como se observa en la 
Fig.~\ref{fig:emittedang}, cuanto mayor es la carga $Z$,
$\overline{\theta}$ es menor y el electrón es emitido hacia adelante. 
Por supuesto, este efecto es más notorio en el rango de energías 
intermedias que a altas energías.

Para ilustrar esta fenómeno, consideremos el impacto de C$^{6+}$ con una 
energía de 500~keV sobre oxígeno. En la primera aproximación de Born, la 
teoría estima emisión de electrones con energías medias de $46.7$ eV y 
ángulos medios de 78\textdegree. En cambio, en la aproximación CDW, los 
electrones eyectados tienen energías medias de $55.7$~eV y ángulos de 
emisión iguales a 60\textdegree. Estos resultados sugieren una 
penetración más profunda de los electrones secundarios que la FBA, con 
una orientación más cercana a la dirección del ion. La correción de la 
dirección de emisión se puede atribuir al efecto de captura del 
continuo, incluido en la CDW.

La Fig.~\ref{fig:emittedang} también proporciona una descripción 
ilustrativa del comportamiento de los antiprotones: el proyectil repele 
a los electrones a bajas energías, produciendo un ángulo de emisión 
medio $\overline{\theta}_{\alpha}\ge$~90\textdegree (emisión hacia 
atrás). Nótese la diferencia en el ángulo de emisión por impacto de 
protones y antiprotones dada por la CDW, con respecto a la primera 
aproximación de Born; este fenómeno constituye un efecto Barkas 
angular~\cite{Sigmud:03}.

%%%%%%%%%%%%%%%%%%%%%%%%%%%%%%%%%%%%%%%%%%%%%%%%%%%%%%%%%%%%%%%%%%%%%%%%
\section{El modelo estequiométrico}
\label{sec:SSM}
%%%%%%%%%%%%%%%%%%%%%%%%%%%%%%%%%%%%%%%%%%%%%%%%%%%%%%%%%%%%%%%%%%%%%%%%

El modelo estequiométrico simple (\acs{ssm}) que se propone para 
calcular secciones eficaces moleculares totales de ionización está 
basado en la aproximación de átomo independiente, también llamada regla 
aditiva de Bragg. Este modelo supone que los átomos que componen una 
molécula $M$ interactúan con el proyectil incidente pero no entre sí. 
Así, si la molécula $M$ está compuesta por $n_{\alpha}$ átomos del 
elemento $\alpha$, el modelo estequiométrico aproxima la sección eficaz 
total de ionización de la molécula $\sigma_M$ como la suma de secciones 
eficaces totales de ionización de los átomos aislados 
$\sigma_{\alpha}$ ponderada por $n_{\alpha}$, 
\begin{equation}
\sigma_{M}=\sum_{\alpha} n_{\alpha}\,\sigma_{\alpha}\,.  
\label{eq:sumion}
\end{equation}
Los blancos moleculares que se examinan a lo largo de este capítulo se 
clasifican en tres familias: CH, CHNO, y ADN, como se muestra en la 
Tabla~\ref{tab:families}.

\begin{table}
\begin{center}
\begin{tabularx}{\textwidth}{
>{\centering\arraybackslash}p{0.3\textwidth}
>{\centering\arraybackslash}p{0.3\textwidth}
>{\centering\arraybackslash}p{0.3\textwidth}}
\rowcolor{mydarkgray} 
CH & CHNO & DNA \\
CH$_4$ (metano) & C$_5$H$_5$N (piridina)       & C$_5$H$_5$N$_5$ (adenina) \\
\rowcolor{mygray} 
C$_2$H$_2$ (acetileno) & C$_4$H$_4$N$_2$ (pirimidina)     & C$_4$H$_5$N$_3$O (citosina) \\
C$_2$H$_4$ (eteno)     & C$_2$H$_7$N     (dimetilamina)   & C$_5$H$_5$N$_5$O (guanina) \\
\rowcolor{mygray} 
C$_2$H$_6$ (etano)     & CH$_5$N         (monometilamina) & C$_5$H$_6$N$_2$O$_2$ (timina) \\
C$_6$H$_6$ (benceno)   & C$_4$H$_8$O     (THF)            & C$_4$H$_4$N$_2$O$_2$ (uracilo) \\
\rowcolor{mygray} 
 & & H$_2$O (agua) \\
\end{tabularx}
\caption[Blancos moleculares examinados y clasificados en tres 
familias.]
{Blancos moleculares de interés examinados en el presente trabajo y 
clasificados en tres familias.}
\label{tab:families}
\end{center}
\end{table}

\begin{figure}
\centering
\includegraphics[width=0.9\textwidth]{ionmol/adn1.eps}
\caption[Sección eficaz total de ionización reducida por $Z$ (Parte I).]
{Sección eficaz total de ionización reducida $\sigma_{M}/Z^2$ como 
una función de la energía de impacto del ion. Curvas: cálculos teóricos 
SSM--CDW. Símbolos: datos experimentales de impacto de 
protón~\cite{Iriki:11}, 
C$^{4+}$~\cite{Sens:20}, C$^{6+}$~\cite{Bhattacharjee:19}, y 
e$^-$~\cite{Rahman:16} con conversión de equivelocidad.}
\label{fig:crossDNA_1}
\end{figure} 

En la Fig.~\ref{fig:crossDNA_1} se reportan las secciones eficaces de 
ionización totales reducidas con la carga del ion incidente, 
$\sigma_M/Z^2$, de las nucleobases del ADN --adenina, citosina, guanina 
y timina-- debido al impacto de iones de carga múltiple, que se obtienen 
empleando de la aproximación SSM y el método CDW. Para adenina, los 
cálculos teóricos tienen un excelente acuerdo con los datos 
experimentales disponibles para el impacto de protones~\cite{Iriki:11}. 
Los cálculos del modelo SSM--CDW coinciden con las mediciones de 
C$^{4+}$~\cite{Sens:20} sobre adenina, dentro del margen de error. Si 
bien estos datos son preliminares (publicados recientemente en la 
Ref.~\cite{Sens:20}), la 
comparación es alentadora en cuanto a la respuesta del modelo para la 
ionización por proyectiles de carga múltiple. En cambio, los resultados 
teóricos de ionización de adenina debido a C$^{6+}$ discrepan con el 
valor experimental disponible~\cite{Bhattacharjee:19} en un factor 2. 
Esta discrepancia es intrigante ya que la medición se encuentra en el 
rango de altas energías, donde aún la FBA modela muy bien a los 
experimentos.

No se han encontrado en la bibliografía datos experimentales de 
ionización por impacto de iones de carga múltiple para el resto de las 
nucleobases de ADN. En la Fig.~\ref{fig:crossDNA_1} se incluyen 
mediciones de ionización por impacto de electrones~\cite{Rahman:16}, con 
la correspondiente conversión de equivelocidad, para energías incidentes 
superiores a 300~eV. Los cálculos SSM--CDW producen resultados por 
debajo de los valores experimentales. Sin embargo, estos concuerdan muy
bien con otros modelos teóricos~\cite{mozejko2003,tan2018}. Más aún, 
cálculos recientes de Zein y colaboradores~\cite{Zein:21}, que emplean 
el modelo de encuentro binario de Bethe, sostienen esta tendencia.

\begin{figure}
\centering
\includegraphics[width=0.9\textwidth]{ionmol/adn2.eps}
\caption[Sección eficaz total de ionización reducida por $Z$ 
(Parte II).]
{Sección eficaz total de ionización reducida $\sigma_{M}/Z^2$ como 
una función de la energía de impacto del ion. 
Curvas: cálculos teóricos SSM--CDW. 
Símbolos: datos experimentales de impacto de protón en 
uracilo~\cite{itoh2013}, 
pirimidina~\cite{wolff2014}, THF~\cite{wang2016} y agua~\cite{Luna2007,
Bolorizadeh86,H_Rudd85,toburen80}. Impacto de C$^{4+}$~\cite{Sens:20} y 
C$^{4+}$, C$^{6+}$, O$^{6+}$, F$^{6+}$, O$^{8+}$, y F$^{8+}$ en 
uracilo~\cite{agnihotri2012,agnihotri2013}. Ionización de agua por 
impacto de He$^{2+}$~\cite{Ohsawa05,He_Rudd85,toburen80}, 
C$^{6+}$~\cite{DalCappello:09,Bhattacharjee:17} y 
O$^{8+}$~\cite{Bhattacharjee:16}. 
Impacto de e$^-$~\cite{bug2017,wolf2019,fuss2009} con conversión de 
equivelocidad.}
\label{fig:crossDNA_2}
\end{figure} 

Las secciones eficaces de ionización total reducidas $\sigma_M/Z^2$ 
para uracilo, pirimidina, THF y agua se muestran en la 
Fig.~\ref{fig:crossDNA_2}. En uracilo, los resultados SSM--CDW tienen 
buen acuerdo con las mediciones experimentales de ionización por impacto 
de protones de Itoh~\textit{et al.}~\cite{itoh2013}. Sin embargo, para 
el mismo blanco, los resultados teóricos difieren en un factor 2 de las 
mediciones experimentales de 
Agnihotri \textit{et al.}~\cite{agnihotri2012,agnihotri2013} para el 
impacto de iones de carga múltiple. No obstante, cabe señalar que los 
valores teóricos coinciden con los cálculos de Champion, 
Rivarola y colaboradores~\cite{agnihotri2012,champion2012}. Los cálculos 
de Sarkadi~\cite{sarkadi2016}, que emplean el método de trayectorias 
clásicas Monte Carlo, resultan amplitudes mayores a estos valores 
experimentales. En la figura también se incluye una reciente medición 
experimental preliminar de Sens \textit{et al.}~\cite{Sens:20} para la 
ionización en uracilo debido al impacto de C$^{4+}$, que coincide con la 
teoría. Si bien los resultados son preliminares, estas discrepancias 
podría indicar algún problema con los valores experimentales de 
Agnihotri y colaboradores. 

Para la ionización de pirimidina, los cálculos teóricos se comparan con 
los datos experimentales de ionización por impacto de protones de Wolff 
\textit{et al.}~\cite{wolff2014} y por impacto de 
electrones~\cite{bug2017}. El modelo SSM--CDW concuerda con las 
mediciones de electrones incidentes a energías superiores a 500~keV. Sin 
embargo, los cálculos sobrestiman significativamente las valores 
experimentales de protones. Para la la molécula de THF, se encuentran 
disponibles una mayor cantidad de resultados experimentales de 
ionización por impacto de H$^+$~\cite{wang2016} y de 
e$^-$~\cite{bug2017,wolf2019,fuss2009}. Los resultados que se obtienen 
de la combinación del SSM y las secciones eficaces atómicas CDW muestran 
un buen acuerdo general con estas mediciones.

En el caso de la molécula de agua, los datos experimentales son más 
abundantes que para el resto de los blancos. Las curvas teóricas 
coinciden con estos valores~\cite{Luna2007,Bolorizadeh86,H_Rudd85,
Ohsawa05,He_Rudd85,toburen80,Bhattacharjee:16} para todos los iones de 
carga múltiple examinados aquí a excepción de
C$^{6+}$~\cite{DalCappello:09,Bhattacharjee:17}, donde la teoría 
sobrestima la ionización en un factor 2 en el rango de altas energías. 
Estas discrepancias podrían suponer fallas en el experimento. 
A diferencia de las nucleobases, pirimidina y THF, la estequiometría del 
agua es más simple; sin embargo, observamos que nuestro modelo responde 
bien incluso para calcular la ionización de moléculas simples debido a 
la incidencia de iones pesados, tal como es el caso de O$^{6+}$.

%%%%%%%%%%%%%%%%%%%%%%%%%%%%%%%%%%%%%%%%%%%%%%%%%%%%%%%%%%%%%%%%%%%%%%%%
\section{Reglas de escala}
\label{sec:scaling}

%%%%%%%%%%%%%%%%%%%%%%%%%%%%%%%%%%%%%%%%%%%%%%%%%%%%%%%%%%%%%%%%%%%%%%%%
\subsection{Escala con electrones activos del blanco}
\label{subsec:ne_scaling}

%%%%%%%%%%%%%%%%%%%%%%%%%%%%%%%%%%%%%%%%%%%%%%%%%%%%%%%%%%%%%%%%%%%%%%%%
\subsubsection{Regla de Toburen}
\label{subsec:toburen}
%%%%%%%%%%%%%%%%%%%%%%%%%%%%%%%%%%%%%%%%%%%%%%%%%%%%%%%%%%%%%%%%%%%%%%%%

Toburen y colaboradores~\cite{Toburen:75,Toburen:76} propusieron el 
primer modelo fenomenológico completo y simple para la eyección de 
electrones de moléculas complejas. Los autores encontraron conveniente 
escalar la sección eficaz de ionización experimental en términos del 
número $n_e$ de electrones ligados débilmente, también llamados activos. 
En general, $n_e$ está dado por la cantidad de electrones en la capa de 
valencia. Así, por ejemplo, para C, N y O, este valor es igual al número 
total de electrones del átomo menos el número de electrones en la capa 
K. 

Siguiendo a Toburen \textit{et al.}, se define la sección eficaz de 
ionización reducida por el número de electrones débilmente ligado 
$\sigma_e$ para la molécula $M$ como
\begin{equation}
\sigma_e\equiv\frac{\sigma_M}{n_e}\,, 
\label{eq:cross-ne} 
\end{equation}
donde $n_e$ es el número total de electrones activos en el proceso 
colisional, 
\begin{equation}
n_e=\sum_{\alpha}n_{\alpha}\,\nu_{\alpha}\,,
\end{equation} 
tal que $n_{\alpha}$ el número de atómos $\alpha$ que componen la 
molécula y $\nu_{\alpha}$ el número de electrones débilmente ligado en 
cada átomo $\alpha$. Los números que asigna Toburen para los blancos 
atómicos considerados en este trabajo están dados por 
\begin{equation}
\nu_{\alpha}^T=\left\{ 
\begin{array}{ll}
1, & \text{para H,} \\
4, & \text{para C,} \\ 
5, & \text{para N,} \\ 
6, & \text{para O}\,.
\end{array}\right.
\label{eq:neToburen} 
\end{equation} 

La regla de Toburen se puede enunciar diciendo que $\sigma_{e}$ es un 
parámetro universal, que depende únicamente de la velocidad y naturaleza
del ion incidente. Esta regla es aplicable en la región de $0.25$ a 
5~MeV/amu. A muy altas energías, la probabilidad de ionizar electrones 
de la capa K es relevante y el número total de electrones activos será, 
por consiguiente, diferente. Una dependencia similar con el número de 
electrones débilmente ligados fue enunciada por Itoh y 
colaboradores~\cite{itoh2013} 
para el impacto de protones sobre uracilo y adenina.

Para comprobar la validez de la ley de escala de Toburen, las secciones 
eficaces CDW son divididas por el número de electrones activos de 
Toburen, dando lugar a las secciones eficaces reducidas $\sigma_{e}^T$, 
para los blancos moleculares que se enlistan en la 
Tabla~\ref{tab:families}. Los resultados de esta aproximación se 
muestran en la Fig.~\ref{fig:newscaling}(a) en función de la energía de 
impacto, para diferentes proyectiles de carga múltiple. La figura 
verifica que la regla de Toburen se cumple a altas energías. Sin 
embargo, dado que la banda de resultados es muy ancha, no se puede 
considerar a su desempeño como satisfactorio.

\begin{figure}[t]
\centering
\includegraphics[width=0.9\textwidth]{ionmol/CDWscaling.eps}
\caption[Sección eficaz de ionización molecular reducida por $n_e$.]
{Sección eficaz de ionización molecular reducida con el número de 
electrones débilmente ligados usando 
(a)~los números de Toburen $\nu_{\alpha}^T$, y 
(b) los números efectivos $\nu_{\alpha}'$ propuestos para las 
moléculas enlistadas en la Tabla~\ref{tab:families}. 
Símbolos: datos experimentales de las Figs.~\ref{fig:crossDNA_1} y 
\ref{fig:crossDNA_2}.}
\label{fig:newscaling}
\end{figure}

%%%%%%%%%%%%%%%%%%%%%%%%%%%%%%%%%%%%%%%%%%%%%%%%%%%%%%%%%%%%%%%%%%%%%%%%
\subsubsection{Números efectivos}
\label{subsec:CDW}
%%%%%%%%%%%%%%%%%%%%%%%%%%%%%%%%%%%%%%%%%%%%%%%%%%%%%%%%%%%%%%%%%%%%%%%%

Para verificar la regla de Toburen en los cálculos SSM--CDW, 
en la Fig.~\ref{fig:neCDW} se ilustran las secciones eficaces de 
ionización CDW de los átomos H, C, N y O debido al impacto de H$^+$ 
reducidas por el número de electrones activos de cada átomo. La 
Fig.~\ref{fig:neCDW}(a) muestra los cálculos de ionización CDW divididos 
por los números de Toburen dados por la Ec.~(\ref{eq:neToburen}). La 
regla de Toburen no se verifica de manera apropiada por los cálculos 
CDW: para un valor de energía dado, los valores teóricos de ionización 
reducidos $\sigma_{\alpha}/\nu_{\alpha}^T$ de cada blanco atómico no son 
constantes. Por otro lado, considerando los valores teóricos de 
ionización de oxígeno dados en la Fig.~\ref{fig:atomscaling}, se observa 
que estos son muy similares a las secciones eficaces del carbono. Estos 
cálculos parecen sugerir que el número de electrones activos en el 
átomo de O es 4 en lugar de 6. De la misma manera, el número de 
electrones activos para N, que se obtiene de los resultados teóricos 
CDW, es diferente al sugerido por Toburen~(\ref{eq:neToburen}). 
La Fig.~\ref{fig:neCDW}(b) muestra los nuevos resultados, que se 
obtienen asumiendo los valores corregidos de $\nu_{\alpha}$. El estudio 
sistemático de los resultados CDW para el resto de los proyectiles 
verifica una mejora a la regla de Toburen. Así, el número total de 
electrones activos de la molécula $M$ en la Ec.~(\ref{eq:cross-ne}) está 
ahora dado por 
\begin{equation}
n_e'=\sum_{\alpha}n_{\alpha}\,\nu_{\alpha}'\,,
\label{eq:neprima}
\end{equation}
donde $\nu_{\alpha}'$ son los números de electrones activos 
por átomo, que se obtienen de ajustar las secciones eficaces de 
ionización CDW atómicas. Estos números efectivos están dados por
\begin{equation}
\nu_{\alpha }' \sim\left\{ 
\begin{array}{ll}
1, & \text{para H,} \\
4, & \text{para C, N, y O}\,. \\ 
\end{array}
\right. 
\label{eq:neCDW}
\end{equation}

\begin{figure}[t]
\centering
\includegraphics[width=0.9\textwidth]{ionmol/neCDW.eps}
\caption[Sección eficaz de ionización atómica reducida por $n_e$.]
{Sección eficaz de ionización atómica por impacto de H$^+$ reducida con 
el número de electrones débilmente ligados usando 
(a)~los números de Toburen $\nu_{\alpha}^T$, y 
(b) los números efectivos $\nu_{\alpha}'$.}
\label{fig:neCDW}
\end{figure}

Empleando la Ec.~(\ref{eq:neCDW}), se definen nuevos números de 
electrones activos $n_e'$ para las moléculas consideradas hasta ahora.  
La Tabla~\ref{tab:ne_molecules} muestra los valores optimizados $n_e'$ y 
los números de Toburen $n_e$ para una gran variedad de sistemas 
moleculares. La principal diferencia entre los números de Toburen y los 
números efectivos que se proponen aquí se debe a la cantidad de 
electrones activos del oxígeno.
Las secciones eficaces moleculares divididas por $n_e'$ se muestran en 
la Fig.~\ref{fig:newscaling}(b). Se incluyen los datos experimentales 
para la ionización de 
adenina~\cite{Iriki:11,Sens:20,Bhattacharjee:19}, 
uracilo~\cite{itoh2013,Sens:20}, 
pirimidina~\cite{wolff2014}, 
THF~\cite{wang2016} y 
agua~\cite{Luna2007,Bolorizadeh86,H_Rudd85,toburen80,Ohsawa05,He_Rudd85,
DalCappello:09,Bhattacharjee:17,Bhattacharjee:16} por impacto de iones 
de carga múltiple. También se ilustran mediciones de ionización por 
impacto de electrones con conversión de equivelocidad en 
pirimidina~\cite{bug2017} y THF~\cite{bug2017,wolf2019,fuss2009}. 
La comparación con los valores experimentales validan la optimización 
propuesta en esta Tesis. 

\begin{table}
\begin{center}
\begin{tabularx}{\textwidth}{
>{\centering\arraybackslash}p{0.13\textwidth}
>{\centering\arraybackslash}p{0.17\textwidth}
>{\centering\arraybackslash}p{0.04\textwidth}
>{\centering\arraybackslash}p{0.04\textwidth}
>{\centering\arraybackslash}p{0.13\textwidth}
>{\centering\arraybackslash}p{0.17\textwidth}
>{\centering\arraybackslash}p{0.04\textwidth}
>{\centering\arraybackslash}p{0.04\textwidth}}
\rowcolor{mydarkgray} 
Molécula        & Nombre      & $n_e$ & $n_e'$ & 
Molécula        & Nombre      & $n_e$ & $n_e'$ \\
H$_2$           & Dihidrógeno & 2      & 2     & 
C$_4$H$_4$N$_2$ & Pirimidina  & 30     & 28    \\
\rowcolor{mygray} 
H$_2$O          & Agua        & 8      & 6     & 
C$_6$H$_6$      & Etano       & 30     & 30    \\
NH$_3$          & Amoníaco    & 8      & 7     & 
C$_4$H$_4$N$_2$O$_2$ & Uracilo & 40    & 36    \\
\rowcolor{mygray} 
CH$_4$          & Metano      & 8      & 8     & 
C$_4$H$_5$N$_3$O & Citosina   & 42     & 37    \\
CH$_5$N         & Metilamina  & 14     & 13    & 
C$_5$H$_6$N$_2$O$_2$ & Timina & 48     & 42    \\
\rowcolor{mygray} 
C$_2$H$_7$N     & Etilamina   & 20     & 19    & 
C$_5$H$_5$N$_5$ & Adenina     & 50     & 45    \\
C$_4$H$_8$O     & THF         & 30     & 28    & 
C$_5$H$_5$N$_5$O & Guanina    & 56     & 49    \\ \\
\end{tabularx}
\caption[Números de electrones activos moleculares de Toburen y CDW.]
{Número de electrones débilmente ligados según Toburen $n_e$ y 
optimización CDW~$n_e'$ para algunos blancos moleculares de interés 
biológico.}
\label{tab:ne_molecules}
\end{center}
\end{table}

\begin{figure}[t]
\centering
\includegraphics[width=0.9\textwidth]{ionmol/scale_ne.eps}
\caption[Ionización por impacto de protón en términos de $n_e'$.]
{Sección eficaz de ionización por impacto de protón a $0.5$, 1, y 2~MeV 
en términos del número de electrones activos dado por la 
Tabla~\ref{tab:ne_molecules}. Curvas: cálculos teóricos SSM--CDW. 
Símbolos: datos experimentales de 
\mbox{\Large$\circ$}~adenina~\cite{Iriki:11}, 
$\triangle$ uracilo~\cite{itoh2013}, 
$\bigtriangledown$ pirimidina~\cite{wolff2014}, 
$\blacktriangle$ C$_2$H$_7$N, CH$_5$N, metano y amoníaco~\cite{Lynch:76},
\mbox{\scriptsize$\bigstar$} amoníaco y H$_2$~\cite{Rudd:85}, y 
\mbox{\Large$\bullet$} agua~\cite{Luna2007}.}
\label{fig:recta}
\end{figure}

La escala definida por los números efectivos se puede examinar de forma 
alternativa dibujando las secciones eficaces de ionización de las 
moléculas en función de $n_e'$ para ciertos valores de energía. Los 
resultados SSM--CDW se muestran en la Fig.~\ref{fig:recta} para 
energías de impacto de $0.5$, 1 y 2~MeV. Como puede observarse, las 
secciones eficaces de ionización calculadas para todas las moléculas 
presentan una dependencia lineal con el número de electrones efectivos 
$n_e'$ de la Tabla~\ref{tab:ne_molecules}. Se obtienen resultados 
similares incluso para $E=10$~MeV (no los incluimos por claridad en la 
figura). La comparación con los datos experimentales disponibles muestra 
una buena concordancia general, desde moléculas simples~\cite{Lynch:76,
Rudd:85,Luna2007}, tales como H$_2$, CH$_4$ y NH$_3$, hasta las más 
complejas~\cite{Iriki:11,itoh2013,wolff2014}, como la adenina. Para los 
datos de ionización por impacto de electrones, los valores 
experimentales se interpolaron entre vecinos cercanos. 

Será interesante verificar la regla de escala optimizada que se propone
aquí con experimentos futuros, principalmente para estados de carga de 
proyectiles más altos. Por lo pronto, incluso los cálculos recientes 
de Vera \textit{et al.}~\cite{deVera:20} para la ionización de 
biomateriales por impacto de electrones confirman satisfactoriamente
esta regla de escala en todo el rango de energías. Por otro lado, en el 
reciente trabajo de L\"udde~\textit{et al.}~\cite{Ludde:19} sobre la 
ionización de moléculas biológicas por impacto de protones, los autores 
hallaron los mismos valores de escala para N y O. El acuerdo con este 
modelo independiente refuerza la ley de escala dada por los números 
efectivos de electrones activos.

%%%%%%%%%%%%%%%%%%%%%%%%%%%%%%%%%%%%%%%%%%%%%%%%%%%%%%%%%%%%%%%%%%%%%%%%
\subsection{Escala con la carga del ion}
\label{sec:zscaling}
%%%%%%%%%%%%%%%%%%%%%%%%%%%%%%%%%%%%%%%%%%%%%%%%%%%%%%%%%%%%%%%%%%%%%%%%

A energías de impacto intermedias, la regla $Z^2$ no es válida y se 
deben considerar otras escalas. En la literatura, se encuentran dos 
tipos de leyes de escala con la carga $Z$ del ion incidente, aplicables 
en este rango de energías de impacto. La regla sugerida por Janev y 
Presnyakov~\cite{Janev:80} considera $\sigma/Z$ en función de $E/Z$ como 
la forma reducida \textit{natural} de la sección eficaz de ionización 
$\sigma$ y la energía de ion incidente $E$. Más recientemente, 
Montenegro y colaboradores~\cite{Dubois:13,Montenegro:13} propusieron 
una expresión alternativa, la cual sugiere que la sección eficaz es una 
función de $Z^2/E$ a altas energías,
\begin{equation}
\frac{\sigma}{Z^{\alpha}}=f\left(\frac{E}{Z^{2-\alpha}}\right),
\label{eq:Montenegro}
\end{equation}
que mantiene la relación $Z^2/E$ para cualquier valor del parámetro 
$\alpha$. Los autores proponen el valor $\alpha=4/3$ para la ionización 
de He y H$_2$ debido al impacto de diversos iones 
cargados~\cite{Dubois:13}. 

Siguiendo el trabajo de Montenegro y colaboradores, en el presente 
trabajo se optimiza el parámetro~$\alpha$ según los valores que brinda 
el modelo SSM--CDW. Como resultado de dicho ajuste, se obtiene un valor
de $\alpha=1.2$. La validez de este resultado es apreciable en las 
Figs.~\ref{fig:zreduced-adn1} (adenina, citosina, timina y guanina) 
y~\ref{fig:zreduced-adn2} (uracilo, pirimidina, THF y agua), donde 
--para cada blanco-- las curvas SSM--CDW correspondientes a los 
diferentes iones se superponen. Es notable como 
el escaleo de los resultados teóricos SSM--CDW es válido incluso para 
energías de impacto cercanas al máximo de las secciones eficaces, que 
corresponden a rangos de energías incidentes desde 50 keV/amu para H$^+$ 
hasta 250 keV/amu para O$^{+8}$.

\begin{figure}
\centering
\includegraphics[width=0.9\textwidth]{ionmol/adn1_zscale.eps}
\caption[Sección eficaz de ionización reducida por $Z$ y $\alpha$ 
(Parte I).]
{Sección eficaz de ionización reducida $\sigma/Z^{\alpha}$ como función
de la energía incidente del ion $E/Z^{2-\alpha}$ con $\alpha=1.2$. 
Curvas: resultados teóricos SSM--CDW. 
Símbolos: datos experimentales de la Fig.~\ref{fig:crossDNA_1}.}
\label{fig:zreduced-adn1}
\end{figure} 

\begin{figure}
\centering
\includegraphics[width=0.9\textwidth]{ionmol/adn2_zscale.eps}
\caption[Sección eficaz de ionización reducida por $Z$ y $\alpha$ 
(Parte II).]
{Sección eficaz de ionización reducida $\sigma/Z^{\alpha}$ como función
de la energía incidente del ion $E/Z^{2-\alpha}$ con $\alpha=1.2$. 
Curvas: resultados teóricos SSM--CDW. 
Símbolos: datos experimentales de la Fig.~\ref{fig:crossDNA_2}.}
\label{fig:zreduced-adn2}
\end{figure} 

Los datos experimentales disponibles para los sistemas ion-blanco bajo 
estudio~\cite{Iriki:11,Sens:20,Bhattacharjee:19,itoh2013,wolff2014,
wang2016,agnihotri2012,agnihotri2013,Luna2007,Bolorizadeh86,H_Rudd85,
He_Rudd85,toburen80,Ohsawa05,Bhattacharjee:17,DalCappello:09,
Bhattacharjee:16} también son examinados con la regla de escala 
$Z^\alpha$. Al igual que en las figuras previas, en el caso de los 
blancos con pocos o ningún dato experimental se incluyen las secciones 
eficaces experimentales de ionización por impacto de 
electrones~\cite{Rahman:16,bug2017,wolf2019,fuss2009} a grandes 
velocidades con la conversión correspondiente. Como se observa, la 
mayoría de los datos experimentales en la Figs.~\ref{fig:zreduced-adn1} 
y \ref{fig:zreduced-adn2} confirman el escaleo sugerido aquí, incluso 
para O$^{+8}$ en agua~\cite{Bhattacharjee:16}. 

%%%%%%%%%%%%%%%%%%%%%%%%%%%%%%%%%%%%%%%%%%%%%%%%%%%%%%%%%%%%%%%%%%%%%%%%
\subsection{Escala con electrones activos y carga del ion}
\label{sec:nez_scaling}
%%%%%%%%%%%%%%%%%%%%%%%%%%%%%%%%%%%%%%%%%%%%%%%%%%%%%%%%%%%%%%%%%%%%%%%%

Considerando la reducción con la carga del ion incidente $Z^\alpha$ y  
el número de electrones activos del blanco $n_e'$, se introduce la 
sección eficaz de ionización molecular independiente $\tilde{\sigma}$, 
que se expresa como función de $\tilde{E}=E/Z^{2-\alpha}$, y está dada 
por 
\begin{equation}
\tilde{\sigma}\left(\tilde{E}\right)=\frac{\sigma_e}{Z^{\alpha}}
=\frac{\sigma_M}{n_e'\,Z^{\alpha}}\,,
\label{eq:u-scaling}
\end{equation}
donde $\sigma_M$ es la sección eficaz de ionización de un blanco 
molecular, $\alpha=1.2$ es el parámetro de ajuste y $n_e'$ es el número 
de electrones activos de la molécula dado por las 
Ecs.~(\ref{eq:neprima}) y (\ref{eq:neCDW}). La Fig.~\ref{fig:zalpha} 
muestra los valores teóricos y experimentales de $\tilde{\sigma}$ para 
todos los sistemas moleculares de la Tabla~\ref{tab:families}. Como se 
observa, la regla de escala combinada permite expresar los cálculos 
teóricos y mediciones de forma independiente, tanto de la naturaleza del 
ion incidente, como de la complejidad del blanco molecular. Los 
resultados SSM--CDW se ubican en una banda estrecha válida para 
cualquier ion incidente (reducida con $Z^\alpha$) en cualquier 
molécula (contraída con el número de electrones activos $n_e'$) con una 
dispersión de aproximadamente $\pm 20\%$. Los resultados teóricos 
SSM--CDW permiten sugerir una expresión paramétrica
\begin{equation}
\Sigma(E)= \frac{a_0}{E} \ln \left( a_1 E - a_2 \right)\,,
\end{equation}
donde $a_0=0.04541$, $a_1=105.3$ y $a_2=2.314$, que ajusta los valores 
teóricos y experimentales con una dispersión del $\pm 35\%$. La curva 
paramétrica se muestra en la figura con una línea discontinua, mientras 
que la dispersión se ilustra con un área gris. Nótese que no hemos 
incluido en esta figura los resultados para uracilo de las 
Refs.~\cite{agnihotri2012,agnihotri2013}. 

\begin{figure}[t]
\centering
\includegraphics[width=0.9\textwidth]{ionmol/Zne_scaling.eps}
\caption[Sección eficaz de ionización reducida por $Z$ y $n_e$.]
{Sección eficaz de ionización reducida con la carga $Z$ del ion 
incidente y con el número de electrones activo $n_e$ del blanco 
molecular, dado por la Ec.~(\ref{eq:u-scaling}) con $\alpha=1.2$. 
Curvas: cálculos teóricos SSM--CDW (líneas sólidas) y ajuste paramétrico 
(línea discontinua). Símbolos: datos experimentales de las 
Figs.~\ref{fig:crossDNA_1} y \ref{fig:crossDNA_2}.}
\label{fig:zalpha}
\end{figure} 

Bajo la hipótesis de que la regla de escala independiente propuesta aquí 
es válida para cualquier combinación ion--molécula, se aplica la regla 
universal a un conjunto de valores experimentales no considerados en la
formulación de nuestro modelo. En la Fig.~\ref{fig:zalpha} se muestran 
las mediciones reducidas de Rudd~\textit{et al.}~\cite{Rudd:85,Rudd:83} 
para H$^{+}$ y He$^{+2}$ en N$_2$, O$_2$, CH$_4$, CO y CO$_2$, y los 
recientes experimentos de Luna~\textit{et al.} \cite{Luna2019} de 
H$^{+}$ en CH$_4$. Los valores experimentales que se incluyen en la 
figura están dentro de la incertidumbre provista por la regla general.

El buen acuerdo entre los resultados dados por la escala independiente y 
los datos experimentales disponibles que se ilustran en la 
Fig.~\ref{fig:zalpha} resume los principales resultados de este 
capítulo. El modelo muestra ser eficaz no solo para calcular la 
ionización de los sistemas ion--blanco estudiados aquí sino también 
para reproducir una gran variedad de sistemas colisionales. Aunque los 
resultados teóricos SSM--CDW son válidos para energías mayores a 
aquellas donde ocurre el máximo de la sección eficaz de ionización, 
se puede observar en la Fig.~\ref{fig:zalpha} que los datos 
experimentales verifican la regla de escala aún para valores de energía 
menores. Se espera que nuevas mediciones experimentales para otros 
iones y moléculas verifiquen el modelo teórico propuesto.

%%%%%%%%%%%%%%%%%%%%%%%%%%%%%%%%%%%%%%%%%%%%%%%%%%%%%%%%%%%%%%%%%%%%%%%%
\section{Estructura molecular de los blancos}
\label{sec:molcalculations}
%%%%%%%%%%%%%%%%%%%%%%%%%%%%%%%%%%%%%%%%%%%%%%%%%%%%%%%%%%%%%%%%%%%%%%%%

Finalmente, para estudiar el rango de validez del SSM, se calcula la 
estructura molecular de las cinco nucleobases --adenina, citosina, 
guanina, timina y uracilo-- empleando el código 
{\sc gamess}~\cite{gamess}. Los 
cálculos de energía se realizaron implementando el método restringido de 
Hartree--Fock con optimización de geometría y el conjunto de bases 
Gaussianas 3-21G. 

\begin{figure}[t]
\centering
\includegraphics[width=0.9\textwidth]{ionmol/levelsDNA.eps}
\caption[Energías de ligadura moleculares teóricas de ADN y ARN.]
{Energías de ligadura moleculares teóricas de adenina, citosina, 
guanina, timina, y uracilo, comparado con los valores correspondientes 
de los átomos que las constituyen.}
\label{fig:ADNbindener}
\end{figure}

En la Fig.~\ref{fig:ADNbindener} se muestran las energías de ligadura 
molecular de los electrones de valencia para las nucleobases de ADN y 
uracilo. Las energías de ligadura del orbital molecular más alto (HOMO) 
que se obtienen concuerdan con los valores 
experimentales~\cite{Hush,Verkin,Dougherty} en un 2\% para todas las 
moléculas consideradas. En la izquierda de la Fig.~\ref{fig:ADNbindener}, 
se muestran las energías atómicas de Hartree-Fock de los elementos 
constituyentes. Esta comparación proporciona un esquema general sobre la 
distribución de los electrones débilmente ligados en las moléculas. Se 
traza una línea discontinua alrededor de $-26$~eV para separar la banda 
molecular en dos. Los niveles de energía atómica por encima de esta 
línea pueden considerarse como los correspondientes a los electrones 
débilmente ligados de la Ec.~(\ref{eq:neCDW}). Por ejemplo, los 
electrones $2s$ y $2p$ del carbono se ubican por encima de la línea 
discontinua, que corresponde a los cuatro electrones dados por los 
números efectivos. En el caso de oxígeno, 
solo los cuatro electrones de los orbitales 2p se encuentran por encima 
de la línea divisoria propuesta, que se corresponde con el número de 
electrones débilmente ligados dado por los valores efectivos. El caso 
del átomo de nitrógeno no es tan directo; el número $\nu_{N}'=4$ 
sugiere que sólo uno de los dos electrones de la capa $2s$ contribuye a
la estructura molecular.

%%%%%%%%%%%%%%%%%%%%%%%%%%%%%%%%%%%%%%%%%%%%%%%%%%%%%%%%%%%%%%%%%%%%%%%%
\subsection{Modelo estequiométrico modificado}
%%%%%%%%%%%%%%%%%%%%%%%%%%%%%%%%%%%%%%%%%%%%%%%%%%%%%%%%%%%%%%%%%%%%%%%%

El modelo estequiométrico propuesto en la Sección~\ref{sec:SSM} 
considera a la molécula $M$ como un conjunto de átomos neutros aislados, 
lo cual es completamente irreal. Se puede sugerir una primera mejora al 
modelo asumiendo que los átomos no son efectivamente neutros y que 
tienen una distribución electrónica dispar dentro de la molécula. Esta 
característica puede expresarse mediante una carga efectiva $q_{\alpha}$ 
por átomo. Entre la gran variedad de distribuciones de 
carga~\cite{lee2003} existentes, en este nuevo modelo se considera la 
carga de Mulliken.

\begin{table}
\begin{center}
\begin{tabularx}{\textwidth}{
>{\centering\arraybackslash}p{0.15\textwidth}
>{\centering\arraybackslash}p{0.08\textwidth}
>{\centering\arraybackslash}p{0.08\textwidth}
>{\centering\arraybackslash}p{0.08\textwidth}
>{\centering\arraybackslash}p{0.08\textwidth}
>{\centering\arraybackslash}p{0.32\textwidth}}
\rowcolor{mydarkgray} 
Molécula & C & H & N & O & Estequiometría de carga \\
Adenina & $+0.32$ & $+0.23$ & $-0.55$ &       & 
C$_{4.92}$H$_{4.77}$N$_{5.14}$ \\ 
\rowcolor{mygray} 
Citosina & $+0.28$ & $4+0.21$ & $-0.56$ & $-0.53$ & 
C$_{3.93}$H$_{4.79}$N$_{3.14}$O$_{1.13}$ \\ 
Guanina & $+0.46$ & $+0.20$ & $-0.58$ & $-0.36$ & 
C$_{4.89}$H$_{4.80}$N$_{5.15}$O$_{1.09}$ \\ 
\rowcolor{mygray} 
Timina & $+0.20$ & $+0.19$ & $-0.54$ & $-0.52$ & 
C$_{4.95}$H$_{5.81}$N$_{2.13}$O$_{2.13}$ \\ 
Uracilo & $+0.31$ & $+0.22$ & $-0.59$ & $-0.47$ & 
C$_{3.92}$H$_{3.78}$N$_{2.15}$O$_{2.12}$ \\ 
\end{tabularx}
\caption[Cargas efectivas medias de Mulliken por átomo]
{Cargas efectivas medias de Mulliken por átomo $q_{\alpha}$, y nueva 
fórmula estequiométrica definida por la Ec.~(\ref{eq:newstoi}) para 
cinco moléculas de ADN.}
\label{tab:newstoi}
\end{center}
\end{table}

Para tomar en cuenta este efecto, se considera que el número total 
de electrones $Q_{\alpha }$ en el elemento $\alpha$ se distribuye de 
forma dispar sobre todos los átomos $\alpha$. Por lo tanto, cada 
elemento $\alpha$ tendrá una carga $q_{\alpha}=Q_{\alpha}/n_{\alpha}$ 
asociada, que puede ser positiva o negativa. Este valor dependerá de la 
electronegatividad relativa respecto a los otros 
átomos~\cite{rappe1991}. 
Siguiendo esta idea, se puede estimar el número fraccional de átomos por 
molécula $n_{\alpha}'$, el cual está dado por 
\begin{equation}
n_{\alpha }^{\prime }=n_{\alpha }-\frac{q_{\alpha }}{\nu_{\alpha }'}
\label{eq:newstoi}
\end{equation}
En el caso de átomos neutrales, $q_{\alpha}=0$ y 
$n_{\alpha}'=n_{\alpha}$, como dispone el SSM. En la 
Tabla~\ref{tab:newstoi}, se muestra un valor promedio de carga efectiva 
por átomo $q_{\alpha}$ para C, H, N, y O en las cinco nucleobases, que 
se obtienen de los cálculos de estructura molecular descriptos 
en la sección anterior.

Implementando la Ec.~(\ref{eq:newstoi}), es posible determinar una nueva 
fórmula estequiométrica de carga, que se muestra en la última columna de 
la Tabla~\ref{tab:newstoi}). Ahora, en lugar de tener un número entero 
de átomos $n_{\alpha}$, se tiene un número fraccional dado por 
$n_{\alpha}'$. Considerando estos valores, se calculan nuevas secciones 
eficaces moleculares, 
$\sigma'_{M}=\sum_{\alpha}n_{\alpha}'\sigma_{\alpha}$. Los errores 
relativos de las secciones eficaces de ionización con el SSM modificado 
para las bases de ADN de la Tabla~\ref{tab:newstoi} presentan 
diferencias menores al 3\% respecto a los valores previos. Esta 
comparación sugiere que el modelo estequiométrico es un modelo simple 
pero robusto, capaz de modelar exitosamente este tipo de moléculas 
complejas.


