\chapter{Ionización de átomos y moléculas: método de inversión depurada}
\label{chap:iondim}

%%%%%%%%%%%%%%%%%%%%%%%%%%%%%%%%%%%%%%%%%%%%%%%%%%%%%%%%%%%%%%%%%%%%%%%%
\section{Introducción}
%%%%%%%%%%%%%%%%%%%%%%%%%%%%%%%%%%%%%%%%%%%%%%%%%%%%%%%%%%%%%%%%%%%%%%%%

Los procesos inelásticos que surgen debido a la interacción de un 
proyectil con la materia se estudian de forma extensa en la teoría 
cuántica. A grandes rasgos, los métodos desarrollados para calcular 
estos procesos se pueden clasificar en dos grupos: las aproximaciones 
perturbativas, entre las que se destacan las de 
Born~\cite{Bates:62,McDowell:61} y de onda 
distorsionada~\cite{Crothers:10,Rivarola:87}, y el grupo de los llamados 
métodos completamente cuánticos o ``exactos''~\cite{Pindzola:07,
Burke:11,Bray:17,Zatsarinny:04,McCurdy:04}. 

Usualmente, la estructura de sistemas atómicos multielectrónicos se 
obtiene de resolver las ecuaciones de Schr\"odinger empleando el modelo 
de partículas independientes en conjunción con la aproximación de campo 
central~\cite{Bransden:03,Cowan:81}. En este marco, el cálculo de 
procesos inelásticos simples se realiza asumiendo la aproximación de 
electrón activo. En la ionización, el electrón activo está inicialmente 
ligado y después de la colisión se encuentra libre. La descripción de los 
estados ligados es relativamente simple mientras que la representación 
de los continuos presenta cierta dificultad. Se han desarrollado 
diversas metodologías para el diseño de potenciales efectivos en blancos 
atómicos~\cite{Hibbert:82,Gombas:56,Green:69,Klapisch:71,Phillips:59,
Herman:63,Dalgarno:70,Bayliss:77,Cowan:76,Lee:77} y 
moleculares~\cite{Menchero:10,Granados:16}. Es conveniente contar con un 
potencial efectivo local ya que permite obtener en forma consistente las 
funciones de onda de las partículas interactuantes antes y después de la 
colisión.

En este capítulo se estudia la estructura de átomos y moléculas simples 
mediante potenciales efectivos que resultan de la implementación del 
\textbf{Método de Inversión Depurada}~(\acs{dim})~\cite{Mendez:16,
Mendez:18,Mendez:19dim}. Este método consiste en obtener potenciales 
centrales mediante soluciones conocidas del blanco. El DIM es general 
y se puede aplicar a soluciones que surgen de diversas aproximaciones. 
Con el fin de ilustrar su implementación, las soluciones que se utilizan 
en este trabajo están dadas por la teoría de Hartree--Fock~(\acs{hf}). 

En este capítulo, el objetivo principal es ilustrar el uso efectivo de 
los potenciales DIM en la teoría de colisiones atómicas. Para ello, se 
examina la ionización de blancos atómicos y moleculares debido al 
impacto de fotones e iones. Tanto el desarrollo teórico del DIM como su 
implementación numérica y aplicación en diversos sistemas 
multielectrónicos fueron realizadas originalmente por la autora y 
colaboradores. 

%%%%%%%%%%%%%%%%%%%%%%%%%%%%%%%%%%%%%%%%%%%%%%%%%%%%%%%%%%%%%%%%%%%%%%%%
\section{Método de inversión depurada (DIM)}
%%%%%%%%%%%%%%%%%%%%%%%%%%%%%%%%%%%%%%%%%%%%%%%%%%%%%%%%%%%%%%%%%%%%%%%%
\label{sec:dimatomos}

La ecuación de Schr\"odinger de un sistema de $N$ electrones y carga $Z$ 
en la aproximación de campo central está dada por
\begin{equation}
 \left[ -\frac{1}{2}\frac{d^2}{dr^2} + \frac{l(l+1)}{2r^2} +
 V(r) \right] u_{nl}(r) = \varepsilon_{nl} \, u_{nl}(r)\,,
\label{eq:eqSchroRadial}
\end{equation}
donde $V(r)$ es el potencial que gobierna la dinámica, $u_{nl}$ es la 
función radial reducida y $\varepsilon_{nl}$ la energía del orbital 
$nl$. 
El método de inversión consiste en resolver el problema inverso de la 
Ec.~(\ref{eq:eqSchroRadial}): suponiendo que las soluciones $u_{nl}$ y 
$\varepsilon_{nl}$ se conocen, es posible definir un \textit{potencial 
invertido} que las generan
\begin{equation}
V_{nl}(r) = 
\frac{1}{2}\frac{1}{u_{nl}(r)} \frac{d^2\,u_{nl}(r)}{dr^{2}} - 
\frac{l(l+1)}{2r^{2}}+\varepsilon_{nl} \,.
\label{eq:Vinv}
\end{equation}
Este procedimiento se ha estudiado en la teoría del funcional de la 
densidad~(\acs{dft} por sus siglas en inglés), en donde se invierte la
ecuación de Kohn--Sham y las densidades del estado 
fundamental~\cite{Wu:03,Gaiduk:13,Ryabinkin:15,Schipper:97,deSilva:12,
Kananenka:13,Jacob:11}. El método de inversión también fue sugerido por 
Hilton~\textit{et al.} para cálculo de procesos de 
fotoionización~\cite{Hilton:77,Suzer:77,Hilton:79,Hilton:80,Crljen:87}, 
los cuales se basaron en investigaciones previas sobre polarizabilidad 
atómica~\cite{Sternheimer:54,Dalgarno:59}. Sin embargo, estos trabajos 
se enfocan en los resultados de secciones eficaces y no presentan 
detalles acerca de la calidad de los potenciales y las funciones de onda 
resultantes. 

El método de inversión depurada (DIM) asume que el potencial invertido 
tiene una forma Coulombiana, y es conveniente definir una \textit{carga 
invertida} tal que
\begin{equation}
Z_{nl}(r) \equiv -r \, V_{nl}(r) \,.
\label{eq:Zinv}
\end{equation}
El comportamiento general del potencial y de su correspondiente carga 
efectiva se ilustran en la Fig.~\ref{fig:potycharge}; la carga deberá 
ser suave y cumplir con condiciones de borde definidos por la naturaleza 
del blanco a describir: 
en el origen la carga debe ser igual a la carga nuclear del átomo y 
asintóticamente, debido al apantallamiento electrónico, ésta debe ser
igual a uno. Esto es,
\begin{equation}
Z_{nl}(r) \, \rightarrow 
\bigg\{ 
\begin{array}{ll}
Z  \ \  & \ \ \text{:\ \ }r  \rightarrow 0 \,, \\ 
1           & \ \ \text{:\ \ }r  \rightarrow \infty \,.
\end{array}
\label{eq:Zasympt}
\end{equation} 
La carga invertida se ajusta con una expresión analítica que cumple con 
estas condiciones de borde. En átomos, la carga DIM está dada por
\begin{equation}
Z_{nl}^{\mathrm{DIM}}(r)= \sum_{j=1}^{n} z_j e^{-\alpha_j r}+1 \,,
\label{eq:atomzDIM}
\end{equation}
donde $\Sigma_j z_j=Z-1$. Los parámetros $\left\{z_j,\alpha_j\right\}$ 
definen un potencial de prueba que se optimiza hasta reproducir las 
soluciones iniciales $u_{nl}$ y $\varepsilon_{nl}$ de manera precisa. 
Debido a la presencia de la función de onda en el denominador de la 
Ec.~(\ref{eq:Vinv}), es posible que las cargas invertidas presenten 
problemas numéricos. Una forma de evitarlos es restringiendo la región 
de ajuste, descartando comportamientos tales como polos o divergencias. 

\begin{figure}[t]
\centering
\includegraphics[width=0.9\textwidth]{dim/pot-charge.eps}
\caption[Características físicas del potencial y carga efectiva.]
{Ilustración de las características físicas esperadas del (a) potencial 
y (b) carga efectiva para el átomo de carga nuclear $Z$.}
\label{fig:potycharge}
\end{figure}

La mayoría de los métodos de funcional de la densidad están basados en 
un principio variacional que minimiza la energía. Sin 
descartar su importancia, la energía es sólo uno de los observables que 
caracteriza un estado cuántico. Más aún, usando diferentes funciones de 
prueba (de formas variadas) e implementando un método variacional, es 
posible reproducir la misma energía final. Por ejemplo, Bartschat 
\textit{et al.}~\cite{Albright:93,Bartschat:96} muestran que dos 
potenciales diferentes (uno conteniendo intercambio electrónico y otro 
despreciándolo) conducen a energías similares y precisas de la serie de 
Rydberg, en varios sistemas de cuasi-un electrón. Sin embargo, estos 
potenciales conducen a grandes discrepancias cuando se implementan en 
cálculos de dispersión \cite{BartschatBray:96}. Por lo tanto, además de 
los valores de energía, se incluyen en la optimización DIM valores 
medios de $\langle r^k \rangle$. La inclusión de estos 
observables permite caracterizar la precisión del orbital DIM cerca 
($k=-1$) y lejos ($k=1$) del origen. Así, la función de costo $J$ que se 
define para la optimización de los potenciales DIM está dada por los 
errores relativos de cada una de estas cantidades
\begin{equation}
J=\sum_{i=1}^3 \frac{X_i-\widebar{X}_i}{X_i}\,,
\label{eq:fncosto-dim}
\end{equation}
donde la suma se hace sobre 
$X=\left[\varepsilon,\langle r \rangle,\langle 1/r \rangle\right]$, 
siendo $X$ los valores conocidos y $\widebar{X}$ los resultados que se 
obtienen de resolver la Ec.~(\ref{eq:eqSchroRadial}) con el potencial 
paramétrico. 

La variación del conjunto de parámetros $\left\{z_j,\alpha_j\right\}$ no 
es sistematizable mediante métodos convencionales debido a tres 
características del problema. En primer lugar, el hiper-espacio de 
parámetros que se define consta de típicamente de una decena de 
dimensiones. Por lo tanto, la implementación de algunas técnicas de 
optimización, tales como búsqueda de grilla, resultan numéricamente 
impracticables. Además, la función de costo no está dada por una 
expresión analítica, por lo que las técnicas basadas en el gradiente de 
la función también se descartan. Finalmente, la hiper-superficie 
definida por el costo no es convexa, entonces, la aplicación de métodos 
convencionables, que no dependen de la derivada, generalmente sólo 
encuentran mínimos locales. Por estos motivos, el ajuste de los 
parámetros sólo puede hacerse en forma manual, lo que  
requiere habilidad, pericia y experiencia. Actualmente, la autora y 
colaboradores se encuentran trabajando en incorporar al DIM métodos de 
optimización usados en el campo del aprendizaje 
automatizado~\cite{DiFilippo:19}.

%%%%%%%%%%%%%%%%%%%%%%%%%%%%%%%%%%%%%%%%%%%%%%%%%%%%%%%%%%%%%%%%%%%%%%%%
\subsection{DIM y soluciones Hartree--Fock}
%%%%%%%%%%%%%%%%%%%%%%%%%%%%%%%%%%%%%%%%%%%%%%%%%%%%%%%%%%%%%%%%%%%%%%%%
\label{subsec:invHF}

El método de inversión depurada es general y puede implementarse con 
cualquier tipo de soluciones. Con el fin de ilustrar su aplicación, en 
esta presentación se usan funciones de onda y energías obtenidas 
mediante la teoría de Hartree--Fock. La utilización del método HF 
presenta una gran ventaja: la teoría se conoce en detalle y se logran 
soluciones con alta precisión. De esta forma, puede hacerse una 
comparación estricta con los resultados del DIM, verificando la calidad 
de los potenciales que proporciona el método.

\begin{figure}[t]
\centering
\includegraphics[width=0.9\textwidth]{dim/dim_2sK.eps} 
\caption[Orbital radial y carga efectiva correspondiente.]
{(a) Orbital radial $u_{2s}^{\mathrm{HF}}$ del estado fundamental de K.
(b) Cargas invertida $Z_{2s}^{\mathrm{HF}}$ (línea discontinua) 
y depurada $Z_{2s}^{\mathrm{DIM}}$ (línea sólida).}
\label{fig:2sK}
\end{figure}

A pesar de que el procedimiento de inversión dado por la 
Ec.~(\ref{eq:Vinv}) es directo, su implementación mediante soluciones HF 
no produce, en general, cargas invertidas suaves. Por ejemplo, en la 
Fig.~\ref{fig:2sK} se muestra (a)~el orbital $u_{2s}^{\mathrm{HF}}$ del 
átomo de potasio en su estado fundamental y (b)~su correspondiente carga 
invertida $Z_{2s}^{\mathrm{HF}}$ (línea discontinua). También se muestra 
la carga $Z_{2s}^{\mathrm{DIM}}$ (línea sólida), que se obtiene luego de 
ajustar la carga invertida mediante la Ec.~(\ref{eq:atomzDIM}). El 
orbital $2s$ tiene dos nodos: uno de ellos, en $r\approx 0.111$~a.u., es
genuino, mientras que el nodo en \mbox{$r\approx 5.79$~a.u.} es espurio. 
Se usa el término genuino para denotar los nodos que cumplen la relación 
del número cuántico radial $n_r=n-l-1$. Los otros nodos (espurios) no 
han suscitado demasiada atención ya que aparecen a grandes distancias, 
en regiones donde la amplitud del orbital es despreciable (trataremos en 
detalle este problema al final del capítulo). Dado que la función de 
onda se encuentra en el denominador de la Ec.~(\ref{eq:Vinv}), estos 
nodos producen polos en la carga invertida. La inversión directa 
presenta otro problema numérico que la torna impracticable; 
$Z_{2s}^{\mathrm{HF}}$ tiene una divergencia pronunciada para grandes 
distancias radiales (en la figura, por ejemplo, para $r\geq 1$~a.u.). El 
origen de este problema se discutirá más adelante, en la 
Sección~\ref{sec:discusionHF}. 

\begin{figure}[t]
\centering
\begin{tikzpicture}[remember picture]
%  \tikzset{shift={(current page.center)}}
\node[process,fill=green!15] (inv) 
          {Inversión directa};
\node[process] (region) at (inv) [xshift=0cm,yshift=-1.5cm] 
          {Definición de región de ajuste};
\node[process] (eqnorm) at (region) [xshift=0cm,yshift=-1.5cm] 
          {Definición de semillas};
\node[process,text width=4.8cm] (diag) 
          at (eqnorm) [xshift=0cm,yshift=-2cm] 
          {Diagonalización y \\cálculo de
          valores medios};
\node[process] (costo) at (diag) [xshift=-2.5cm,yshift=-2.3cm] 
          {Cálculo de costo};
\node[process] (var) at (diag) [xshift=2.5cm,yshift=-2.3cm] 
          {Variación de parámetros};
\node[decision] (converge) at (costo) [xshift=-4cm,yshift=0cm] 
          {¿Convergió?};
\node[process,fill=blue!20] (dim) at (diag) [xshift=0cm,yshift=-5.2cm] 
          {Potencial DIM};
\draw[arrow] (inv) -- (region);
\draw[arrow] (region) -- (eqnorm);
\draw[arrow] (eqnorm) -- (diag);
\draw[arrow,bend right=33] (diag.west) 
                        to ([xshift=-0.75cm,yshift=0cm]{costo.north});
\draw[arrow,bend right=53] ([xshift=-0.3cm,yshift=0cm]{costo.south}) 
                        to ([xshift=0.3cm,yshift=0cm]{var.south});
\draw[arrow,bend right=33] ([xshift=0.75cm,yshift=0cm]{var.north}) 
                        to (diag.east);
\draw[arrow,dashed] (costo) -- (converge);
\draw[arrow,dashed] (converge) |- (region.west) node [near start,left] 
                    {No};
\draw[arrow,dashed] (converge) |- (dim.west) node [near start,right] 
                    {Sí};
\end{tikzpicture}
\caption{Procedimiento de optimización del potencial DIM.}
\label{fig:procDIM}
\end{figure}


El procedimiento general para la obtención de los potenciales DIM 
empleando orbitales HF se esquematiza en la Fig.~\ref{fig:procDIM}. Para 
el orbital $nl$ de un blanco dado, se implementa la Ec.~(\ref{eq:Vinv}). 
Luego, se define una región de ajuste sobre la carga invertida 
resultante. La clave de una optimización exitosa está dada por la 
correcta definición de esta región: tiene que ser lo más extensa 
posible, descartando por completo cualquier divergencia numérica. El 
paso siguiente en la optimización consiste en definir cuidadosamente una 
semilla inicial para los parámetros $\left\{z_j,\alpha_j\right\}$. Se 
sabe que en un proceso de optimización, la elección de estos valores es 
fundamental. En este trabajo, las semillas se obtienen mediante la 
resolución de la ecuación normal definida por el problema (ver detalles 
en Apéndice~\ref{app:ecnormal}). Los valores resultantes determinan un 
potencial de prueba con el que se resuelve la 
Ec.~(\ref{eq:eqSchroRadial}). Utilizando las soluciones de la 
diagonalización, se calculan los valores medios que definen la función 
de costo~(\ref{eq:fncosto-dim}), que se minimiza variando los parámetros 
en forma iterativa, hasta converger a los valores de partida (en este 
caso, los calculados con las funciones de HF). Si las soluciones DIM 
de~(\ref{eq:Vinv}) no convergen a los valores correctos, se define una 
nueva región de ajuste de la carga invertida y se reinicia el 
procedimiento.

%%%%%%%%%%%%%%%%%%%%%%%%%%%%%%%%%%%%%%%%%%%%%%%%%%%%%%%%%%%%%%%%%%%%%%%%
\subsection{Corolarios de DIM-HF}
%%%%%%%%%%%%%%%%%%%%%%%%%%%%%%%%%%%%%%%%%%%%%%%%%%%%%%%%%%%%%%%%%%%%%%%%
\label{sec:corolarios}

%=======================================================================
\subsubsection*{Energía total}
%=======================================================================

El método DIM permite derivar una expresión para el cálculo de la 
energía total del sistema, en términos del potencial y sus soluciones,
\begin{equation}
E^{\mathrm{DIM}} = \sum\limits_{nl} 
\left[ 
\varepsilon_{nl}^{\mathrm{DIM}} - 
\frac{1}{2}\int  \rho_{nl}^{\mathrm{DIM}}(r)
\left( V_{nl}^{\mathrm{DIM}}(r) + \frac{Z_{N}}{r}\right) dr \,
\right] \, ,
\label{eq:Etotal}
\end{equation}
donde la densidad es 
$\rho_{nl}^{\mathrm{DIM}}(r)=|u_{nl}^{\mathrm{DIM}}(r)|^2$. 
Como se ha visto, los potenciales DIM de un átomo se optimizan 
independientemente para cada orbital $nl$. Sin embargo, la energía total 
es una magnitud global, que refleja el desempeño colectivo de todas las 
optimizaciones individuales.

%=======================================================================
\subsubsection*{Potenciales de intercambio}
%=======================================================================

En la teoría de Hartree--Fock, es posible determinar el potencial de 
intercambio orbital de un blanco aplicando el operador de Fock. El 
potencial resultante no es local ya que depende del resto de los 
orbitales. La primera aproximación de potencial local fue propuesta por 
Slater~\cite{Slater:51}, en la cual se promedia la densidad de carga. 
Otra técnica, propuesta por Sharp y Horton~\cite{Sharp:53}, consiste en 
aproximar el operador de intercambio con un potencial local tal que, 
mediante un método variacional, se minimiza la energía. Desde entonces, 
se han desarollado diversos métodos más elaborados que permiten 
determinar potenciales de intercambio~\cite{Krieger:92,Gorling:92,
Yang:02,Staroverov:06,Ryabinkin:13}. Sin embargo, estos potenciales son 
difíciles de expresar mediante fórmulas analíticas simples y no se 
pueden implementar en problemas colisionales como los que se tratan en 
este trabajo.

Debido a que la teoría de Hartree--Fock incluye el término de 
intercambio electrónico de manera exacta, el método de inversión 
depurada permite definir potenciales ``exactos'' 
$V_{nl}^{\mathrm{x}}(r)$ para cada orbital $nl$. Suponiendo que el 
potencial DIM se puede expresar como
\begin{equation}
V_{nl}^{\mathrm{DIM}}(r) = -\frac{Z}{r} + V^{\mathrm{H}}(r) 
+ V_{nl}^{\mathrm{x}}(r) \, , 
\label{eq:VDIM}
\end{equation}
donde $V^{\mathrm{H}}$ es el potencial directo local debido a la 
repulsión electrostática electrónica en el esquema de HF y 
$V_{nl}^{\mathrm{x}}$ es el potencial de intercambio orbital. Entonces,
\begin{equation}
V_{nl}^{\mathrm{x}}(r)=V_{nl}^{\mathrm{DIM}}(r)+\frac{Z}{r}
-\int{ \frac{\rho^{\mathrm{HF}}(r^{\prime})  }
{\left| \mathbf{r} - \mathbf{r^{\prime}} \right|}} \, 
d \mathbf{r^{\prime}} \, ,
\label{eq:exchange-potential}
\end{equation}
donde $\rho^{\mathrm{HF}}$ es la densidad electrónica total que se
calcula con los orbitales HF.

%=======================================================================
\subsubsection*{Energías de intercambio}
%=======================================================================

Utilizando los potenciales de intercambio DIM, se puede definir la 
energía de intercambio total DIM $E^{\mathrm{x}}$ como
\begin{equation}
E^{\mathrm{x}} = \sum_{nl}\varepsilon_{nl}^{\mathrm{x}} = 
\sum_{nl}\left[\frac{1}{2}\int{\rho^{\mathrm{HF}}_{nl}(r) \, \, 
V_{nl}^{\mathrm{x}}}(r) \, dr \, \right]\,,
\label{eq:exchange-energy}
\end{equation}
donde $\varepsilon_{nl}^{\mathrm{x}}$ son las energías de intercambio 
correspondientes a cada orbital $nl$. La expresión de la energía de 
intercambio total DIM permite comparar la calidad de los 
potenciales~(\ref{eq:exchange-potential}) con valores de referencia. En 
definitiva, esta relación representa otra cantidad global con la que se 
puede evaluar cuantitativamente al método.

%%%%%%%%%%%%%%%%%%%%%%%%%%%%%%%%%%%%%%%%%%%%%%%%%%%%%%%%%%%%%%%%%%%%%%%%
\section{DIM en moléculas}
%%%%%%%%%%%%%%%%%%%%%%%%%%%%%%%%%%%%%%%%%%%%%%%%%%%%%%%%%%%%%%%%%%%%%%%%
\label{sec:dimmoleculas}

El método de inversión depurada puede extenderse para obtener 
potenciales efectivos de moléculas simples. Para ello, primero se 
define el método mediante el cual se calculan los orbitales.
Luego, se plantea el esquema de inversión de las soluciones y el método 
de depuración con el que se obtienen los potenciales moleculares. 

%%%%%%%%%%%%%%%%%%%%%%%%%%%%%%%%%%%%%%%%%%%%%%%%%%%%%%%%%%%%%%%%%%%%%%%%
\subsection{Descripción de moléculas}
%%%%%%%%%%%%%%%%%%%%%%%%%%%%%%%%%%%%%%%%%%%%%%%%%%%%%%%%%%%%%%%%%%%%%%%%
\label{sec:moleculas}

La descripción de la estructura electrónica de sistemas moleculares 
constituye un desafío desde el punto de vista teórico debido a su 
geometría multicéntrica~\cite{Helgaker:00,Schaefer:04}. 
En el marco de la aproximación de Born--Oppenheimer, el Hamiltoniano 
molecular no--relativista en el que sólo se consideran fuerzas 
Coulombianas puede escribirse como
\begin{equation}
H=-\sum_{i=1}^N \frac{1}{2} \nabla^2_{\mathbf{r}_i} 
  +\sum_{i<j=1}^N \frac{1}{\left|\mathbf{r}_i-\mathbf{r}_j\right|} 
  -\sum_{i=1}^N \sum_{\alpha=1}^n \frac{Z_{\alpha}}{
    \left|\mathbf{r}_i-\mathbf{r}_{\alpha}\right|} 
\label{eq:BOhamiltonian}
\end{equation}
donde los índices $i,j$ recorren los electrones de la molécula y 
$\alpha$ actúa sobre todos los núcleos. 
Teniendo en cuenta las mismas razones esgrimidas en el caso atómico, la 
ecuación de Schr\"odinger correspondiente, $H\Psi=E\Psi$, se resuelve 
implementando el método de Hartree--Fock. Los códigos computacionales 
más utilizados, emplean bases finitas para la representación de los 
orbitales moleculares. Estos orbitales se expresan como una combinación 
lineal de orbitales atómicos, 
tal que
\begin{equation}
\Psi_i(\mathbf{r})=\sum_j c_{ji} \, \phi_j(\mathbf{r})\,.
\end{equation}
A su vez, los orbitales atómicos $\phi(\mathbf{r})$ se construyen 
empleando conjuntos de base, por ejemplo, de orbitales tipo Gaussianos. 
Esta forma de construcción de los orbitales moleculares es conveniente 
ya que el producto de dos Gaussianas centradas en dos átomos diferentes 
resulta en una suma finita de Gaussianas centradas en un punto a lo 
largo del eje que las conecta. De esta manera, las integrales de 
múltiples centros se pueden simplificar considerablemente, lo que 
proporciona una gran ventaja computacional.

%%%%%%%%%%%%%%%%%%%%%%%%%%%%%%%%%%%%%%%%%%%%%%%%%%%%%%%%%%%%%%%%%%%%%%%%
\subsection{Inversión con bases Gaussianas}
%%%%%%%%%%%%%%%%%%%%%%%%%%%%%%%%%%%%%%%%%%%%%%%%%%%%%%%%%%%%%%%%%%%%%%%%
\label{sec:invmol}

En general, los orbitales moleculares se expanden alrededor de los 
núcleos atómicos que componen la molécula. Mediante la utilización de 
funciones Gaussianas esféricas, los orbitales moleculares se expanden en 
un único centro, y la expresión del potencial molecular invertido es 
análoga a la Ec.~(\ref{eq:Vinv}). 

La representación de los orbitales moleculares por conjuntos de base de 
orbitales tipo Gaussianos introduce nuevas dificultades numéricas en el 
procedimiento de inversión. Además de las divergencias asintóticas y los 
polos, los potenciales invertidos presentan prominentes 
oscilaciones~\cite{Schipper:97,Jacob:11,Gaiduk:13}. Estas 
características se deben a la presencia de ondulaciones en los 
orbitales, que surgen debido al número finito de elementos en la base, y 
son amplificadas por la derivada segunda de los orbitales, necesaria 
para obtener el potencial. En algunos casos, las oscilaciones en 
regiones cercanas a átomos electronegativos son enormes (por ejemplo, 
el cloro). La aparición de estas oscilaciones en los potenciales 
invertidos requiere del desarrollo de nuevas técnicas de depuración.

%%%%%%%%%%%%%%%%%%%%%%%%%%%%%%%%%%%%%%%%%%%%%%%%%%%%%%%%%%%%%%%%%%%%%%%%
\subsection{Depuración y perfiles de oscilación}
%%%%%%%%%%%%%%%%%%%%%%%%%%%%%%%%%%%%%%%%%%%%%%%%%%%%%%%%%%%%%%%%%%%%%%%%
\label{sec:invmol}

Los patrones de oscilación en los potenciales invertidos moleculares 
varían según el conjunto de base utilizado para representar las 
soluciones. Dado que los orbitales moleculares se expresan como 
combinación lineal de orbitales atómicos, es conveniente definir 
perfiles de oscilación por átomo, por base y por orbital. Para el átomo 
$\alpha$ que se describe implementando un conjunto de base (BS), el 
perfil de oscilación del orbital $nl$ está dado por
\begin{equation}
 p_{nl}^{\mbox{\scriptsize BS}} = Z_{nl}^{\mbox{\scriptsize BS}}-
 Z_{nl} \,,
 \label{eq:oscillation-prof}
\end{equation}
donde $Z_{nl}^{\mbox{\scriptsize BS}}$ es la carga invertida del orbital
$nl$ del átomo $\alpha$ cuando se usa el conjunto de bases ``BS'' y 
$Z_{nl}$ es la carga invertida correspondiente cuando los orbitales se 
obtienen mediante algún método de referencia en el cual desaparezcan 
(o en el cual disminuyan notoriamente) estas oscilaciones. La 
implementación de la Ec.~(\ref{eq:oscillation-prof}) requiere que las 
soluciones usadas en la inversión estén dadas por el mismo esquema de 
aproximación (por ejemplo, la teoría de Hartree--Fock). De esta manera, 
nos aseguramos que las oscilaciones provienen exclusivamente del 
carácter limitado de la base escogida.

Una vez definidos los $nl$ perfiles de oscilación atómicos de la base, 
estos se pueden utilizar directamente para eliminar las oscilaciones de 
los potenciales moleculares. Por ejemplo, para un hidruro XH$_n$, será  
necesario definir los perfiles de oscilación del átomo X y del 
hidrógeno. Luego, se implementa un procedimiento de depuración similar 
al que se utiliza en átomos. La diferencia en este caso, es que las 
expresiones analíticas para la carga deben tener en cuenta estructuras 
particulares de los compuestos. Por ejemplo, para hidruros,
\begin{eqnarray}
 Z(r) = \sum_j z_j e^{-\alpha_j r} 
 + z_{\mbox{\scriptsize H}} e^{-(\ln r - \ln \beta)^2/(2\gamma)} 
 + 1\,.
 \label{eq:molzDIM}
\end{eqnarray}
En contraste con la aproximación propuesta para los átomos, el segundo 
término en la expresión~(\ref{eq:molzDIM}) se incluye para tener en 
cuenta el efecto de carga producido por los hidrógenos. Este nuevo 
término permite ajustar convenientemente tanto la ubicación como el 
ancho de los potenciales hidrogénicos apantallados, sin afectar el valor 
correcto de la carga en el origen.

Para ilustrar este procedimiento, consideremos el orbital $1s$ del átomo 
de carbono. Primero, se resuelven las ecuaciones de Hartree--Fock usando 
el conjunto de bases \mbox{6-311G} con el código 
\textsc{gamess}~\cite{Schmidt:93,Gordon:05}. Aplicando la 
Ec.~(\ref{eq:Vinv}), se obtiene la carga invertida correspondiente. La 
carga resultante $Z_{1s}^{\mbox{\scriptsize 6-311G}}$ se muestra en la 
Fig.~\ref{fig:1sCarbon}(a) con una línea raya-punto verde. La carga 
tiene oscilaciones en toda su extensión radial, divergiendo para valores 
grandes de $r$. Se realiza el mismo cálculo con el conjunto de base 
universal Gaussiano (\acs{ugbs}), que tiene un número significativamente 
mayor de funciones primitivas. La carga invertida correspondiente 
$Z_{1s}^{\mbox{\scriptsize UGBS}}$ se presenta en la figura con una 
línea discontinua celeste. A pesar que esta carga efectiva aún diverge 
cerca de $r\approx1\,$a.u., las oscilaciones en la región media 
desaparecen. Luego, se resuelven las ecuaciones diferenciales de 
Hartree--Fock para el átomo de carbono usando el método de diferencias 
finitas (\acs{fd}) con el código \textsc{hf} de C. Froese 
Fischer~\cite{FroeseFischer:97}. La carga invertida 
$Z_{1s}^{\mbox{\scriptsize FD}}$ correspondiente se exhibe con una línea 
sólida roja en la Fig.~\ref{fig:1sCarbon}(a). Como es de esperar, esta 
carga no presenta oscilaciones, ya que no se utilizó un método espectral 
para construir los orbitales. Por comparación, también se incluye la 
carga $1s$ del carbono, resultante de implementar el DIM, con línea de 
puntos. Los perfiles de oscilación correspondientes al orbital $1s$ de 
los conjuntos de base \mbox{6-311G} y UGBS se obtienen empleando la 
Ec.~(\ref{eq:oscillation-prof}) y se muestran en la 
Fig.~\ref{fig:1sCarbon}(b). Los perfiles de oscilación para un conjunto 
de base atómico son únicos. Una vez que se determinan, se pueden 
sustraer de los potenciales moleculares invertidos de manera directa. 
Luego, se usa la Ec.~(\ref{eq:molzDIM}) y los parámetros se optimizan 
siguiendo la metodología presentada en la Sección~\ref{sec:dimatomos}.

\begin{figure}[t]
\centering
\includegraphics[width=0.9\textwidth]{dim/carbon_prof.eps}
\caption[Inversión de orbitales descriptos con conjuntos de base 
finitos.]
{(a) Cargas efectivas invertidas del orbital $1s$ del átomo de carbón.
(b)~Perfiles de oscilación de los conjuntos de base.}
\label{fig:1sCarbon}
\end{figure}

%%%%%%%%%%%%%%%%%%%%%%%%%%%%%%%%%%%%%%%%%%%%%%%%%%%%%%%%%%%%%%%%%%%%%%%%
\section{Resultados}
%%%%%%%%%%%%%%%%%%%%%%%%%%%%%%%%%%%%%%%%%%%%%%%%%%%%%%%%%%%%%%%%%%%%%%%%
\label{sec:dimresultados}

A continuación se presentan los resultados obtenidos mediante la 
implementación del método de inversión depurada para describir blancos
atómicos y moleculares. A lo largo de este trabajo de investigación se 
ha estudiado una amplia variedad de sistemas multielectrónicos mediante 
el método DIM, publicados en las Refs.~\cite{Mendez:16,Mendez:19dim,
Mendez:18}. De estos resultados, se seleccionan cuatro blancos para 
examinar en detalle en este capítulo: helio, nitrógeno, neón y metano. 
Luego, el DIM se combina con la primera aproximación de Born para 
calcular procesos inelásticos en átomos y moléculas simples. 

%%%%%%%%%%%%%%%%%%%%%%%%%%%%%%%%%%%%%%%%%%%%%%%%%%%%%%%%%%%%%%%%%%%%%%%%
\subsection{Estructura electrónica de blancos}
%%%%%%%%%%%%%%%%%%%%%%%%%%%%%%%%%%%%%%%%%%%%%%%%%%%%%%%%%%%%%%%%%%%%%%%%
\label{subsec:dimtarget}

Los orbitales de Hartree--Fock $u_{nl}^{\mathrm{HF}}$ y sus 
correspondientes energías $\varepsilon^{\mathrm{HF}}$, que se 
usan en la Ec.~(\ref{eq:Vinv}), se calculan usando indistintamente los 
códigos \textsc{hf} de C. Froese Fischer~\cite{FroeseFischer:97}, y 
\textsc{nrhf} de W. Johnson~\cite{Johnson:07}. Si bien ambos códigos 
resuelven la estructura de blancos no relativistas utilizando 
diferencias finitas, los algoritmos, métodos y grillas numéricas 
que emplean son diferentes. 

%=======================================================================
\subsubsection{Helio}
%=======================================================================

En primer lugar, se muestran los resultados obtenidos al implementar
el método de inversión depurada en el átomo de helio en su estado
fundamental. La inversión del orbital $1s$ no presenta ninguna 
dificultad numérica, ya que no tiene nodos. Como veremos más adelante, 
tampoco diverge a grandes distancias. Debido a la simplicidad del 
blanco, es posible ajustar la carga invertida con sólo unos pocos 
parámetros. La carga invertida $Z_{1s}^{\mathrm{HF}}$ y la carga 
invertida depurada $Z_{1s}^{\mathrm{DIM}}$ se muestran en la 
Fig.~\ref{fig:Hepots}(a) con línea discontinua y sólida, 
respectivamente. La carga DIM, dada por la Ec.~(\ref{eq:atomzDIM}), 
puede optimizarse con sólo tres parámetros, los cuales se dan en la 
Tabla~\ref{tab:params-atoms}. 

\begin{figure}[t]
\centering
\includegraphics[width=0.9\textwidth]{figures/dim/Hepots.eps}
\caption[Cargas efectivas y potencial de intercambio DIM de He.]
{(a) Cargas efectivas $1s$ del He ($^1$S) obtenidas mediante inversión 
directa (línea discontinua) e inversión depurada (línea sólida). 
(b) Potencial de intercambio DIM (línea sólida) y OPM (línea punteada).}
\label{fig:Hepots}
\end{figure}

Para verificar la calidad de la estructura atómica dada por el potencial 
DIM, en la Tabla~\ref{tab:results-atoms} se presenta una comparación 
entre los valores de energías (total y orbital) y los radios medios 
obtenidos utilizando el método de inversión depurada (fila superior), y  
los valores originales de HF correspondientes (fila inferior). El 
potencial DIM reproduce el valor de energía total, dada por la 
Ec.~(\ref{eq:Etotal}), en un $0.003\%$. La energía orbital $1s$ coincide 
con la energía de HF en 6 cifras significativas. Los valores medios de 
$u_{1s}^{\mathrm{DIM}}$ también muestran una excelente concordancia, 
tanto para la región cercana al origen, $\langle 1/r\rangle$, como en la 
región lejana, $\langle r\rangle$.

El potencial de intercambio orbital DIM del átomo de helio, definido por 
la Ec.~(\ref{eq:exchange-potential}), se muestra en la 
Fig.~\ref{fig:Hepots}(b). También se muestran los resultados del 
\textit{optimized potential method} (\acs{opm}) que se obtienen 
empleando el código \textsc{atomopm} de Talman~\cite{Talman:76,
Talman:89}, ampliamente usado en la comunidad DFT. Los potenciales DIM y 
OPM coinciden en todo el rango espacial. Las energías total y orbital de 
intercambio, dadas por la Ec.~(\ref{eq:exchange-energy}), del estado 
fundamental del helio se presentan en la Tabla~\ref{tab:exchange-atoms}. 
La energía total concuerda muy bien con el valor de intercambio exacto 
atómico de Hartree--Fock (EAHF, por sus siglas en 
inglés)~\cite{Becke:88}.

%=======================================================================
\subsubsection{Nitrógeno}
%=======================================================================

\begin{figure}[t]
\centering
\includegraphics[width=0.9\textwidth]{figures/dim/N4S_DIM.eps}
\caption[Cargas efectivas DIM de N.]
{Cargas efectivas $1s$, $2s$ y $2p$ del término $^4$S de N obtenidas 
mediante inversión directa (líneas discontinuas) e inversión depurada 
(líneas sólidas).}
\label{fig:Nzeff}
\end{figure}

\begin{figure}[t]
\centering
\includegraphics[width=0.9\textwidth]{figures/dim/N_Vx.eps}
\caption[Potenciales de intercambio DIM de N.]
{Potenciales de intercambio DIM de los términos $^4$S (línea sólida), 
$^2$D (línea discontinua) y $^2$P (línea punto-raya), y valores OPM 
(línea punteada) de N.}
\label{fig:NVx}
\end{figure}

Los excelentes resultados obtenidos mediante la implementación de DIM en 
el átomo de helio pueden atribuirse a la simplicidad del orbital y a 
su simetría esférica. Para evaluar la capacidad del DIM para describir 
blancos con más electrones y de capa abierta, se considera el átomo de 
nitrógeno. La configuración electrónica del estado fundamental del 
nitrógeno $2p^3$ da lugar a tres términos: $^4$S, $^2$D y $^2$P. Cada 
uno de ellos establece una densidad electrónica diferente. En la 
Fig.~\ref{fig:Nzeff} se muestran las cargas de los orbitales $1s$, $2s$ 
y $2p$ resultantes de la inversión directa (líneas discontinuas) y del 
método de inversión depurada (líneas continuas) del término de energía 
más bajo de N. La inversión del orbital $u_{1s}^{\mathrm{HF}}$ diverge 
en la región asintótica, mientras que el nodo genuino del orbital $2s$
produce un polo en $r\approx 0.32$~a.u..

Después de optimizar cuidadosamente los potenciales, se obtienen los 
parámetros de las cargas DIM correspondientes a los términos $^4$S, 
$^2$D y $^2$P, que se dan en la Tabla~\ref{tab:params-atoms}. Las 
soluciones obtenidas al resolver la Ec.~(\ref{eq:eqSchroRadial}) con los 
potenciales DIM se presentan en la Tabla~\ref{tab:results-atoms}. Las 
energías totales, orbitales y valores radiales medios reproducen los 
valores HF hasta en un $0.05\%$, $1\times 10^{-4}\%$ y $0.3\%$, 
respectivamente. 

En la Fig.~\ref{fig:NVx} se muestran los potenciales de intercambio 
$1s$, $2s$ y $2p$ de los términos $^4$S (línea sólida), $^2$D (línea 
discontinua) y $^2$P (línea raya-punto) de N. Los potenciales 
$V_{nl}^{\mathrm{x}}$ se comparan con los resultados OPM (línea 
punteada). Los potenciales de intercambio del orbital $1s$ de cada 
término se comportan de manera similar. El potencial OPM concuerda con 
los valores DIM cerca del origen y asintóticamente. En el caso del 
orbital $2s$, los potenciales de intercambio correspondientes a cada 
término se comportan de manera diferente en el origen. Para valores de 
$r>0.5$~a.u., los potenciales DIM y OPM coinciden. Nótese que el 
potencial OPM concuerda muy bien con el potencial $V_{2s}^{\mathrm{x}}$ 
del término $^2$D. Finalmente, los potenciales de intercambio DIM del 
$2p$ también se comportan de manera similar en los tres términos. Sin 
embargo, los potenciales DIM y OPM convergen sólo en la región 
asintótica. 

\begin{table}
\begin{center}
\begin{tabular}{
>{\centering\arraybackslash}p{0.12\textwidth}
>{\centering\arraybackslash}p{0.05\textwidth}
>{\centering\arraybackslash}p{0.22\textwidth}
>{\centering\arraybackslash}p{0.22\textwidth}
>{\centering\arraybackslash}p{0.22\textwidth}}
\rowcolor{mydarkgray} 
   &       & $nl$ & $z_j$        & $\alpha_j$   \\
%%%%%%%%%%%%%%%%%%%% Helio %%%%%%%%%%%%%%%%%%%%
He & $^1$S & $1s$ &  $1.3175$ & $2.5003$  \\\rowcolor{mygray} 
   &       &      & $-0.3175$ & $5.0437$  \\ 
%%%%%%%%%%%%%%%%%% Nitrogeno %%%%%%%%%%%%%%%%%%
N  & $^4$S & $1s$ & $5.2563$ & $1.2621$  \\\rowcolor{mygray} 
   &       &      & $0.7437$ & $8.0284$  \\ 
   &       & $2s$ & $2.7136$ & $0.8947$ \\\rowcolor{mygray} 
   &       &      & $2.4528$ & $3.5127$  \\
   &       &      & $0.8336$ & $3.3865$  \\ \rowcolor{mygray} 
   &       & $2p$ & $3.6435$ & $1.2407$  \\ 
   &       &      & $2.0550$ & $5.3514$  \\\rowcolor{mygray} 
   &       &      & $0.3015$ & $0.2866$ \\
   & $^2$D & $1s$ & $5.1664$ & $1.2241$  \\\rowcolor{mygray} 
   &       &      & $0.8137$ & $7.5680$  \\ 
   &       & $2s$ & $3.7478$ & $2.8531$  \\\rowcolor{mygray} 
   &       &      & $1.8541$ & $1.0311$  \\ 
   &       &      & $0.3981$ & $0.2397$ \\\rowcolor{mygray} 
   &       & $2p$ & $4.0105$ & $1.2874$  \\ 
   &       &      & $1.8552$ & $5.7086$  \\\rowcolor{mygray} 
   &       &      & $0.1343$ & $0.2680$ \\
   & $^2$P & $1s$ & $5.1864$ & $1.2178$  \\\rowcolor{mygray} 
   &       &      & $0.8137$ & $7.5674$  \\ 
   &       & $2s$ & $3.6700$ & $3.1495$  \\\rowcolor{mygray} 
   &       &      & $1.4394$ & $0.7404$  \\ 
   &       &      & $0.8907$ & $0.8306$ \\\rowcolor{mygray} 
   &       & $2p$ & $2.3280$ & $1.4093$  \\ 
   &       &      & $1.8977$ & $1.1656$  \\\rowcolor{mygray} 
   &       &      & $1.7743$ & $5.6878$ \\
%%%%%%%%%%%%%%%%%%%%%%%%%%%%%%%%%%%%%%%%%%%%%%%
Ne & $^1$S & $1s$ & $7.3677$ & $2.4173$ \\\rowcolor{mygray} 
   &       &      & $1.3004$ & $0.1264$ \\
   &       &      & $0.3320$ & $13.1582$ \\\rowcolor{mygray} 
   &       & $2s$ & $0.2977$ & $17.9939$ \\
   &       &      & $0.6681$ & $0.0673$ \\\rowcolor{mygray} 
   &       &      & $8.0342$ & $2.4722$ \\
   &       & $2p$ & $1.3531$ & $8.5695$ \\\rowcolor{mygray} 
   &       &      & $0.3359$ & $0.4649$ \\
   &       &      & $7.3111$ & $2.0906$ \\
\end{tabular}
\caption[Parámetros de la carga efectiva de He, N y Ne.]
{Parámetros de la carga efectiva $Z_{1s}^{\mathrm{ DIM}}$ de He ($^1$S), 
N ($^4$S, $^2$D, $^2$P) y Ne ($^1$S).}
\label{tab:params-atoms}
\end{center}
\end{table}

\begin{table}
\begin{center}
\begin{tabular}{
>{\centering\arraybackslash}p{0.07\textwidth}
>{\centering\arraybackslash}p{0.03\textwidth}
>{\centering\arraybackslash}p{0.15\textwidth}
>{\centering\arraybackslash}p{0.10\textwidth}
>{\centering\arraybackslash}p{0.15\textwidth}
>{\centering\arraybackslash}p{0.15\textwidth}
>{\centering\arraybackslash}p{0.15\textwidth}}
\rowcolor{mydarkgray} 
   & & $E$ & $nl$ & $\varepsilon_{nl}$ & $\left<r\right>_{nl}$ 
   & $\left<1/r\right>_{nl}$ \\
He & $^1$S & $-2.8616$   & $1s$ & $-0.9180$  & $0.9273$ & $1.6873$ \\
\rowcolor{mygray} 
   &       & $-2.8617$   &      & $-0.9180$  & $0.9273$ & $1.6873$ \\
N  & $^4$S & $-54.3762$  & $1s$ & $-15.6291$ & $0.2283$ & $6.6487$ \\
\rowcolor{mygray} 
   &       & $-54.4009$  &      & $-15.6291$ & $0.2283$ & $6.6532$ \\
   &       &             & $2s$ & $-0.9453$  & $1.3345$ & $1.0804$ \\
   \rowcolor{mygray} 
   &       &             &      & $-0.9453$  & $1.3323$ & $1.0782$ \\
   &       &             & $2p$ & $-0.5676$  & $1.4127$ & $0.9550$ \\
   \rowcolor{mygray} 
   &       &             &      & $-0.5676$  & $1.4096$ & $0.9577$ \\
   & $^2$D & $-54.2756$  & $1s$ & $-15.6664$ & $0.2283$ & $6.6493$ \\
   \rowcolor{mygray} 
   &       & $-54.2962$  &      & $-15.6664$ & $0.2283$ & $6.6539$ \\
   &       &             & $2s$ & $-0.9637$  & $1.3292$ & $1.0874$ \\
   \rowcolor{mygray} 
   &       &             &      & $-0.9637$  & $1.3263$ & $1.0832$ \\
   &       &             & $2p$ & $-0.5087$  & $1.4488$ & $0.9388$ \\
   \rowcolor{mygray} 
   &       &             &      & $-0.5087$  & $1.4466$ & $0.9421$ \\
   & $^2$P & $-54.2086$  & $1s$ & $-15.6916$ & $0.2282$ & $6.6504$ \\
   \rowcolor{mygray} 
   &       & $-54.2281$  &      & $-15.6916$ & $0.2282$ & $6.6543$ \\
   &       &             & $2s$ & $-0.9763$  & $1.3256$ & $1.0871$ \\
   \rowcolor{mygray} 
   &       &             &      & $-0.9763$  & $1.3223$ & $1.0866$ \\
   &       &             & $2p$ & $-0.4713$  & $1.4718$ & $0.9298$ \\
   \rowcolor{mygray} 
   &       &             &      & $-0.4713$  & $1.4730$ & $0.9316$ \\
Ne & $^1$S & $-128.4978$ & $1s$ & $-32.7725$ & $0.1575$ & $9.6215$ \\
\rowcolor{mygray} 
   &       & $-128.5475$ &      & $-32.7724$ & $0.1576$ & $9.6181$ \\
   &       &             & $2s$ & $-1.9304$  & $0.8913$ & $1.6408$ \\
   \rowcolor{mygray} 
   &       &             &      & $-1.9304$  & $0.8921$ & $1.6326$ \\  
   &       &             & $2p$ & $-0.8504$  & $0.9678$ & $1.4303$ \\
   \rowcolor{mygray} 
   &       &             &      & $-0.8504$  & $0.9653$ & $1.4354$ \\
\end{tabular}
\caption[Energías y radios medios de He, N y Ne.]
{Energías totales, energías orbitales y radios medios de He ($^1$S), 
N ($^4$S, $^2$D, $^2$P) y Ne ($^1$S) obtenidos con el método de 
inversión depurada (filas superiores) y con el método de HF (filas 
inferiores).}
\label{tab:results-atoms}
\end{center}
\end{table}

\begin{table}
\begin{center}
\begin{tabular}{
>{\centering\arraybackslash}p{0.07\textwidth}
>{\centering\arraybackslash}p{0.03\textwidth}
>{\centering\arraybackslash}p{0.14\textwidth}
>{\centering\arraybackslash}p{0.14\textwidth}
>{\centering\arraybackslash}p{0.14\textwidth}
>{\centering\arraybackslash}p{0.14\textwidth}
>{\centering\arraybackslash}p{0.14\textwidth}}
\rowcolor{mydarkgray} 
   &      & $1s$     & $2s$    & $3s$  & Total  & EAHF~\cite{Becke:88}\\
He & $^1$S & $-0.5129$ &           &           & $-1.0258$ & $-1.026$ \\
\rowcolor{mygray} 
N  & $^4$S & $-2.1175$ & $-0.4776$ & $-0.4711$ & $-6.6034$ & $-6.596$ \\
   & $^2$D & $-2.1175$ & $-0.4777$ & $-0.4262$ & $-6.4688$ & \\
   \rowcolor{mygray} 
   & $^2$P & $-2.1175$ & $-0.4780$ & $-0.3973$ & $-6.3827$ & \\
Ne & $^1$S & $-3.1106$ & $-0.8620$ & $-0.6938$ & $-12.1080$& $-12.105$\\
\rowcolor{mygray} 
\end{tabular}
\caption[Energías de intercambio total y orbitales de He, N y Ne.]
{Energías de intercambio de los orbitales DIM y total de He~($^1$S), 
N~($^4$S, $^2$D, $^2$P) y Ne~($^1$S).}
\label{tab:exchange-atoms}
\end{center}
\end{table}

Los valores de energía de intercambio orbitales y total de la 
configuración fundamental de N se muestran en la 
Tabla~\ref{tab:exchange-atoms}. La energía de intercambio orbital $1s$ 
de todos los términos son iguales, como se espera para un orbital de 
capa cerrada. De manera similar, las energías correspondientes al 
orbital $2s$ varían ligeramente, con una dispersión del $0.08\%$. 
Dado que la capa $2p$ está abierta, la energía de intercambio del 
orbital varía significativamente en los diferentes términos, con una 
dispersión de hasta 18\%. Empleando la Ec.~(\ref{eq:exchange-energy}), 
se calculan las energías de intercambio total. La energía del término 
más bajo y el valor EAHF presenta un acuerdo cercano al $0.1\%$.

%=======================================================================
\subsubsection{Neón}
%=======================================================================

La implementación del DIM en neón es análoga al caso del nitrógeno. En
este caso, la capa de valencia del átomo está completa. Los resultados 
de la optimización de los parámetros que definen las cargas efectivas 
DIM del átomo de neón se muestran en la Tabla~\ref{tab:params-atoms}. La 
comparación entre las soluciones de los potenciales DIM y el método de 
Hartree--Fock se dan en la Tabla~\ref{tab:results-atoms}. El acuerdo en 
energías orbitales es excelente, del orden de $10^{-5}$, 
mientras que DIM reproduce los valores medios de los orbitales HF en 
aproximadamente $0.1\%$. Por otro lado, la energía total tiene una 
dispersión del $0.04\%$ respecto a HF. Las energías DIM de intercambio 
para cada orbital, y las de intercambio total para cada término se 
presentan en la Tabla~\ref{tab:exchange-atoms}. Las energías 
$E^{\mathrm{x}}$ y los valores EAHF concuerdan en $0.04\%$.

%=======================================================================
\subsubsection{Metano}
%=======================================================================

Para ilustrar la aplicación del DIM en moléculas se escogió como ejemplo 
el metano. Este hidruro es lo suficientemente simétrico como para poder 
ser descripto por un potencial angular promediado~\cite{Granados:16}. 
Los orbitales moleculares HF de CH$_4$ se calculan usando los conjuntos 
de bases UGBS del carbono y del hidrógeno. Estas bases sólo consideran 
momentos angulares hasta $L=1$. El cálculo de estructura electrónica de 
metano con estos conjuntos de base deberían incluir funciones de 
polarización (por lo menos hasta las funciones $d$), con el fin de 
incrementar la precisión de las energías 
moleculares~\cite{Rothenberg:71,Hariharan:72}. Sin embargo, para poder 
estudiar en forma aislada únicamente los efectos de las bases finitas, 
estas funciones no se incluyen aquí. 

\begin{figure}[t]
\centering
\includegraphics[width=0.9\textwidth]{figures/dim/ch4_dim.eps}
\caption[Cargas invertidas y depuradas de metano.]
{Cargas efectivas de CH$_4$ de los orbitales moleculares $1a_1$, $2a_2$ 
y $2t_1$ obtenidas mediante el conjunto de base UGBS; inversión directa 
(líneas discontinuas) e inversión depurada (líneas sólidas).}
\label{fig:ch4zeff}
\end{figure}

Las cargas resultantes de la inversión directa de los orbitales UGBS 
se muestran en la Fig.~\ref{fig:ch4zeff} con líneas discontinuas. Dado 
que los orbitales moleculares se obtienen mediante combinaciones 
lineales de orbitales atómicos de hidrógeno y carbono, para remover los 
efectos de las bases se deben calcular los perfiles de oscilación de 
cada uno de los átomos constituyentes. Se emplea la 
Ec.~(\ref{eq:oscillation-prof}) para determinar los perfiles 
$p_{1s}^{\mbox{\scriptsize UGBS}}$, $p_{2s}^{\mbox{\scriptsize UGBS}}$ y 
$p_{2p}^{\mbox{\scriptsize UGBS}}$ del carbono. Luego, se sustraen los 
perfiles $p_{nl}^{\mbox{\scriptsize UGBS}}$ de las correspondientes 
cargas invertidas $Z_{nl}^{\mbox{\scriptsize UGBS}}$. Así, se eliminan 
completamente las oscilaciones de todos los orbitales, excepto para el 
$2a_2$. Este orbital presenta minúsculas fluctuaciones residuales debido 
a la base del hidrógeno, que no afectan al subsiguiente proceso de 
depuración.

Los parámetros de las cargas moleculares DIM definidos por la 
Ec.~(\ref{eq:molzDIM}) se muestran en la Tabla~\ref{tab:ch4parameters}. 
Las cargas correspondientes se ilustran en la Fig.~\ref{fig:ch4zeff} con 
líneas sólidas. En este caso, la función de costo~(\ref{eq:fncosto-dim}) 
a minimizar considera como valores de referencia a las energías y radios 
medios de los orbitales moleculares dados por Moccia~\cite{Moccia:69}. 
Las energías orbitales DIM reproducen estos valores hasta la cuarta 
cifra significativa, como se indica en la Tabla~\ref{tab:ch4parameters}. 
Por otro lado, los radios medios $\langle r\rangle$ y 
$\langle 1/r\rangle$ obtenidos con los potenciales moleculares DIM están 
dentro del $1\%$ de los valores de Moccia.

\begin{table}[t]
\centering
\begin{tabular}{
>{\centering\arraybackslash}p{0.13\textwidth}
>{\centering\arraybackslash}p{0.13\textwidth}
>{\centering\arraybackslash}p{0.13\textwidth}
>{\centering\arraybackslash}p{0.13\textwidth}
>{\centering\arraybackslash}p{0.13\textwidth}
>{\centering\arraybackslash}p{0.13\textwidth}}
\rowcolor{mydarkgray} 
   $nl$ & $E$        & $z_j$        & $\alpha_j$   & $\beta$&$\gamma$\\
$1a_1$  & $-11.1949$ & $1.925280$ & $0.641982$ & & \\
\rowcolor{mygray} 
        &            & $0.953120$ & $5.571510$ & & \\
        &            & $2.121600$ & $1.500440$ & & \\
\rowcolor{mygray} 
$2a_2$  & $-0.9204$  & $2.912200$ & $3.149990$ & & \\
        &            & $2.087800$ & $0.771371$ & & \\
\rowcolor{mygray} 
        &            & $1.23640$  &            & $2.329570$&$0.053420$\\
$2t_1$  & $-0.5042$  & $0.901953$ & $2.895140$ & & \\
\rowcolor{mygray} 
        &            & $1.112030$ & $0.388649$ & & \\
        &            & $2.986017$ & $2.931210$ & & \\
\rowcolor{mygray} 
        &            & $1.301820$ &            & $2.169850$&$0.012616$\\ 
\end{tabular}
\caption[Energías y parámetros de ajuste de cargas efectivas de metano.]
{Energías orbitales moleculares y parámetros de ajuste de cargas 
efectivas de metano.}
\label{tab:ch4parameters}
\end{table}

%%%%%%%%%%%%%%%%%%%%%%%%%%%%%%%%%%%%%%%%%%%%%%%%%%%%%%%%%%%%%%%%%%%%%%%%
\subsection{Procesos colisionales simples}
%%%%%%%%%%%%%%%%%%%%%%%%%%%%%%%%%%%%%%%%%%%%%%%%%%%%%%%%%%%%%%%%%%%%%%%%
\label{subsec:procol}

Una de las motivaciones fundamentales para el desarrollo del método DIM 
es su utilización en cálculos de procesos colisionales. Para comprobar 
su utilidad efectiva, en esta sección, se calcula la ionización de 
atómos multielectrónicos y moléculas simples debido al impacto de 
protones y fotones. En estos cálculos, se asume que el Hamiltoniano del 
sistema está determinado por la dinámica del proyectil, el blanco y el 
electrón activo. El marco teórico de la ionización está dada por la 
primera aproximación de Born (FBA). Para un rango intermedio-alto de la 
energía del proyectil, el primer orden perturbativo resulta suficiente 
para obtener resultados razonables. Por simplicidad, denominamos la 
combinación del DIM para representar la estructura del blanco, y el 
cálculo de la ionización utilizando el primer orden perturbativo, como 
ionización DIM-FBA. 

%=======================================================================
\subsubsection{Fotoionización}
%=======================================================================

Para demostrar la aplicabilidad del DIM en procesos colisionales 
simples, se emplean los potenciales en el cálculo de fotoionización de 
blancos atómicos y moleculares. Entre los sistemas 
estudiados~\cite{Mendez:19dim}, se escoge ilustrar los 
resultados para helio, nitrógeno, neón y metano. En la 
Fig.~\ref{fig:photoDIM} se muestran las secciones eficaces totales de 
fotoionización DIM-FBA con líneas sólidas. Los resultados teóricos para 
helio y nitrógeno coinciden de manera excelente con los valores 
experimentales (símbolos)~\cite{Samson:90,Henke:93,Stolte:16} a bajas, 
medias y altas energías del fotón incidente. En el caso del átomo de 
neón, surgen algunas discrepancias con las mediciones (símbolos) 
\cite{Henke:93,Samson:02} a energías bajas e intermedias del proyectil. 
Este comportamiento sugiere la necesidad de incluir, en los cálculos de 
la fotoionización, correcciones de mayor orden que incluyan efectos de 
múltiples cuerpos, tales como la relajación de los orbitales 
debido a la creación de un hueco electrónico, respuestas colectivas de 
electrones de capas internas~\cite{Ederer:64} o efectos de correlación.

\begin{figure}
\centering
\includegraphics[width=0.9\textwidth]{dim/fotoDIM-part1.eps} 

\vspace{-1.15cm}
\includegraphics[width=0.9\textwidth]{dim/fotoDIM-part2.eps}
\caption[Fotoionización de He, N, Ne y CH$_4$.]
{Sección eficaz de fotoionización de He, N, Ne y CH$_4$. Curvas: 
cálculos teóricos DIM-FBA. Símbolos: datos 
experimentales~\cite{Samson:90,Henke:93,Stolte:16,Samson:02,
Lukirskii:64,Henke:82,Samson:89}.}
\label{fig:photoDIM}
\end{figure}

La orientación molecular es importante para determinar la sección eficaz 
en un proceso colisional. Sin embargo, en la configuración experimental, 
las moléculas en estado gaseoso generalmente tienen orientaciones 
aleatorias. Esto justifica la aproximación asumida en el método DIM, 
consistente en promediar esféricamente a los potenciales. En la región 
de energías entre $\sim$15 y 300 eV se aprecia la contribuación de la 
capa de valencia a la fotoionización, mientras que la discontinuidad en 
$0.3$~keV corresponde al umbral de ionización del orbital molecular 
$1a_1$. Los resultados teóricos del modelo DIM-FBA para la sección 
eficaz total de fotonización de CH$_4$ se encuentran en buen acuerdo con 
datos experimentales~\cite{Lukirskii:64,Henke:82,Samson:89} en el 
rango de altas energías y cerca del umbral de ionización de la capa 
$2p$. Para fotoenergías bajas e intermedias, el acuerdo entre los
cálculos DIM-FBA y las mediciones no es bueno. Los efectos 
colectivos mencionados anteriormente no han sido considerados en estos 
valores, y probablemente sean los responsables de las discrepancias 
observadas. En la fotoionización de un electrón perteneciente al orbital 
interno $1a_1$, estos efectos no son tan signficativos. Por ello, con 
sólo el primer orden de aproximación, se obtiene buen acuerdo con los 
datos experimentales. 

%=======================================================================
\subsubsection{Ionización por impacto de protones}
%=======================================================================

\begin{figure}[t]
\centering
\includegraphics[width=0.9\textwidth]{figures/dim/ionDIM.eps}
\caption[Ionización por impacto de protón de N y CH$_4$.]
{Sección eficaz de ionización por impacto de protones para N y 
CH$_4$. Línea sólida: cálculos teóricos DIM-FBA. Símbolos: datos 
experimentales de ionización por impacto de 
protones~\cite{Rudd:83,Rudd:85} y electrones~\cite{Brook:78} con 
conversión de equivelocidad.}
\label{fig:iondim}
\end{figure}

El uso efectivo de los potenciales DIM también se estudia analizando  
la ionización por impacto de protones en blancos multielectrónicos. De 
los cálculos realizados, se eligen dos ejemplos para ilustrar este 
proceso. Las secciones eficaces totales de ionización por impacto de 
protones para N y CH$_4$, que se obtienen en el marco del modelo 
DIM-FBA, se muestran en la Fig.~\ref{fig:iondim}. En la figura se 
ilustran los datos experimentales por impacto de 
protones~\cite{Rudd:83,Rudd:85} y, en el caso de nitrógeno, se incluyen 
también resultados de ionización por impacto de 
electrones~\cite{Brook:78}. Los valores de energía incidente de los 
electrones se convierten según el principio de equivelocidad, donde 
$m_p\approx 1836\,m_e$. Por ejemplo, electrones que inciden sobre el 
blanco con $0.54$~keV son equivalentes en energía a protones incidentes 
con aproximadamente 1~MeV. Para valores de energía mayores a 400~keV, 
las secciones eficaces de ambos proyectiles coinciden. El acuerdo entre 
los cálculos teóricas y los datos experimentales es muy bueno en la 
región de altas energías, donde tiene validez la primera aproximación de 
Born. 


%%%%%%%%%%%%%%%%%%%%%%%%%%%%%%%%%%%%%%%%%%%%%%%%%%%%%%%%%%%%%%%%%%%%%%%%
\section{Discusión: HF vs. DIM}
%%%%%%%%%%%%%%%%%%%%%%%%%%%%%%%%%%%%%%%%%%%%%%%%%%%%%%%%%%%%%%%%%%%%%%%%
\label{sec:discusionHF}

Las publicaciones de los resultados de este trabajo~\cite{Mendez:16,
Mendez:19dim,Mendez:18,Mitnik:19} han derivado en interesantes 
discusiones sobre la validez de las hipótesis del DIM respecto al 
comportamiento asintótico de las funciones de onda, y la posible 
existencia de potenciales locales. En esta sección se repasan estas 
discusiones, examinando la teoría de Hartree--Fock y el origen de los 
problemas numéricos que surgen del cálculo de los potenciales 
invertidos. Las hipótesis sobre las que se basa el método de inversión 
depurada permiten resolver estos problemas, por lo que se podría 
considerar al DIM como una alternativa superadora para representar en 
forma precisa orbitales atómicos mediante de potenciales locales.

Uno de los aspectos principales de esta discusión concierne a la posible 
existencia (o la negación teórica) de un potencial local que pueda 
derivar a las funciones de onda originales de HF. Por ejemplo, el 
trabajo de Amusia \textit{et al.}~\cite{Amusia:04} concluye que esto es 
imposible ya que la densidad electrónica de Hartree--Fock no es 
$v$-representable y, por lo tanto, no puede surgir de un cálculo 
asumiendo partículas no-interactuantes bajo el efecto de un potencial 
local. Sin embargo, como se demuestra en esta Tesis, el método de 
inversión depurada ha permitido obtener potenciales efectivos que 
reproducen con gran precisión las soluciones HF.

%%%%%%%%%%%%%%%%%%%%%%%%%%%%%%%%%%%%%%%%%%%%%%%%%%%%%%%%%%%%%%%%%%%%%%%%
\subsection{Nodos genuinos}
%%%%%%%%%%%%%%%%%%%%%%%%%%%%%%%%%%%%%%%%%%%%%%%%%%%%%%%%%%%%%%%%%%%%%%%%
\label{subsec:nodosHF}

\begin{figure}[t]
\centering
\includegraphics[width=0.9\textwidth]{dim/example_2sMg.eps} 
\caption[Orbital radial y su derivada segunda.]
{Orbital radial $u_{2s}^{\mathrm{HF}}$ del estado fundamental de Mg y su 
derivada segunda escalada.}
\label{fig:example2sMg}
\end{figure}

Dado que el DIM se basa en la Ec.~(\ref{eq:Vinv}), en la cual la función 
de onda se encuentra en el denominador, el potencial invertido debería 
diverger en cada uno de los nodos del orbital excepto cuando se cumpla 
con una condición específica: que las derivadas segundas se anulen 
exactamente en esos mismos puntos. Esta condición no se asume en ninguna 
instancia de la teoría de Hartree--Fock. Sin embargo, durante esta 
investigación se han encontrado resultados que apuntan en esta 
dirección. Por ejemplo, en la Fig.~\ref{fig:example2sMg} se muestra la 
función $u_{2s}^{\mathrm{HF}}$ del Mg y su derivada segunda (escalada 
por un factor). Las dos raíces de $u_{2s}^{\mathrm{HF}}''$ son puntos de 
inflexión y se corresponden a (1)~el nodo radial y (2)~el punto de 
retorno clásico en la función orbital. A primera vista, el nodo del 
orbital y la primera raíz de la derivada segunda parecen coincidir. Una 
inspección más cercana (ver recuadro) muestra que ambas funciones no se 
anulan exactamente en el mismo punto. Definiendo $\Delta r$ como la 
distancia entre los nodos del orbital y de su derivada segunda, se 
encuentra una pequeña distancia $\Delta r=1\times 10^{-3}$~a.u.. 
Inspeccionando una gran cantidad de átomos, se encuentra que esta 
comportamiento es general, y existe siempre una gran proximidad entre 
los ceros de las funciones y de sus derivadas segundas. Esta 
característica hace suponer que la cercanía entre los nodos genuinos de 
los orbitales HF y las correspondientes raíces de su derivada segunda 
no es casual, y que los nodos genuinos en la teoría de Hartree--Fock 
podrían también ser puntos de inflexión. 

\begin{figure}[t]
\centering
\includegraphics[width=0.9\textwidth]{dim/dr_2sMg.eps} 
\caption{Dependecia de $\Delta r$ con el orden de aproximación numérica, 
para el orbital $2s$ del átomo de potasio. (a)~Primer orden y 200 
puntos, (b)~400 puntos; (c)~octavo orden y 1000 puntos.}
\label{fig:dr2sMg}
\end{figure}

Para indagar esta hipótesis, se diseña un experimento numérico que 
consiste en examinar el comportamiento del valor $\Delta r$ resolviendo 
las ecuaciones de HF con diferentes órdenes de aproximación. La calidad 
de estos métodos se controla variando el orden de precisión de los 
algoritmos y la densidad de puntos de las grillas numéricas empleadas en 
los cálculos. La Fig.~\ref{fig:dr2sMg} muestra $u_{2s}^{\mathrm{HF}}$ de 
Mg (línea sólida) y su derivada segunda (línea discontinua) en las 
proximidades del nodo, implementando tres grados de aproximación 
distintos. Los cálculos menos precisos se muestran en la 
Fig.~\ref{fig:dr2sMg}(a), donde se realiza una aproximación de primer 
orden y se usa una grilla numérica de 200 puntos (mínimo valor necesario 
para obtener convergencia), resultando en 
$\Delta r=8\times 10^{-3}$~a.u.. Aumentando el número de puntos a 400, 
este valor se reduce a $\Delta r=4\times 10^{-3}$~a.u., como se muestra 
en la Fig.~\ref{fig:dr2sMg}(b). Por último, se incrementa el número de 
puntos a 1000 y se usa un algoritmo de Adams-Moulton de octavo 
orden~\cite{Johnson:07}. La Fig.~\ref{fig:dr2sMg}(c) muestra el mejor 
resultado obtenido, en el cual la distancia entre los ceros de la 
función y su derivada se reduce al valor 
$\Delta r=1\times 10^{-3}$~a.u.. Se realizó un cálculo adicional usando 
el OPM, que se muestra en la Fig.~\ref{fig:dr2sMg}(c) con una línea 
raya-punto. Como se observa, su segunda derivada 
$u_{2s}^{\mathrm{OPM}}''$ (línea punto-raya-punto) es estrictamente cero 
en el nodo. Este resultado se debe a que el potencial OPM es local. 

El presente experimento computacional sugiere que los nodos de los 
orbitales HF podrían ser también puntos de inflexión. Para probar 
efectivamente esta hipótesis, será necesario considerar algoritmos con 
órdenes de aproximación mayores. De confirmarse, la teoría de 
Hartree--Fock podría modificarse agregando una restricción adicional al 
procedimiento variacional autoconsistente, que imponga que los nodos 
sean además puntos de inflexión. El cumplimiento de esta restricción 
aseguraría la existencia de un potencial local, en principio, sin polos. 

%%%%%%%%%%%%%%%%%%%%%%%%%%%%%%%%%%%%%%%%%%%%%%%%%%%%%%%%%%%%%%%%%%%%%%%%
\subsection{Decaimiento exponencial}
%%%%%%%%%%%%%%%%%%%%%%%%%%%%%%%%%%%%%%%%%%%%%%%%%%%%%%%%%%%%%%%%%%%%%%%%
\label{subsec:decaimientoHF}

Otro aspecto teórico que ha derivado en interesantes discusiones dentro 
de la comunidad se refiere a las condiciones de borde~(\ref{eq:Zasympt}) 
que el DIM impone a los potenciales invertidos. Esta discusión ha 
llegado a hacerse pública en un comentario~\cite{Cinal:19}, donde se 
objeta la validez del método de inversión depurada por este motivo.

En el DIM, el potencial se asume de tipo Coulombiano, con una carga 
efectiva en el origen determinada únicamente por la carga del núcleo. A 
grandes distancias, se supone que todos los electrones apantallan al 
núcleo, dejando como resultado una carga efectiva igual a 1. De esta 
forma, a grandes distancias, los potenciales están dados por
\begin{equation}
\lim_{r\rightarrow\infty} V_{nl}^{\mathrm{DIM}}(r) = -\frac{1}{r}\,.
\label{eq:VDIMasympt}
\end{equation}
Los orbitales correspondientes tienen un decaimiento asintótico 
denonimado en la literatura de ``tipo Hartree''~\cite{Casida:89},
\begin{equation}
\lim_{r \rightarrow \infty} u_{nl}^{\mathrm{DIM}}(r) \propto
\exp(- \sqrt{- 2 \varepsilon_{nl}^{\mathrm{HF}} } r ) \,.
\label{eq:uDIMasympt}
\end{equation}
El término ``tipo-Hartree'' puede resultar confuso ya que la energía 
del orbital $\varepsilon_{nl}^{\mathrm{HF}}$ ha sido calculada 
considerando el término de intercambio. 

Sin embargo, en la teoría de HF, los orbitales tienen un comportamiento 
asintótico diferente, dictado por la energía del orbital molecular de 
mayor ocupación (\acs{homo}) $\varepsilon_{\mathrm{HOMO}}^{\mathrm{HF}}$ 
\cite{Handy:69,Handler:80,Ishida:92},
\begin{equation}
\lim_{r \rightarrow \infty} u_{nl}^{\mathrm{HF}}(r) \propto
\exp(- \sqrt{- 2 \varepsilon_{\mathrm{HOMO}}^{\mathrm{HF}} } r )  \, .
\label{eq:uHFasympt}
\end{equation}
Este comportamiento se debe al potencial \textit{exacto} de 
Hartree--Fock~\cite{Cinal:10}, que está dado por
\begin{equation}
\lim_{r\rightarrow\infty} V_{nl}^{\mathrm{HF}}(r) \approx
-\left(\varepsilon_{\mathrm{HOMO}}^{\mathrm{HF}}
-\varepsilon_{nl}^{\mathrm{HF}}\right)+\frac{q_{nl}}{r}\,,
\label{eq:VHFasympt}
\end{equation}
donde $\varepsilon_{nl}$ son las energías orbitales de HF y $q_{nl}$ es 
un coeficiente que depende del orbital y puede ser distinto de $-1$.

El comportamiento asintótico de los orbitales~(\ref{eq:uHFasympt}) y el 
potencial exacto~(\ref{eq:VHFasympt}) de Hartree--Fock se han demostrado 
rigurosamente. Sin embargo, muchos autores han puesto en tela de juicio 
su relevancia, interpelando si este comportamiento \textit{tiene un 
significado físico o si es sólo un artefacto del método de 
Hartree--Fock}\myfnote{}~\cite{Handy:69}. Handler y Smith reconocieron 
estos cuestionamientos señalando que \textit{la cuestión del 
comportamiento asintótico es de interés intrínseco de la 
teoría}\myfnote{}~\cite{Handler:80}. Weber y Parr estudiaron en 
profundidad este problema: limitaron los efectos de intercambio entre 
electrones a una ``esfera de influencia'' finita y asumieron que 
\textit{el electrón que se extrae de un átomo es de alguna manera 
distinguible por el hecho de la separación. De cierta 
forma, este electrón ve al ion como una carga positiva clásica si la 
distancia entre ellos es lo suficientemente grande y, por lo tanto, en 
este contexto, los efectos de intercambio pueden 
despreciarse}\myfnote{}~\cite{Weber:71}.

\begin{figure}[t]
\centering
\includegraphics[width=0.9\textwidth]{dim/V1s_K.eps} 
\caption[Comportamiento asintótico de los potenciales.]
{Comportamiento asintótico de los potenciales invertido HF (línea 
discontinua) y DIM (línea sólida) de orbital $1s$ del átomo de K.}
\label{fig:V1sK}
\end{figure}

A continuación, se presenta un ejemplo que permite ilustrar estos 
conceptos en forma concreta. En la Fig.~\ref{fig:V1sK} se muestran los 
potenciales $V_{1s}^{\mathrm{DIM}}$ y $V_{1s}^{\mathrm{HF}}$ del átomo 
de potasio con líneas sólida y discontinua, respectivamente. El punto de 
retorno clásico del orbital $1s$ se ilustra con una línea vertical 
punteada: a partir de allí, el orbital decae exponencialmente. En la 
región asintótica, el potencial invertido tiende a una constante, 
siguiendo la Ec.~(\ref{eq:VHFasympt}). Por ello, cuando los potenciales 
se reescriben en términos de cargas efectivas, estas últimas divergen 
(ver recuadro). Sin embargo, el potencial DIM tiene, por definición, una 
forma Coulombiana y la carga efectiva tiende a 1 a grandes distancias, 
siguiendo la Ec.~(\ref{eq:VDIMasympt}).

\begin{figure}[t]
\centering
\includegraphics[width=0.9\textwidth]{dim/Lns_K.eps} 
\caption[Comportamiento asintótico de los orbitales HF.]
{Comportamiento asintótico de los orbitales $s$ de HF (líneas 
discontinuas y sólidas) y DIM (líneas punteadas) del átomo de K. La 
expresión $L_{nl}$ corresponde a la Ec.~(\ref{eq:Lnl}).}
\label{fig:LnsK}
\end{figure}

En la región asintótica, la amplitud de los orbitales es minúscula y 
resulta conveniente examinar su comportamiento en detalle a través de la 
derivada logarítmica. Se define la función
\begin{equation}
L_{nl}(r) \equiv r \frac{d \log{u_{nl}}}{d r}\,,
\label{eq:Lnl}
\end{equation}
que debería tener una forma lineal para funciones $u_{nl}$ que decaen 
exponencialmente. Siguiendo el ejemplo anterior, la Fig.~\ref{fig:LnsK} 
ilustra la derivada logarítmica de los orbitales HF $ns$ del átomo de 
potasio. Los orbitales HF se muestran con líneas sólidas (capas internas 
$1s$, $2s$ y $3s$) y discontinuas (capa de valencia $4s$). Los polos de 
las funciones $L_{nl}(r)$ corrresponden a los nodos de las funciones 
$u_{nl}$. La relación del número cuántico radial establece que los 
orbitales $1s$ no tienen nodos, mientras que los orbitales $2s$ tienen 
sólo un único nodo. Sin embargo, en el caso de potasio, el orbital $1s$ 
presenta dos polos, i.e., dos nodos espurios, en $r\approx 0.99$~a.u. y 
$5.68$~a.u.. Por otra parte, el orbital $2s$ tiene un nodo espurio en 
$5.78$~a.u.. Estos nodos se discuten en la siguiente sección.

A grandes distancias, $r>10$~a.u., todos los orbitales de K presentan el 
comportamiento de HF dado por la Ec.~(\ref{eq:uHFasympt}): los orbitales 
de las capas internas y el \acs{homo} son indistinguibles. En la 
Fig.~\ref{fig:LnsK} se incluye, además, el comportamiento de los 
orbitales DIM $1s$, $2s$ y $3s$ (líneas punteadas) en la región 
asintótica, que presentan el decaimiento exponencial de ``tipo 
Hartree''. Se puede observar que si bien la teoría de Hartree--Fock 
establece otro comportamiento asintótico, comporten el mismo tipo de 
decaimiento exponencial --tipo Hartree, Ec.~\ref{eq:uDIMasympt}-- en un 
vasto rango espacial. En este caso, estos rangos se extienden desde el 
punto de retorno hasta distancias $r\approx 0.4$, $1.5$, y 5~a.u., para 
los orbitales $1s$, $2s$ y $3s$, respectivamente. Se señala estos 
valores, ya que entre el origen y esos puntos se concentra más del 97\% 
de la densidad de estos orbitales. Por lo tanto, el 
comportamiento físico más importante de los orbitales en la región 
externa está dictada por la expresión de ``tipo Hartree'', en lugar del 
comportamiento exacto de la teoría de Hartree--Fock. Esta es quizás la 
razón por la que Weber, Handy y Parr~\cite{Weber:70} sugirieron 
el comportamiento de largo alcance dado por~(\ref{eq:uDIMasympt}), 
que produce resultados más precisos que los obtenidos por la teoría 
formal de HF. Casida y Chong~\cite{Casida:89} también llegaron a las 
mismas conclusiones basándose en experimentos de ionización electrónica. 

La discusión sobre cuál condición asintótica resulta más apropiada, 
todavía no está saldada. La definición de las condiciones de borde que 
se imponen en el DIM concuerdan con Weber, Handy y Parr, que sugieren 
que el comportamiento a grandes distancias~(\ref{eq:VDIMasympt}) es 
preferible y produce resultados más precisos que los que se obtienen 
utilizando la teoría formal de Hartree--Fock. 

%%%%%%%%%%%%%%%%%%%%%%%%%%%%%%%%%%%%%%%%%%%%%%%%%%%%%%%%%%%%%%%%%%%%%%%%
\subsection{Nodos espurios}
%%%%%%%%%%%%%%%%%%%%%%%%%%%%%%%%%%%%%%%%%%%%%%%%%%%%%%%%%%%%%%%%%%%%%%%%
\label{subsec:espuriosHF}

Las implicancias de asumir el potencial exacto de HF a grandes 
distancias tiene efectos inesperados en los orbitales radiales: surgen
oscilaciones espurias en la región asintótica. Hay escasas referencias 
en la teoría de Hartree--Fock sobre la existencia de los nodos espurios. 
Por ejemplo, Fischer~\cite{FroeseFischer:97} (discípula de Hartree y 
autora del código computacional HF más utilizado de la física atómica) 
los menciona brevemente, aunque sin proporcionar detalles posteriores. 

La existencia de los nodos espurios en los orbitales HF no se debe a 
problemas numéricos sino que es inherente al método. Lo hemos 
comprobado calculando orbitales con diferentes métodos y programas 
computacionales. En todos los cálculos realizados con \textsc{hf} y 
\textsc{nrhf}, se encontraron los mismos nodos espurios exactamente en 
los mismos puntos. Incluso coinciden con los escasos reportes que se 
encuentran en la literatura, empleando otros métodos para resolver las 
ecuaciones de HF. En general, en los cálculos que utilizan técnicas 
espectrales no sólo aparecen estos nodos espurios, sino que se agregan 
otros nodos adicionales. Dado que las funciones de base decaen 
exponencialmente, las expansiones dejan de tener sentido a grandes 
distancias. En estas regiones, las amplitudes de los 
orbitales son despreciables. La teoría de Hartree--Fock se basa en el 
principio variacional, por lo cual el comportamiento asintótico de los 
orbitales tiene un efecto insignificante en los valores medios, 
particularmente, de las energías. Por lo tanto, los nodos espurios no 
tienen consecuencias prácticas y pueden ser ignorados, sin consecuencias 
notorias. Sin embargo, en el esquema de inversión, el ancho y la 
amplitud de 
los nodos espurios afectan significativamente las regiones más internas 
de la carga invertida. Por ejemplo, la inversión del nodo espurio del 
orbital $2s$ del átomo de potasio ubicado en $r\approx 5.78$~a.u., que 
se muestra en el recuadro de la Fig.~\ref{fig:2sK}(a), afecta a la carga 
de manera tal que esta empieza a diverger a partir $r\approx 1$~a.u., 
aún cuando la amplitud del orbital no es lo suficientemente pequeña. 

Quizás el cuestionamiento más importante al DIM está dado por el hecho 
que el método adopta el comportamiento de ``tipo Hartree'' en lugar del 
comportamiento asintótico que dicta la teoría de Hartree--Fock. El 
trabajo de Weber \textit{et al.}~\cite{Weber:70} plantea una importante 
conjetura al respecto. Los autores teorizaron que todos los nodos 
espurios desaparecerían si los orbitales tuvieran un 
comportamiento asintótico de ``tipo Hartree'', en lugar del límite 
exacto~(\ref{eq:uHFasympt}). La condición de borde asintótica del DIM 
sostiene esta hipótesis, por lo tanto, los orbitales DIM no presentan 
nodos espurios.

\begin{center}
\rule[0.5ex]{0.8\linewidth}{0.5pt}
\end{center}

Las discusiones de esta Sección se pueden resumir en las siguientes 
afirmaciones:
\begin{itemize}
\item La existencia de un potencial local de Hartree--Fock es 
cuestionable ya que la densidad de HF no es $v$-representable.
\item La hipótesis de que los nodos de los orbitales y las raíces 
correspondientes de su derivada segunda deben coincidir aseguraría, en 
principio, la existencia de un potencial local sin nodos.
\item El límite asintótico correcto que deben seguir las funciones 
radiales orbitales es todavía objeto de discusión. 
\item La existencia de los nodos espurios se debe al término de 
intercambio que sobrevive a grandes distancias, y dicta el 
comportamiento del potencial exacto en la teoría de HF.
\item Los orbitales DIM cumplen las condiciones asintóticas de ``tipo 
Hartree'' y no presentan nodos espurios, los cuales son intrínsecos a la 
teoría de HF.
\item A diferencia del método HF, el límite asintótico del potencial DIM 
permite representar el apantallamiento electrónico en forma correcta.
\end{itemize}
Con estos argumentos, se puede justificar la utilización de las 
condiciones de borde del potencial, y el comportamiento ``tipo Hartree'' 
de los orbitales, asumidas en el DIM. Además, el método asegura 
potenciales locales, que reproducen las soluciones HF en forma precisa
y se pueden utilizar para describir procesos colisionales simples, lo 
que constituye el objetivo principal de este capítulo.

