\chapter{Ionización de átomos y moléculas simples}
\label{chap:iondim}

%%%%%%%%%%%%%%%%%%%%%%%%%%%%%%%%%%%%%%%%%%%%%%%%%%%%%%%%%%%%%%%%%%%%%%%%
\section{Introducción}
%%%%%%%%%%%%%%%%%%%%%%%%%%%%%%%%%%%%%%%%%%%%%%%%%%%%%%%%%%%%%%%%%%%%%%%%

%% Métodos para describir blancos
%%  - Estructura electrónica: HF, DFT 
%%  - Potenciales efectivos (Herman-Skillman, OEP, parametric potential)
%% Métodos para describir procesos colisionales
%%  - Born
%% Lo que se hizo en este trabajo
%%  - DIM en átomos
%%  - DIM en moléculas
%%  - FBA para ionización/fotoionización en atomos y moléculas simples

La obtención de las probabilidades de transición en un proceso 
colisional inelastico requiere la correcta representación de los estados 
inicial y final del blanco. En general, la resolución de las ecuaciones 
de Schr\"odinger de sistemas multielectrónicos implementan el modelo 
de partículas independientes (\acs{ipm}) en conjunción con la 
aproximación de campo central (\acs{cfa})~\cite{Bransden:03,Cowan:81}. 
Métodos tales como la teoría de Hatree--Fock~(\acs{hf}) o la teoría del 
funcional de la densidad~(\acs{dft}) tienen a la CFA como hipótesis 
fundamental. Por otro lado, los procesos colisionales simples usualmente 
se basan en la aproximación de un electrón activo (\acs{sae}). El método 
SAE asume que en la colisión sólo un electrón del blanco interactúa con 
el proyectil, mientras que el resto de los electrones actúan como 
espectadores. En la ionización, el electrón activo se encuentra 
inicialmente ligado, mientras que luego de la colisión con el proyectil, 
éste se mueve en el continuo. La representación de los estados ligados 
es relativamente directa mientras que la representación de los estados 
continuos suele presentar cierta dificultad.

En la aproximación de campo central, la existencia de un potencial 
efectivo local que permita obtener de forma directa las funciones de 
onda de las partículas interactuantes en el proceso colisional es ideal.
Con este fin, se ha desarrollado una gran variedad de metodologías para 
el diseño de potenciales efectivos en blancos atómicos~\cite{Hibbert:82} 
y moleculares~\cite{Menchero:10,Granados:16}. Entre ellos se destacan 
algunos modelos paramétricos~\cite{Gombas:56,Green:69,Klapisch:71}, 
basados en el formalismo de HF~\cite{Phillips:59,Herman:63}, de 
polarización del núcleo electrónico~\cite{Dalgarno:70,Bayliss:77} y 
relativistas~\cite{Cowan:76,Lee:77}. Otra técnica conocida es la 
determinación de potenciales centrales a partir de funciones de onda y/o 
densidades electrónicas, conocida como método de inversión. Este 
procedimiento ha sido estudiado de forma extensa en la 
DFT~\cite{Wu:03,Gaiduk:13,Ryabinkin:15,Schipper:97,deSilva:12,
Kananenka:13,Mura:97,Jacob:11}. 
%DFT~\cite{Wu:03,Gaiduk:13,Ryabinkin:15}. 
%En algunos trabajos, los autores proponen un 
%potencial KS particular y resuelven el sistema de ecuaciones 
%correspondiente, obteniendo orbitales KS~\cite{Schipper:97,deSilva:12,
%Kananenka:13}. A través de la inversión, obtienen un potencial KS 
%reconstruido, que coincide con el original en casi en toda su extensión 
%excepto en algunas regiones cerca del origen, donde aparecen grandes 
%oscilaciones. En algunos casos el potencial reconstruido está tan 
%distorcionado que es imposible de reconocerlo~\cite{Mura:97,Jacob:11}. 
En la teoría de colisiones, el método de inversión fue sugerido por 
Hilton \textit{et al.}, en aplicaciones circunscriptas al cálculo de 
procesos de fotoionización en átomos~\cite{Hilton:77,Suzer:77}, 
agua~\cite{Hilton:79} y otras moléculas~\cite{Hilton:80,Crljen:87}. A su 
vez, estos trabajos se refieren a un trabajo previo en polarizabilidad 
atómica llevada a cabo por Sternheimer~\cite{Sternheimer:54} y Dalgarno 
y Parkinson~\cite{Dalgarno:59}. Sin embargo, los autores se enfocaron en 
los resultados de secciones eficaces de fotoionización y no proveyeron 
detalles acerca de la calidad de los potenciales y las funciones de 
onda implementados.

Por otro lado, una amplia variedad de métodos de primeros principios han 
sido desarrollados para predecir secciones eficaces de ionización debido 
al impacto de partículas cargadas, incluyendo modelos 
clásicos~\cite{Catlow:67}, semi-clásicos~\cite{falta}, y mecanico-cuánticos. 
En este último grupo, destacamos tres grados de aproximación; la primera 
aproximación de Born~(\acs{fba})~\cite{Bates:62,McDowell:61}, los 
métodos de onda continua distorcionada~(\acs{cdw})~\cite{Crothers:10,
Rivarola:87} y los de acoplamiento cercano~(\acs{cc})~\cite{Pindzola:07,
Burke:11,Pindzola:16,Bray:17}, que constituyen el estado del arte.

En este Capítulo presentamos el método de inversión depurada (\acs{dim}),
que consiste en la inversión y optimización de potenciales efectivos, 
con el fin de obtener una descripción precisa de la estructura de atómos 
y moléculas simples. La descripción de los blancos apartir del DIM es 
luego examinada por su desempeño en la ionización debido a fotones e 
iones cargados. En la Sección~\ref{sec:dimatomos} desarrollamos el DIM 
para sistemas atómicos. El método reduce el sistema de ecuaciones 
acopladas de muchos electrones a un sistema de ecuaciones de tipo 
Kohn--Sham. Los potenciales son obtenidos mediante la inversión directa 
de las ecuaciones resultantes. En nuestra implementación, las soluciones 
de estas ecuaciones están dadas por la teoría de HF. Luego, los 
potenciales invertidos son sometidos a una depuración. Esta técnica 
consiste en descartar regiones del potencial con rasgos inconsistentes, 
heredados por el procedimiento de inversión, y la imposición de 
condiciones de borde apropiadas de forma analítica. De esta manera, los 
potenciales se parametrizan y optimizan mediante simple expresiones. 

A pesar de que el procedimiento de inversión es directo, su 
implementación no necesariamente produce cargas efectivas suaves. Por 
ejemplo, los nodos de los orbitales HF son traducidos al potencial por 
la inversión como grandes polos ficticios. Más aún, en orbitales sin 
nodos, el rápido decaimiento asintótico de algunos orbitales genera 
inmensas divergencias numéricas. En la Sección~\ref{subsec:probinv} 
investigamos el origen teórico de las inconsistencias devenidas de la 
inversión. En particular, revisamos la posibilidad de que los nodos de 
los orbitales sean además puntos de inflexión a partir de un experimento 
numérico. Además, examinamos el decaimiento exponencial dictado por la teoría de Hartree--Fock y su relación con la divergencia de las cargas a grandes distancias.
% Dos conclusiones surgen de nuestro estudio; por un lado, 
% los órdenes de aproximación implementados no son suficientes para 
% obtener orbitales a partir de los cuales se pueden desprender 
% potenciales efectivos de forma directa. Por otro lado, nuestro estudio 
% parece sugerir que la exitosa implementación del método de inversión 
% con orbitales HF requiere una condición extra en la minimización de la 
% energía en la teoría de Hartree--Fock. \textit{no se si la 
% conclusiones van en la introducción}}
% esto es, que los nodos de los orbitales HF también sean turning points.

La descripción de sistemas moleculares constituye un verdadero desafío 
debido a su simetría no esférica y multicéntrica. En el último siglo se 
han desarrollado diversas aproximaciones teóricas semi-empíricas y de 
primeros principios para tal fin~\cite{Szabo:96,Helgaker:00,
Schaefer:04}. En la Sección~\ref{sec:dimmoleculas} 
presentamos una extensión del método DIM para sistemas moleculares 
simples, proveyendo una nueva expresión paramétrica para potenciales 
moleculares. A diferencia del caso atómico, los orbitales moleculares 
son descritos a partir de bases gaussianas, y la implementación de 
conjuntos de base~(\acs{bs}) requiere una nueva técnica de depuración 
debido a las oscilaciones introducidas por la base. 

El objetivo  principal de este trabajo es ilustrar el uso efectivo de 
potenciales DIM en la teoría de colisiones atómicas. Para este fin 
realizaremos ciertas simplificaciones: 
\begin{enumerate}
\item Los cálculos están restringidos a los hamiltonianos que describen
sólo el proyectil, el blanco y el electrón activo;
\item Los elementos de la matriz de transición se consideran en primer 
orden perturbativo. Si el primer orden tiene un muy mal desempeño, 
consideramos que no tendría sentido extender el cálculo a términos 
mayores de la serie. 
\end{enumerate}
Los procesos colisionales que examinaremos serán la ionización de atómos 
multielectrónicos y moléculas con pocos átomos debido al impacto de 
protones y fotones. Por simplicidad, restringiremos nuestros cálculos 
al marco de la FBA (ver Apéndice~\ref{app:born}), que se conoce  
reproduce razonablemente las secciones eficaces experimentales de 
diversos procesos en el rango intermedio--alto de energías incidentes 
del proyectil. Más aún, dentro de este rango de energía y orden de 
aproximación, los orbitales de Hartree--Fock proporcionan una buena 
descripción del blanco y el límite a altas energías de las secciones 
eficaces es el correcto. No es nuestra intención hacer un repaso 
detallado de los métodos teóricos existentes, descritos extensamente en 
la bibliografía, y sus resultados. 

En la Sección~\ref{subsec:dimtarget} se presentan resultados de la  
implementación del DIM para describir la estructura de blancos atómicos 
y moleculares. El uso de los potenciales DIM y la FBA para describir la 
ionización simple de átomos y moléculas por impacto de partículas 
cargadas y fotones se muestra en la Sección~\ref{subsec:dimion}.
En total, se examinaran cuatro blancos: helio, nitrógeno, neón y metano. 
En este contexto, inspeccionaremos la influencia de la descripción del 
blanco mediante el potencial DIM en las secciones eficaces cuando la 
complejidad del blanco incrementa según el número de electrones.

\begin{comment}
\end{comment}

%%%%%%%%%%%%%%%%%%%%%%%%%%%%%%%%%%%%%%%%%%%%%%%%%%%%%%%%%%%%%%%%%%%%%%%%
\section{Descripción de blancos atómicos}
%%%%%%%%%%%%%%%%%%%%%%%%%%%%%%%%%%%%%%%%%%%%%%%%%%%%%%%%%%%%%%%%%%%%%%%%
\label{sec:atomos}

La ecuación de Schrödinger para un sistema de $n$ electrones y 
un núcleo de carga $Z$ se escribe como
\begin{equation}
\left[
\sum_{i=1}^N \left(-\frac{1}{2}\nabla^2_{{\mathbf r}_i}
                   -\frac{Z}{{\mathbf r}_i}\right) + 
\sum_{i<j=1}^N \frac{1}{|{\mathbf r}_i - {\mathbf r}_j |} 
\right] \Psi\left(q_1,q_2,\cdots,q_N\right) 
= E\, \Psi\left(q_1,q_2,\cdots,q_N\right) 
\, ,
\end{equation}
donde $q_i$ está compuesto por la coordenada espacial $\mathbf{r}_i$ y 
de espín $\chi_i$ del electrón $i$--ésimo. El tratamiento explícito de 
la ecuación de Schr\"odinger en los casos de iones multielectrónicos es 
una tarea, literalmente, imposible de realizar. Por lo tanto, se debe 
recurrir a aproximaciones que permitan describir al sistema de forma 
precisa. Uno de los métodos más implementados con este fin está dado por 
la teoría de Hartree--Fock. 

En la aproximación de Hartree--Fock, se asume --en concordancia con el 
modelo de partícula independiente y el principio de exclusión de Pauli-- 
que la función de onda $N$-electrónica es un determinante de Slater 
$\Phi$. En otras palabras, ésta es un producto antisimétrico de 
espín--orbitales electrónicos individuales $\phi$. El determinante de 
Slater óptimo se obtiene usando el método variacional para determinar 
los mejores espín--orbitales electrónicos individuales. Así, el método 
de HF es un caso particular del método variacional, donde la función de 
prueba para los $N$ electrones atómicos es un determinante de Slater 
cuyos espín--orbitales individuales están optimizados. La función de 
onda de los $N$ electrones atómicos $\Psi(q_1,q_2,\dots,q_N)$, solución 
de la ecuación de Schrödinger, puede ser representada sólo por una suma 
infinita de determinantes de Slater, de manera que esta aproximación 
puede ser considerada como un primer paso en la determinación de 
funciones de onda atómicas y las energías. 

 % Bransden pags 388, 389, 394, 395
Las ecuaciones de Hartree--Fock se pueden reescribir de forma compacta, 
\begin{equation}
\left[-\frac{1}{2}\nabla_{\mathbf{r}_i}^2-\frac{Z}{r_i}
+\mathcal{V}^{\mathrm{dir}}(\mathbf{r}_i)
-\mathcal{V}^{\mathrm{int}}(q_i) \right]
\phi_{\lambda}(q_i)=\varepsilon_{\lambda}\,\phi_{\lambda}(q_i)\,,
\label{eq:compactHFeqs}
\end{equation}
donde los operadores directo y de intercambio están dados por
\begin{align}
\mathcal{V}^{\mathrm{dir}}(\mathbf{r}_i) &
=\sum_\mu \mathcal{V}_\mu^{\mathrm{dir}}(\mathbf{r}_i)
=\int\frac{\phi_{\mu}^*(\mathbf{r}_j)\phi_{\mu}(\mathbf{r}_j)\, 
d\mathbf{r}_j}{r_{ij}} \,, \\
\mathcal{V}^{\mathrm{int}}(q_i) 
&=\sum_\mu \mathcal{V}_\mu^{\mathrm{int}}(q_i) \,,\\
\mathcal{V}_\mu^{\mathrm{int}} \phi_{\lambda}(q_i) &= \left[
\int\frac{\phi_{\mu}^*(q_j)\phi_{\lambda}(q_j)\,dq_j}{r_{ij}} \right] 
\phi_\mu(q_i)\,.
\end{align}
En el caso de átomos con capas cerradas, asumiendo que los orbitales 
espaciales se pueden separar en sus partes radial y angular
\begin{equation}
\phi_{nlm}(\mathbf{r})=\frac{1}{r}u_{nl}(r)Y_l^m(\theta,\phi)\,,
\label{eq:centralfield-wave}
\end{equation}
las ecuaciones radiales de HF son exactamente iguales a las ecuaciones 
de Schr\"odinger bajo la aproximación de campo central, 
\begin{equation}
 \left[ -\frac{1}{2}\frac{d^2}{dr^2} + \frac{l(l+1)}{2r^2} +
 V(r) \right] u_{nl}(r) = \varepsilon_{nl} \, u_{nl}(r)\,.
\label{eq:eqSchroRadial}
\end{equation}

%A partir de ahora, simplificaremos entonces las 
%expresiones, y sólo consideraremos a las funciones radiales 
%$u(r)$ en lugar de $u({\mathbf r})$. Por lo tanto, 
%$V_{\mu}^{\mathrm{ dir}}({\mathbf r})$ pasa a ser 
%$V_{\mu}^{\mathrm{ dir}}(r)$, y 
%el potencial directo $V^{\mathrm{ dir}}({\mathbf r})$ es 
%$V^{\mathrm{ dir}}(r)$, etc..

%%%%%%%%%%%%%%%%%%%%%%%%%%%%%%%%%%%%%%%%%%%%%%%%%%%%%%%%%%%%%%%%%%%%%%%%
\section{El método de la inversión depurada}
%%%%%%%%%%%%%%%%%%%%%%%%%%%%%%%%%%%%%%%%%%%%%%%%%%%%%%%%%%%%%%%%%%%%%%%%
\label{sec:dimatomos}

El método de inversión depurada consiste en suponer que los orbitales
de átomos multielectrónicos pueden ser representados por las soluciones
de un sistema de ecuaciones de un electrón activo para el cual existe 
un potencial efectivo que gobierna la dinámica del átomo. 
En un blanco atómico cualquiera, suponiendo que las soluciones de la 
Ec.~(\ref{eq:eqSchroRadial}) pueden ser aproximadas por las soluciones 
HF $u_{nl}^{\mathrm{HF}}$ y $\varepsilon_{nl}^{\mathrm{HF}}$, existe un 
potencial efectivo local correspodiente $V_{nl}^{\mathrm{HF}}$ que las 
genera. Así, mediante esta suposición, convertimos las ecuaciones HF en 
un conjunto de ecuaciones de tipo Kohn--Sham cuyas soluciones están 
dadas por la teoría de Hartree--Fock,
\begin{equation}
\left[ 
-\frac{1}{2}\frac{d^{2}}{dr^{2}} + \frac{l(l+1)}{2r^{2}} + 
V_{nl}^{\mathrm{HF}}(r) 
\right] u_{nl}^{\mathrm{HF}}(r)
   = \varepsilon_{nl}^{\mathrm{HF}}\, u_{nl}^{\mathrm{HF}}(r) \, .
\label{eq:KS}
\end{equation}
A su vez, y debido a la naturaleza de las soluciones, el potencial 
efectivo generador
\begin{equation}
V_{nl}^{\mathrm{HF}}(r) = V^{\mathrm{ext}}(r) + 
V^{\mathrm{dir}}(r) + V_{nl}^{\mathrm{int}}(r) \, ,  
\label{eq:veff}
\end{equation}
está compuesto por un potencial externo $V^{\mathrm{ext}}$, 
el potencial directo o de Hartree $V^{\mathrm{dir}}$, y el potencial
de intercambio orbital $V_{nl}^{\mathrm{int}}$. A diferencia de los
potenciales más generales, este potencial no tiene el término de 
correlación electrónica ya que las soluciones HF no lo incluyen. En el 
caso de un átomo en fase gaseosa, el potencial externo está queda 
determinado por el campo coulombiano del núcleo, mientras que el 
potencial de Hartree constituye la repulsión electrostática electrónica. 


%%%%%%%%%%%%%%%%%%%%%%%%%%%%%%%%%%%%%%%%%%%%%%%%%%%%%%%%%%%%%%%%%%%%%%%%
\subsection{La inversión directa}
%%%%%%%%%%%%%%%%%%%%%%%%%%%%%%%%%%%%%%%%%%%%%%%%%%%%%%%%%%%%%%%%%%%%%%%%
\label{subsec:inversion}

Dado que los orbitales $u_{nl}^{\mathrm{HF}}$ y energías  
$\varepsilon^{\mathrm{HF}}$ de Hartree--Fock son conocidas (calculadas 
numéricamente con los códigos {\sc hf} de C. F. 
Fischer~\cite{FroeseFischer:97}, y {\sc nrhf} de 
W. Johnson~\cite{Johnson:07}), es posible implementar el método de 
inversión~(\acs{IM}) a la Ec.~(\ref{eq:KS}). Así, obtenemos el 
\textit{potencial invertido HF} 
\begin{equation}
V_{nl}^{\mathrm{HF}}(r) = 
\frac{1}{2}\frac{1}{u_{nl}^{\mathrm{HF}}(r)}
\frac{d^2}{dr^{2}}u_{nl}^{\mathrm{HF}}(r) - 
\frac{l(l+1)}{2r^{2}}+\varepsilon _{nl}^{\mathrm{HF}} \,,
\label{eq:VHF}
\end{equation}
el cual queda totalmente determinado por las soluciones HF,
$u_{nl}^{\mathrm{HF}}$ y $\varepsilon_{nl}^{\mathrm{HF}}$.

\begin{figure}[t!]
\centering
\includegraphics[width=0.93\textwidth]{dim/pot-charge.eps}
\vspace{-0.3cm}
\caption[Características físicas del potencial y carga efectiva.]
{Ilustración de las características físicas esperadas del (a) potencial 
y (b) carga efectiva del blanco.}
\label{fig:potycharge}
\end{figure}

Inspeccionando el comportamiento de los potenciales invertidos, notamos 
que éstos tienen una forma coulombiana. Suponiendo que todos los 
potenciales invertidos siguen este comportamiento, ilustrado en el panel
izquierdo de la Fig.~\ref{fig:potycharge}, es conveniente definir una 
\textit{carga invertida HF} 
\begin{equation}
Z_{nl}^{\mathrm{HF}}(r) \equiv -r \, V_{nl}^{\mathrm{HF}}(r) \,.
\label{eq:Zeff}
\end{equation}
Esta carga efectiva, esquematizada en el panel derecho de la 
Fig.~\ref{fig:potycharge}, deberá ser suave y cumplir con condiciones de 
borde definidos por la naturaleza del blanco a describir. Esto es, en el 
origen la carga deberá ser igual a la carga nuclear del átomo, mientras 
que asintóticamente la carga debe ser igual a uno, que resulta del 
apantallamiento electrónico.

%%%%%%%%%%%%%%%%%%%%%%%%%%%%%%%%%%%%%%%%%%%%%%%%%%%%%%%%%%%%%%%%%%%%%%%%
\subsection{Problemas de la inversión directa}
%%%%%%%%%%%%%%%%%%%%%%%%%%%%%%%%%%%%%%%%%%%%%%%%%%%%%%%%%%%%%%%%%%%%%%%%
\label{subsec:probinv}

\begin{figure}
\centering
\includegraphics[width=0.88\textwidth]{dim/dim_2sK.eps} 
\vspace{-0.3cm}
\caption[Orbital radial y carga efectiva correspondiente.]
{(a) Orbital radial $u_{2s}^{\mathrm{HF}}$ del estado fundamental de K.
%con nodo genuino ($0.11$~a.u.) y nodo espurio ($5.79$~a.u.). 
(b) Cargas invertida $Z_{2s}^{\mathrm{HF}}$ (línea discontinua) 
y depurada $Z_{2s}^{\mathrm{DIM}}$ (línea sólida).}
\label{fig:2sK}
\end{figure}

A pesar de que el procedimiento de inversión dado por la 
Ec.~(\ref{eq:VHF}) es directo, su implementación no necesariamente 
produce cargas efectivas suaves. En la Fig.~\ref{fig:2sK} se muestra 
(a)~el orbital $u_{2s}^{\mathrm{HF}}$ del átomo de potasio en su estado 
fundamental y (b) su correspondiente carga invertida 
$Z_{2s}^{\mathrm{HF}}$ (línea discontinua).
% y depurada $Z_{2s}^{\mathrm{DIM}}$ (línea sólida). 
El orbital $2s$ tiene dos nodos: un nodo genuino en 
$r\approx 0.111$~a.u. y un nodo espúreo en $r\approx 5.79$~a.u.. Usamos 
el término genuino para denotar los nodos que se deben estrictamente de 
la resolución de la ecuación de Schr\"odinger y cumplen la relación 
$r_n=n-l-1$. Los nodos espúreos aparecen a grandes distancias, en 
regiones donde la amplitud del orbital es muy pequeña. Ambos nodos son 
traducidos a la carga invertida como polos; el polo genuino 
(correspondiente al nodo genuino) tiene una amplitud pequeña, mientras 
que el polo espúreo es tan grande que está fuera de escala. Además, la 
carga $Z_{2s}^{\mathrm{HF}}$ presenta una divergencia pronunciada para 
valores $r>1$~a.u.. Las justificaciones numéricas a la presencia de 
estos defectos son simples. Los polos surgen porque el orbital radial 
$2s$ en el denominador de la Ec.~(\ref{eq:VHF}) y su derivada segunda no 
se anulan entre sí en los nodos, mientras que la divergencia asintótica 
tiene origen en el coeficiente del término exponencial que sigue la 
función $u_{2s}^{\mathrm{HF}}$ a grandes distancias.

%%% Fallas
%%% Nodos que no son puntos de inflexión  -> Derivada 2da
%%% Nodos que son espúreos (no localidad) -> Fischer
%%% Divergencia a grandes r               -> Hartree

% Defectos: 
% Los defectos de las cargas invertidas surgen del propio método de Hartree--Fock; los nodos genuinos no son estrictamente puntos de  inflexión, el decaimiento exponencial de los orbitales sigue el comportamiento orbitales tipo Hartree, mientras que el método autoconsistente conduce a la aparición de nodos espúreos.

En general, las cargas resultantes de la inversión de los orbitales HF 
tienen asociadas alguno de los tres defectos. A partir de dos ejemplos, 
examinamos cada uno de ellos y su transfondo téorico a través un 
experimento numérico. Sin embargo, el análisis se puede generalizar para 
los orbitales HF de cualquier átomo no relativista.

%%%%%%%%%%%%%%%%%%%%%%%%%%%%%%%%%%%%%%%%%%%%%%%%%%%%%%%%%%%%%%%%%%%%%%%%
\subsubsection{Nodos genuinos}
%%%%%%%%%%%%%%%%%%%%%%%%%%%%%%%%%%%%%%%%%%%%%%%%%%%%%%%%%%%%%%%%%%%%%%%%

Para entender la existencia del polo genuino, vamos a examinar el 
orbital $2s$ del átomo de magnesio y sus puntos de inflexión. En la 
Fig.~\ref{fig:example2sMg} se muestra la función $u_{2s}^{\mathrm{HF}}$ 
y su derivada segunda numérica (escalada por un factor). Las dos raices 
de $u_{2s}^{\mathrm{HF}}''$ son puntos de inflexión de 
$u_{2s}^{\mathrm{HF}}$ y se corresponden a (1)~el nodo genuino y (2)~el 
punto de retorno clásico. A primera vista, el nodo genuino y el primer 
punto de inflexión parecen coincidir; sin embargo, si examinamos el 
recuadro vemos que ese no es el caso. Definiendo $\Delta r$ como la 
distancia entre el nodo del orbital y la primera raiz de su derivada 
segunda, encontramos que hay una pequeña distancia 
$\Delta r=1\times 10^{-3}$~a.u. entre las primeras raices de ambas 
funciones. Si bien no existe ninguna restricción en la teoría que fuerce 
a los nodos genuinos HF a ser también puntos de inflexión, este fenómeno 
se repite en todos los orbitales con nodos de los átomos de la tabla 
periódica descritos mediante la teoría de Hartree--Fock. 

\begin{figure}
\vspace{-0.4cm}
\centering
\includegraphics[width=0.85\textwidth]{dim/example_2sMg.eps} 
\vspace{-0.45cm}
\caption[Orbital radial y su derivada segunda.]
{Orbital radial $u_{2s}^{\mathrm{HF}}$ del estado fundamental de Mg y su 
derivada segunda escalada.}
\label{fig:example2sMg}
%\end{figure}

\vspace{0.4cm}
%\begin{figure}
%\centering
\includegraphics[width=0.85\textwidth]{dim/dr_2sMg.eps} 
\vspace{-0.45cm}
\caption[Dependecia de $\Delta r$ del orden de aproximación numérica.]
{Dependecia de $\Delta r$ del orden de aproximación numérica en el 
orbital $2s$ del átomo de potasio. (a) Primer orden y 200 puntos, (b) 
400 puntos; (c) octavo orden y 1000 puntos.
a algoritmos.}
\label{fig:dr2sMg}
\end{figure}

Creemos que la cercanía entre los nodos genuinos de los orbitales y las 
correspondientes raices de su segunda derivada no es casual, y que los 
nodos genuinos en la teoría de Hartree--Fock deben ser puntos de 
inflexión. El experimento numérico diseñado para indagar esta hipótesis 
consiste en realizar varias aproximaciones, con mejoras sucesivas en su 
precisión, examinando el comportamiento del valor $\Delta r$ resultante. 
La calidad de los métodos numéricos usados para resolver las ecuaciones 
de HF se pueden evaluar a través de la variación del orden de precisión 
de los algoritmos y la densidad de puntos de las grillas numéricas. En 
este experimento usamos el método lineal de pasos múltiples de Adams--
Moulton para las ecuaciones diferenciales y el método de diferenciación 
Lagrangiana para las derivadas. La metodología propuesta se implementa 
modificando el código \textsc{nrhf} de Johnson~\cite{Johnson:07}, que 
utiliza aproximaciones de octavo orden por defecto. No obstante, los 
mismos resultados y conclusiones se obtienen con el código~\textsc{hf} 
de Fischer~\cite{FroeseFischer:97}.

La Fig.~\ref{fig:dr2sMg} muestra $u_{2s}^{\mathrm{HF}}$ de Mg (línea 
sólida) y su segunda derivada numérica (línea discontinua) en las 
proximidades del nodo implementando tres grados de aproximación 
distintos en los métodos numéricos. Los cálculos menos precisos se 
muestran en la  Fig.~\ref{fig:dr2sMg}(a). En este caso se ha 
implementando el primer orden de los algoritmos numéricos y una grilla 
numérica de 200 puntos (mínimo valor necesario para obtener 
convergencia). Esta aproximación resulto en el mayor valor para 
$\Delta r$, siendo éste igual a $8\times 10^{-3}$~a.u.. Aumentando el 
número de puntos a 400, este valor se reduce a 
$\Delta r=4\times 10^{-3}$~a.u., como se muestra en la 
Fig.~\ref{fig:dr2sMg}(b). Incrementando el número de puntos y el orden 
de aproximación de los algoritmos, en la Fig.~\ref{fig:dr2sMg}(c) se 
muestra el mejor resultado posible, donde 
$\Delta r=1\times 10^{-3}$~a.u.. Éste es el mayor grado de precisión 
alcanzado con los métodos numéricos descritos; aún considerando un 
número mayor de puntos en la grilla numérica, los resultados no varían. 

Realizamos un cálculo adicional usando el método del potencial efectivo 
optimizado (\acs{oep}) desarrollado por Talman~\cite{Sharp:53,Talman:76,
Talman:89}. La Fig.~\ref{fig:dr2sMg}(c) también muestra el orbital 
$u_{2s}^{\mathrm{OEP}}$ de Mg cerca del nodo con una línea raya-punto. 
Sin embargo, y debido al caracter local del potencial, su segunda 
derivada $u_{2s}^{\mathrm{OEP}}''$ (línea punto-raya-punto) es cero en 
el nodo. 

\textcolor{red}{Es posible que la no localidad del método de Hartree--
Fock sea responsable de que los nodos genuinos en los orbitales no sean 
puntos de inflexión.} Una exploración más en detalle, con mayores 
órdenes de aproximación en los métodos numéricos, sería necesaria para 
descartar esta hipótesis. La excelente reproducción de los orbitales HF 
mediante el potencial local OEP parece sugerir que esta premisa es 
correcta. De ser así, se podría agregar una restricción adicional al 
procedimiento variacional autoconsistente de Hartree--Fock. En 
definitiva, el cumplimiento de esta restricción aseguraría un potencial 
local, en principio, sin polos.

%%%%%%%%%%%%%%%%%%%%%%%%%%%%%%%%%%%%%%%%%%%%%%%%%%%%%%%%%%%%%%%%%%%%%%%%
\subsubsection{Decaimiento exponencial}
%%%%%%%%%%%%%%%%%%%%%%%%%%%%%%%%%%%%%%%%%%%%%%%%%%%%%%%%%%%%%%%%%%%%%%%%

Los orbitales de los electrones ligados decaen exponencialmente para 
distancias mayores al punto de retorno clásico. El punto de retorno 
clásico se define como la posición en la cual la energía es igual al 
potencial efectivo. El decaimiento asintótico de los orbitales HF han 
sido estudiados por varios autores desde que Handy, Maron y Silverstone 
(HMS)~\cite{Handy:69} demostraron que a grandes distancias $r$, la parte 
radial de las funciones orbitales están determinadas por la energía del 
orbital molecular de mayor ocupación (\acs{homo}) 
$\varepsilon_{\mathrm{HOMO}}^{\mathrm{HF}}$. Esto es 
\begin{equation}
\lim_{r \rightarrow \infty} u_{nl}^{\mathrm{HF}}(r) =  
\exp(- \sqrt{- 2 \varepsilon_{\mathrm{HOMO}}^{\mathrm{HF}} } r )  \, .
\label{eq:rHF}
\end{equation}
Este comportamiento fue confirmado por derivaciones posteriores, por 
ejemplo, los trabajos de HMS~\cite{Handler:80}, y de Ishida y 
Ohno~\cite{Ishida:92}. Por otro lado, los orbitales que corresponden a 
potenciales esféricos tienen un decaimiento asintótico de tipo 
Hartree~\cite{Casida:89},
\begin{equation}
\lim_{r \rightarrow \infty} u_{nl}^{\mathrm{DIM}}(r) =  
\exp(- \sqrt{- 2 \varepsilon_{nl}^{\mathrm{HF}} } r ) \,.
\label{eq:rHlike}
\end{equation}
El término ``tipo-Hartree'' puede resultar confuso ya que la energía 
del orbital $\varepsilon_{nl}^{\mathrm{HF}}$ se corresponde a valores 
donde se ha considerado el término de intercambio. El comportamiento 
asintótico de los orbitales HF se puede examinar en detalle a través de 
la derivada logaritmica de los orbitales radiales, 
\begin{equation}
L_{nl}(r) \equiv r \frac{d \log{u_{nl}}}{d r}\,,
\label{eq:Lnl}
\end{equation}
que se comporta de forma lineal para funciones $u_{nl}$ que decaen 
exponencialmente. 

La Fig.~\ref{fig:LnsK} muestra la derivada logarítmica de los orbitales 
HF de momento angular $s$ del átomo de potasio. Los orbitales HF se dan 
con líneas discontinuas (capas internas) y sólidas (capa de valencia). A 
grandes distancias, estos orbitales presentan el comportamiento de 
Hartree--Fock dado por la Ec.~\ref{eq:rHF}: los orbitales de las capas 
internas siguen el decaimiento asintótico del \acs{homo}. Además, 
incluimos el comportamiento de tipo Hartree correspondiente a cada 
orbital (líneas punteadas). Observamos que las funciones 
$u_{nl}^{\mathrm{HF}}$ tienen este decaimiento exponencial a partir del 
punto de retorno clásico de cada orbital y hasta $0.4$~a.u., $1.5$~a.u. 
y 5~a.u. en los orbitales $1s$, $2s$ y $3s$, respectivamente. 

Como habiamos anticipado, las divergencias en la carga invertida se 
deben al coeficiente del término exponencial de $u_{nl}(r)$ a grandes 
distancias. Por ejemplo, el comportamiento asintótico del potencial 
invertido~(\ref{eq:VHF}) correspondiente a los orbitales $s$ de K está 
dado por
\begin{equation}
\lim_{r \rightarrow \infty} V_{ns}^{\mathrm{HF}}(r)=
-\left(\varepsilon_{\mathrm{HOMO}}^{\mathrm{HF}}
-\varepsilon_{ns}^{\mathrm{HF}}\right) \,,
\label{eq:asintoticoVHF}
\end{equation}
que es siempre distinto de cero, excepto para el \acs{homo}.

Por otro lado, los polos de $L_{nl}(r)$ de la Fig.~\ref{fig:LnsK}  
corrresponden a los nodos de $u_{nl}^{\mathrm{HF}}$. En general, los 
nodos de dos orbitales con el mismo momento angular que no coinciden no 
surgen de la imposición de ortogonalidad y son espúreos. 

\begin{figure}
\centering
\includegraphics[width=0.9\textwidth]{dim/Lns_K.eps} 
\caption[Comportamiento asintótico de los orbitales HF.]
{Comportamiento asintótico de los orbitales HF según $L_{nl}$, dada por 
la Ec.~\ref{eq:Lnl}, de los orbitales $s$ del átomo de K.}
\label{fig:LnsK}
\end{figure}

%%%%%%%%%%%%%%%%%%%%%%%%%%%%%%%%%%%%%%%%%%%%%%%%%%%%%%%%%%%%%%%%%%%%%%%%
\subsubsection{Nodos espúreos}
%%%%%%%%%%%%%%%%%%%%%%%%%%%%%%%%%%%%%%%%%%%%%%%%%%%%%%%%%%%%%%%%%%%%%%%%
\label{subsubsec:espureos}

Los orbitales de HF pueden tener oscilaciones y, por lo tanto, nodos 
espúreos para valores grandes de $r$. La existencia de estas 
oscilaciones han sido señaladas por Fischer~\cite{FroeseFischer:97}, y 
no son consecuencia de la implementación numérica sino que es inherente 
al método de Hartree--Fock. Estas oscilaciones pueden encontrarse en al 
menos un orbital de los elementos de la tabla periódica, desde el Mg en 
adelante. \textcolor{red}{Es probable que este comportamiento se deba al 
término de intercambio a grandes distancias.} Los nodos espúreos 
aparecen en regiones donde la amplitud del orbital es muy pequeña y su 
existencia, por lo general, puede ser ignorada.

En el ejemplo de la Fig.~\ref{fig:LnsK} vemos que el orbital $1s$ del K 
tiene dos nodos espúreos en $0.99$~a.u. y $5.68$~a.u., mientras que el 
orbital $2s$ tiene un nodo espúreo en $5.78$~a.u.. 
\textcolor{red}{Notamos que los nodos espúreos aparecen, en principio, 
como resultado del cambio en el decaimiento asintótico de tipo 
Hartree~(\ref{eq:rHlike}) al de Hartree--Fock~(\ref{eq:rHF}), que 
incluye el intercambio.}

%%%%%%%%%%%%%%%%%%%%%%%%%%%%%%%%%%%%%%%%%%%%%%%%%%%%%%%%%%%%%%%%%%%%%%%%
\subsection{Método de depuración}
%%%%%%%%%%%%%%%%%%%%%%%%%%%%%%%%%%%%%%%%%%%%%%%%%%%%%%%%%%%%%%%%%%%%%%%%
\label{subsec:depuracion}

Para poder sortear las dificultades numéricas que resultan de la 
implementación de la inversión, hemos desarrollado el método de 
depuración. La depuración consiste en optimizar cargas efectivas en 
lugar de potenciales efectivos. Así, somos capaces de restringir los 
valores de cualquier potencial a tener las condiciones de borde 
correctas, forzando el comportamiento de la carga efectiva invertida 
según
\begin{equation}
Z_{nl}^{\mathrm{DIM}}(r) \, \rightarrow 
\bigg\{ 
\begin{array}{ll}
Z_{N}  \ \  & \ \ \text{as\ \ }r  \rightarrow 0\  \\ 
1           & \ \ \text{as\ \ }r  \rightarrow \infty \ 
\end{array}\,,
\label{eq:Zasympt}
\end{equation}
donde $Z_N$ es la carga nuclear del blanco atómico. Una vez que se 
define la carga en los bordes, podemos obtener una expresión analítica
suave para $Z_{nl}^{\mathrm{DIM}}$, ajustando $Z_{nl}^{\mathrm{HF}}$ en
el mayor rango posible, exceptuando las zonas de comportamiento errático.
Estas condiciones se pueden cumplir imponiéndole a la carga efectiva 
DIM que siga la expresión analítica dada por
\begin{equation}
Z_{nl}^{\mathrm{DIM}}(r)= \sum_{j=1}^{n} z_j e^{-\alpha_j r}+1 \,,
\label{eq:atomzDIM}
\end{equation}
donde $\Sigma_j z_j=Z_N-1$. Luego, los parámetros $z_j$ y $\alpha_j$ 
son optimizados de manera tal que reproduzcan las soluciones de HF de 
manera precisa.

%%%%%%%%%%%%%%%%%%%%%%%%%%%%%%%%%%%%%%%%%%%%%%%%%%%%%%%%%%%%%%%%%%%%%%%%
\subsubsection{Optimización de la carga DIM}
%%%%%%%%%%%%%%%%%%%%%%%%%%%%%%%%%%%%%%%%%%%%%%%%%%%%%%%%%%%%%%%%%%%%%%%%
\label{subsec:optDIM}

Un asunto muy importante en la optimización del potencial está dado por 
la autoconsistencia dentro de los códigos numéricos implementados para 
el cálculo de las soluciones a ser invertidas. Para tal fin, hemos 
estudiado los códigos de Hartree--Fock~\cite{FroeseFischer:97,
Johnson:07} y hemos implementado las grillas numéricas específicas de 
cada código, incluyendo los mismos métodos para el cálculo de derivadas. 

El ajuste del conjunto de parámetros $\left\{z_j,\alpha_j\right\}$ 
requiere un gran nivel de experiencia y detalle. El procedimiento 
general para la obtención de los potenciales DIM se esquematiza en la 
Fig.~\ref{fig:procedimientoDIM}. La clave de una optimización exitosa 
está dada por la definición de la región de ajuste: tiene que ser lo más 
extensa posible pero descartando los puntos cercanos a los polos y 
divergencias. De esta manera, la carga DIM $Z_{nl}^{\mathrm{DIM}}$ se 
superpone con la carga invertida $Z_{nl}^{\mathrm{HF}}$ a lo largo de la 
región interna bien comportada, permitiendo un ajuste preciso. La 
segunda parte de la optimización consiste en la obtención de una semilla 
para los parámetros $\left\{z_j,\alpha_j\right\}$. Estos valores son 
obtenidos mediante la resolución de la ecuación normal definida por el 
problema (ver detalles en Apéndice~\ref{app:ecnormal}). Los valores de 
las semillas obtenidas definen un potencial efectivo inicial con el que 
se resuelve la Ec.~(\ref{eq:eqSchroRadial}). Las soluciones obtenidas 
nos permiten definir una función de costo, la cual es minimizada 
variando los parámetros del problema de forma iterativa. 

\begin{figure}[t]
\centering
\begin{tikzpicture}[remember picture]
%  \tikzset{shift={(current page.center)}}
  \node[process] (inv) {Inversión directa};
  \node[process] (rem) at (inv) [xshift=0cm,yshift=-1.5cm] 
            {Remoción de divergencias};
  \node[process] (eqnorm) at (rem) [xshift=0cm,yshift=-1.5cm] 
            {Ecuación normal};
  \node[process] (var) at (eqnorm) [xshift=0cm,yshift=-1.5cm] 
            {Variación de parámetros};
  \node[process] (diag) at (var) [xshift=-2.5cm,yshift=-2.3cm] 
            {Diagonalización};
  \node[process] (costo) at (var) [xshift=2.5cm,yshift=-2.3cm] 
            {Cálculo de costo};
  \node[decision] (converge) at (costo) [xshift=4cm,yshift=0cm] 
            {¿Convergió?};
  \node[process, fill=blue!20] 
  (dim) at (converge) [xshift=0cm,yshift=-3.2cm] 
            {Potencial DIM};
  \draw [arrow] (inv) -- (rem);
  \draw [arrow] (rem) -- (eqnorm);
  \draw [arrow] (eqnorm) -- (var);
  \draw [arrow, bend right=33] 
            (var.west) 
            to ([xshift=-0.5cm,yshift=0cm]{diag.north});
  \draw [arrow, bend right=53] 
            ([xshift=-0.25cm,yshift=0cm]{diag.south}) 
            to ([xshift=0.25cm,yshift=0cm]{costo.south});
  \draw [arrow, bend right=33] 
            ([xshift=0.5cm,yshift=0cm]{costo.north}) 
            to (var.east);
  \draw [arrow, dashed] (costo) -- (converge);
  \draw [arrow, dashed] 
  (converge) |- (rem.east) node [near start,left] {No};
  \draw [arrow, dashed] (converge) -- (dim.north)  node [midway,right] {Sí};
\end{tikzpicture}
\caption{Procedimiento de optimización del potencial DIM.}
\label{fig:procedimientoDIM}
\end{figure}

Es importante remarcar que la mayoría de los métodos de funcional de la 
densidad están basados en un principio variacional que minimiza el 
funcional de energía. Sin descartar su importancia, la energía es solo
uno de los observables que caracteriza un estado cuántico. Más aún, a 
partir de diferentes funciones de prueba (de formas variadas) e 
implementando un método variacional, es posible reproducir la misma 
energía final. Un simple ejemplo es dado por 
Bartschat~\cite{Albright:93,Bartschat:96}, en donde dos potenciales
diferentes (uno conteniendo intercambio electrónico y otro 
despreciándolo) conducen a energías similares y precisas de la serie de 
Rydberg en varios sistemas de cuasi-un electrón. Sin embargo, examinando 
estos potenciales éstos muestran grandes discrepancias cuando son 
implementados en cálculos de dispersión \cite{BartschatBray:96}. Por lo 
tanto, además de los valores de energía, hemos incluido en nuestra 
optimización valores medios de $\langle r^{n} \rangle$, tal que 
$n=-1,1$. La inclusión de estos observables nos permite caracterizar la 
precisión del orbital DIM acerca ($n=-1$) y lejos ($n=1$) del origen. 
Así, la función de costo que define nuestra optimización está dada por 
los errores relativos de cada una de estas cantidades
\begin{equation}
J=\sum_X \frac{X^{\mathrm{HF}}-\widebar{X}}{X^{\mathrm{HF}}}\,,
\label{eq:fncosto-dim}
\end{equation}
donde la suma se hace sobre 
$X=\left[E,\langle r \rangle,\langle 1/r \rangle\right]$, siendo 
$X^{\mathrm{HF}}$ los valores de HF y $\widebar{X}$ los resultados 
obtenidos de la diagonización del potencial de prueba. 

A partir de este procedimiento se obtiene la carga invertida depurada 
$Z_{2s}^{\mathrm{DIM}}(r)$ correspondiente al orbital $2s$ del átomo 
de potasio que se muestra con línea sólida en la Fig.~\ref{fig:2sK}(b).
Como se ve en la figura, se satisfacen las condiciones de borde dadas 
por la Ec.~\ref{eq:Zasympt}, donde en el origen $Z_{2s} = 19$, y 
asintóticamente $Z_{2s} \rightarrow 1$.

%%%%%%%%%%%%%%%%%%%%%%%%%%%%%%%%%%%%%%%%%%%%%%%%%%%%%%%%%%%%%%%%%%%%%%%%
\section{Descripción de blancos moleculares}
%%%%%%%%%%%%%%%%%%%%%%%%%%%%%%%%%%%%%%%%%%%%%%%%%%%%%%%%%%%%%%%%%%%%%%%%
\label{sec:moleculas}

En el marco de la aproximación de Born--Oppenheimer, el Hamiltoniano 
molecular no--relativista en el que se consideran sólo las fuerzas Coulombianas puede escribirse como
\begin{equation}
H = - \sum_{i=1}^N \frac{1}{2} \nabla^2_{\mathbf{r}_i} 
    - \sum_{i=1}^N \sum_{\alpha=1}^n \frac{Z_{\alpha}}{
    \left|\mathbf{r}_i-\mathbf{r}_{\alpha}\right|} 
    + \sum_{i<j=1}^N \frac{1}{\left|\mathbf{r}_i-\mathbf{r}_j\right|} 
    + \sum_{\alpha<\beta=1}^n \frac{z_{\alpha}z_{\beta}}{
    \left|\mathbf{r}_{\alpha}-\mathbf{r}_{\beta}\right|}\,,
\label{eq:gralmolHamiltonian}
\end{equation}
donde los índices $i,j$ van sobre todos los electrones y $\alpha,\beta$ 
sobre todos los núcleos. Si consideramos moléculas de tipo $X\!H_n$, el 
Hamiltoniano dado por~\ref{eq:gralmolHamiltonian} se reduce a 
\begin{equation}
H = -\sum_{i=1}^N \frac{1}{2} \nabla^2_{\mathbf{r}_i} 
    - \sum_{i=1}^N \frac{Z_N}{r_i} 
    + \sum_{i=1}^N V_H(r_i)
    + \sum_{i<j}^N \frac{1}{r_{ij}}\,,
\end{equation}
donde
\begin{equation}
V_H(r_i)=
-\sum_{j=1}^{n} \frac{1}{\left|\mathbf{r}_i-\mathbf{R}_{H_j}\right|}\,,
\label{eq:Vhidrogenos}
\end{equation}
siendo $Z_N$ la carga nuclear del átomo más pesado $X$, y 
$\mathbf{R}_{H_j}$ son las coordenadas de los hidrógenos respecto al 
átomo pesado $X$. La ecuación de Schr\"odinger correspondiente, 
$H\Psi=E\Psi$, se resuelve en el marco de la aproximación de campo 
central, donde los orbitales que componen la función de onda se expresan 
según la Ec.~(\ref{eq:centralfield-wave}). En general, los orbitales y 
energías de sistemas moleculares también se obtienen a partir del método 
restringido de Hartree--Fock. El cálculo de las ecuaciones HF 
generalmente implementan conjuntos de bases (\acs{bs}) finitas para la 
representación de los orbitales moleculares (\acsp{mo}). Usualmente, los 
MOs son expresados como una combinación lineal de orbitales atómicos 
(\acs{lcao}), 
\begin{equation}
 \Psi_i=\sum_j c_{ji} \phi_j\,.
\end{equation}
A su vez, los orbitales atómicos pueden ser construidos a partir de 
conjuntos de base de orbitales, por ejemplo, tipo Gaussianos 
(\acsp{gto}).

%%%%%%%%%%%%%%%%%%%%%%%%%%%%%%%%%%%%%%%%%%%%%%%%%%%%%%%%%%%%%%%%%%%%%%%%
\subsection{DIM en moléculas}
%%%%%%%%%%%%%%%%%%%%%%%%%%%%%%%%%%%%%%%%%%%%%%%%%%%%%%%%%%%%%%%%%%%%%%%%
\label{sec:dimmoleculas}

En esta Sección extendemos el método de la inversión depurada para 
obtener potenciales efectivos moleculares. Primero, definimos el método 
a través del cual obtenemos los orbitales moleculares. Luego, planteamos 
el esquema de inversión de los orbitales HF. Finalmente, presentamos las 
modificaciones al método de depuración mediante el cual obtenemos 
potenciales de hidruros.

La expresión del potencial molecular invertido es análoga a la 
Ec.~(\ref{eq:VHF}). Sin embargo, para el caso molecular, las soluciones 
se expresan en conjuntos de base GTO. 
%\textcolor{red}{Acá me falta incluir la explicación de como paso el 
%cálculo de múltiples centros a OCE. No sé si ponerlo.} 
%
La representación dada por los BS introduce nuevas dificultades en el 
método de inversión. Además de las divergencias asintóticas y los polos, 
los potenciales invertidos moleculares presentan grandes 
oscilaciones~\cite{Schipper:97,Mura:97,Jacob:11,Gaiduk:13}. Estas 
oscilaciones prominentes surgen debido a la presencia de ondulaciones en 
los orbitales moleculares, las cuales a su vez se deben al número finito 
de elementos en el conjunto de la base. La segunda derivada de los 
orbitales, necesaria para evaluar la fórmula inversa, amplifica estos 
rasgos. En algunos casos, las oscilaciones son enormes; por ejemplo, 
cerca de átomos electronegativos como el cloro. La aparición de estas 
oscilaciones en los potenciales invertidos nos obliga a incorporar 
medidas adicionales en el procedimiento de depuración.

Para ilustrar el procedimiento de depuración, consideramos el orbital 
$1s$ del átomo de carbono. Primero, resolvemos las ecuaciones de 
Hartree--Fock usando el conjunto de bases \mbox{6-311G} con el código 
{\sc gamess} code~\cite{Schmidt:93,Gordon:05}. Luego, aplicando la  
Ec.~(\ref{eq:VHF}), obtenemos la carga invertida correspondiente. La 
carga resultante $Z_{1s}^{\mbox{\scriptsize 6-311G}}$ se muestra en la 
Fig.~\ref{fig:1sCarbon}(a) con una línea raya-punto. La carga presenta 
oscilaciones en toda su extensión radial, divergiendo para valores 
grandes de $r$. Se realiza el mismo cálculo con el conjunto de base 
universal gaussiano (\acs{ugbs}), que tiene un número significativamente 
mayor de funciones primitivas. La carga invertida correspondiente 
$Z_{1s}^{\mbox{\scriptsize UGBS}}$ se muestra en la figura con una línea 
discontinua. A pesar que la carga aún diverge cerca de 
$r\approx1\,$a.u., las oscilaciones ahora están circunscriptas cerca del 
núcleo. Finalmente, se resuelven las ecuaciones diferenciales de 
Hartree--Fock para el átomo de carbono usando el método de diferencias 
finitas (\acs{fd}). La carga invertida $Z_{1s}^{\mbox{\scriptsize FD}}$ 
que se obtiene usando las soluciones de este método se muestra con una 
línea sólida en la Fig.~\ref{fig:1sCarbon}(a). Como es de esperar, esta 
carga invertida no presenta oscilaciones, ya que no se han usado 
conjuntos de base para construir los orbitales. Sin embargo, como hemos 
visto en la Sección anterior, la carga diverge para $r>1\,$a.u., debido 
a las características del método de Hartree--Fock. 

\begin{figure}[t]
\centering
\includegraphics[width=0.9\textwidth]{dim/carbon_prof.eps}
\caption[Inversión de funciones de onda descritas con conjuntos de base.]
{(a) Cargas efectivas invertidas del orbital $1s$ del átomo de carbón.
(b)~Perfiles de oscilación de los conjuntos de base.}
\label{fig:1sCarbon}
\end{figure}

Los patrones de oscilación varían según el conjunto de base utilizado
para representar las soluciones. Definimos los perfiles de oscilación
de cada BS como
\begin{equation}
 p_{nl}^{\mbox{\scriptsize BS}} = Z_{nl}^{\mbox{\scriptsize BS}}-
 Z_{nl}^{\mbox{\scriptsize FD}} \,,
 \label{eq:oscillation-prof}
\end{equation}
donde $Z_{nl}^{\mbox{\scriptsize BS}}$ es la carga invertida del átomo
cuando se usa el conjunto de bases ``BS'' y 
$Z_{nl}^{\mbox{\scriptsize FD}}$ es la carga efectiva que se obtiene a 
partir de la inversión de las soluciones del método de diferencias 
finitas. En el caso del carbono, los conjuntos de base consideradas para 
calcular el orbital $1s$ fueron \mbox{6-311G} y UGBS. Los perfiles de 
oscilación correspondientes a dicho orbital, usando la 
Ec.~(\ref{eq:oscillation-prof}), se muestran en la 
Fig.~\ref{fig:1sCarbon}(b). Dado que los perfiles de oscilación para 
cada conjunto de base atómico son característicos, una vez determinados, 
éstos pueden ser implementados para remover las oscilaciones de los 
cálculos moleculares. 

Una vez que las oscilaciones son removidas de los cálculos moleculares, 
se procede a implentar el procedimiento de depuración descrito en la
Sección~\ref{subsec:depuracion}. Debido a la estructura de las moléculas 
consideras, definimos una nueva ecuación parámetrica para las cargas 
moleculares DIM,
\begin{eqnarray}
 Z(r) = \sum_j z_j e^{-\alpha_j r} 
 + z_{\mbox{\scriptsize H}} e^{-(\ln r - \ln \beta)^2/(2\gamma)} 
 + 1\,.
 \label{eq:molzDIM}
\end{eqnarray}
En constraste con la aproximación propuesta para los 
átomos~(\ref{eq:atomzDIM}), incluimos un segundo término en la fórmula 
analítica de la carga por cuenta de $V_H(r)$, dado 
por~(\ref{eq:Vhidrogenos}). Esta expresión nos permite ajustar 
convenientemente tanto la ubicación como el ancho de los potenciales 
hidrogénicos apantallados sin afectar el valor correcto de la carga en 
el origen.

%%%%%%%%%%%%%%%%%%%%%%%%%%%%%%%%%%%%%%%%%%%%%%%%%%%%%%%%%%%%%%%%%%%%%%%%
\section{Resultados}
%%%%%%%%%%%%%%%%%%%%%%%%%%%%%%%%%%%%%%%%%%%%%%%%%%%%%%%%%%%%%%%%%%%%%%%%
\label{sec:dimresultados}

En esta Sección examinamos los resultados obtenidos a partir de la 
implementación de nuestro método de potencial efectivo para la 
descripción de blancos atómicos y moleculares. La aplicabilidad del 
método de inversión depurada en blancos de estructura compleja es 
evaluada a través del estudio de la ionización por impacto de protones y 
fotones en átomos y moléculas. Analizaremos el desempeño de los  
potenciales DIM ante la primera aproximación de Born para predecir 
secciones eficaces totales de ionización de helio, nitrógeno, neón y 
metano.

%%%%%%%%%%%%%%%%%%%%%%%%%%%%%%%%%%%%%%%%%%%%%%%%%%%%%%%%%%%%%%%%%%%%%%%%
\subsection{Estructura electrónica de blancos}
%%%%%%%%%%%%%%%%%%%%%%%%%%%%%%%%%%%%%%%%%%%%%%%%%%%%%%%%%%%%%%%%%%%%%%%%
\label{subsec:dimtarget}

%%%%%%%%%%%%%%%%%%%%%%%%%%%%%%%%%%%%%%%%%%%%%%%%%%%%%%%%%%%%%%%%%%%%%%%%
\subsubsection*{Helio}
%%%%%%%%%%%%%%%%%%%%%%%%%%%%%%%%%%%%%%%%%%%%%%%%%%%%%%%%%%%%%%%%%%%%%%%%

En primer lugar, mostraremos los resultados obtenidos de implementar
el método de inversión depurada en el átomo de helio en su estado
fundamental. Dado que el orbital $1s$ no tiene nodos y que éste 
decae exponencialmente con la energía del orbital HOMO, la inversión 
directa de este orbital no presenta ninguna de las complicaciones 
numéricas presentadas en la Sección~\ref{subsec:probinv}. Además, la 
simpleza del átomo de helio nos permite ajustar la carga invertida con 
un número reducido de parámetros. Así, es posible optimizar la carga 
efectiva $Z_{1s}^{\mathrm{ DIM}}$, dada por la Ec.~(\ref{eq:atomzDIM}), 
con sólo tres parámetros, los cuales se dan en la 
Tabla~\ref{tab:params-atoms}.

Para verificar la calidad de la estructura atómica dada por el potencial 
DIM obtenido, en la Tabla~\ref{tab:results-atoms} presentamos una 
comparación entre los valores de energías y radios medios obtenidos 
utilizando el método de inversión depurada con sus correspondientes 
valores obtenidos mediante el método de Hartree--Fock. El potencial 
$V_{1s}^{\mathrm{ DIM}}$ es capaz de producir un orbital 
$u_{1s}^{\mathrm{ DIM}}$ cuya energía coincide con la energía original 
de HF en 6 cifras significativas. Los valores medios también muestran 
una excelente concordancia, tanto para la región cercana al origen, 
$\langle r \rangle$, como en los que muestrean la calidad de la región 
asintótica, $\langle 1/r\rangle$.

%%%%%%%%%%%%%% RESULTADOS %%%%%%%%%%%%%%
\begin{table}[t]
\begin{center}
\begin{tabularx}{\textwidth}{
>{\centering\arraybackslash}p{0.2\textwidth}
>{\centering\arraybackslash}p{0.2\textwidth}
>{\centering\arraybackslash}p{0.2\textwidth}
>{\centering\arraybackslash}p{0.2\textwidth}}
%\begin{tabular}{*{5}{c}}
\rowcolor{mydarkgray} 
         & $nl$ & $z$        & $\alpha$   \\
%%%%%%%%%%%%%%%%%%%% Helio %%%%%%%%%%%%%%%%%%%%
He $^1$S & $1s$ &  $1.31745$ & $2.50032$  \\ 
\rowcolor{mygray} 
         &      & $-0.31745$ & $5.04372$  \\ 
%%%%%%%%%%%%%%%%%% Nitrogeno %%%%%%%%%%%%%%%%%%
N $^4$S  & $1s$ & $5.25634$  & $1.26207$  \\ 
\rowcolor{mygray} 
         &      & $0.74366$ & $8.02844$  \\ 
%%%%%%%%%%%%%%%%%%%%%%%%%%%%%%%%%%%%%%%%%%%%%%%
         & $2s$ & $2.45281$  & $3.51271$  \\ 
\rowcolor{mygray} 
         &      & $0.83357$ & $3.38654$  \\ 
         &      & $2.71362$  & $0.89470$ \\ 
\rowcolor{mygray} 
%%%%%%%%%%%%%%%%%%%%%%%%%%%%%%%%%%%%%%%%%%%%%%%
         & $2p$ & $3.64345$  & $1.24069$  \\ 
         &      & $2.05501$  & $5.35135$  \\ 
\rowcolor{mygray} 
         &      & $0.30154$ & $0.28661$ \\
%%%%%%%%%%%%%%%%%%%%%%%%%%%%%%%%%%%%%%%%%%%%%%%
Ne $^1$S & $1s$ & $7.36769$ & $2.41728$ \\
\rowcolor{mygray} 
         &      & $1.30036$ & $0.12640$ \\
         &      & $0.33195$ & $13.15820$ \\
%%%%%%%%%%%%%%%%%%%%%%%%%%%%%%%%%%%%%%%%%%%%%%%
\rowcolor{mygray} 
         & $2s$ & $0.29774$ & $17.99390$ \\
         &      & $0.66808$ & $0.06729$ \\
\rowcolor{mygray} 
         &      & $8.03418$  & $2.47221$ \\
%%%%%%%%%%%%%%%%%%%%%%%%%%%%%%%%%%%%%%%%%%%%%%%
         & $2p$ & $1.35305$ & $8.56948$ \\
\rowcolor{mygray} 
         &      & $0.33588$ & $0.46494$ \\
         &      & $7.31107$ & $2.09063$ \\
\end{tabularx}
\caption[Parámetros de la carga efectiva de He, N y Ne.]
{Parámetros de la carga efectiva $Z_{1s}^{\mathrm{ DIM}}$ de He~$^1$S, 
N~$^4$S y Ne~$^1$S.}
\label{tab:params-atoms}
\end{center}
\end{table}
\begin{table}[t]
\begin{center}
\begin{tabularx}{\textwidth}{
>{\centering\arraybackslash}p{0.16\textwidth}
>{\centering\arraybackslash}p{0.10\textwidth}
>{\centering\arraybackslash}p{0.16\textwidth}
>{\centering\arraybackslash}p{0.16\textwidth}
>{\centering\arraybackslash}p{0.16\textwidth}
>{\centering\arraybackslash}p{0.06\textwidth}}
\rowcolor{mydarkgray} 
         & $nl$ & Energía & $\left<r\right>$ & $\left<1/r\right>$ & \\
He $^1$S & $1s$ & $-0.91796$  & $0.92727$ & $1.68728$ & DIM\\
\rowcolor{mygray} 
         &      & $-0.91796$  & $0.92731$ & $1.68725$ & HF \\
N $^4$S  & $1s$ & $-15.62906$  & $0.22830$ & $6.64863$ &  \\
\rowcolor{mygray} 
         &      & $-15.62906$  & $0.22830$ & $6.65324$ &   \\
         & $2s$ & $-0.94532$  & $1.33447$ & $1.08037$ & \\
\rowcolor{mygray} 
         &      & $-0.94532$  & $1.33228$ & $1.07818$  &  \\
         & $2p$ & $-0.56759$  & $1.41268$ & $0.95498$  & \\
\rowcolor{mygray} 
         &      & $-0.56759$  & $1.40963$ & $0.95769$   & \\
Ne $^1$S & $1s$ & $-32.77245$ & $0.15749$ & $9.62145$  & \\
\rowcolor{mygray} 
         &      & $-32.77244$ & $0.15763$ & $9.61805$ &  \\
         & $2s$ & $-1.93039$  & $0.89134$ & $1.64077$ &  \\
\rowcolor{mygray} 
         &      & $-1.93039$  & $0.89211$ & $1.63255$ &  \\  
         & $2p$ & $-0.85041$  & $0.96776$ & $1.43025$  & \\
\rowcolor{mygray} 
         &      & $-0.85041$  & $0.96527$ & $1.43535$  & \\
\end{tabularx}
\caption[Energías y radios medios obtenidos con DIM y HF en He.]
{Energías y radios medios obtenidos con el método de inversión depurada 
(filas superiores) y con el método de HF (filas inferiores) de 
He~$^1$S, N~$^4$S y Ne~$^1$S.}
\label{tab:results-atoms}
\end{center}
\end{table}
\begin{figure}[t]
\centering
\includegraphics[width=0.9\textwidth]{figures/dim/N_dim.eps}
\caption[Cargas efectivas DIM de nitrógeno.]
{Cargas efectivas $1s$, $2s$ y $2p$ del término $2\,^4$S de N obtenidas 
mediante inversión directa (lineas discontinuas) e inversión depurada 
(líneas sólidas).}
\label{fig:Nzeff}
\end{figure}

%%%%%%%%%%%%%%%%%%%%%%%%%%%%%%%%%%%%%%%%%%%%%%%%%%%%%%%%%%%%%%%%%%%%%%%%
\subsubsection*{Nitrógeno}
%%%%%%%%%%%%%%%%%%%%%%%%%%%%%%%%%%%%%%%%%%%%%%%%%%%%%%%%%%%%%%%%%%%%%%%%

Los excelentes resultados obtenidos a partir de la implementación de DIM 
en el átomo de helio pueden atribuirse a la simplicidad del orbital y su 
simetría esférica. Para evaluar la destreza del método de inversión 
depurada para describir blancos con más electrones y de capa abierta, 
consideramos ahora el átomo de nitrógeno. 

La configuración más baja del nitrógeno $2p^3$ da lugar a tres términos 
diferentes: $^4$S, $^2$D, $^2$P. Cada uno de estos términos está 
descrito por una densidad electrónica diferente. Consideramos aquí 
las soluciones de HF del término de menor energía. En la  
Fig.~\ref{fig:Nzeff} se muestran las cargas de los orbitales $1s$, $2s$ 
y $2p$ obtenidas a partir del método de inversión (líneas discontinuas) 
y el método de inversión depurada (líneas continuas). Como se estudiado 
en la Sección~\ref{subsec:probinv}, la inversión del orbital 
$u_{1s}^{\mathrm{HF}}$ diverge en la región asintótica debido al 
comportamiento del potencial dado por la Ec.~\ref{eq:asintoticoVHF}. 
Además, el nodo genuino del orbital $2s$, en $r\approx 0.32$~a.u., es 
traducido por el método de inversión (IM) como un polo a la carga. 

La implementación del método de inversión depurada permite ajustar 
analíticamente la carga invertida en el mayor rango posible. Luego de 
una optimización cuidadosa del blanco se obtienen los parámetros de las 
cargas DIM correspondientes el término $^4$S que se dan en la 
Tabla~\ref{tab:params-atoms}. Resolviendo la ecuación de Schr\"odinger 
para un electrón (\ref{eq:eqSchroRadial}) con las cargas efectivas DIM 
es posible reproducir las energías orbitales hasta la quinta cifra 
significativa de los valores de Hartree--Fock y los valores medios 
$\langle r \rangle$ y $\langle 1/r \rangle$ hasta en un $0.2\%$, como se 
muestra en la Tabla~\ref{tab:results-atoms}.  


%%%%%%%%%%%%%%%%%%%%%%%%%%%%%%%%%%%%%%%%%%%%%%%%%%%%%%%%%%%%%%%%%%%%%%%%
\subsubsection*{Neón}
%%%%%%%%%%%%%%%%%%%%%%%%%%%%%%%%%%%%%%%%%%%%%%%%%%%%%%%%%%%%%%%%%%%%%%%%

La implementación del DIM en neón es análoga al caso de nitrógeno. Los 
resultados de la optimización de los parámetros que definen las cargas 
efectivas, dadas por la Ec.~(\ref{eq:atomzDIM}), del átomo de neón se 
muestran en la Tabla~\ref{tab:params-atoms}. La comparación entre las 
soluciones de los potenciales DIM y el método de Hartree--Fock se dan en 
la Tabla~\ref{tab:results-atoms}. El acuerdo en energías orbitales es 
excelente, del orden de $1\times 10^{-5}$, mientras que DIM reproduce 
los valores medios de los orbitales HF en aproximadamente $0.1\%$. 

%%%%%%%%%%%%%%%%%%%%%%%%%%%%%%%%%%%%%%%%%%%%%%%%%%%%%%%%%%%%%%%%%%%%%%%%
\subsubsection*{Metano}
%%%%%%%%%%%%%%%%%%%%%%%%%%%%%%%%%%%%%%%%%%%%%%%%%%%%%%%%%%%%%%%%%%%%%%%%

La expansión del método de inversión depurada a sistemas moleculares es 
aplicada a la molécula de metano. Este hidruro es altamente simétrico y, 
por lo tanto, puede ser descrito por un potencial angular  
promediado~\cite{Granados:16}. 

\begin{figure}[t]
\centering
\includegraphics[width=0.9\textwidth]{figures/dim/ch4_dim.eps}
\caption[Cargas efectivas DIM de metano.]
{Cargas efectivas de CH$_4$; inversión directa (lineas discontinuas)
e inversión depurada (líneas sólidas).}
\label{fig:ch4zeff}
\end{figure}

Los orbitales moleculares de HF de CH$_4$ se calcularon usando los 
conjuntos de bases UGBS del carbono y el hidrógeno, los cuales 
consideran momentos angulares hasta $L=1$. El cálculo de estructura 
electrónico de metano con estos conjuntos de base deberían incluir 
funciones de polarización (por lo menos hasta las funciones $d$), con el 
fin de incrementar la precisión de las energías  
moleculares~\cite{Rothenberg:71,Hariharan:72}. Sin embargo, para aislar 
los efectos de los conjuntos de base, realizamos los cálculos de 
perfiles de oscilación descritos en la Sección~\ref{sec:dimmoleculas} y 
los orbitales moleculares en el mismo esquema. Las cargas obtenidas 
mediante la inversión directa se muestran en la Fig.~\ref{fig:ch4zeff} 
con líneas discontinuas. Dado que los orbitales moleculares están 
escritos por combinaciones lineales de orbitales atómicos del carbono y 
el hidrógeno, las oscilaciones de las cargas invertidas se deben al 
número finito de funciones primitivas en el conjunto de base de cada 
átomo. Para remover estas oscilaciones, se deben determinar los perfiles 
de oscilación producidas por el conjunto de base de los átomos 
constituyentes. Usamos la Ec.~(\ref{eq:oscillation-prof}) para 
determinar los perfiles $p_{1s}^{\mbox{\scriptsize UGBS}}$, 
$p_{2s}^{\mbox{\scriptsize UGBS}}$ y $p_{2p}^{\mbox{\scriptsize UGBS}}$ 
del carbono. Así, se puden remover los perfiles 
$p_{nl}^{\mbox{\scriptsize UGBS}}$ de las correspondientes cargas 
invertidas $Z_{i}^{\mbox{\scriptsize UGBS}}$ del metano. Las 
oscilaciones se remueven completamente para todos los orbitales excepto 
el $2a_2$, que presenta pequeñas fluctuaciones residuales debido a la 
base del hidrógeno. Ya que estas ondulaciones son mínimas y se ubican 
cerca del núcleo, podemos despreciarlas y procedemos a implementar le 
método de depuración descrita en la Sección~\ref{sec:dimmoleculas}. 

Los parámetros optimizados de las cargas moleculares DIM, dadas por la 
Ec.~(\ref{eq:molzDIM}), se dan en la Tabla~\ref{tab:ch4parameters}. Las 
cargas correspondientes se muestran en la Fig.~\ref{fig:ch4zeff} con 
líneas sólidas.  En este caso, para la construcción de función de 
costo~(\ref{eq:fncosto-dim}) minimizada se consideraron los valores de 
energía y los valores medios de los MOs dados por 
Moccia~\cite{Moccia:69}. Los valores de energía obtenidos con ellas 
reproducen las originales hasta la cuarta cifra significativa y se dan 
en la Tabla~\ref{tab:ch4parameters}. Por otro lado, los valores medios 
$\langle r\rangle$ y $\langle 1/r\rangle$ obtenidos con los potenciales 
moleculares DIM son en promedio del $0.3\%$ y menores al $1\%$.

\begin{table}[t]
\centering
\begin{tabular}{
>{\centering\arraybackslash}p{0.13\textwidth}
>{\centering\arraybackslash}p{0.13\textwidth}
>{\centering\arraybackslash}p{0.13\textwidth}
>{\centering\arraybackslash}p{0.13\textwidth}
>{\centering\arraybackslash}p{0.13\textwidth}
>{\centering\arraybackslash}p{0.13\textwidth}}
\rowcolor{mydarkgray} 
   $nl$ & $E$        & $z$        & $\alpha$   & $\beta$ & $\gamma$ \\
$1a_1$  & $-11.1949$ & $1.925280$ & $0.641982$ & & \\
\rowcolor{mygray} 
        &            & $0.953120$ & $5.571510$ & & \\
        &            & $2.121600$ & $1.500440$ & & \\
\rowcolor{mygray} 
$2a_2$  & $-0.9204$  & $2.912200$ & $3.149990$ & & \\
        &            & $2.087800$ & $0.771371$ & & \\
\rowcolor{mygray} 
        &            & $1.23640$  &            & $2.329570$ & $0.053420$ \\
$2t_1$  & $-0.5042$  & $0.901953$ & $2.895140$ & & \\
\rowcolor{mygray} 
        &            & $1.112030$ & $0.388649$ & & \\
        &            & $2.986017$ & $2.931210$ & & \\
\rowcolor{mygray} 
        &            & $1.301820$ &            & $2.169850$ & $0.012616$ \\ 
\end{tabular}
\caption[Energías y parámetros de ajuste de cargas efectivas de metano.]
{Energías orbitales moleculares y parámetros de ajuste de cargas efectivas dadas por la Ec.~(\ref{eq:molzDIM}) de metano.}
\label{tab:ch4parameters}
\end{table}

%%%%%%%%%%%%%%%%%%%%%%%%%%%%%%%%%%%%%%%%%%%%%%%%%%%%%%%%%%%%%%%%%%%%%%%%
\subsection{Procesos colisionales simples}
%%%%%%%%%%%%%%%%%%%%%%%%%%%%%%%%%%%%%%%%%%%%%%%%%%%%%%%%%%%%%%%%%%%%%%%%

En esta Sección analizaremos la respuesta del potencial efectivo DIM, 
obtenido mediante el método de inversión depurada, para describir la 
estructura de blancos atómicos y molecular en el proceso de 
fotoionización de tres blancos atómicos --helio, nitrógeno y neón-- y un 
blanco molecular simple: metano. La estructura de los blancos, con 
estado de carga neutro y en sus estados fundamentales de menor energía, 
está dada por los resultados de la Sección~\ref{subsec:dimtarget}. 
En los blancos moleculares, la orientación molecular es importante para 
determinar la sección eficaz en un proceso colisional. Sin embargo, en 
sistemas gasesosos, las moléculas tienen orientaciones aleatorias que no 
están predefinidas en el experimento. Por lo tanto, la descripción 
promediada esféricamente de los sistemas moleculares asumida por el 
potencial DIM está en concordancia con la configuración del blanco.
Los procesos colisionales examinados en esta Sección se describen a 
primer orden empleando la primera aproximación de Born (FBA). 
Denominamos la combinación de la descripción del blanco mediante el 
potencial efectivo DIM y el modelado de la fotoionización a primer orden 
como fotoionización DIM-FBA.  

%%%%%%%%%%%%%%%%%%%%%%%%%%%%%%%%%%%%%%%%%%%%%%%%%%%%%%%%%%%%%%%%%%%%%%%%
\subsubsection{Fotoionización}
%%%%%%%%%%%%%%%%%%%%%%%%%%%%%%%%%%%%%%%%%%%%%%%%%%%%%%%%%%%%%%%%%%%%%%%%
\label{subsec:foto}

Las secciones eficaces de ionización simple por impacto de fotón según 
el modelo DIM-FBA para helio, nitrógeno, neón y metano se muestran en 
la Fig.~\ref{fig:photoDIM} con líneas 
sólidas. Los resultados teóricos DIM-FBA para helio y nitrógeno 
coinciden de manera excelente con los valores experimentales 
(símbolos)~\cite{Samson:90,Henke:93,Stolte:16} a bajas, medias y altas 
energías del fotón incidente. En el caso del átomo de neón, algunas 
discrepancias con las mediciones (símbolos) \cite{Henke:93,Samson:02} 
empiezan a surgir a energías bajas e intermedias del projectil. Este 
comportamiento sugiere la necesidad de incluir en la descripción de la 
fotoionización correcciones de mayor orden que incluyan efectos de 
múltiples cuerpos que puedan ser relevantes tales como la relajación de 
los orbitales debido a la creación de un hueco electrónico, respuestas 
colectivas de electrones de capas internas~\cite{Ederer:64} y efectos de 
correlación.

\begin{figure}
\centering
\includegraphics[width=0.92\textwidth]{dim/fotoDIM-part1.eps} 

\vspace{-1.15cm}
\includegraphics[width=0.92\textwidth]{dim/fotoDIM-part2.eps}
\caption[Fotoionización de He, N, Ne y CH$_4$.]
{Sección eficaz total de fotoionización de un electrón de He~$^1$S, 
N~$^4$S, Ne~$^1$S y CH$_4$. Curvas: cálculos teóricos DIM-FBA. Símbolos: 
datos experimentales~\cite{Samson:90,Henke:93,Stolte:16,Samson:02,
Lukirskii:64,Henke:82,Samson:89}.}
\label{fig:photoDIM}
\end{figure}

La predicción del modelo DIM-FBA para la sección eficaz total de 
fotonización de CH$_4$ se encuentra en buen acuerdo con valores 
experimentales en el rango de altas energías y cerca del umbral. La 
curva entre aproximadamente 15 y 300 eV muestra la fotoionización de la 
capa eterna $n=2$, mientras que la discontinuidad en $0.3$~keV 
corresponde al umbral del orbital molecular $1a_1$. Para fotoenergías 
bajas e intermedias, el acuerdo entre nuestras predicciones y los datos 
experimentales~\cite{Lukirskii:64,Henke:82,Samson:89} no es bueno. 
Fenónemos tales como la relajación de los orbitales moleculares, 
posibles contribuciones colectivas y efectos de correlación deben ser  
considerados en futuros cálculos. Por otro lado, para la fotoionización 
de un electrón perteneciente al orbital interno $1a_1$, estos efectos no 
son tan signficativos, y obtenemos buen acuerdo con los valores 
experimentales disponibles. 

%=======================================================================
\subsubsection{Ionización por impacto de iones}
\label{subsec:dimion}

\begin{figure}
\centering
\includegraphics[width=0.9\textwidth]{figures/dim/ionDIM.eps}
\caption[Ionización por impacto de protón de N y CH$_4$.]
{Sección eficaz total de ionización de un electrón por impacto de protón 
de N~$^4$S y CH$_4$. Línea sólida: cálculos teóricos DIM-FBA. 
Símbolos: datos experimentales de ionización por impacto de 
protón~\cite{Rudd:83,Rudd:85} y electrón~\cite{Brook:78} con conversión 
de equivelocidad.}
\label{fig:iondim}
\end{figure}

Los resultados de ionización por impacto de protón en N~$^4$S y CH$_4$, 
en el marco del modelo DIM-FBA, se muestra en la Fig.~\ref{fig:iondim}. 
El acuerdo entre nuestras predicciones teóricas y las datos 
experimentales es muy bueno en la región de altas energías, donde tiene 
validez la primera aproximación de Born. En el caso de nitrógeno, 
incluimos también datos experimentales de ionización por impacto de 
electrón. Para valores de energía mayores a 400~keV, se espera que la 
sección eficaz de ambos proyectiles coincida. 

%=======================================================================
\section{Conclusiones}
\label{conclusion}

En este Capítulo desarrollamos el método de inversión depurada (DIM) 
para obtener potenciales efectivos que permitan describir la estructura 
electrónica de blancos atómicos y moleculares de manera precisa. La 
disponibilidad de estos potenciales permite conocer los estados 
iniciales y finales del blanco en una colisión de manera directa. 

Los potenciales DIM se obtienen a partir de la inversión de ecuaciones 
de un electrón con soluciones de Hartree--Fock. Los potenciales 
resultantes presentan defectos numéricos (polos y divergencias), los 
cuales son examinados en detalle. Encontramos que el origen de estos 
defectos se ubica en la teoría de Hartree--Fock. Los polos se deben a 
que los nodos genuinos de los orbitales HF no son puntos de inflexión. 
Si bien esta característica de los orbitales no es explícita en la 
teoría, la obtención de potenciales sin polos así lo requiere. Las 
divergencias asintóticas del potencial se deben al coeficiente del 
decaimiento exponencial de los orbitales HF. La teoría de HF establece 
que éstos decaen con la energía del HOMO. Sin embargo, la obtención de 
potenciales con correcto comportamiento asintótico mediante el esquema 
de inversión requiere que cada orbital decaiga con la energía del dicho 
orbital. También se encontraron oscilaciones en los orbitales de las 
capas internas de átomos con carga nuclear $Z\ge 12$, que dan lugar a  
nodos espúreos. Es posible que estas oscilaciones se deban al término de 
intercambio a grandes distancias. Para tratar los defectos encontrados 
en la inversión, se desarrolló el método de depuración, que consiste en 
ajustar las cargas invertidas, en regiones sin polos ni divergencias, 
mediante una expresión analítica. Los parámetros que definen esta 
expresión son optimizados cuidadosamente hasta reproducir las soluciones 
iniciales.

El método de inversión depurada para átomos fue extendido para 
moléculas. Debido a que los orbitales moleculares se expresan a partir 
de conjuntos de base finitas, las soluciones presentan ondulaciones casi 
imperceptiles. La implementación de la inversión traduce estas pequeñas 
fluctuaciones como grandes oscilaciones en la carga molecular. Debido a 
esto, se requieren pasos adicionales en el método de depuración, los 
cuales incluyen la determinación de perfiles de oscilación de los 
conjuntos de base atómicas utilizadas en el cálculo molecular. 

Implementamos el método DIM para obtener potenciales efectivos que 
reproducen las soluciones de HF de forma precisa en tres blancos 
atómicos --helio, nitrógeno y neón-- y un sistema molecular de simetría 
esférica: metano. La efectividad del DIM para describir la estructura de 
blancos en una colisión fue examinado a partir de la primera 
aproximación de Born. Los potenciales de He, N, Ne y CH$_4$ se 
implementaron para calcular, en conjunción con la FBA, secciones 
eficaces de ionización por el impacto de protones y fotones. Ambos 
procesos son reproducidos en términos generales con buena concordancia 
con los datos experimentales disponibles. Las discrepancias principales 
se atribuyen al hecho de que nuestro cálculo sólo considera el primer 
orden perturbativo. Será necesario implementar métodos perturbativos con 
mayor orden de aproximación para examinar la validez del método DIM en 
la región de energías intermedias.





