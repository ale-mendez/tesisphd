\chapter{Conclusiones}
\label{chap:conclusiones}

%%%%%%%%%%%%%%%%%%%%%%%%%%%%%%%%%%%%%%%%%%%%%%%%%%%%%%%%%%%%%%%%%%%%%%%%

En el Capítulo~\ref{chap:iondim} se desarrolló el método de inversión 
depurada (DIM) para obtener potenciales efectivos que permitan describir 
la estructura electrónica de blancos atómicos y moleculares simples de 
manera precisa. Estos potenciales permiten determinar los estados 
iniciales y finales del blanco en una colisión. El DIM es general y 
aplicable a soluciones que resultan de diversas aproximaciones. En este 
trabajo se mostró la implementación del DIM a partir de soluciones dadas 
por la teoría de Hartree--Fock. Los potenciales resultantes de la 
inversión de orbitales HF presentan defectos numéricos (polos y 
divergencias). Siguiendo el método de depuración, las cargas invertidas 
atómicas y moleculares se ajustaron con expresiones analíticas simples 
restringiendo las regiones donde surgen defectos numéricos debido a la 
inversión. Los parámetros que definen esta expresión son optimizados 
cuidadosamente hasta reproducir las soluciones iniciales. 
El DIM se implementó para obtener potenciales efectivos y de intercambio 
que reproducen las soluciones de HF de forma precisa en diversos 
sistemas multielectrónicos. Los resultados se ejemplificaron con cuatro
blancos: helio, nitrógeno, neón y metano. Las soluciones que se obtienen 
de estos potenciales reproducen los valores de referencia con gran 
precisión. La uso efectivo del DIM para describir la estructura de 
sistemas atómicos y moleculares simples en una colisión fue examinado a 
partir de la primera aproximación de Born. Los 
potenciales de He, N, Ne y CH$_4$ se implementaron para calcular, en 
conjunción con la FBA, secciones eficaces de ionización por el impacto 
de protones y fotones. En términos generales, ambos procesos reproducen 
con buena concordancia los datos experimentales disponibles. Las 
principales discrepancias se atribuyen al hecho de que el modelo 
colisional teórico sólo considera el primer orden perturbativo. 


%%%%%%%%%%%%%%%%%%%%%%%%%%%%%%%%%%%%%%%%%%%%%%%%%%%%%%%%%%%%%%%%%%%%%%%%

En este Capítulo~\ref{chap:ionmol}, se estudió la ionización de blancos 
moleculares de interés biológico debido al impacto de iones de carga 
múltiple. La aproximación teórica propuesta combina el modelo 
estequiométrico simple (SSM), para describir la molécula, y el método de 
onda distorsionada con estado inicial de Eikonal para blancos atómicos 
(descriptos con potenciales DIM), para modelar el proceso colisional. Se 
calcularon secciones eficaces de ionización para 80 sistemas colisiones
compuestos por blancos moleculares de interés biológico --incluyendo 
las nucleobases del ADN y ARN-- y 6 proyectiles: antiprotones, H$^{+}$, 
He$^{2+}$, Be$^{4+}$, C$^{6+}$, y O$^{8+}$. 
Se realizaron cálculos de estructura molecular para las nucleobases y se 
propuso un modelo estequiométrico modificado utilizando la carga de 
Mulliken. El extenso análisis de los resultados 
presentados refuerzan la validez del modelo SSM--CDW para tratar la 
ionización de moléculas complejas por el impacto de iones de carga 
múltiple en el rango de energía intermedia a alta. 
A partir de los resultados SSM--CDW, se propusieron tres reglas de 
escala para las secciones eficaces de ionización. 
La primer ley de escala consideró la naturaleza del blanco y la sección 
eficaz de ionización se reduce con el número de electrones débilmente 
ligados. En la segunda regla, se examinaron las secciones eficaces según 
la naturaleza del proyectil (carga del ion incidente). La última ley de 
escala se obtuvo de la conjunción de las dos primeras: la sección eficaz 
de ionización total se reduce con el número efectivo de electrones 
activos y la carga del ion. Esta combinación condujo a una 
ley de escala de tipo universal, independiente de la carga del ion y el 
blanco molecular. Las tres reglas de escala propuestas se implementaron 
para los 80 sistemas colisionales examinados y se compararon con datos 
experimentales disponibles, con buenos resultados. 
Además, se encontró que la regla de escala universal propuesta es válida 
incluso para energías fuera del rango de validez del método SSM--CDW.

%%%%%%%%%%%%%%%%%%%%%%%%%%%%%%%%%%%%%%%%%%%%%%%%%%%%%%%%%%%%%%%%%%%%%%%%

En el Capítulo~\ref{chap:heavy} se estudió la estructura de nueve 
blancos pesados con cargas nucleares entre 40 y 78. Se resolvió la 
ecuación de Dirac mediante el método de potencial paramétrico 
implementado en el paquete de códigos {\sc hullac}. El acuerdo entre los 
resultados teóricos presentes y valores experimentales es muy bueno. Las 
discrepancias encontradas se entienden considerando que los cálculos 
teóricos suponen a los blancos como átomos aislados, mientras que los 
mediciones se obtuvieron en sistemas sólidos. Las energías de ligadura y 
orbitales radiales relativistas resultantes permitieron definir valores 
teóricos para los radios de Wigner--Seitz y energías de Fermi de los 
blancos. Estos parámetros se compararon con valores derivados de 
experimentos disponibles, con un buen acuerdo entre ellos. En los 
lantánidos y algunos metales de transición pesados, el análisis de la 
estructura del blanco condujo a proponer nuevos valores para los 
parámetros del FEG. A partir de métodos de primeros principios, los 
cálculos de estructura se utilizaron para predecir secciones eficaces de 
frenado. El modelo teórico implementado combina la respuesta por 
separado de los electrones ligados y de valencia en un extenso rango de 
energías. La aproximación de plasma local por capa se utilizó 
para describir la energía transferida a los electrones ligados $1s$-$4f$ 
y dos modelos diferentes para modelar la respuesta de las capas de 
valencia: el modelo de potencial apantallado con condición de cúspide, 
para energías por debajo del umbral de excitaciones de plasmón, y el 
formalismo dieléctrico de Mermin-Lindhard, para valores de energías por 
encima de este umbral. Se calcularon las secciones eficaces de frenado 
debido al impacto de protones en Hf, Ta y Pt. Los resultados teóricos se 
compararon con datos experimentales disponibles y modelos no 
relativistas (de primeros principios y semi-empíricos). En general, las 
predicciones de fuerza de frenado presentes son superiores. Esto se 
debe, en gran medida, a la correcta descripción de la estructura de los 
blancos dada por los cálculos relativistas.

%%%%%%%%%%%%%%%%%%%%%%%%%%%%%%%%%%%%%%%%%%%%%%%%%%%%%%%%%%%%%%%%%%%%%%%%

En el Capítulo~\ref{chap:bayeopt} se examinó la estructura de blancos 
neutros en procesos de excitación por impacto de electrones. El esquema 
de optimización de la estructura se realizó en tres 
etapas: Primero, se estudió la influencia de las configuraciones 
electrónicas en las secciones eficaces de excitación del blanco y se 
encontró una fuerte dependencia con el número de niveles y tipo de 
orbitales incluidos en dicha representación. Luego, se definió el 
potencial paramétrico que proporcionó los mejores valores de energías 
orbitales y oscillator strenghts. Por último, la optimización consistió 
en la variación de los parámetros $\lambda_{nl}$ que definen el potencial 
con el fin de minimizar una función de costo definida a priori. El 
objetivo principal de esta investigación constituyó en explorar la 
última parte de la optimización de forma automática. Para esto, se 
introdujo el método de inferencia Bayesiana con procesos Gaussianos. El
modelo de optimización automática se implementó en el átomo de Be ya que 
constituye un caso particularmente complejo y, debido a su gran 
importancia en Astrofísica y Fusión, ha sido largamente estudiado y 
documentado. Los cálculos GP de energías de los primeros 11 términos del 
Be y los oscillator strengths de las transiciones dipolares entre estos 
términos se compararon con valores de referencia con muy excelentes 
resultados. La implementación del RMPS a partir del blanco optimizado 
con GP permitió obtener una buena representación de las excitaciones por 
impacto de electrones. Se encontraron significativas correcciones 
respecto a las secciones eficaces del blanco sin optimizar. Los valores 
teóricos GP--RMPS obtenidos presentan pseudoresonancias, las cuales son 
características de sistemas atómicos donde la representación de 
pseudo-estados no se ha convergido completamente. Sin embargo, la 
estructura atómica de 27 configuraciones definida permitió validar el 
método de optimización automática y probó ser suficiente para los 
objetivos del presente trabajo. 

%%%%%%%%%%%%%%%%%%%%%%%%%%%%%%%%%%%%%%%%%%%%%%%%%%%%%%%%%%%%%%%%%%%%%%%%

En este trabajo de investigación se exploró la optimización de blancos
atómicos y moleculares para el cálculo de procesos colisionales. Se 
mostró que esta tarea no es simple ni directa; según la naturaleza del 
blanco, se implementaron diferentes metodologías, desde métodos de 
inversión hasta el aprendizaje automatizado. Las metodologías 
presentadas permitieron describir una amplia variedad de sistemas 
multielectrónicos: desde átomos livianos, 
pertenecientes a las primeras filas de la tabla periódica, hasta blancos 
pesados, tales como lantánidos y metales de transición de los 
periodos 5 y 6. También se estudiaron moléculas simples, compuestas por 
unos cuantos átomos, y sistemas complejos de interés biológico, tales 
como las nucleobases de ADN y ARN. La optimización de los blancos 
permitieron describir de forma precisa diversos procesos inelásticos 
debido al impacto de partículas cargadas y fotones, los cuales a su vez
se describieron mediante métodos teóricos con diferentes órdenes de 
aproximación --desde aproximaciones perturbativas de primer orden hasta 
métodos cuánticos considerados el estado del arte. 




