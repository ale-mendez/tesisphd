\chapter{Conclusiones}
\label{chap:conclusiones}

%%%%%%%%%%%%%%%%%%%%%%%%%%%%%%%%%%%%%%%%%%%%%%%%%%%%%%%%%%%%%%%%%%%%%%%%

En este trabajo de investigación se exploró la optimización de blancos
atómicos y moleculares en diversos procesos colisionales. A lo largo 
del trabajo se presentaron métodos que permitieron describir una amplia 
variedad de sistemas multielectrónicos: desde átomos livianos, 
pertenecientes a las primeras filas de la tabla periódica, hasta blancos 
pesados, tales como lantánidos y metales de transición de los 
periodos 5 y 6. También se estudiaron moléculas simples, compuestas por 
unos cuantos átomos, y sistemas complejos de interés biológico, tales 
como las nucleobases de ADN y ARN. Estos blancos fueron examinados en 
procesos inelásticos debido al impacto de partículas cargadas y fotones, 
a partir de métodos teóricos con diversos órdenes de aproximación, desde 
aproximaciones perturbativas de primer orden hasta métodos cuánticos 
considerados el estado del arte. 

En el Capítulo~\ref{chap:iondim} se desarrolló el método de inversión 
depurada (DIM) para obtener potenciales efectivos que permitan describir 
la estructura electrónica de blancos atómicos y moleculares simples de 
manera precisa. Estos potenciales permiten determinar los estados 
iniciales y finales del blanco en una colisión. El DIM es general y 
aplicable a soluciones que resultan de diversas aproximaciones. En este 
trabajo se mostró la implementación del DIM a partir de soluciones dadas 
por la teoría de Hartree--Fock. Los potenciales resultantes de la 
inversión de orbitales HF presentan defectos numéricos (polos y 
divergencias). Siguiendo el método de depuración, las cargas invertidas 
atómicas y moleculares se ajustaron con expresiones analíticas simples 
restringiendo las regiones donde surgen defectos numéricos debido a la 
inversión. Los parámetros que definen esta expresión son optimizados 
cuidadosamente hasta reproducir las soluciones iniciales. 
El DIM se implementó para obtener potenciales efectivos y de intercambio 
que reproducen las soluciones de HF de forma precisa en diversos 
sistemas multielectrónicos. Los resultados se ejemplificaron con cuatro
blancos: helio, nitrógeno, neón y metano. Las soluciones que se obtienen 
de estos potenciales reproducen los valores de referencia con gran 
precisión. La uso efectivo del DIM para describir la estructura de 
sistemas atómicos y moleculares simples en una colisión fue examinado a 
partir de la primera aproximación de Born. Los 
potenciales de He, N, Ne y CH$_4$ se implementaron para calcular, en 
conjunción con la FBA, secciones eficaces de ionización por el impacto 
de protones y fotones. En términos generales, ambos procesos reproducen 
con buena concordancia los datos experimentales disponibles. Las 
principales discrepancias se atribuyen al hecho de que el modelo 
colisional teórico sólo considera el primer orden perturbativo. 


%%%%%%%%%%%%%%%%%%%%%%%%%%%%%%%%%%%%%%%%%%%%%%%%%%%%%%%%%%%%%%%%%%%%%%%%

En este Capítulo~\ref{chap:ionmol}, se estudió la ionización de blancos 
moleculares de interés biológico debido al impacto de iones de carga 
múltiple. La aproximación teórica propuesta combina el modelo 
estequiométrico simple (SSM), para describir la molécula, y el método de 
onda distorsionada con estado inicial de Eikonal para blancos atómicos 
(descriptos con potenciales DIM), para modelar el proceso colisional. Se 
calcularon secciones eficaces de ionización para 80 sistemas colisiones
compuestos por blancos moleculares de interés biológico --incluyendo 
las nucleobases del ADN y ARN-- y 6 proyectiles: antiprotones, H$^{+}$, 
He$^{2+}$, Be$^{4+}$, C$^{6+}$, y O$^{8+}$. 
Se realizaron cálculos de estructura molecular para las nucleobases y se 
propuso un modelo estequiométrico modificado utilizando la carga de 
Mulliken. El extenso análisis de los resultados 
presentados refuerzan la validez del modelo SSM--CDW para tratar la 
ionización de moléculas complejas por el impacto de iones de carga 
múltiple en el rango de energía intermedia a alta. 
A partir de los resultados SSM--CDW, se propusieron tres reglas de 
escala para las secciones eficaces de ionización. 
La primer ley de escala consideró la naturaleza del blanco y la sección 
eficaz de ionización se reduce con el número de electrones débilmente 
ligados. En la segunda regla, se examinaron las secciones eficaces según 
la naturaleza del proyectil (carga del ion incidente). La última ley de 
escala se obtuvo de la conjunción de las dos primeras: la sección eficaz 
de ionización total se reduce con el número efectivo de electrones 
activos y la carga del ion. Esta combinación condujo a una 
ley de escala de tipo universal, independiente de la carga del ion y el 
blanco molecular. Las tres reglas de escala propuestas se implementaron 
para los 80 sistemas colisionales examinados y se compararon con datos 
experimentales disponibles, con buenos resultados. 
Además, se encontró que la regla de escala universal propuesta es válida 
incluso para energías fuera del rango de validez del método SSM--CDW.

%%%%%%%%%%%%%%%%%%%%%%%%%%%%%%%%%%%%%%%%%%%%%%%%%%%%%%%%%%%%%%%%%%%%%%%%

En el Capítulo~\ref{chap:heavy} se estudió la estructura de nueve 
blancos pesados con cargas nucleares entre 40 y 78. Se resolvió la 
ecuación de Dirac mediante el método de potencial paramétrico 
implementado en el paquete de códigos {\sc hullac}. El acuerdo entre los 
resultados teóricos presentes y valores experimentales es muy bueno. Las 
discrepancias encontradas se entienden considerando que los cálculos 
teóricos suponen a los blancos como átomos aislados, mientras que los 
mediciones se obtuvieron en sistemas sólidos. Las energías de ligadura y 
orbitales radiales relativistas resultantes permitieron definir valores 
teóricos para los radios de Wigner--Seitz y energías de Fermi de los 
blancos. Estos parámetros se compararon con valores derivados de 
experimentos disponibles, con un buen acuerdo entre ellos. En los 
lantánidos y algunos metales de transición pesados, el análisis de la 
estructura del blanco condujo a proponer nuevos valores para los 
parámetros del FEG. A partir de métodos de primeros principios, los 
cálculos de estructura se utilizaron para predecir secciones eficaces de 
frenado. El modelo teórico implementado combina la respuesta por 
separado de los electrones ligados y de valencia en un extenso rango de 
energías. La aproximación de plasma local por capa se utilizó 
para describir la energía transferida a los electrones ligados $1s$-$4f$ 
y dos modelos diferentes para modelar la respuesta de las capas de 
valencia: el modelo de potencial apantallado con condición de cúspide, 
para energías por debajo del umbral de excitaciones de plasmón, y el 
formalismo dieléctrico de Mermin-Lindhard, para valores de energías por 
encima de este umbral. Se calcularon las secciones eficaces de frenado 
debido al impacto de protones en Hf, Ta y Pt. Los resultados teóricos se 
compararon con datos experimentales disponibles y modelos no 
relativistas (de primeros principios y semi-empíricos). En general, las 
predicciones de fuerza de frenado presentes son superiores. Esto se 
debe, en gran medida, a la correcta descripción de la estructura de los 
blancos dada por los cálculos relativistas.

%%%%%%%%%%%%%%%%%%%%%%%%%%%%%%%%%%%%%%%%%%%%%%%%%%%%%%%%%%%%%%%%%%%%%%%%

En el Capítulo~\ref{chap:bayeopt} se examinó la estructura de blancos 
neutros en procesos de excitación por impacto de electrones. Se 
describió brevemente el método de $R$-Matrix utilizado para el cálculo 
del proceso colisional y los códigos que se emplean para tal fin. 
Los métodos utilizados para la representación del blanco (la expansión 
de interacción de configuraciones y de diversos potenciales modelos) se
presentaron brevemente. 
El esquema de optimización de la estructura se definió a través de tres 
cantidades fundamentales: (1) las configuraciones definidas en la 
expansión del CI, (2) el potencial modelo elegido para obtener los 
orbitales radiales de los electrones ligados y (3) los parámetros que 
definen dichos potenciales. Cada una de estas variables se examinó de 
forma secuencial. En primer lugar, se estudió la influencia de las 
configuraciones electrónicas en las secciones eficaces de excitación del 
blanco y se encontró una fuerte dependencia con el número de niveles y 
tipo de orbitales incluidos en dicha representación. Luego de definir 
las configuraciones necesarias para obtener valores aceptables para 
diversastransiciones entre estados ligados, se eligió el potencial 
paramétrico que proporcionara los mejores valores de energías orbitales 
y oscillator strenghts. La última etapa de la optimización consistió en 
la variación de los parámetros que definen los potenciales con el fin de 
minimizar una función de costo definida a priori. En general, todas 
estas etapas necesarias para la optimización se ejectuan de forma manual 
y conlleva meses de trabajo. 

El objetivo principal de esta investigación constituyó en estudiar la 
optimización de átomos neutros en procesos de excitación por impacto de 
electrones. La última etapa de la optimización se examino en detalle con 
el fin de automatizar la variación de los parámetros $\lambda_{nl}$. 
Esto se realizó implementando modelos matemáticos que son recurrentes en 
el campo del aprendizaje automatizado. El método de inferencia Bayesiana 
con procesos Gaussianos se introdujo en el modelo de optimización de la 
estructura atómica con éxito. Se eligió el átomo de Be para validar la 
metodología ya que constituye un caso particularmente complejo y, debido 
a su gran importancia en Astrofísica y Fusión, ha sido largamente 
estudiado y documentado. En total, se ajustaron 16 parámetros de dos 
potenciales en forma automática y con mínima intervención humana. Se 
reprodujeron valores de energías totales y de excitación de los primeros 
11 términos del Be con gran precisión. En general, la comparación entre 
los oscillator strengths teóricos de las transiciones dipolares de estos 
términos con datos experimentales produjo resultados satisfactorios.

La implementación del RMPS a partir del blanco optimizado con GP 
permitió obtener una buena representación de las excitaciones por 
impacto de electrones. Se encontraron significativas correcciones 
respecto a las secciones eficaces del blanco sin optimizar. Los valores 
teóricos RMPS obtenidos presentan pseudoresonancias, las cuales son 
características de sistemas atómicos donde la representación de 
pseudo-estados no se ha convergido completamente. 
El costo computacional del cálculo RMPS es una de las variables 
determinantes en la predicción del proceso colisional. La elección de 
una estructura atómica pequeña que permita validar el método de 
optimización automática probó ser suficiente para los objetivos del 
presente trabajo. 



