\chapter{Conclusiones}
\label{chap:conclusiones}

En este trabajo de investigación se exploró la optimización de sistemas 
multielectrónicos para el cálculo de colisionales inelásticas. Según 
el proceso colisional y la naturaleza del blanco a examinar, se 
implementaron cuatro metodologías diferentes:
\begin{itemize}
\item 
La estructura electrónica de átomos y moléculas se estudió mediante el 
método de inversión depurada (DIM). Esta novedosa aproximación permitió 
obtener potenciales efectivos que describen exitosamente la estructura 
de diversos blancos. Los potenciales DIM se utilizaron, en conjunción 
con dos teorías perturbativas (FBA y CDW), para calcular la ionización 
debido al impacto de proyectiles (fotones y protones). En términos 
generales, el DIM resultó una técnica eficaz para calcular correctamente 
los procesos inelásticos analizados.

\item
Además, se examinó la ionización debido al impacto de iones desnudos en 
blancos moleculares complejos compuestos por H, C, N y O. Para esto, se 
introdujo un modelo estequiométrico simple (SSM), que permite descibir 
una molécula como una combinación lineal de los átomos que la componen. 
A pesar de la simplicidad de este modelo, se demostró que esta 
aproximación es robusta y confiable para predecir a primer orden las 
secciones eficaces de ionización para decenas de sistemas colisionales, 
incluyendo las nucleobases del ADN y ARN. A partir de los cálculos 
realizados, se encontró una ley de escala independiente que permite 
aproximar este proceso para, en principio, cualquier sistemas 
molécula-proyectil.

\item
En el caso de los atómos pesados, se vió la importancia de elegir el 
método apropiado para describir los efectos relativistas. Implementando 
el método de potencial paramétrico de Klapisch y optimizando las 
configuraciones electrónicas elegidas para representar los blancos, se 
obtuvieron estructuras atómicas precisas de lantánidos y metales de 
transición pesados. Se mostró la necesidad de cálculos relativistas
apropiados para predicciones correctamente la fuerza de frenado debido a 
protones en diversos blancos.

\item
En la última parte de este trabajo, se examinó la excitación por impacto 
de electrones de átomos neutros mediante el método de $R$-matrix. 
En este caso, el esquema de optimización de la estructura atómica 
depende del ajuste manual de los parámetros del problema. 
La complejidad del cálculo se resolvió empleando la optimización 
Bayesiana mediante procesos Gaussianos, que permitió ajustar de forma 
automática la estructura del blanco. Esta implementación mostró 
ser un éxito tanto para reproducir la estructura del blanco (energías 
de excitación, totales y oscillator strengths) como las secciones 
eficaces de excitación, con un costo computacional relativamente bajo.
\end{itemize}
%
Las metodologías empleadas en esta Tesis permitieron calcular de forma 
precisa la estructura electrónica de una amplia variedad de sistemas 
atómicos y moleculares. Estos blancos se examinaron en procesos 
inelásticos descriptos mediante métodos teóricos con diferentes órdenes 
de aproximación --desde aproximaciones perturbativas de primer orden 
hasta métodos completamente cuánticos. Del trabajo presentado se 
concluye que sólo mediante la correcta optimización de los blancos se 
puede calcular de forma precisa, en el rango de validez definido, los 
procesos inelásticos a los que estos son sometidos.


