\chapter*{Conclusiones}%
\addcontentsline{toc}{chapter}{Conclusiones}%

En este trabajo de investigación se presentaron métodos de 
optimización de sistemas multielectrónicos para el cálculo de 
colisionales inelásticas. Según el proceso colisional y la naturaleza 
del blanco, se formularon cuatro metodologías diferentes:
\begin{itemize}
\item 
Se desarrolló el método de inversión depurada (DIM) para estudiar la 
estructura electrónica de átomos y moléculas simples. Esta novedosa 
aproximación permitió obtener potenciales efectivos que describen 
exitosamente la estructura de diversos blancos. Los potenciales DIM se 
utilizaron, en conjunción con dos teorías perturbativas (FBA y CDW), 
para calcular la ionización debido al impacto de iones. En términos 
generales, el DIM resultó una técnica eficaz para calcular correctamente 
los procesos inelásticos analizados.

\item
Se formuló el modelo estequiométrico simple que permitió calcular la 
ionización debido al impacto de iones de carga múltiple en blancos 
moleculares complejos compuestos por H, C, N y O. A pesar de la 
simplicidad del método, se demostró que esta aproximación es robusta y 
confiable para estimar a primer orden las secciones eficaces de 
ionización para decenas de sistemas colisionales, incluyendo las 
nucleobases del ADN y ARN. Utilizando estos nuevos resultados, se 
enunció una ley de escala independiente que permite aproximar este 
proceso, en principio, para cualquier sistema colisional molecular.

\item
En el caso de los atómos pesados, se demostró la importancia de elegir el 
método apropiado para describir los efectos relativistas. Implementando 
el método de potencial paramétrico de Klapisch y optimizando las 
configuraciones electrónicas elegidas para representar los blancos, se 
obtuvieron estructuras atómicas precisas de lantánidos y metales de 
transición pesados. Se mostró que los efectos relativistas también 
afectan los valores de las energías de ligadura de las capas externas. 
La utilización de las soluciones relativistas presentes en el cálculo de 
fuerza de frenado debido a protones fueron determinantes para poder 
explicar recientes mediciones.

\item
Se diseñó un método de optimización para ajustar la estructura de 
átomos neutros en la excitación por impacto de electrones. El cálculo 
de las probabilidades de transición se realizó utilizando un método 
complementamente cuántico. El esquema de optimización de la estructura 
depende de la selección correcta de las configuracions electrónicas a 
introducir en el cálculo, de la elección de potenciales, y del ajuste 
de diversos parámetros. La complejidad del cálculo se resolvió empleando 
la inferencia Bayesiana mediante procesos Gaussianos, que permitió 
ajustar de forma automática estos parámetros. El método de 
optimización desarrollado resultó exitoso para reproducir las energías 
(de excitación y totales), intensidades de oscilador y secciones 
eficaces de excitación, con un costo computacional relativamente bajo.
\end{itemize}
%
En conclusión, los métodos desarrollados cumplieron satisfactoriamente 
los objetivos de esta Tesis. Las metodologías formuladas permitieron 
calcular en forma precisa la estructura electrónica de una amplia 
variedad de sistemas atómicos y moleculares. Además, estas técnicas 
mostraron tener
un buen desempeño en el cálculo de procesos inelásticos descriptos 
mediante métodos teóricos con diferentes órdenes de aproximación --desde 
perturbativas de primer orden hasta métodos completamente cuánticos. 


