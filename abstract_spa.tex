\chapter*{}%
\addcontentsline{toc}{chapter}{Resumen}%

\begin{center}
\begin{large}
\textbf{Optimización de blancos atómicos y moleculares \\ en procesos colisionales}
\end{large}
\end{center}

\vspace{1.5cm}
En esta Tesis, se estudian procesos colisionales inelásticos que surgen 
de la interacción de iones y electrones con blancos atómicos y 
moleculares. Los sistemas examinados a lo largo de este trabajo de 
investigación tienen diversas aplicaciones, que van desde el diagnóstico 
de plasmas astrofísicos, caracterización y mejoramiento del diseño de 
materiales, hasta la evaluación de daño por iones y radiación. 
En el tratamiento de problemas colisionales, es de fundamental importancia 
contar una precisa representación de los blancos. Si bien el desarrollo 
teórico que permite conocer la dinámica de átomos y moléculas está 
determinada por la solución de la ecuación de Schr\"odinger 
correspondiente, es sabido que la resolución exacta de sistemas 
complejos es una tarea imposible de realizar. Por ello, hemos 
desarollado diversas técnicas, cada una de ellas orientada a un problema
colisional diferente. Estos desarrollos incluyen un ``método de inversión 
depurada" (DIM) para la obtención de potenciales efectivos atómicos y 
moleculares, un método Bayesiano para la optimización de la estructura 
del blanco en cálculos de colisiones electrónicas, la implementación 
de métodos relativistas para el frenado de iones en elementos pesados, 
y un modelo estequiométrico para el cálculo de ionización en moleculas 
biológicas.
Todos estos métodos han sido desarrollados desde primeros principios y
son implementados originalmente en este trabajo de Doctorado.

%En esta Tesis se estudian problemas colisionales con aplicaciones diversas, que van desde el estudio de plasmas astrofísicos, frenamiento de iones en tierras raras, emisión electrónica en metales, hasta ionización en blancos moleculares. Es de fundamental importancia, en todos estos casos, obtener una  representacion de los blancos con gran precision. Para ello, desarrollamos distintas tecnicas, orientadas hacia cada una de las aplicaciones mencionadas. Estas incluyen un "metodo de inversion depurada" (DIM) para la obtencion de potenciales efectivos, un metodo Bayesiano para la optimizacion de la estructura del blanco en calculos de colisiones electronicas, la implementacion de metodos relativistas para el frenado de iones en elementos pesados, y un metodo estequiometrico, para el calculo de ionizacion en moleculas biologicas. Todos estos métodos han sido desarrollados desde primeros principios, e implementados originalmente en este trabajo de Doctorado.


\vspace{1cm}
\noindent
Palabras claves:  
Átomos,
Moléculas,
Optimización de blancos,
Procesos colisionales
