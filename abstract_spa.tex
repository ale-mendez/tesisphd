\chapter*{}%
\addcontentsline{toc}{chapter}{Resumen}%

\vspace{-2cm}
\begin{center}
\begin{large}
\textbf{Optimización de blancos atómicos y moleculares \\ 
en procesos colisionales}
\end{large}
\end{center}

\vspace{1cm}
El objetivo de esta Tesis consiste en desarrollar novedosas metodologías 
que permitan 
obtener datos precisos de estructura de blancos atómicos y moleculares 
para su posterior empleo en el cálculo de procesos inelásticos. A lo 
largo de este trabajo, se estudian átomos livianos y pesados, así como 
moléculas simples y complejas, incluyendo las nucleobases. Entre las 
metodologías formuladas se incluye un método 
de inversión para la obtención de potenciales efectivos, que se utilizan 
para calcular la ionización de átomos y moléculas simples por el 
impacto de protones y fotones. Se presenta un modelo estequiométrico 
para describir sistemas moleculares complejos, que permite calcular la 
ionización de moléculas con interés biológico debido al impacto de iones 
de carga múltiple. Basados en la inferencia Bayesiana, ampliamente usada 
en el aprendizaje automatizado, se diseña una metodología para ajustar 
observables en colisiones atómicas. Para el 
cálculo de frenado de iones en átomos pesados, se prescribe un método 
relativista perturbativo junto a la optimización de las configuraciones
electrónicas. En todos los casos, las técnicas propuestas permiten 
describir en forma precisa la estructura electrónica de los blancos, 
mejorando significativamente la calidad y eficiencia de los cálculos 
colisionales.

\vspace{1cm}
\noindent
Palabras claves: 
Colisiones atómicas y moleculares, 
Estructuras electrónicas, 
Ionización de átomos y moléculas, 
Potenciales efectivos
