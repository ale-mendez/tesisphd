
\begin{thebibliography}{9}

%%%%%%%%%%%%%%%%%%%%%%%%%%%%%%%%%%%%%%%%%%%%%%%%%%%%%%%%%%%%%%%%%%%%%%%%
% CAPITULO 2
%%%%%%%%%%%%%%%%%%%%%%%%%%%%%%%%%%%%%%%%%%%%%%%%%%%%%%%%%%%%%%%%%%%%%%%%

% Introducción

\bibitem{HohenberKohn:64}
P. Hohenberg, W. Kohn, 
Phys. Rev., \textbf{136}, B864 (1964).

\bibitem{KohnSham:65}
W. Kohn, L. J. Sham, 
Phys. Rev. \textbf{140}, A1133 (1965).

\bibitem{Becke:14} 
A. D. Becke,
J. Chem. Phys. \textbf{140}, 18A301 (2014).

\bibitem{Bartlett:10} 
R. J. Bartlett, 
Mol. Phys. \textbf{108}, 3299-3311 (2010).

\bibitem{Verma:12} 
P. Verma, R. J. Bartlett,
J. Chem. Phys. \textbf{137}, 134102 (2012).

\bibitem{Slater:51}
J. C. Slater, 
Phys. Rev. \textbf{81}, 385 (1951).

\bibitem{Sharp:53} 
R. T. Sharp, G. K. Horton,
Phys. Rev. \textbf{90}, 317 (1953).

\bibitem{Talman:76} 
J. D. Talman, W. F. Shadwick, 
Phys. Rev. A \textbf{14}, 36-40 (1976).

\bibitem{Talman:89} 
J. D. Talman, 
Comput. Phys. Commun. \textbf{54}, 85-94 (1989).

\bibitem{Krieger:92}
J. B. Krieger, Y. Li, G. J. Iafrate, 
Phys. Rev. A \textbf{45}, 101-126 (1992).

\bibitem{Gorling:92}
A. G\"orling,
Phys. Rev. A \textbf{46}, 3753-3757 (1992).

\bibitem{Yang:02}
W. Yang, Q. Wu,
Phys. Rev. Lett. \textbf{89}, 143002 (2002).

\bibitem{Staroverov:06}
V. N. Staroverov, G. E. Scuseria, E. R. Davidson,
J. Chem. Phys. \textbf{124}, 11103 (2006).

\bibitem{Ryabinkin:13}
I. G. Ryabinkin, A. A. Kananenka, V. N. Staroverov,
Phys. Rev. Lett. \textbf{111}, 013001 (2013).

\bibitem{abinit}
{\sc abinit},  
\url{www.abinit.org}

\bibitem{Vanderbilt}
Vanderbilt Ultra--Soft Pseudopotential,  
\url{www.physics.rutgers.edu/~dhv/uspp/}

\bibitem{Gaiduk:13}
A. P. Gaiduk, I. G. Ryabinkin, V. N. Staroverov,
J. Chem. Theory Comput. \textbf{9}, 3959 (2013).

\bibitem{Wu:03}
Q. Wu, W. Yang,
J. Chem. Phys. \textbf{118}, 2498 (2003).

\bibitem{Ryabinkin:15}
I. G. Ryabinkin, S. V. Kohut, V. N. Staroverov,
Phys. Rev. Lett. \textbf{115}, 083001 (2015).

\bibitem{Mura:97} 
M. E. Mura, P. J. Knowles, C. A. Reynolds,
J. Chem. Phys. \textbf{106}, 9659 (1997).

\bibitem{Umrigar:94} 
C. J. Umrigar, X. Gonze,
Phys. Rev. A \textbf{50}, 3827 (1994).

\bibitem{Gritsenko:97} 
O. V. Gritsenko, E. J. Baerends, 
Theor. Chem. Acc. \textbf{96}, 44 (1997).

\bibitem{Filippi:94} 
C. Filippi, C. J. Umrigar, M. Taut, 
J. Chem. Phys. \textbf{100}, 1290 (1994).

\bibitem{Schipper:97} 
P. R. T. Schipper, O. V. Gritsenko, E. J. Baerends,
Theor. Chem. Acc. \textbf{98}, 16 (1997).

\bibitem{deSilva:12}
P. de Silva, T. A. Wesolowski,
Phys. Rev. A \textbf{85}, 032518 (2012).

\bibitem{Kananenka:13} 
A. A. Kananenka, S. V. Kohut, A. P. Gaiduk, I. G. Ryabinkin, 
J. Chem. Phys. \textbf{139}, 074112 (2013).

\bibitem{Jacob:11} 
C. R. Jacob,
J. Chem. Phys. \textbf{135}, 244102 (2011).

\bibitem{Hilton:77} 
P. R. Hilton, S. Nordholm, N. S. Hush, 
J. Chem. Phys. \textbf{67}, 5213 (1997).

\bibitem{Suzer:77} 
S. S{\"u}zer, P. R. Hilton, N. S. Hush, S. Nordholm,
J. Elect. Spec. Rel. Phen. \textbf{12}, 357 (1977).

\bibitem{Hilton:79} 
P. R. Hilton, S. Nordholm, N. S. Hush,
Chem. Phys. Lett. \textbf{64}, 515 (1979).

\bibitem{Hilton:80} 
P. R. Hilton, S. Nordholm, N. S. Hush, 
J. Elect. Spec. Rel. Phen. \textbf{18}, 101 (1980).

\bibitem{Crljen:87} 
{\v Z}. Crljen, G. Wendin,
Phys. Rev. A \textbf{35}, 1571 (1987).

\bibitem{Sternheimer:54} 
R. M. Sternheimer, 
Phys. Rev. \textbf{96}, 951 (1954).

\bibitem{Dalgarno:59} 
A. Dalgarno, D. Parkinson,
Proc. R. Soc. Lond. A \textbf{250}, 422 (1959).

\bibitem{Mendez:16}
A. M. P. Mendez, D. M. Mitnik, J. E. Miraglia, 
Int. J. Quantum Chem. \textbf{116}, 1882--1890 (2016).

\bibitem{Mendez:18}
A. M. P. Mendez, D. M. Mitnik, J. E. Miraglia, 
Adv. Quant. Chem. \textbf{76}, 117--132 (2018).

\bibitem{Bates:62}
D. R. Bates, 
\textit{Theoretical Treatment of Collisions between Atomic Systems.
In At. Mol. Process.};
Bates, D. R., Ed;
Pure and Applied Physics;
Elsevier, 1962;
Vol.~13, pp 549--621.

\bibitem{McDowell:61}
M. R. C. McDowell, G. Peach, 
Phys. Rev. \textbf{121}, 1383--1387 (1961).

\bibitem{Pindzola:07}
M. S. Pindzola, F. Robicheaux, S. D. Loch, J. C. Berengut, T. Topcu, 
J. Colgan, M. Foster, D. C. Griffin, C. P. Ballance, D. R. Schultz,
T. Minami, N. R. Badnell, M. C. Witthoeft, D. R. Plante, D. M. Mitnik, 
J. A. Ludlow, U. Kleiman, 
%The time-dependent close-coupling method for  atomic and molecular collision processes.
J. Phys. B \textbf{40}, R39-R60 (2007).

\bibitem{Burke:11}
P. G. Burke, 
\textit{R--Matrix Theory of Atomic Collisions.}
Springer--Verlag Berlin Heidelberg, 2011.

\bibitem{Bray:17}
I. Bray, I. B. Abdurakhmanov, J. J. Bailey, A. W. Bray, D. V. Fursa,
A. S. Kadyrov, C. M. Rawlins, J. S. Savage, A. T. Stelbovics, M. C. Zammit,
J. Phy. B \textbf{50}, 202001 (2017).

\bibitem{Pindzola:16}
M. S. Pindzola, J. Colgan, F. Robicheaux, T.-G. Lee, M. F. Ciappina,
M. Foster, J. A. Ludlow, S. A. Abdel-Naby,
Time-Dependent Close--Coupling Calculations for Ion--Impact Ionization of Atoms and Molecules. 
In \textit{Advances In Atomic, Molecular, and Optical Physics};
Arimondo, E.; Lin, C. C.; Yelin, S. F., Ed,; 
Academic Press, 2016; Vol. 65,; pp 291--319.

\bibitem{Szabo:96}
A. Szabo, N. S. Ostlund,
\textit{Modern Quantum Chemistry: Introduction to Advanced Electronic 
Structure Theory},
Dover Publications, Inc.: Mineola, New York, 1996.

\bibitem{Helgaker:00}
T. Helgaker, P. J{\o}rgensen, J. Olsen,
\textit{Molecular Electronic-Structure Theory},
John Wiley {\&} Sons, Ltd: Chichester, UK, 2000.

\bibitem{Schaefer:04}
H. F. Schaefer III,
\textit{Quantum Chemistry: The Development of Ab Initio Methods in
Molecular Electronic Structure Theory},
Dover Publications, Inc: Mineola, New York, 2004.

\begin{comment}
% atomos hidrogenicos

\bibitem{Cowan1981} 
R.D. Cowan, The Theory of Atomic Structure and Spectra}, 
University of California Press (1981). 

\bibitem{Brandsen1983} 
B.H. Brandsen and C.J. Joachin, 
Physics of atoms and molecules}, 
Longman Scientific and Technical (1984).
\end{comment}

\bibitem{FroeseFischer:97}
C. Froese Fischer, T. Brage, P. J\"onsson,
\textit{Computational Atomic Structure: An MCHF Approach},
Institute of Physics Publishing: Bristol, UK, 1997.

\bibitem{Johnson:07}
W. R. Johnson, 
\textit{Atomic Structure Theory: Lectures on Atomic Physics},
Springer--Verlag Berlin Heidelberg, 2007.

\bibitem{Albright:93} 
B. J. Albright, K. Bartschat, P. R. Flicek,
J. Phys. B \textbf{26}, 337 (1993).

\bibitem{Bartschat:96} 
K. Bartschat, 
Computational Atomic Physics,
Springer--Verlag, 1996; Chapter II.

\bibitem{BartschatBray:96} 
K. Bartschat, I. Bray, 
J. Phys. B \textbf{29}, 271 (1996).

% dim en moleculas

\bibitem{Schipper:97}
P. R. T. Schipper, O. V. Gritsenko, E. J. Baerends, 
Theor. Chem. Accounts: Theory, Comput. Model. \textbf{98}, 16--24 (1997).

\bibitem{Mura:97}
M. E. Mura, P. J. Knowles, C. A. Reynolds, 
J. Chem. Phys. \textbf{106}, 9659--9667 (1997).

\bibitem{Jacob:11}
C. R.  Jacob, 
J. Chem. Phys. \textbf{135}, 244102 (2011).

\bibitem{Gaiduk:13}
A. P. Gaiduk, I. G. Ryabinkin, V. N. Staroverov, 
J. Chem. Theory Comput. \textbf{9}, 3959--3964 (2013).

\bibitem{Schmidt:93}
M. W. Schmidt, K. K. Baldridge, J. A. Boatz, S. T. Elbert, M. S. Gordon, 
J. H. Jensen, S. Koseki, N. Matsunaga, K. A. Nguyen, S. Su, T. L. Windus, 
M. Dupuis, J. A. Montgomery, 
%General atomic and molecular electronic structure system.
J. Comput. Chem. \textbf{14}, 1347--1363 (1993).

\bibitem{Gordon:05}
M. S. Gordon, M. W. Schmidt, 
Advances in electronic structure theory: GAMESS a decade later. 
In \textit{Theory Appl. Comput. Chem.}; 
Dykstra, C. E.; Frenking, G.; Kim, K. S.; Scuseria, G. E. Eds;
Elsevier: Amsterdam, 2005; pp 1167--1189.

% experimentos fotoionizacion de N y Ne

\bibitem{Henke:93}
B. L. Henke, E. M. Gullikson, J. C. Davis, 
At. Data Nucl. Data Tables \textbf{54}, 181--342 (1993).

\bibitem{Samson:90}
Samson, J. A. R.; Angel, G. C.
Phys. Rev. A \textbf{42}, 1307--1312 (1990).

\bibitem{Samson:02}
J. A. R. Samson, W. C. Stolte, 
J. Electron Spectros. Relat. Phenomena \textbf{123}, 265--276 (2002).

\bibitem{Stolte:16}
W. C. Stolte, V. Jonauskas, D. W. Lindle, M. M. Sant'Anna, D. W. Savin, 
Astrophys. J. \textbf{818}, 149 (2016).

\bibitem{Ederer:64}
D. L. Ederer, 
Phys. Rev. Lett. \textbf{13}, 760--762 (1964).

% experimentos fotoionizacion ch4

\bibitem{Granados:16}
C. M. Granados--Castro, 
Application of Generalized Sturmian Basis Functions to Molecular Systems.
Tesis de Doctorado, Universit\'e de Lorraine, Metz, France y 
Universidad Nacional del Sur, Bah\'ia Blanca, Argentina, 2016.

\bibitem{Rothenberg:71}
S. Rothenberg, H. F. Schaefer, 
J. Chem. Phys. 54, 2764--2766 (1971).

\bibitem{Hariharan:72}
P. C. Hariharan, J. A. Pople, 
Chem. Phys. Lett. 16, 217--219 (1972).

\bibitem{Lukirskii:64}
A. P. Lukirskii, I. A. Brytov, T. M. Zimkina, 
Optika i spektr. 17, 234 (1964).

\bibitem{Henke:82}
B. L. Henke, P. Lee, T. J. Tanaka, R. L. Shimabukuro, B. K. Fujikawa, 
At. Data Nucl. Data Tables 27, 1--144 (1982).

\bibitem{Samson:89}
J. A. R. Samson, G. N. Haddad, T. Masuoka, P. N. Pareek, D. A. L. Kilcoyne, 
J. Chem. Phys. 90, 6925--6932 (1989).

% experimentos ionizacion ch4

\bibitem{Rudd:83}
M. E. Rudd, R. D. DuBois, L. H. Toburen, C. A. Ratcliffe, T. V. Goffe, 
Phys. Rev. A 28, 3244--3257 (1983).

\bibitem{Rudd:85}
M. E. Rudd, Y. K. Kim, D. H. Madison, J. W. Gallagher, 
Rev. Mod. Phys. 57, 965--994 (1985).

%%%%%%%%%%%%%%%%%%%%%%%%%%%%%%%%%%%%%%%%%%%%%%%%%%%%%%%%%%%%%%%%%%%%%%%%
% IONIZACION DE MOLECULAS: MODELO ESTEQUIOMETRICO
%%%%%%%%%%%%%%%%%%%%%%%%%%%%%%%%%%%%%%%%%%%%%%%%%%%%%%%%%%%%%%%%%%%%%%%%

\bibitem{Mohamad2017}
O. Mohamad, B. J. Sishc, J. Saha, A. Pompos, A. Rahimi, M. D. Story, 
A. J. Davis, D. N. Kim, 
%Carbon Ion Radiotherapy: A Review of Clinical Experiences and Preclinical Research, with an Emphasis on DNA Damage/Repair. 
Cancers \textbf{9}, 66 (2017).

\bibitem{galassi2000}
M. E. Galasssi, R. D. Rivarola, M. Beuve, G. H. Olivera, P. D. Fainstein, 
Phys. Rev. A \textbf{62}, 022701 (2000).

\bibitem{ludde2016}
H. J. L\"udde, A. Achenbach, T. Kalkbrenner, H.-C. Jankowiak, T. Kirchner,
Eur. Phys. J. D \textbf{70}, 82 (2016).

\bibitem{ludde2018}
H. J. L\"udde, M. Horbatsch, T. Kirchner,
Eur. Phys. J. B \textbf{91}, 99 (2018).

\bibitem{fainstein1988}
P. D. Fainstein, V. H. Ponce, R. D. Rivarola,
J. Phys. B: At. Mol. Opt. Phys. \textbf{21}, 287 (1988).

\bibitem{miraglia2008} 
J. E. Miraglia, M. S. Gravielle,
%Ionization of the He, Ne, Ar, Kr, and Xe isoelectronic series by proton impact. 
Phys Rev A \textbf{78}, 052705 (2008)

\bibitem{miraglia2009} 
J. E. Miraglia, 
%Ionization of He, Ne, Ar, Kr, and Xe by proton impact: Single differential distributions. 
Phys. Rev. A \textbf{79}, 022708 (2009).

\bibitem{Denifl2011}
S. Denifl, T. D. Märk, P. Scheier,
The Role of Secondary Electrons in Radiation Damage. 
En \textit{Radiation Damage in Biomolecular Systems. Biological and Medical Physics, Biomedical Engineering.} 
Eds: García Gómez-Tejedor G., Fuss M. 
Springer, Dordrecht (2012) 

\bibitem{toburen1975} 
W. E. Wilson, L. H. Toburen,
%Electron emission from proton --hydrocarbon-molecule collisions at 0.3--2.0 MeV. 
Phys. Rev. A \textbf{11}, 1303 (1975).

\bibitem{toburen1976} 
D. J. Lynch, L. H. Toburen, W. E. Wilson,
%Electron emission from methane, ammonia, monomethylamine, and dimethylamine by 0.25 to 2.0 MeV protons. 
J. Chem. Phys. \textbf{64}, 2616 (1976).

\bibitem{gamess}
M. W. Schmidt, K. K. Baldridge, J. A. Boatz, S. T. Elbert, M. S. Gordon, 
J. H. Jensen, S. Koseki, N. Matsunaga, K. A. Nguyen, S. J. Su, T. L. Windus, 
M. Dupuis, J. A. Montgomery 
J. Comput. Chem. \textbf{14}, 1347-1363 (1993).

\bibitem{salvat1995}
F. Salvat, J. M. Fern\'andez-Varea, W. Williamson,
Comput. Phys. Commun. \textbf{90}, 151--168 (1995)

\bibitem{montanari2017} 
C. C. Montanari, J. E. Miraglia,
%Ionization probabilities of Ne, Ar, Kr, and Xe by proton impact for different initial states and impact energies. 
Nucl. Instr. Meth. Phys. Res. B \textbf{407}, 236--243 (2017).

\bibitem{miraglia2019} 
J. E. Miraglia,
%Shell-to-shell ionization cross sections of antiprotons, H$^{+}$, He$^{2+},$ Be$^{4+},$ C$^{6+}$ and O$^{8+}$ on H, C, N, O, P, and S atoms,
\href{https://arxiv.org/abs/1909.13682}{arXiv:1909.13682 [physics.atom-ph]}.

\bibitem{surdutovic2018} 
E. Surdutovich, A. V. Solov'yov, 
%Multiscale approach to the physics of radiation damage with ions. 
arXiv:1312.0897v, (2013)

\bibitem{abril2015} 
P. de Vera, I. Abril, R. Garcia-Molina, A. V. Solov'yov,
%Ionization of biomolecular targets by ion impact: input data for radiobiological applications. 
J. Phys.: Conf. Ser. \textbf{438}, 012015 (2013).

\bibitem{Rudd1992} 
M. E. Rudd, Y.-K. Kim,, D. H. Madison, T. J. Gay,
%Electron production in proton collisions with atoms and molecules: energy distributions. 
Rev. Mod. Phys. \textbf{64}, 441--490 (1992).

\bibitem{iriki2011}
Y. Iriki, Y. Kikuchi, M. Imai, A. Ito,
Phys. Rev. A \textbf{84}, 052719 (2011).

\bibitem{rahman2016}
M. A. Rahman , E. Krishnakumar,
J. Chem. Phys. \textbf{144}, 161102 (2016).

\bibitem{mozejko2003}
P. Mozejko, L. Sanche, 
%Cross section calculations for electron scattering from DNA and RNA bases.
Radiat Environ. Biophys \textbf{42}, 201 (2003).

\bibitem{tan2018}
H. Q. Tan, Z. Mi, A. A. Bettiol, 
%Simple and universal model for electron-impact ionization of complex biomolecules, 
Phys. Rev. E \textbf{97}, 032403 (2018)

\bibitem{itoh2013} 
A. Itoh, Y. Iriki, M. Imai, C. Champion, R. D. Rivarola, 
%Cross sections for ionization of uracil by MeV-energy-proton impact, 
Phys. Rev. A \textbf{88}, 052711 (2013).

\bibitem{agnihotri2012}
A. N. Agnihotri, S. Kasthurirangan, S. Nandi, A.
Kumar, M. E. Galassi, R. D. Rivarola, O. Foj\'{o}n, C. Champion, J. Hanssen,
H. Lekadir, P. F. Weck, L. C. Tribedi.,
%Ionization of uracil in collisions with highly charged carbon and oxygen ions of energy 100 keV to 78 MeV. 
Phys. Rev. A \textbf{85}, 032711 (2012).

\bibitem{agnihotri2013}
A. N. Agnihotri, S. Kasthurirangan, S. Nandi, A. Kumar, C. Champion, 
H. Lekadir, J. Hanssen, P. F. Weck, M. E. Galassi, R. D. Rivarola, 
O. Fojon, C. Tribedi, 
%Absolute total ionization cross sections of uracil (C$_4$H$_4$N$_2$O$_2$) in collisions with MeV energy highly charged carbon, oxygen and fluorine ions
J. Phys. B \textbf{46}, 185201 (2013).

\bibitem{champion2012} 
C. Champion, M. E. Galassi, O. Foj\'{o}n, H. Lekadir, J. Hanssen, 
R. D. Rivarola, P. F. Weck, A. N. Agnihotri, S. Nandi, L. C. Tribedi,
%Ionization of RNA-uracil by highly charged carbon ions.
J. Phys.: Conf. Ser. \textbf{373}, 012004 (2012).

\bibitem{wolff2014}
W. Wolff, H. Luna, L. Sigaud, A. C. Tavares, E. C. Montenegro,
%Absolute total and partial dissociative cross sections of pyrimidine at electron and proton intermediate impact velocities
J. Chem. Phys. \textbf{140}, 064309 (2014).

\bibitem{bug2017}
M. U. Bug, W. Y. Baek, H. Rabus, C. Villagrasa, S. Meylan, A. B. Rosenfeld,
%An electron-impact cross section data set (10 eV--1 keV) of DNA constituents based on consistent experimental data: A requisite for Monte Carlo simulations,
Rad. Phys. Chem. \textbf{130}, 459--479 (2017).

\bibitem{wang2016}
M. Wang, B. Rudek, D. Bennett, P. de Vera, M. Bug, T. Buhr, W. Y. Baek, 
G. Hilgers, H. Rabus, 
%Cross sections for ionization of tetrahydrofuran by protons at energies between 300 and 3000 keV
Phys. Rev. A \textbf{93}, 052711 (2016).

\bibitem{wolf2019}
W. Wolff, B. Rudek, L. A. da Silva, G. Hilgers, E. C. Montenegro, 
M. G. P. Homem,
%Absolute ionization and dissociation cross sections of tetrahydrofuran: Fragmentation--ion production mechanisms
J. Chem. Phys. \textbf{151}, 064304 (2019).

\bibitem{fuss2009}
M. Fuss, A. Muñoz, J. C. Oller, F. Blanco, D. Almeida, P. Limão-Vieira, 
T. P. D. Do, M. J. Brunger, G. Garc\'{i}a,
%Electron-scattering cross sections for collisions with tetrahydrofuran from 50 to 5000 eV
Phys. Rev. A \textbf{80}, 052709 (2009).

\bibitem{janev1980}
R. K. Janev, L. P. Presnyakov, 
J. Phys. B \textbf{13}, 4233 (1980).

\bibitem{dubois2013}
R. D. DuBois, E. C. Montenegro, G. M. Sigaud,
AIP Conference Proceeding \textbf{1525}, 679 (2013).

\bibitem{lynch1976}
D. J. Lynch, L. H. Toburen, W. E. Wilson,
%Electron emission from methane, ammonia, monomethylamine, and dimethylamine by 0.25 to 2.0 MeV protons
J. Chem. Phys. \textbf{64}, 2616 (1976).

\bibitem{rudd1985}
M.E. Rudd, Y.-K. Kim, D.H. Madison, J.W. Gallagher,
%Electron production in proton collisions: total cross sections,
Review of Modern Physics, \textbf{57}, 965--994 (1985).

\bibitem{luna2007}
H. Luna, A. L. F. de Barros, J. A. Wyer, S. W. J. Scully, J. Lecointre, 
P. M. Y. Garcia, G. M. Sigaud, A. C. F. Santos, V. Senthil, M. B. Shah, 
C. J. Latimer, E. C. Montenegro,
%Water-molecule dissociation by proton and hydrogen impact,
Phys. Rev. A \textbf{75}, 042711 (2007).

\bibitem{ludde2019}
H. J. L\"udde, M. Horbatsch, T. Kirchner,
J. Phys. B \textbf{52}, 195203 (2019).


\bibitem{Hush}
N. S. Hush, A. S. Cheung,  
%Ionization potentials and donor properties of nucleic acid bases and related compounds, 
Chem. Phys. Lett., \textbf{34}, 11 (1975).

\bibitem{Verkin}
B. I. Verkin, L. F. Sukodub, I. K. Yanson, 
%Ionization potentials of nitrogenous bases of of nucleic acids, 
Dokl. Akad. Nauk SSSR, \textbf{228}, 1452 (1976).

\bibitem{Dougherty}
D. Dougherty, E. S. Younathan, R. Voll, S. Abdulnur, S. P. McGlynn,
%Photoelectron spectroscopy of some biological molecules, 
J. Electron Spectrosc. Relat. Phenom., \textbf{13}, 379 (1978).

\bibitem{lee2003} 
J.-G. Lee, H. Y. Jeong, H. Lee, 
%Charges of Large Molecules Using Reassociation of Fragments. 
Bull. Korean Chem. Soc. \textbf{24}, 369 (2003).

\bibitem{rappe1991} 
A. K. Rappe, A. K., W. A. Goddard III,
J. Phys. Chem. \textbf{95}, 3358 (1991).


%%%%%%%%%%%%%%%%%%%%%%%%%%%%%%%%%%%%%%%%%%%%%%%%%%%%%%%%%%%%%%%%%%%%%%%%
%                          ATOMOS RELATIVISTAS
%%%%%%%%%%%%%%%%%%%%%%%%%%%%%%%%%%%%%%%%%%%%%%%%%%%%%%%%%%%%%%%%%%%%%%%%

%%%%%%%%%%%%%%%%%%%%%%%%%%%% Introducción %%%%%%%%%%%%%%%%%%%%%%%%%%%%%%

%Fundamentals of Stopping Power and Energy Loss processes
\bibitem{Chu:01} 
W. K. Chu, J. W. Mayer, M. A. Nicolet,
\textit{Backscattering Spectrometry}
(Academic Press, New York, 1978).

\bibitem{Sigmund:06} 
P. Sigmund, 
\textit{Particle Penetration and Radiation Effects. General Aspects and 
Stopping of Swift Point Charges}.
(Springer Series in Solid-State Sciences, Springer, Berlin, 2006), Vol. 151.

\bibitem{Schardt:10} 
D. Schardt, T. Els\"asser, D. Schulz-Ertner, 
Rev. Mod. Phys. \textbf{82},  383-425 (2010).

\bibitem{iaea_codes} 
Códigos disponibles para cálculos de potencial de frenado se pueden 
encontrar en \href{https://www-nds.iaea.org/stopping/stopping\_prog.html}
{www-nds.iaea.org/stopping/stopping\_prog.html}

\bibitem{Paul:03}
H. Paul, A. Schinner,
At. Data Nucl. Data Tables  \textbf{85}, 377-452 (2003).

\bibitem{Diwan:15} 
P.K. Diwan, S. Kumar, 
Nucl. Instr. and Meth. B \textbf{359}, 78-84 (2015).

\bibitem{Damache:04} 
S. Damache, S. Ouichaoui, A. Belhout, A. Medouni, I. Toumert, 
Nucl. Instr. and Meth. B \textbf{225}, 449-463 (2004).

\bibitem{Damache:02} 
D. Moussa, S. Damache, S. Ouichaoui, 
Nucl. Instr. and Meth. B \textbf{268}, 1754-1758 (2010); 
\textbf{343},  44-47 (2015).

%%%%%%%%%%%%%%%%%%%%%%%%%%%% Teoría Hullac %%%%%%%%%%%%%%%%%%%%%%%%%%%%%
\bibitem{Klapisch:77}
M. Klapisch, B. Fraenkel, J.L. Schwob, J. Oreg,
J. Opt. Soc. Am. \textbf{62}, 148 (1977).

\bibitem{Koenig:72}
E. Koenig,
Physica (Utrecht) \textbf{62}, 93 (1972).

\bibitem{Klapisch:71}
M. Klapisch,
Comput. Phys. Comm. \textbf{2}, 239 (1971).

\bibitem{Klapisch:67}
M. Klapisch,
Comput. Rend. Acad. Sci. \textbf{265}, 914 (1967).
%%%%%%%%%%%%%%%%%%%%%%%%%%%%%%%%%%%%%%%%%%%%%%%%%%%%%%%%%%%%%%%%%%%%%%%%
\bibitem{Roth:17}
D. Roth, B. Bruckner, M. V. Moro, S. Gruber, D. Goebl, J. I. Juaristi,
M. Alducin, R. Steinberger, J. Duchoslav, \mbox{D. Primetzhofer}, P. Bauer,
Phys. Rev. Lett. \textbf{118}, 103401 (2017).

% Semi-empirical and Theoretical approach
\bibitem{Montanari:13}
C. C. Montanari, J. E. Miraglia,
Adv. Quantum Chem. \textbf{65}, 165 (2013).

\bibitem{Montanari:17} 
C.C. Montanari, J.E. Miraglia, 
Phys. Rev. A \textbf{96}, 012707 (2017).

\bibitem{Mermin:70} 
N.D. Mermin, 
Phys. Rev. B \textbf{1}, 2362 (1970).

\bibitem{Mendez:19} 
A.M.P. Mendez, C.C. Montanari, D.M. Mitnik, 
Nucl. Instrum. Meth. B \textbf{460}, 114-118 (2019).

\bibitem{Grande:01} 
G. Schiwietz, P. L. Grande, 
Nucl. Instrum. Meth. B \textbf{175-177}, 125-131 (2001); 
Código CasP, disponible en \href{https://www.casp-program.org}{www.casp-program.org}

\bibitem{casp52} 
G. Schiwietz, P. L. Grande,
Nucl. Instr. and Meth. B \textbf{273}, 1-5 (2012); 
P.L. Grande, G. Schiwietz, 
Phys. Rev. A \textbf{58}, 3796 (1998).

\bibitem{DPASS20} 
A. Schinner, P. Sigmund, 
Nucl. Instrum. Meth. B \textbf{460}, 19 (2019); 
P.Sigmund, A. Schinner, 
Eur. Phys. J. D \textbf{12}, 425 (2000). 
Código DPASS, disponible en \href{https://www.sdu.dk/en/DPASS/}{www.sdu.dk/en/DPASS/}

\bibitem{Ziegler01} 
J.F. Ziegler, J.P. Biersack, M. D. Ziegler, 
\textit{SRIM, The Stopping and Range of Ions in Matter}, 
(SRIM Co. Maryland, USA, 2008); 
SRIM2013, Computer Program and Manual. Disponible en \href{https://www.srim.org}{www.srim.org}.

\bibitem{ICRU49} 
ICRU report 49, \textit{Stopping Powers and Ranges for Protons and Alpha Particles},
International Commission on Radiation Units and Measurements (1993).

\bibitem{BarShalom:01}
A. Bar--Shalom, M. Klapisch, J. Oreg,
J. Quant. Spectrosc. Radiat. Transf. \textbf{71}, 169 (2001).

\bibitem{Williams:95}
G. Williams en \href{http://xdb.lbl.gov/Section1/Sec\_1-1.html}{xdb.lbl.gov/Section1/Sec\_1-1.html}

\begin{comment}
\bibitem{Montanari:09}
C. C. Montanari, C. D. Archubi, D. M. Mitnik, J. E. Miraglia,
Phys. Rev. A \textbf{79}, 032903 (2009);

\bibitem{Montanari:11}
C.C. Montanari, D. M. Mitnik, J. E. Miraglia,
Rad. Eff. Defects Sol. \textbf{166}, 338 (2011).

\bibitem{Oswald:18}
M. Oswal, Sunil Kumar, Udai Singh, G. Singhe, K. P. Singh, D. Mehta,
D. Mitnik, C. C. Montanari, T.Nandi,
Nucl. Instr. Meth. Phys.
Res. B \textbf{416}, 110 (2018).
\end{comment}
% heavy atoms

\bibitem{Desclaux:73}
J. P. Desclaux,
Atomic Data and Nuclear Data Tables \textbf{12}, 311 (1973).
% %
\bibitem{werner}
W. S. M. Werner, K. Glantschnig, C. Ambrosch--Draxl,
J. Phys. Chem. Ref. Data \textbf{38} 1013-1092 (2009).

\bibitem{lynch}
D. W. Lynch, C. G. Olson, J. H. Weaver,
Phys. Rev. B \textbf{11}, 3617 (1975).

\bibitem{isaacson}
D. Isaacson,
\textit{Compilation of rs values}, New York University Rep. No. 02698
(National Auxiliary Publication Service, NY 1975).

\bibitem{romaniello}
P. Romaniello, P. L. de Boeij, F. Carbone, D. van der Marel,
Phys. Rev. B \textbf{73}, 075115 (2006).

\bibitem{Montanari:19}
A. M. P. Mendez, C. C. Montanari, D. M. Mitnik, J. E. Miraglia,
\textit{en preparación}.

%%%%%%%%%%%%%%%%%%%%%%%%%%%%%%%%%%%%%%%%%%%%%%%%%%%%%%%%%%%%%%%%%%%%%%%%
% STOPPING DE HAFNIO
%%%%%%%%%%%%%%%%%%%%%%%%%%%%%%%%%%%%%%%%%%%%%%%%%%%%%%%%%%%%%%%%%%%%%%%%

%=======================================================================
%% Use of Hafnium and Stopping Power of Hafnium
\bibitem{Sirotinin} 
E. I. Sirotinin, A. F. Tulinov, V. A. Khodyrev, V. N. Mizgulin, 
Nucl. Instr. and Meth. B \textbf{4}, 337-345 (1984).

\bibitem{Abril} 
I. Abril, M. Behar, R. Garcia Molina, R. C. Fadanelli, L. C. C. M. Nagamine, 
P. L. Grande, L. Sch\"unemann, C. D. Denton, N. R. Arista, E. B. Saitovich,
Eur. Phys. J. D \text{54}, 65-70 (2009).

\bibitem{Behar} 
M. Behar, R. C. Fadanelli, I. Abril, R. Garcia-Molina, C. D. Denton, 
L. C. C. M. Nagamine, N. R. Arista, 
Phys. Rev. A \textbf{80},  062901 (2009).

\bibitem{Primetzhofer} 
D. Primetzhofer, 
Nucl. Instr. and Meth. B \textbf{320}, 100-103 (2014).

\bibitem{Roth}
D. Roth, B. Bruckner, G. Undeutsch, V. Paneta, A. I. Mardare, 
C. L. McGahan, M. Dosmailov, J. I. Juaristi, M. Alducin, 
J. D. Pedarnig, R. F. Haglund, Jr., D. Primetzhofer, P. Bauer
Phys. Rev. Lett. \textbf{119}, 163401 (2017).

%=======================================================================
% Why Hafnium is so important:
\bibitem{Choi} 
J. H. Choi, Y. Mao, J. P. Chang, 
Mat. Sci. Eng. R \textbf{72}, 97-136 (2011).

\bibitem{Robertson} 
J. Robertson, R. M. Wallace, 
Mat. Sci. Eng. R \textbf{88}, 1-41 (2015).

%=======================================================================
%Importance of Stopping Power for Ion Beam Analysis
\bibitem{Alfassi01} 
Z. B. Alfassi,
\textit{Non-destructive elemental analysis}
(Blackwell Publishing, Oxford, 2001).

\bibitem{Tesmer01} 
J. R. Tesmer, M. Nastasi, J. C. Barbour, C. J. Maggiore, J. W. Mayer,
\textit{Handbook of Modern Ion Beam Material Analysis}.
(Materials Research Society, Pittsburgh, 1995).

\bibitem{HPaul03} 
\href{https://www-nds.iaea.org/stopping/}{www-nds.iaea.org/stopping/}.

\bibitem{mondim17} 
C. C. Montanari, P. Dimitriou, 
Nucl. Instr. and Meth. B \textbf{408},  50-55 (2017).

\bibitem{Roth17} 
D. Roth, B. Bruckner, M. V. Moro, S. Gruber, D. Goebl, J. I. Juaristi, 
M. Alducin,R. Steinberger, J. Duchoslav, D. Primetzhofer, P. Bauer, 
Phys. Rev. Lett. \textbf{118}, 103401 (2017).

%=======================================================================
\begin{comment}
\bibitem{zenodo} 
C.C. Montanari \textit{et al.} DOI: 10.5281/zenodo.3678785 
\end{comment}
%=======================================================================
% Experimental
\bibitem{Miranda01} 
P. A. Miranda, A. Sep\'ulveda, J. R. Morales, T. Rodriguez, E. Burgos, 
H. Fern\'andez,
Nucl. Instr. and Meth. B \textbf{318}, 292-296  (2014).

\bibitem{Lebow} 
Lebow Company. 5960 Mandarin Ave. Goleta CA, 93117, USA.

%=======================================================================
% Transmission Method
\bibitem{Sun01} 
G. Sun, M. D\"{o}belli, A.M. M\"{u}ller, M. Stocker, M. Suter, L. Wacker, 
Nucl. Instr. and Meth. B \textbf{256}, 586-590 (2007).

\bibitem{Raisanen01} 
J. Raisanen, U. Watjen, A.J.M. Plompen, F. Munnik, 
Nucl. Instr. and Meth. \textbf{B} 118, 1-6  (1996).

\bibitem{Schulz01} 
F. Schulz, J. Shchuchinzky, 
Nucl. Instr. and Meth. B \textbf{12},  90-94 (1985).

\bibitem{Chilton} 
A.B. Chilton, J.N. Cooper, J.C. Harris, 
Phys. Rev. \textbf{93}, 413-418  (1954).

\bibitem{Rajatora} 
M. Rajatora, K. Vakevainen, T. Ahlgre, E. Rauhala, J. Raisanen, K. Rakennus, 
Nucl. Instr. and Meth. B \textbf{119}, 457-462 (1996).

%=======================================================================
%X Theoretical Approach (Montanari)
\bibitem{Echenique:81} 
P. M. Echenique, R. M. Nieminen, R. H. Ritchie, 
Sol. State Comm. \textbf{37}, 779-781 (1981).

\bibitem{Nagy:89} 
I. Nagy, A. Arnau, P. M. Echenique, E. Zaremba, 
Phys. Rev. B \textbf{40}, 11983 (1989).

\bibitem{lynch75} 
D. W. Lynch, C. G. Olson, J. H. Weaver, 
Phys. Rev. B \textbf{11}, 3617-3624 (1975).

\bibitem{suppression} 
C. C. Montanari, J. E. Miraglia, N. R. Arista, 
Phys. Rev. A \textbf{62}, 052902 (2000).

\bibitem{Hf_arxiv} 
A. M. P. Mendez, C. C. Montanari, D. M. Mitnik, 
\textit{Slater-type orbital expansion of neutral hafnium, numerical 
solution of the relativistic Dirac equation}, 
available soon in arXiv.org. 

\bibitem{badnell97} 
N. R. Badnell, 
Comput. Phys. Commun. \textbf{182}, 1528 (2011).

\bibitem{mon09} 
C. C. Montanari, D. M. Mitnik, C. D. Archubi, J. E. Miraglia, 
Phys. Rev. A \textbf{70}, 032903 (2009); 
Phys. Rev. A \textbf{80}, 012901 (2009).

\bibitem{lindhard53} 
J. Lindhard, M. Scharff,  
Mat. Fys. Medd. Dan. Vid. Selsk  \textbf{27}, 1 (1953).

\bibitem{chu72} 
W. K. Chu, D. Powers, 
Rev. Lett. A \textbf{40}, 23 (1972).

\end{thebibliography}
