
\begin{thebibliography}{9}

\bibitem{Mendez:16}
A. M. P. Mendez, D. M. Mitnik, J. E. Miraglia, 
Int. J. Quantum Chem. \textbf{116}, 1882--1890 (2016).

\bibitem{Mendez:19dim}
A. M. P. Mendez, D. M. Mitnik, J. E. Miraglia, 
Adv. Quant. Chem. \textbf{8}, 179--200 (2019).

\bibitem{Mendez:18}
A. M. P. Mendez, D. M. Mitnik, J. E. Miraglia, 
Adv. Quant. Chem. \textbf{76}, 117--132 (2018).

\bibitem{Mitnik:19}
D. M. Mitnik, A. M. P. Mendez, J. E. Miraglia, 
Int. J. Quantum Chem. \textbf{120}, e26102 (2020).

\bibitem{Mendez:20ionmol}
A.M.P. Mendez, C.C. Montanari, J.E. Miraglia,
J. Phys. B, \textbf{53}, 055201 (2020).

\bibitem{Mendez:20scale}
A.M.P. Mendez, C.C. Montanari, J.E. Miraglia,
J. Phys. B, \textbf{53}, 175202 (2020). 

\bibitem{Mendez:19relat} 
A.M.P. Mendez, C.C. Montanari, D.M. Mitnik, 
Nucl. Instrum. Methods Phys. Res., B \textbf{460}, 114-118 (2019).

\bibitem{Montanari:20}
C. C. Montanari, P. A. Miranda, E. Alves, A. M. P. Mendez,
D. M. Mitnik, J. E. Miraglia, R. Correa, J. Wachter, M. Aguilera, 
N. Catarino, R. C. da Silva,
Phys. Rev. A \textbf{101}, 062701 (2020). 

%\bibitem{Oswald:20}
%M. Oswald, S. Kumar, U. Singh, G. Singh, K.P.Singh, D. Mehta, 
%A. M. P. Mendez, D. M. Mitnik, C. C. Montanari, D. Mitra, T. Nandi,
%Radiat. Phys. Chem.  \textbf{176}, 108809 (2020).

\bibitem{Mendez:20baye}
A.M.P. Mendez, J.I. Di Filippo, S.D. López, D.M. Mitnik,
J. Phys.: Conf. Ser. \textbf{1412}, 132027 (2020).

\bibitem{Mendez:prep}
A.M.P. Mendez, D.M. Mitnik, en preparación.

%%%%%%%%%%%%%%%%%%%%%%%%%%%%%%%%%%%%%%%%%%%%%%%%%%%%%%%%%%%%%%%%%%%%%%%%
% CAPITULO 2
%%%%%%%%%%%%%%%%%%%%%%%%%%%%%%%%%%%%%%%%%%%%%%%%%%%%%%%%%%%%%%%%%%%%%%%%

%%%% Introducción %%%% 

\bibitem{Bates:62}
D. R. Bates, 
\textit{Theoretical Treatment of Collisions between Atomic Systems.
In At. Mol. Process.};
Bates, D. R., Ed;
Pure and Applied Physics;
Elsevier, 1962;
Vol.~13, pp 549--621.

\bibitem{McDowell:61}
M. R. C. McDowell, G. Peach, 
Phys. Rev. \textbf{121}, 1383--1387 (1961).

\bibitem{Crothers:10}
D. S. F. Crothers,
% An introduction to Continuum Distorted Wave theory
J Atom. Mol. Opt. Phys. vol. 2010, 604572 (2010).

\bibitem{Rivarola:87}
R. D. Rivarola, P. D. Fainstein,
Nucl. Instrum. Methods Phys. Res., B \textbf{24-25}, 240--242 (1987).

\bibitem{Pindzola:07}
M. S. Pindzola, F. Robicheaux, S. D. Loch, J. C. Berengut, T. Topcu, 
J. Colgan, M. Foster, D. C. Griffin, C. P. Ballance, D. R. Schultz,
T. Minami, N. R. Badnell, M. C. Witthoeft, D. R. Plante, D. M. Mitnik, 
J. A. Ludlow, U. Kleiman, 
%The time-dependent close-coupling method for  atomic and molecular collision processes.
J. Phys. B \textbf{40}, R39-R60 (2007).

\bibitem{Burke:11}
P. G. Burke, 
\textit{R--Matrix Theory of Atomic Collisions.}
Springer--Verlag Berlin Heidelberg, 2011.

\bibitem{Bray:17}
I. Bray, I. B. Abdurakhmanov, J. J. Bailey, A. W. Bray, D. V. Fursa,
A. S. Kadyrov, C. M. Rawlins, J. S. Savage, A. T. Stelbovics, M. C. Zammit,
J. Phy. B \textbf{50}, 202001 (2017).

\bibitem{Zatsarinny:04}
O. Zatsarinny, K. Bartschat,
J. Phys. B \textbf{37}, 2173 (2004).

\bibitem{McCurdy:04}
C. W. McCurdy,M.  Baertschy, T. N. Rescigno,
J. Phys. B \textbf{37}, R137 (2004).

\bibitem{Bransden:03}
B. H. Bransden, C. J. Joachain,
``Physics of atoms and molecules'', 2da edición (2003),
Pearson Education Limited, Harlow, Inglaterra.

\bibitem{Cowan:81}
R. D. Cowan,
``The Theory of Atomic Structure and Spectra'', 1ra edición (1981),
University of California Press. Berkeley, Estados Unidos.

%%%% Potenciales %%%% 

\bibitem{Hibbert:82}
A. Hibbert,
``Advances in atomic and molecular physics'', Vol. 18 (1982).
Eds: D. Bates, B. Bederson. Elsevier Science \& Technology Books.

\bibitem{Gombas:56}
P. Gombás, 
Handb. Phys. \textbf{36}, 109--231, Berlin: Springer-Verlag, (1956).

\bibitem{Green:69}
A. E. S. Green, D. L. Sellin, A. S. Zachor,
Phys. Rev. \textbf{184}, 1 (1969).

\bibitem{Klapisch:71}
M. Klapisch,
Comput. Phys. Comm. \textbf{2}, 239 (1971).

\bibitem{Phillips:59}
J. C. Phillips, L. Kleinmann,
Phys. Rev. \textbf{116}, 287 (1959).

\bibitem{Herman:63}
F. Hermanm, S. Skillman,
``Atomic Structure Calculations''. 
Prentice-Hall, New York, USA (1963).

\bibitem{Dalgarno:70}
A. Dalgarno, C. Bottcher, G. A. Victor, 
Chem. Phys. Lett. \textbf{7}, 265 (1970).

\bibitem{Bayliss:77}
W. E. Bayliss,
J. Phys. B \textbf{10}, L583 (1977).

\bibitem{Cowan:76}
R. D. Cowan, D. C. Griffin, 
J. Opt. Soc. Am. \textbf{66}, 1010 (1976).

\bibitem{Lee:77}
Y. S. Lee, W. C. Ermler, K. S. Pitzer, 
J. Chem. Phys. \textbf{67}, 5861 (1977).

\bibitem{Menchero:10}
L. Fernández--Menchero and S. Otranto, 
Phys. Rev. A {\bf 82}, 022712 (2010).

\bibitem{Granados:16}
C. M. Granados--Castro, 
``Application of Generalized Sturmian Basis Functions to
Molecular Systems''. 
Universit\'e de Lorraine, Metz, France y Universidad Nacional del Sur, Bah\'ia Blanca, Argentina, 2016.

%%% inversion

\bibitem{Wu:03}
Q. Wu, W. Yang,
J. Chem. Phys. \textbf{118}, 2498 (2003).

\bibitem{Ryabinkin:15}
I. G. Ryabinkin, S. V. Kohut, V. N. Staroverov,
Phys. Rev. Lett. \textbf{115}, 083001 (2015).

\bibitem{deSilva:12}
P. de Silva, T. A. Wesolowski,
Phys. Rev. A \textbf{85}, 032518 (2012).

\bibitem{Kananenka:13} 
A. A. Kananenka, S. V. Kohut, A. P. Gaiduk, I. G. Ryabinkin, 
J. Chem. Phys. \textbf{139}, 074112 (2013).

\bibitem{Schipper:97} 
P. R. T. Schipper, O. V. Gritsenko, E. J. Baerends,
Theor. Chem. Acc. \textbf{98}, 16 (1997).
%Theor. Chem. Accounts: Theory, Comput. Model. \textbf{98}, 16--24 (1997).

\bibitem{Jacob:11} 
C. R. Jacob,
J. Chem. Phys. \textbf{135}, 244102 (2011).

\bibitem{Gaiduk:13}
A. P. Gaiduk, I. G. Ryabinkin, V. N. Staroverov, 
J. Chem. Theory Comput. \textbf{9}, 3959--3964 (2013).

\bibitem{Hilton:77} 
P. R. Hilton, S. Nordholm, N. S. Hush, 
J. Chem. Phys. \textbf{67}, 5213 (1997).

\bibitem{Suzer:77} 
S. S{\"u}zer, P. R. Hilton, N. S. Hush, S. Nordholm,
J. Elect. Spec. Rel. Phen. \textbf{12}, 357 (1977).

\bibitem{Hilton:79} 
P. R. Hilton, S. Nordholm, N. S. Hush,
Chem. Phys. Lett. \textbf{64}, 515 (1979).

\bibitem{Hilton:80} 
P. R. Hilton, S. Nordholm, N. S. Hush, 
J. Elect. Spec. Rel. Phen. \textbf{18}, 101 (1980).

\bibitem{Crljen:87} 
{\v Z}. Crljen, G. Wendin,
Phys. Rev. A \textbf{35}, 1571 (1987).

\bibitem{Sternheimer:54} 
R. M. Sternheimer, 
Phys. Rev. \textbf{96}, 951 (1954).

\bibitem{Dalgarno:59} 
A. Dalgarno, D. Parkinson,
Proc. R. Soc. Lond. A \textbf{250}, 422 (1959).

\bibitem{Albright:93} 
B. J. Albright, K. Bartschat, P. R. Flicek,
J. Phys. B \textbf{26}, 337 (1993).

\bibitem{Bartschat:96} 
K. Bartschat, 
Computational Atomic Physics,
Springer--Verlag, 1996; Chapter II.

\bibitem{BartschatBray:96} 
K. Bartschat, I. Bray, 
J. Phys. B \textbf{29}, 271 (1996).

\bibitem{DiFilippo:19}
J. I. Di Filippo,
``Métodos de Aprendizaje Automático para la obtención de Potenciales 
Efectivos Atómicos'', Tesis de Licenciatura, Universidad de Buenos Aires
(2019).


% exchange potential

\bibitem{Slater:51}
J. C. Slater, 
Phys. Rev. \textbf{81}, 385 (1951).

\bibitem{Sharp:53} 
R. T. Sharp, G. K. Horton,
Phys. Rev. \textbf{90}, 317 (1953).

\bibitem{Krieger:92}
J. B. Krieger, Y. Li, G. J. Iafrate, 
Phys. Rev. A \textbf{45}, 101-126 (1992).

\bibitem{Gorling:92}
A. G\"orling,
Phys. Rev. A \textbf{46}, 3753-3757 (1992).

\bibitem{Yang:02}
W. Yang, Q. Wu,
Phys. Rev. Lett. \textbf{89}, 143002 (2002).

\bibitem{Staroverov:06}
V. N. Staroverov, G. E. Scuseria, E. R. Davidson,
J. Chem. Phys. \textbf{124}, 11103 (2006).

\bibitem{Ryabinkin:13}
I. G. Ryabinkin, A. A. Kananenka, V. N. Staroverov,
Phys. Rev. Lett. \textbf{111}, 013001 (2013).

\bibitem{Helgaker:00}
T. Helgaker, P. J{\o}rgensen, J. Olsen,
\textit{Molecular Electronic-Structure Theory},
John Wiley {\&} Sons, Ltd: Chichester, UK, 2000.

\bibitem{Schaefer:04}
H. F. Schaefer III,
\textit{Quantum Chemistry: The Development of Ab Initio Methods in
Molecular Electronic Structure Theory},
Dover Publications, Inc: Mineola, New York, 2004.

% dim en moleculas

\bibitem{Schmidt:93}
M. W. Schmidt, K. K. Baldridge, J. A. Boatz, S. T. Elbert, M. S. Gordon, 
J. H. Jensen, S. Koseki, N. Matsunaga, K. A. Nguyen, S. Su, T. L. Windus, 
M. Dupuis, J. A. Montgomery, 
%General atomic and molecular electronic structure system.
J. Comput. Chem. \textbf{14}, 1347--1363 (1993).

\bibitem{Gordon:05}
M. S. Gordon, M. W. Schmidt, 
Advances in electronic structure theory: GAMESS a decade later. 
In \textit{Theory Appl. Comput. Chem.}; 
Dykstra, C. E.; Frenking, G.; Kim, K. S.; Scuseria, G. E. Eds;
Elsevier: Amsterdam, 2005; pp 1167--1189.

%%% Hartree--Fock numérico

\bibitem{FroeseFischer:97}
C. Froese Fischer, T. Brage, P. J\"onsson,
\textit{Computational Atomic Structure: An MCHF Approach},
Institute of Physics Publishing: Bristol, UK, 1997.

\bibitem{Johnson:07}
W. R. Johnson, 
\textit{Atomic Structure Theory: Lectures on Atomic Physics},
Springer--Verlag Berlin Heidelberg, 2007.

%%% OEP %%%

\bibitem{Talman:76} 
J. D. Talman, W. F. Shadwick, 
Phys. Rev. A \textbf{14}, 36-40 (1976).

\bibitem{Talman:89} 
J. D. Talman, 
Comput. Phys. Commun. \textbf{54}, 85-94 (1989).

%%% EAHF

\bibitem{Becke:88}
A.D. Becke, 
%``Density--functional exchange--energy approximation with correct 
%asymptotic behavior", 
Phys. Rev. A \textbf{38}, 3098 (1988).

% dim ch4

\bibitem{Rothenberg:71}
S. Rothenberg, H. F. Schaefer, 
J. Chem. Phys. \textbf{54}, 2764--2766 (1971).

\bibitem{Hariharan:72}
P. C. Hariharan, J. A. Pople, 
Chem. Phys. Lett. \textbf{16}, 217--219 (1972).

\bibitem{Moccia:69}
R. Moccia, 
J. Chem. Phys. \textbf{40}, 2164 (1964).

% experimentos fotoionizacion de He, N, Ne y CH4

\bibitem{Samson:90}
Samson, J. A. R.; Angel, G. C.
Phys. Rev. A \textbf{42}, 1307--1312 (1990).

\bibitem{Stolte:16}
W. C. Stolte, V. Jonauskas, D. W. Lindle, M. M. Sant'Anna, D. W. Savin, 
Astrophys. J. \textbf{818}, 149 (2016).

\bibitem{Henke:93}
B. L. Henke, E. M. Gullikson, J. C. Davis, 
At. Data Nucl. Data Tables \textbf{54}, 181--342 (1993).

\bibitem{Samson:02}
J. A. R. Samson, W. C. Stolte, 
J. Electron Spectros. Relat. Phenomena \textbf{123}, 265--276 (2002).

\bibitem{Ederer:64}
D. L. Ederer, 
Phys. Rev. Lett. \textbf{13}, 760--762 (1964).

\bibitem{Lukirskii:64}
A. P. Lukirskii, I. A. Brytov, T. M. Zimkina, 
Optika i spektr. 17, 234 (1964).

\bibitem{Henke:82}
B. L. Henke, P. Lee, T. J. Tanaka, R. L. Shimabukuro, B. K. Fujikawa, 
At. Data Nucl. Data Tables 27, 1--144 (1982).

\bibitem{Samson:89}
J. A. R. Samson, G. N. Haddad, T. Masuoka, P. N. Pareek, D. A. L. Kilcoyne, 
J. Chem. Phys. 90, 6925--6932 (1989).

% experimentos ionizacion N y CH4

\bibitem{Rudd:85}
M. E. Rudd, Y.-K. Kim, D. H. Madison, J. W. Gallagher, 
Rev. Mod. Phys. \textbf{57}, 965--994 (1985).

\bibitem{Rudd:83}
M. E. Rudd, R. D. DuBois, L. H. Toburen, C. A. Ratcliffe, T. V. Goffe, 
Phys. Rev. A 28, 3244--3257 (1983).

\bibitem{Brook:78}
E. Brook, M. F. A. Harrison, A. C. H. Smith, 
J. Phys. B \textbf{11}, 3115--3132 (1978).

%%% local potential %%%

\bibitem{Amusia:04}
M.Ya. Amusia, A.Z. Msezane, V.R. Shaginyan, and D. Sokolovski, 
%``On the relation between the Hartree--Fock and Kohn--Sham
%approaches'',
Phys. Lett. A {\bf 330}, 10 (2015).

%%% comporamiento asintótico %%%

\bibitem{Cinal:19}
M. Cinal,
%Comment on ``Depurated inversion method for orbital-specific exchange
%potentials'',
Int. J. Quantum Chem. \textbf{120}, e26101 (2019).

\bibitem{Casida:89}
M. E. Casida, D. P. Chong, 
%``Large $r$ approximation for spherically averaged momentum distributions'',
Chem. Phys. \textbf{132}, 391 (1989).

%%% decaimiento asintótico %%%

\bibitem{Handy:69}
N. C. Handy, M. T. Marron, H. J. Silverstone, 
%``Long--range behavior of Hartree--Fock orbitals'',
Phys. Rev. \textbf{180}, 45 (1969).

\bibitem{Handler:80}
G. S. Handler, D. W. Smith, H. J. Silverstone, 
%``Asymptotic behavior of atomic Hartree--Fock orbitals'',
J. Chem. Phys. \textbf{73}(8), 3936 (1980).

\bibitem{Ishida:92}
T. Ishida, K. Ohno, 
%``On the asymptotic behavior of Hartree--Fock orbitals'', 
Theor. Chim. Acta \textbf{81} 355 (1992).

\bibitem{Cinal:10}
M. Cinal,
%``Direct mapping between exchange potentials of Hartree--Fock and 
%Kohn–Sham schemes as origin of orbital proximity'',
J. Chem. Phys. \textbf{132}, 014104 (2010).

\bibitem{Weber:71}
T. A. Weber, R. G. Parr, 
%``Hartree--Fock theory with exchange cutoff'', 
Phys. Rev. A \textbf{3}, 81 (1971).

\bibitem{Weber:70}
T. A. Weber, N. C. Handy, R. G. Parr, 
J. Chem. Phys. \textbf{52}, 1501 (1970).

%%%%%%%%%%%%%%%%%%%%%%%%%%%%%%%%%%%%%%%%%%%%%%%%%%%%%%%%%%%%%%%%%%%%%%%%
% IONIZACION DE MOLECULAS: MODELO ESTEQUIOMETRICO
%%%%%%%%%%%%%%%%%%%%%%%%%%%%%%%%%%%%%%%%%%%%%%%%%%%%%%%%%%%%%%%%%%%%%%%%

\bibitem{Baskar:12}
R. Baskar, K. A. Lee, R. Yeo, K.-W. Yeoh,
%Cancer and radiation therapy: Current advances and future directions. 
Int. J. Med. Sci. \textbf{9}, 193--199 (2012).
%https://doi.org/10.7150/ijms.3635 

\bibitem{Solov:09}
A. V. Solov'yov, E. Surdutovich, E. Scifoni, I. Mishustin, and 
W. Greiner, 
%Physics of ion beam cancer therapy: A multiscale approach
Phys. Rev. E \textbf{79}, 011909 (2009);
% https://link.aps.org/doi/10.1103/PhysRevE.79.011909

\bibitem{Gafur:18} 
N. A. Gafur, M.  Sakakibara, S. Sano, K. A. Sera, 
% \textit{Case Study of Heavy Metal Pollution in Water of Bone River by Artisanal Small-Scale Gold Mine Activities in Eastern Part of Gorontalo, Indonesia}, 
Water \textbf{10}, 1507 (2018).
%doi:10.3390/w10111507.

\bibitem{FerrazDias:13} 
D. Benedetti, E. Nunes, M. Sarmento, C. Porto, C. E. Iochims dos Santos, 
J. Ferraz Dias, J. da Silva,
% \textit{Genetic damage in soybean workers exposed to pesticides: Evaluation with the comet and buccal micronucleus cytome assays},
Mutation Research/Genetic Toxicology and Environmental Mutagenesis,
Volume 752, 28-33 (2013);
% https://doi.org/10.1016/j.mrgentox.2013.01.001.

%\bibitem{Mohamad:17}
%O. Mohamad, B. J. Sishc, J. Saha, A. Pompos, A. Rahimi, M. D. Story, 
%A. J. Davis, D. N. Kim, 
%%Carbon Ion Radiotherapy: A Review of Clinical Experiences and Preclinical Research, with an Emphasis on DNA Damage/Repair. 
%Cancers \textbf{9}, 66 (2017).

%%% otros métodos %%%

\bibitem{Abbas:08}
I. Abbas, C. Champion, B. Zarour, B. Lasri, J. Hanssen,
Phys. Med. Biol. 53, N41-N51 (2008).

\bibitem{Lekadir:09}
H. Lekadir, I. Abbas, C. Champion, O. Fojón, R. D. Rivarola, J. Hanssen,
Phys. Rev. A 79, 0627 10 (2009).

\bibitem{DalCappello:08}
C. Dal Cappello, P. A. Hervieux, I. Charpentier, F. Ruiz-Lopez,
Phys. Rev. A \textbf{78}, 042702 (2008).

\bibitem{Champion:10}
C. Champion, H. Lekadir, M. E. Galassi, O. Fojón, R. D. Rivarola, 
J. Hanssen,
Phys. Med. Biol. \textbf{55}, 6053--6067 (2010).

\bibitem{Galassi:00}
M. E. Galasssi, R. D. Rivarola, M. Beuve, G. H. Olivera, P. D. Fainstein, 
Phys. Rev. A \textbf{62}, 022701 (2000).

\bibitem{Fainstein:88}
P. D. Fainstein, V. H. Ponce, R. D. Rivarola,
J. Phys. B: At. Mol. Opt. Phys. \textbf{21}, 287 (1988).

\bibitem{Miraglia:08} 
J. E. Miraglia, M. S. Gravielle,
%Ionization of the He, Ne, Ar, Kr, and Xe isoelectronic series by proton impact. 
Phys Rev A \textbf{78}, 052705 (2008)

\bibitem{Miraglia:09} 
J. E. Miraglia, 
%Ionization of He, Ne, Ar, Kr, and Xe by proton impact: Single differential distributions. 
Phys. Rev. A \textbf{79}, 022708 (2009).

\bibitem{Montanari:17-iongasesnobles} 
C. C. Montanari, J. E. Miraglia,
%Ionization probabilities of Ne, Ar, Kr, and Xe by proton impact for different initial states and impact energies. 
Nucl. Instrum. Methods Phys. Res., B \textbf{407}, 236--243 (2017).

\bibitem{Galassi:12}
M. E. Galassi, C. Champion, P. F. Weck, R. D. Rivarola, O. Fojón, J. Hanssen,
Phys. Med. Biol. \textbf{57}, 2081--2099 (2012).

\bibitem{champion2012} 
C. Champion, M. E. Galassi, O. Foj\'{o}n, H. Lekadir, J. Hanssen, 
R. D. Rivarola, P. F. Weck, A. N. Agnihotri, S. Nandi, L. C. Tribedi,
%Ionization of RNA-uracil by highly charged carbon ions.
J. Phys.: Conf. Ser. \textbf{373}, 012004 (2012).

\bibitem{agnihotri2012}
A. N. Agnihotri, S. Kasthurirangan, S. Nandi, A.
Kumar, M. E. Galassi, R. D. Rivarola, O. Foj\'{o}n, C. Champion, J. Hanssen,
H. Lekadir, P. F. Weck, L. C. Tribedi.,
%Ionization of uracil in collisions with highly charged carbon and oxygen ions of energy 100 keV to 78 MeV. 
Phys. Rev. A \textbf{85}, 032711 (2012).

\bibitem{agnihotri2013}
A. N. Agnihotri, S. Kasthurirangan, S. Nandi, A. Kumar, C. Champion, 
H. Lekadir, J. Hanssen, P. F. Weck, M. E. Galassi, R. D. Rivarola, 
O. Fojon, C. Tribedi, 
%Absolute total ionization cross sections of uracil (C$_4$H$_4$N$_2$O$_2$) in collisions with MeV energy highly charged carbon, oxygen and fluorine ions
J. Phys. B \textbf{46}, 185201 (2013).

\bibitem{Ludde:16}
H. J. L\"udde, A. Achenbach, T. Kalkbrenner, H.-C. Jankowiak, T. Kirchner,
Eur. Phys. J. D \textbf{70}, 82 (2016).

\bibitem{Ludde:18}
H. J. L\"udde, M. Horbatsch, T. Kirchner,
Eur. Phys. J. B \textbf{91}, 99 (2018).

\bibitem{Ludde:19}
H. J. L\"udde, M. Horbatsch, T. Kirchner,
J. Phys. B \textbf{52}, 195203 (2019).

\bibitem{Ludde:20}
H. J. L\"udde, T. Kalkbrenner, M. Horbatsch, T. Kirchner,
Phys. Rev. A \textbf{101}, 062709 (2020).

\bibitem{salvat1995}
F. Salvat, J. M. Fern\'andez-Varea, W. Williamson,
Comput. Phys. Commun. \textbf{90}, 151--168 (1995)

\bibitem{Shah:81}
M. B. Shah, H. B. Gilbody,
J. Phys. B \textbf{14}, 2361--2377 (1981).

\bibitem{Shah:87}
M. B. Shah, D. S. Elliott, H. B. Gilbody,
J. Phys. B \textbf{20}, 3501--3514 (1987).

\bibitem{Thompson:95}
W. R. Thompson, M. B. Shah, H. B. Gilbody,
J. Phys. B \textbf{28}, 1321--1330 (1995).

\bibitem{Miraglia:19} 
J. E. Miraglia,
%Shell-to-shell ionization cross sections of antiprotons, H$^{+}$, He$^{2+},$ Be$^{4+},$ C$^{6+}$ and O$^{8+}$ on H, C, N, O, P, and S atoms,
%\href{https://arxiv.org/abs/1909.13682}{arXiv:1909.13682 [physics.atom-ph]}.
arXiv:1909.13682 [physics.atom-ph]

\bibitem{Denifl:11}
S. Denifl, T. D. M\"ark, P. Scheier,
The Role of Secondary Electrons in Radiation Damage. En \textit{Radiation 
Damage in Biomolecular Systems. Biological and Medical Physics, 
Biomedical Engineering.} Eds: G. García Gómez-Tejedor, M. Fuss. 
Springer, Dordrecht (2012) 

\bibitem{Champion:15}
C. Champion, M. A. Quinto, J. M. Monti, M. E. Galassi, P. F. Weck, 
O. Fojón, J. Hanssen, R. D. Rivarola, 
Phys. Med. Biol. \textbf{60}, 7805 (2015).

\bibitem{Quinto:17}
M. A. Quinto, J. M. Monti, P. F. Weck, O. Fojón, J. Hanssen, R. D. Rivarola, 
P. Senot, C. Champion,
%Monte Carlo simulation of proton track structure in biological matter. 
Eur. Phys. J. D \textbf{71}, 130 (2017). 
%https://doi.org/10.1140/epjd/e2017-70709-6

\bibitem{Acocer-Avila:19}
M. E. Alcocer-Ávila, M. A. Quinto, J. M. Monti, R. D. Rivarola, C. Champion,
%Proton transport modeling in a realistic biological environment by using 
%TILDA-V. 
Sci Rep \textbf{9}, 14030 (2019). 
%https://doi.org/10.1038/s41598-019-50270-5

\bibitem{Surdutovic:18} 
E. Surdutovich, A. V. Solov'yov, 
%Multiscale approach to the physics of radiation damage with ions. 
Eur. Phys. J. D \textbf{68}, 353 (2014).
%doi:10.1140/epjd/e2014-50004-0 

\bibitem{Abril:15} 
P. de Vera, I. Abril, R. Garcia-Molina, A. V. Solov'yov,
%Ionization of biomolecular targets by ion impact: input data for radiobiological applications. 
J. Phys.: Conf. Ser. \textbf{438}, 012015 (2013).

\bibitem{Rudd:92} 
M. E. Rudd, Y.-K. Kim,, D. H. Madison, T. J. Gay,
%Electron production in proton collisions with atoms and molecules: energy distributions. 
Rev. Mod. Phys. \textbf{64}, 441--490 (1992).

\bibitem{Sigmud:03}
P. Sigmund, A. Schinner,
%Anatomy of Barkas effect
Nucl. Instrum. Methods Phys. Res., B \textbf{}

\bibitem{Iriki:11}
Y. Iriki, Y. Kikuchi, M. Imai, A. Ito,
Phys. Rev. A \textbf{84}, 052719 (2011).

\bibitem{Sens:20}
Nicolas Sens, Tesis doctoral:
``Développement d’une méthode de type `velocity map imaging' pour la 
mesure de sections efficaces d’émission d’électrons par des molécules 
d’intérêt biologique en collision avec des ions''. 
Physique [physics]. Normandie Université, 2020. Français. 
NNT: 2020NORMC213. tel-03093903

\bibitem{Bhattacharjee:19} 
S. Bhattacharjee, C. Bagdia, M. R. Chowdhury, A. Mandal, J. M. Monti, 
R. D. Rivarola, and L. C. Tribedi, 
Phys. Rev. A \textbf{100}, 012703(2019).

\bibitem{Rahman:16}
M. A. Rahman , E. Krishnakumar,
J. Chem. Phys. \textbf{144}, 161102 (2016).

\bibitem{mozejko2003}
P. Mozejko, L. Sanche, 
%Cross section calculations for electron scattering from DNA and RNA bases.
Radiat Environ. Biophys \textbf{42}, 201 (2003).

\bibitem{tan2018}
H. Q. Tan, Z. Mi, A. A. Bettiol, 
%Simple and universal model for electron-impact ionization of complex biomolecules, 
Phys. Rev. E \textbf{97}, 032403 (2018)

\bibitem{Zein:21} 
S. A. Zein, M.-C. Bordage, Z. Francis, G. Macetti, A. Genoni, 
C. Dal Cappello, W.-G. Shin, S. Incerti,
%Electron transport in DNA bases: An extension of the Geant4-DNA Monte Carlo toolkit
Nucl. Instrum. Methods Phys. Res., B \textbf{488}, 70-82 (2021).

\bibitem{itoh2013} 
A. Itoh, Y. Iriki, M. Imai, C. Champion, R. D. Rivarola, 
%Cross sections for ionization of uracil by MeV-energy-proton impact, 
Phys. Rev. A \textbf{88}, 052711 (2013).

\bibitem{sarkadi2016}
L. Sarkadi, 
%Classical trajectory Monte Carlo model calculations for ionization of the uracil molecule by impact of heavy ions
J. Phys. B \textbf{49}, 185203 (2016)

\bibitem{wolff2014}
W. Wolff, H. Luna, L. Sigaud, A. C. Tavares, E. C. Montenegro,
%Absolute total and partial dissociative cross sections of pyrimidine at electron and proton intermediate impact velocities
J. Chem. Phys. \textbf{140}, 064309 (2014).

\bibitem{bug2017}
M. U. Bug, W. Y. Baek, H. Rabus, C. Villagrasa, S. Meylan, A. B. Rosenfeld,
%An electron-impact cross section data set (10 eV--1 keV) of DNA constituents based on consistent experimental data: A requisite for Monte Carlo simulations,
Rad. Phys. Chem. \textbf{130}, 459--479 (2017).

\bibitem{wang2016}
M. Wang, B. Rudek, D. Bennett, P. de Vera, M. Bug, T. Buhr, W. Y. Baek, 
G. Hilgers, H. Rabus, 
%Cross sections for ionization of tetrahydrofuran by protons at energies between 300 and 3000 keV
Phys. Rev. A \textbf{93}, 052711 (2016).

\bibitem{wolf2019}
W. Wolff, B. Rudek, L. A. da Silva, G. Hilgers, E. C. Montenegro, 
M. G. P. Homem,
%Absolute ionization and dissociation cross sections of tetrahydrofuran: Fragmentation--ion production mechanisms
J. Chem. Phys. \textbf{151}, 064304 (2019).

\bibitem{fuss2009}
M. Fuss, A. Muñoz, J. C. Oller, F. Blanco, D. Almeida, P. Limão-Vieira, 
T. P. D. Do, M. J. Brunger, G. Garc\'{i}a,
%Electron-scattering cross sections for collisions with tetrahydrofuran from 50 to 5000 eV
Phys. Rev. A \textbf{80}, 052709 (2009).

% H+ in water-------------------------------------
\bibitem{Luna2007}
H. Luna, A. L. F. de Barros, J. A. Wyer, S. W. J. Scully, J. Lecointre, 
P. M. Y. Garcia, G. M. Sigaud, A. C. F. Santos, V. Senthil, M. B. Shah, 
C. J. Latimer, and E. C. Montenegro,
Phys. Rev. A \textbf{75}, 042711 (2007).

\bibitem{Bolorizadeh86} 
M. A. Bolorizadeh and M. E. Rudd, 
Phys. Rev. A \textbf{33}, 888 (1986). 

\bibitem{H_Rudd85} 
M. E. Rudd, T. V. Goffe, R. D. DuBois, L. H. Toburen, 
Phys. Rev. A \textbf{31}, 492 (1985). 

\bibitem{toburen80} 
L. H. Toburen, W. E. Wilson and R. J. Popowich,
Radiat. Res. \textbf{82}, 27--44 (1980).

% He$^{+2}$ in water---------------------
\bibitem{Ohsawa05}
D. Ohsawa, Y. Sato, Y. Okada, V. P. Shevelko, and F. Soga
Phys. Rev. A \textbf{72}, 062710 (2005).

% He+2 in H2 N2 O2 CO CO2 CH4 N2O
\bibitem{He_Rudd85} 
M. E. Rudd, T. V. Goffe, and A. Itoh, 
Phys. Rev. A \textbf{32}, 2128 (1985).

% Li$^{+3}$ in water---------------------------
%\bibitem{Luna_Li_water} 
%H. Luna, W. Wolff, E. C. Montenegro, Andre C. Tavares, H. J. Ludde, 
%G. Schenk, M. Horbatsch, and T. Kirchner, 
%Phys. Rev. A \textbf{93}, 052705 (2016).  

%O$^{+8}$ in water -------------------------------
\bibitem{Bhattacharjee:16} 
S. Bhattacharjee, S. Biswas, C. Bagdia, M. Roychowdhury, S. Nandi, 
D. Misra, J. M. Monti, C. A. Tachino, R. D. Rivarola, C. Champion and 
L. C. Tribedi, J. 
Phys. B: At. Mol. Opt. Phys. \textbf{49},  065202 (2016).

% C$^{+6}$ in water--------------------------
\bibitem{DalCappello:09}
C. Dal Cappello, C. Champion, O. Boudrioua, H. Lekadir, Y. Sato, 
D. Ohsawa, 
Nuclear Instruments and Methods in Physics Research B 267 (2009) 781--790.

\bibitem{Bhattacharjee:17}
S. Bhattacharjee, S. Biswas, J. M. Monti, R. D. Rivarola, and 
L. C. Tribedi,
Phys. Rev A \textbf{96}, 052707 (2017).

\bibitem{Toburen:75} 
W. E. Wilson, L. H. Toburen,
%Electron emission from proton --hydrocarbon-molecule collisions at 0.3--2.0 MeV. 
Phys. Rev. A \textbf{11}, 1303 (1975).

\bibitem{Toburen:76} 
D. J. Lynch, L. H. Toburen, W. E. Wilson,
%Electron emission from methane, ammonia, monomethylamine, and dimethylamine by 0.25 to 2.0 MeV protons. 
J. Chem. Phys. \textbf{64}, 2616 (1976).

\bibitem{Lynch:76}
D. J. Lynch, L. H. Toburen, W. E. Wilson,
%Electron emission from methane, ammonia, monomethylamine, and dimethylamine by 0.25 to 2.0 MeV protons
J. Chem. Phys. \textbf{64}, 2616 (1976).

% scaling ion x electrones

\bibitem{deVera:20}
P. de Vera, I. Abril, R. Garcia-Molina,
% Excitation and ionisation cross-section in condensed-phase biomaterials 
%by electrons down to very low energy: application to liquid water and 
%genetic building blocks
Phys. Chem. Chem. Phys. (2021). DOI: 10.1039/d0cp04951d

% Z-scaling------------------------------------------------------
\bibitem{Janev:80}
R. K. Janev, L. P. Presnyakov, 
J. Phys. B \textbf{13}, 4233 (1980).

\bibitem{Dubois:13}
R. D. DuBois, E. C. Montenegro, G. M. Sigaud,
AIP Conference Proceeding \textbf{1525}, 679 (2013).

\bibitem{Montenegro:13} 
E. C. Montenegro, G. M. Sigaud, and R. D. DuBois, 
Phys. Rev. A \textbf{87} 012706 (2013).

% H+ in CH4
\bibitem{Luna2019} 
H. Luna, W. Wolff, and E. C. Montenegro, L. Sigaud, 
Phys. Rev. A \textbf{99}, 012709 (2019).

\bibitem{gamess}
M. W. Schmidt, K. K. Baldridge, J. A. Boatz, S. T. Elbert, M. S. Gordon, 
J. H. Jensen, S. Koseki, N. Matsunaga, K. A. Nguyen, S. J. Su, T. L. Windus, 
M. Dupuis, J. A. Montgomery 
J. Comput. Chem. \textbf{14}, 1347-1363 (1993).

\bibitem{Hush}
N. S. Hush, A. S. Cheung,  
%Ionization potentials and donor properties of nucleic acid bases and related compounds, 
Chem. Phys. Lett., \textbf{34}, 11 (1975).

\bibitem{Verkin}
B. I. Verkin, L. F. Sukodub, I. K. Yanson, 
%Ionization potentials of nitrogenous bases of of nucleic acids, 
Dokl. Akad. Nauk SSSR, \textbf{228}, 1452 (1976).

\bibitem{Dougherty}
D. Dougherty, E. S. Younathan, R. Voll, S. Abdulnur, S. P. McGlynn,
%Photoelectron spectroscopy of some biological molecules, 
J. Electron Spectrosc. Relat. Phenom., \textbf{13}, 379 (1978).

\bibitem{lee2003} 
J.-G. Lee, H. Y. Jeong, H. Lee, 
%Charges of Large Molecules Using Reassociation of Fragments. 
Bull. Korean Chem. Soc. \textbf{24}, 369 (2003).

\bibitem{rappe1991} 
A. K. Rappe, A. K., W. A. Goddard III,
J. Phys. Chem. \textbf{95}, 3358 (1991).


%%%%%%%%%%%%%%%%%%%%%%%%%%%%%%%%%%%%%%%%%%%%%%%%%%%%%%%%%%%%%%%%%%%%%%%%
%                          ATOMOS RELATIVISTAS
%%%%%%%%%%%%%%%%%%%%%%%%%%%%%%%%%%%%%%%%%%%%%%%%%%%%%%%%%%%%%%%%%%%%%%%%
%%%%%%%%%%%%%%%%%%%%%%%%%%%% Introducción %%%%%%%%%%%%%%%%%%%%%%%%%%%%%%

\bibitem{Creutzburg:19}
S. Creutzburg, E. Schmidt, P. Kutza, R. Loetzsch, I. Uschmann,
A. Undisz, M. Rettenmayr, F. Gala, G. Zollo, A. Boulle, A.
Debelle, E. Wendler, 
%Defects and mechanical properties in weakly damaged Si ion implanted GaAs,
Phys Rev. B \textbf{99}, 245205 (2019).

\bibitem{Jeynes:16}
J. Jeynes, J. L. Colaux, 
%Thin film depth profiling by ion beam analysis, 
Analyst \textbf{141}, 5944 (2016).

\bibitem{Mayer:20}
M. Mayer, S. M\"oller, M. Rubel et al., 
%Ion beam analysis of fusion plasma-facing materials and components: Facilities and research challenges, 
Nucl. Fusion \textbf{60}, 025001 (2020).

\bibitem{He:17}
B. He, X. J. Meng, Z. G. Wang et al., 
%Ab initio research of energy loss for energetic protons in solid-density Be, 
Phys. Plasmas \textbf{24}, 033110 (2017).

\bibitem{AlcocerAvila:19}
M. E. Alcocer-Ávila, M. A. Quinto, J. M. Monti et al.,
%Proton transport modeling in a realistic biological environment by using TILDA - V , 
Sci. Rep. \textbf{9}, 14030 (2019).

\bibitem{Vera:19}
P. de Vera, R. Garcia-Molina, I. Abril, 
%Simulation of the energy spectra of swift light ion beams after traversing cylin-drical targets: A consistent interpretation of experimental data relevant for hadron therapy, 
Eur. Phys. J. D \textbf{73}, 209 (2019).

\bibitem{Schardt:10} 
D. Schardt, T. Els\"asser, D. Schulz-Ertner, 
%Heavy-ion tumor therapy: Physical and radiobiological benefits
Rev. Mod. Phys. \textbf{82},  383-425 (2010).

\bibitem{Chu:01} 
W. K. Chu, J. W. Mayer, M. A. Nicolet,
\textit{Backscattering Spectrometry}
(Academic Press, New York, 1978).

\bibitem{Sigmund:06} 
P. Sigmund, 
\textit{Particle Penetration and Radiation Effects. General Aspects and 
Stopping of Swift Point Charges}.
(Springer Series in Solid-State Sciences, Springer, Berlin, 2006), Vol. 151.

\bibitem{iaea_codes} 
Códigos disponibles para cálculos de potencial de frenado se pueden 
encontrar en \href{https://www-nds.iaea.org/stopping/stopping\_prog.html}
{www-nds.iaea.org/stopping/stopping\_prog.html}

\bibitem{Paul:03}
H. Paul, A. Schinner,
At. Data Nucl. Data Tables  \textbf{85}, 377-452 (2003).

\bibitem{Diwan:15} 
P.K. Diwan, S. Kumar, 
Nucl. Instrum. Methods Phys. Res., B \textbf{359}, 78-84 (2015).

\bibitem{Damache:04} 
S. Damache, S. Ouichaoui, A. Belhout, A. Medouni, I. Toumert, 
Nucl. Instrum. Methods Phys. Res., B \textbf{225}, 449-463 (2004).

\bibitem{Damache:02} 
D. Moussa, S. Damache, S. Ouichaoui, 
Nucl. Instrum. Methods Phys. Res., B \textbf{268}, 1754-1758 (2010); 
\textbf{343},  44-47 (2015).

\bibitem{Montanari:09} 
C. C. Montanari, D. M. Mitnik, C. D. Archubi, J. E. Miraglia, 
Phys. Rev. A \textbf{79}, 032903 (2009); 
Phys. Rev. A \textbf{80}, 012901 (2009).

\bibitem{Klapisch:67}
M. Klapisch,
Comput. Rend. Acad. Sci. \textbf{265}, 914 (1967).

%\bibitem{Koenig:72}
%E. Koenig,
%Physica (Utrecht) \textbf{62}, 93 (1972).

\bibitem{Klapisch:77}
M. Klapisch, B. Fraenkel, J.L. Schwob, J. Oreg,
J. Opt. Soc. Am. \textbf{62}, 148 (1977).

\bibitem{BarShalom:01}
A. Bar-Shalom, M. Klapisch, J. Oreg,
J. Quant. Spectrosc. Radiat. Transf. \textbf{71}, 169--188 (2001).

\bibitem{Montanari:17} 
C.C. Montanari, J.E. Miraglia, 
Phys. Rev. A \textbf{96}, 012707 (2017).

\bibitem{Mermin:70} 
N.D. Mermin, 
Phys. Rev. B \textbf{1}, 2362 (1970).

\bibitem{Montanari:13}
C. C. Montanari, J. E. Miraglia,
Adv. Quantum Chem. \textbf{65}, 165 (2013).

\bibitem{Echenique:81} 
P. M. Echenique, R. M. Nieminen, R. H. Ritchie, 
Sol. State Comm. \textbf{37}, 779-781 (1981).

\bibitem{Nagy:89} 
I. Nagy, A. Arnau, P. M. Echenique, E. Zaremba, 
Phys. Rev. B \textbf{40}, 11983 (1989).

\bibitem{suppression} 
C. C. Montanari, J. E. Miraglia, and N. R. Arista, 
Phys. Rev. A \textbf{62}, 052902 (2000).

\bibitem{Williams:95}
G. Williams en 
\href{http://xdb.lbl.gov/Section1/Sec\_1-1.html}{xdb.lbl.gov/Section1/Sec\_1-1.html}

\bibitem{Desclaux:73}
J. P. Desclaux,
Atomic Data and Nuclear Data Tables \textbf{12}, 311 (1973).

\bibitem{Werner:09}
W. S. M. Werner, K. Glantschnig, and C. Ambrosch--Draxl,
J. Phys. Chem. Ref. Data \textbf{38} 1013-1092 (2009).

\bibitem{Lynch:75}
D. W. Lynch, C. G. Olson, J. H. Weaver,
Phys. Rev. B \textbf{11}, 3617 (1975).

\bibitem{Isaacson:75}
D. Isaacson,
\textit{Compilation of rs values}, New York University Rep. No. 02698
(National Auxiliary Publication Service, NY 1975).

\bibitem{Romaniello:06}
P. Romaniello, P. L. de Boeij, F. Carbone and D. van der Marel,
Phys. Rev. B \textbf{73}, 075115 (2006).

\bibitem{Strange:99}
N. Strange, A. Svane, W. M. Temmerman, Z. Szotek, H. Winter,
%Understanding the valencyof rare earths from first-principles theory
Nature \textbf{399}, 756--758 (1999).

\bibitem{Bonnelle:15}
C. Bonnelle, N. Spector
``Rare-Earths and Actinides in High Energy Spectroscopy'' 
Progress in Theoretical Chemistry and Physics. Springer (2015).

\bibitem{Roth:17}
D. Roth, B. Bruckner, M. V. Moro, S. Gruber, D. Goebl, J. I. Juaristi,
M. Alducin, R. Steinberger, J. Duchoslav, \mbox{D. Primetzhofer}, P. Bauer,
%Electronic Stopping of Slow Protons in Transition and Rare Earth Metals: 
%Breakdown of the Free Electron Gas Concept
Phys. Rev. Lett. \textbf{118}, 103401 (2017).

\bibitem{Lindhard:53} 
J. Lindhard, M. Scharff,  
Mat. Fys. Medd. Dan. Vid. Selsk  \textbf{27}, 1 (1953).

\bibitem{Chu:72} 
W. K. Chu, D. Powers, 
Rev. Lett. A \textbf{40}, 23 (1972).

\bibitem{Sirotinin:84}
E. I. Sirotinin, A. F. Tulinov, V. A. Khodyrev, V. N. Mizgulin, 
Nucl. Instrum. Methods Phys. Res. B \textbf{4},3 37 (1984).

\bibitem{Grande:01} 
G. Schiwietz, P. L. Grande, 
Nucl. Instrum. Methods Phys. Res., B \textbf{175-177}, 125-131 (2001); 
Código CasP, disponible en \href{https://www.casp-program.org}{www.casp-program.org}

\bibitem{casp52} 
G. Schiwietz, P. L. Grande,
Nucl. Instrum. Methods Phys. Res., B \textbf{273}, 1-5 (2012); 
P.L. Grande, G. Schiwietz, 
Phys. Rev. A \textbf{58}, 3796 (1998).

\bibitem{DPASS20} 
A. Schinner, P. Sigmund, 
Nucl. Instrum. Methods Phys. Res., B \textbf{460}, 19 (2019); 
P.Sigmund, A. Schinner, 
Eur. Phys. J. D \textbf{12}, 425 (2000). 
Código DPASS, disponible en \href{https://www.sdu.dk/en/DPASS/}{www.sdu.dk/en/DPASS/}

\bibitem{Ziegler01} 
J.F. Ziegler, J.P. Biersack, M. D. Ziegler, 
\textit{SRIM, The Stopping and Range of Ions in Matter}, 
(SRIM Co. Maryland, USA, 2008); 
SRIM2013, Computer Program and Manual. Disponible en \href{https://www.srim.org}{www.srim.org}.

\bibitem{ICRU49} 
ICRU report 49, \textit{Stopping Powers and Ranges for Protons and Alpha Particles},
International Commission on Radiation Units and Measurements (1993).

\bibitem{Moro:20}
M. V. Moro, P. Bauer, D. Primetzhofer,
%Experimental electronic stopping cross section of transition metals for 
%light ions: Systematics around the stopping maximum
Phys. Rev. A \textbf{102}, 022808 (2020).

\bibitem{iaea}
C. C. Montanari, H. Paul,
\href{https://www-nds.iaea.org/stopping/}{www-nds.iaea.org/stopping/}.

\bibitem{Shiomi:96}
N. Shiomi-Tsuda, N.Sakamoto, H. Ogawa, 
Nucl. Instrum. Methods Phys. Res. B \textbf{115}, 88 (1996).

\bibitem{Shiomi:94}
N. Shiomi-Tsuda, N. Sakamoto, R. Ishiwari, 
Nucl. Instrum. Methods Phys. Res. B \textbf{93}, 391 (1994).

\bibitem{Bichsel:92}
H. Bichsel, T. Hiraoka, 
Nucl. Instrum. Methods Phys. Res. B \textbf{66}, 345 (1992).

\bibitem{Ogino:88}
K. Ogino, T. Kiyosawa, T. Kiuchi, 
Nucl. Instrum. Methods Phys. Res. B \textbf{33}, 155 (1988).

\bibitem{Krist:83}
Th. Krist, P. Mertens,
Nucl. Instrum. Methods Phys. Res. \textbf{218}, 790 (1983),
Nucl. Instrum. Methods Phys. Res. \textbf{218}, 821 (1983)

\bibitem{Celedon:15}
C. Celedon, E. A Sanchez, L. Salazar Alarcón, J. Guimpel, A. Cortés,
P. Vargas, N. R. Arista,
Nucl. Instrum. Methods Phys. Res. B \textbf{360}, 103 (2015).

\bibitem{Goebl:13}
D. G\"oebl, D. Roth, P. Bauer, 
Phys. Rev. A \textbf{87}, 062903 (2013).

\bibitem{Primetzhofer:12}
D. Primetzhofer
Phys. Rev. B \textbf{86}, 094102 (2012).

\bibitem{Sakamoto:91}
N. Sakamoto, H. Ogawa, M. Mannami, K. Kimura, Y. Susuki, M. Hasegawa,
I. Hatayama, T. Noro, H. Ikegami, 
Rad. Effects Defects in Solids \textbf{117}, 193 (1991).

\bibitem{Ishiwari:79}
R. Ishiwari, N. Shiomi, N. Sakamoto, 
Phys. Lett. A \textbf{75}, 112 (1979).

\bibitem{Ishiwari:74}
R. Ishiwari, N. Shiomi, S. Shirai, Y. Uemura, 
Phys. Lett. A \textbf{48}, 96 (1974).


%=======================================================================
\begin{comment}

\bibitem{Montanari:11}
C.C. Montanari, D. M. Mitnik, J. E. Miraglia,
Rad. Eff. Defects Sol. \textbf{166}, 338 (2011).

\bibitem{Oswald:18}
M. Oswal, Sunil Kumar, Udai Singh, G. Singhe, K. P. Singh, D. Mehta,
D. Mitnik, C. C. Montanari, T.Nandi,
Nucl. Instrum. Methods Phys. Res., B \textbf{416}, 110 (2018).

\bibitem{Abril} 
I. Abril, M. Behar, R. Garcia Molina, R. C. Fadanelli, L. C. C. M. Nagamine, 
P. L. Grande, L. Sch\"unemann, C. D. Denton, N. R. Arista, E. B. Saitovich,
Eur. Phys. J. D \text{54}, 65-70 (2009).

\bibitem{Behar} 
M. Behar, R. C. Fadanelli, I. Abril, R. Garcia-Molina, C. D. Denton, 
L. C. C. M. Nagamine, N. R. Arista, 
Phys. Rev. A \textbf{80},  062901 (2009).

\bibitem{Primetzhofer} 
D. Primetzhofer, 
Nucl. Instrum. Methods Phys. Res., B \textbf{320}, 100-103 (2014).

\bibitem{Roth}
D. Roth, B. Bruckner, G. Undeutsch, V. Paneta, A. I. Mardare, 
C. L. McGahan, M. Dosmailov, J. I. Juaristi, M. Alducin, 
J. D. Pedarnig, R. F. Haglund, Jr., D. Primetzhofer, P. Bauer
%Electronic Stopping of Slow Protons in Oxides: Scaling Properties
Phys. Rev. Lett. \textbf{119}, 163401 (2017).

\bibitem{zenodo} 
C.C. Montanari \textit{et al.} DOI: 10.5281/zenodo.3678785 

%=======================================================================
% Why Hafnium is so important:
\bibitem{Choi} 
J. H. Choi, Y. Mao, J. P. Chang, 
Mat. Sci. Eng. R \textbf{72}, 97-136 (2011).

\bibitem{Robertson} 
J. Robertson, R. M. Wallace, 
Mat. Sci. Eng. R \textbf{88}, 1-41 (2015).

%=======================================================================
%Importance of Stopping Power for Ion Beam Analysis
\bibitem{Alfassi01} 
Z. B. Alfassi,
\textit{Non-destructive elemental analysis}
(Blackwell Publishing, Oxford, 2001).

\bibitem{Tesmer01} 
J. R. Tesmer, M. Nastasi, J. C. Barbour, C. J. Maggiore, J. W. Mayer,
\textit{Handbook of Modern Ion Beam Material Analysis}.
(Materials Research Society, Pittsburgh, 1995).

\bibitem{mondim17} 
C. C. Montanari, P. Dimitriou, 
Nucl. Instrum. Methods Phys. Res., B \textbf{408},  50-55 (2017).

%=======================================================================
% Experimental
\bibitem{Miranda01} 
P. A. Miranda, A. Sep\'ulveda, J. R. Morales, T. Rodriguez, E. Burgos, 
H. Fern\'andez,
Nucl. Instrum. Methods Phys. Res., B \textbf{318}, 292-296  (2014).

\bibitem{Lebow} 
Lebow Company. 5960 Mandarin Ave. Goleta CA, 93117, USA.

%=======================================================================
% Transmission Method
\bibitem{Sun01} 
G. Sun, M. D\"{o}belli, A.M. M\"{u}ller, M. Stocker, M. Suter, L. Wacker, 
Nucl. Instrum. Methods Phys. Res., B \textbf{256}, 586-590 (2007).

\bibitem{Raisanen01} 
J. Raisanen, U. Watjen, A.J.M. Plompen, F. Munnik, 
Nucl. Instrum. Methods Phys. Res., B \textbf{118}, 1-6  (1996).

\bibitem{Schulz01} 
F. Schulz, J. Shchuchinzky, 
Nucl. Instrum. Methods Phys. Res., B \textbf{12},  90-94 (1985).

\bibitem{Chilton} 
A.B. Chilton, J.N. Cooper, J.C. Harris, 
Phys. Rev. \textbf{93}, 413-418  (1954).

\bibitem{Rajatora} 
M. Rajatora, K. Vakevainen, T. Ahlgre, E. Rauhala, J. Raisanen, K. Rakennus, 
Nucl. Instrum. Methods Phys. Res., B \textbf{119}, 457-462 (1996).



\bibitem{Gu:08}
M. F. Gu,
%The flexible atomic code. 
Can. J. Phys. \textbf{86}, 675--689 (2008).

\bibitem{Grant:80}
I. P. Grant, B. J. McKenzie, P. H. Norrington, D. F. Mayers, N. C. Pyper,
%An atomic multiconfigurational Dirac--Fock package
Comput. Phys. Comm. \textbf{21}, 207--231 (1980).




\bibitem{Eissner:69}
W. Eissner, H. Nussbaumer,
J. Phys. B \textbf{2}, 1028--1043 (1969).

\bibitem{Bautista:08}
M. Bautista,
J. Phys. B \textbf{41}, 065701 (2008).


\end{comment}

%%%%%%%%%%%%%%%%%%%%%%%%%%%%%%%%%%%%%%%%%%%%%%%%%%%%%%%%%%%%%%%%%%%%%%%%
%                          R-MATRIX
%%%%%%%%%%%%%%%%%%%%%%%%%%%%%%%%%%%%%%%%%%%%%%%%%%%%%%%%%%%%%%%%%%%%%%%%

%%% intro %%%
%\bibitem{Bray:92}
%I. Bray, A.T. Stelbovics, 
%Phys. Rev. Lett. \textbf{69}, 53 (1992).

\bibitem{Bartschat:04}
K. Bartschat, O. Zatsarinny, I. Bray, D. V. Fursa, A. T. Stelbovics,
%On the convergence of close-coupling results for low-energy electron 
%scattering from magnesium
J. Phys. B \textbf{37}, 2617 (2004).

\bibitem{Zatsarinny:16}
O. Zatsarinny, K. Bartschat, D. V. Fursa, I. Bray,
%Calculations for electron-impact excitation and ionization of beryllium
J. Phys. B \textbf{49}, 235701 (2016).

\bibitem{Be_Ballance:03}
C. P. Ballance, D. C. Griffin, J. Colgan, S. D. Loch, M. S. Pindzola,
%Electron-impact excitation of beryllium and its ions
Phys. Rev. A \textbf{68}, 062705 (2003).

\bibitem{Ballance:03}
C.P. Ballance, N.R. Badnell, D.C. Griffin, S.D. Loch, D.M. Mitnik, 
%The effects of coupling to the target continuum on the electron-impact excitation of Li+, 
J. Phys. B \textbf{36}, 235--246 (2003).

\bibitem{Badnell:03}
N.R. Badnell, D.C. Griffin, D.M. Mitnik, 
%Electron-impact excitation of B+ using the R-matrix with pseudo-states method, 
J. Phys. B \textbf{36}, 1337--1350 (2003).

\bibitem{Mitnik:03}
D.M. Mitnik, D.C. Griffin, C.P. Ballance, N.R. Badnell, 
%An R-matrix with pseudo-states calculation of electron-impact excitation in C2+, 
J. Phys. B \textbf{36}, 717--730 (2003).

\bibitem{Ikeda:07}
K. Ikeda \textit{et al.},
%``ITER Physics Basis'' 
Nucl. Fusion \textbf{47}, 6 (2007).

\bibitem{Rubel:08}
M.J. Rubel \textit{et al.},
J. Phys.: Conf. Ser. \textbf{100}, 062028 (2008).

\bibitem{Deliyannis:00}
C. P. Deliyannis, 
``Lithium and Beryllium as diagnostics of stellar interior physical 
processes.'' En Stellar Clusters and Associations: Convection, Rotation, 
and Dynamos, vol. 198, p. 235 (2000).

\bibitem{Bartschat:97}
K. Bartschat, P. G. Burke, M. P. Scott, 
J. Phys. B \textbf{30}, 5915 (1997).

\bibitem{Colgan:03}
J. Colgan, S. D. Loch, M. S. Pindzola, C. P. Ballance, D. C. Griffin, 
Phys. Rev. A \textbf{68}, 032712 (2003).

\bibitem{Fursa:97}
D. V. Fursa, I. Bray, 
J. Phys. B \textbf{30}, 5895 (1997).

\bibitem{Bray:15}
I. Bray, D. V. Fursa, 
Phys. Conf. Ser. \textbf{576}, 012001 (2015).

\bibitem{Blanco:17}
F. Blanco, F. Ferreira da Silva, P. Lim\~ao-Vieira, G. García, 
Plasma Sources. Sci. Technol. \textbf{26}, 085004 (2017).

\bibitem{Burke:75}
P.G. Burke, W.D. Robb, 
Adv. At. Mol. Phys. \textbf{11}, 143 (1975).

\bibitem{Griffin:07}
D. C. Griffin, M. S. Pindzola,
Adv. At. Mol. Opt. Phys. \textbf{54}, 203 (2007).

\bibitem{QUB-Badnell}
N. R. Badnell,
\url{http://amdpp.phys.strath.ac.uk/tamoc/}.

%\bibitem{Berrington:78}
%K. A. Berrington, P. G. Burke, M. Le Dourneuf, W. D.Robb, K. T. Taylor,
%Vo Ky Lan,
%%A new version of the general program to calculate atomic continuum 
%%processes using the r-matrix method
%Comput. Phys. Commun. \textbf{14} 367-412 (1978).

%\bibitem{Scott:82}
%N. S. Scott, K. T. Taylor,
%%A general program to calculate atomic continuum processes incorporating 
%%model potentials and the Breit-Pauli Hamiltonian within the R-matrix 
%%method
%Comput. Phys. Comm. \textbf{25}, 347--387 (1982).

\bibitem{Mitnik:99}
D. M. Mitnik, M. S. Pindzola, D. C. Griffin, N. R. Badnell, 
J. Phys. B \textbf{32}, L479 (1999).

\bibitem{Mitnik:01}
D. M. Mitnik, D. C. Griffin, N. R. Badnell, 
J. Phys. B \textbf{34}, 4455 (2001).

\bibitem{Ballance:04}
C. P. Ballance, D. C. Griffin, 
J. Phys. B \textbf{37}, 2943 (2004).

\bibitem{Burke:92}
V. M. Burke, P. G. Burke, N. S. Scott, 
Comput. Phys. Commun. \textbf{69}, 76 (1992).

\bibitem{Griffin:98}
D. C. Griffin, N. R. Badnell, M. S. Pindzola, 
%R-matrix electron-impact excitation cross sections in intermediate coupling: 
%an MQDT transformation approach, 
J. Phys. B \textbf{31} 3713--3727 (1998).
%DOI:10.1088/0953-4075/31/16/022

\bibitem{Seaton:85}
M. J. Seaton, 
J. Phys. B \textbf{18}, 2111 (1985).

\bibitem{FernandezMenchero:20}
L. Fernández-Menchero, A. C. Conroy, C. P. Ballance, N. R. Badnell, 
D. M. Mitnik, T. W. Gorczyca, M. J.Seaton,
Comput. Phys. Commun. \textbf{256}, 107489 (2020).

\bibitem{Badnell:11} 
N. R. Badnell, 
Comput. Phys. Commun. \textbf{7}, 1528 (2011).

\bibitem{Burgess:89}
A. Burgess, H. E. Mason, J. A. Tully
Astron. Astrophys. \textbf{217}, 319--328 (1989).

\bibitem{Muller:83}
W. M\"uller, J. Flesch, W. Meyer,
J. Chem. Phys. \textbf{80}, 3297 (1984).

\bibitem{Loughlin:88}
C. Loughlin, G. A. Victor,
Adv. At. Mol. Phys. \textbf{25}, 163 (1988).

\bibitem{Seaton:72}
M. J. Seaton, P. M. H. Wilson,
J. Phys. B \textbf{5}, L1-3 (1972).

\bibitem{Loughlin:73}
C. Loughlin, G. A. Victor,
\textit{Atomic Physics} vol. 3. 
Eds: S. J. Smith, G. K. Walters.
Springer, New York: Plenum (1973).

\bibitem{Migdalek:78}
J. Migdalek, W. E. Baylis,
J. Phys. B \textbf{11}, 17 (1978).

\bibitem{Norcross:76}
D. W. Norcross, M. J. Seaton,
J. Phys. B \textbf{9}, 2983 (1976).

\bibitem{Dipti:19}
F. Dipti \textit{et al.}, %T. Das, K. Bartschat, I. Bray, D.V. Fursa, O. Zatsarinny, C. Ballance, H.-K. Chung, Y. Ralchenko,
‎At. Data Nucl. Data Tables \textbf{127-128}, 1--21 (2019).

\bibitem{Powell:64}
M. J. D. Powell, 
%"An efficient method for finding the minimum of a function of several variables without calculating derivatives". 
Comput. J. \textbf{7}, 155--162 (1964). 
%doi:10.1093/comjnl/7.2.155. 

\bibitem{NumRec:07}
W. H. Press, S. A. Teukolsky, W. T. Vetterling, B. P. Flannery, 
``Numerical Recipes: The Art of Scientific Computing'' 3ra Ed. (2007).

%%% bayes opt - gaussian process %%%

\bibitem{Gelman:13}
A. Gelman, J. B. Carlin, H. S. Stern, D. B. Dunson, A. Vehtari, D. B. Rubin,
\textit{Bayesian Data Analysis}, 3ra Edición. 
Chapman \& Hall/CRC Texts in Statistical Science (2013).

\bibitem{Barber:12}
D. Barber, 
``Bayesian Reasoning and Machine Learning'', 
Cambridge University Press (2012).

\bibitem{Bergstra:11}
J. S. Bergstra, R. Bardenet, Y. Bengio, B. Kégl, 
``Algorithms for Hyper-Parameter Optimization'' in Advances in Neural 
Information Processing Systems 24. Eds: J. Shawe-Taylor, R. S. Zemel, 
P. L. Bartlett, F. Pereira, K. Q. Weinberger, Eds. Curran Associates, 
Inc., 2011, pp. 2546--2554. 

\bibitem{Rasmussen:06}
C. E. Rasmussen, C. K. I. Williams, 
\textit{Gaussian Processes for Machine Learning}. MIT Press (2006).

\bibitem{Murphy:12}
K. P. Murphy, 
``Machine learning: A probabilistic perspective''. 
MIT Press (2012).

\bibitem{GPyOpt}
The GPyOpt authors,
\textit{GPyOpt: A Bayesian Optimization framework in python},
\url{http://github.com/SheffieldML/GPyOpt} (2016).

\bibitem{NIST}
A. Kramida \textit{et al.},
NIST Atomic Spectra Database (version 5.6.1) 

\bibitem{Dalgarno:62}
A. Dalgarno,
Ada. Phys. \textbf{11}, 281--315 (1962).

\bibitem{Sitz:71}
P. Sitz, 
J . Chem. Phys. \textbf{55}, 1481 (1971).

\bibitem{Buttle:67}
P. J. A. Buttle, 
Phys. Rev. \textbf{160}, 719 (1967).

\bibitem{deFreitas:13}
N. de Freitas, 
Machine Learning lectures, UBC, Computer Science Deparment
\url{https://www.cs.ubc.ca/nando/540-2013/lectures.html} (2013).

\end{thebibliography}

