\appendix

%%%%%%%%%%%%%%%%%%%%%%%%%%%%%%%%%%%%%%%%%%%%%%%%%%%%%%%%%%%%%%%%%%%%%%%
\chapter{La ecuación normal}
\label{app:ecnormal}

El procedimiento de ajuste dado por la ecuación normal consiste en 
minimizar la función
\begin{equation}
 \beta(x_i) = y(x_i) - f(x_i,\lambda_1,\lambda_2,\cdots,\lambda_m)\,,
\end{equation}
que es la diferencia entre la curva a ser ajustada, $y(r)$, y la función
analítica implementada para tal fin, $f(r)$. En el esquema de la inversión
depurada, $y(r) = Z_{nl}^{\mathrm{HF}}(r)$ corresponde a la carga invertida,
y $f(r) = Z_{nl}^{\mathrm{DIM}}(r)$ corresponde a la forma analítica que 
se ha fijado para la carga (la ecuación \ref{eq:atomzDIM} para átomos y 
la ecuación \ref{eq:molzDIM} para moléculas). Con el fin de minimizar
$\beta(x_i)$ respecto a los $m$ parámetros $\lambda_j$ que determinan 
$f$, definimos los elementos de matriz $A_{ij}$,
\begin{equation}
  A_{ij} \equiv \frac{d\beta(x_i)}{d\lambda_j} =
 \frac{df(x_i,\lambda_1,\lambda_2,\cdots, \lambda_m)}{d\lambda_j}
\end{equation}
Así, obtenemos el sistema de ecuaciones
\begin{equation}
 \left[
 \begin{array}{c}
  d\beta(x_1) \\
  d\beta(x_2) \\
  \vdots \\
  d\beta(x_n) \\
 \end{array}
 \right] =
 \left[
 \begin{array}{cccc}
  A_{11} & A_{12} & \cdots & A_{1m} \\
  A_{21} & A_{22} & \cdots & A_{2m} \\
  \vdots & \vdots & \ddots & \vdots \\
  A_{n1} & A_{n2} & \cdots & A_{nm} \\
 \end{array}
 \right]
 \left[
 \begin{array}{c}
 d\lambda_1 \\
 d\lambda_2 \\
 \vdots \\
 d\lambda_m \\
 \end{array}
 \right] \,.
\end{equation}
Multiplicando ambos lados de la ecuación por la matrix traspuesta $[A]^{T}$,
\begin{equation}
  \left[ A \right]^T \left[ A \right]\left[ d\lambda \right] =
  \left[ A \right]^T \left[ d\beta \right]\,,
\end{equation}
somos capaces de obtener un sistema de ecuaciones que se puede resolver
con rutinas numéricas estándar. Así, la solución $[d\lambda]$ nos permite 
obtener los mejores parámetros que minimizan $[\beta]$.

