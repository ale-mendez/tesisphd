\appendix

%%%%%%%%%%%%%%%%%%%%%%%%%%%%%%%%%%%%%%%%%%%%%%%%%%%%%%%%%%%%%%%%%%%%%%%%
\chapter{Ecuación normal}
%%%%%%%%%%%%%%%%%%%%%%%%%%%%%%%%%%%%%%%%%%%%%%%%%%%%%%%%%%%%%%%%%%%%%%%%
\label{app:ecnormal}

Las semillas iniciales dadas por la ecuación normal se obtienen al 
minimizar la función
\begin{equation}
 \beta(x_i) = y(x_i) - f(x_i,\lambda_1,\lambda_2,\cdots,\lambda_m)\,,
\end{equation}
que es la diferencia entre la curva a ser ajustada, $y(r)$, y la función
analítica implementada para tal fin, $f(r)$. En el esquema de la inversión
depurada, $y(r) = Z_{nl}^{\mathrm{HF}}(r)$ corresponde a la carga invertida,
y $f(r) = Z_{nl}^{\mathrm{DIM}}(r)$ corresponde a la forma analítica que 
se ha fijado para la carga (la ecuación \ref{eq:atomzDIM} para átomos y 
la ecuación \ref{eq:molzDIM} para moléculas). Con el fin de minimizar
$\beta(x_i)$ respecto a los $m$ parámetros $\lambda_j$ que determinan 
$f$, definimos los elementos de matriz $A_{ij}$,
\begin{equation}
  A_{ij} \equiv \frac{d\beta(x_i)}{d\lambda_j} =
 \frac{df(x_i,\lambda_1,\lambda_2,\cdots, \lambda_m)}{d\lambda_j}
\end{equation}
Así, obtenemos el sistema de ecuaciones
\begin{equation}
 \left[
 \begin{array}{c}
  d\beta(x_1) \\
  d\beta(x_2) \\
  \vdots \\
  d\beta(x_n) \\
 \end{array}
 \right] =
 \left[
 \begin{array}{cccc}
  A_{11} & A_{12} & \cdots & A_{1m} \\
  A_{21} & A_{22} & \cdots & A_{2m} \\
  \vdots & \vdots & \ddots & \vdots \\
  A_{n1} & A_{n2} & \cdots & A_{nm} \\
 \end{array}
 \right]
 \left[
 \begin{array}{c}
 d\lambda_1 \\
 d\lambda_2 \\
 \vdots \\
 d\lambda_m \\
 \end{array}
 \right] \,.
\end{equation}
Multiplicando ambos lados de la ecuación por la matrix traspuesta $[A]^{T}$,
\begin{equation}
  \left[ A \right]^T \left[ A \right]\left[ d\lambda \right] =
  \left[ A \right]^T \left[ d\beta \right]\,,
\end{equation}
se obtiene un sistema de ecuaciones que se resuelve con rutinas numéricas 
estándar. Así, la solución $[d\lambda]$ permite obtener los parámetros 
iniciales que minimizan $[\beta]$.

%%%%%%%%%%%%%%%%%%%%%%%%%%%%%%%%%%%%%%%%%%%%%%%%%%%%%%%%%%%%%%%%%%%%%%%%
\chapter{Método de R-Matrix}
%%%%%%%%%%%%%%%%%%%%%%%%%%%%%%%%%%%%%%%%%%%%%%%%%%%%%%%%%%%%%%%%%%%%%%%%
\label{app:rmatrix}

En la región interna, la función de onda total $\Psi_E$ del sistema 
electrónico $(N+1)$ para cualquier valor de energía $E$ se expande en 
términos de las funciones de base de R-Matrix $\Psi_k$ como
\begin{equation}
\Psi_E=\sum_k A_{Ek}\Psi_k\,.
\label{eq:RM-wavefn}
\end{equation}
La funciones de base están dadas por
\begin{equation}
\Psi_k=\mathcal{A}\sum_{ij}a_{ijk}\bar{\Phi}_i u_j+\sum_j b_{jk}\phi_j\,,
\label{eq:RM-basisfn}
\end{equation}
donde las funciones $u_j$ constituyen un conjunto finito de orbitales de 
base, que se usan para representar las funciones de onda del continuo,
$\bar{\Phi}_i$ son las funciones de onda del blanco con $N$ electrones,
$\phi_j$ son las funciones de onda del sistema $(N+1)$ que se construyen 
con los orbitales del blanco, y el operador $\mathcal{A}$ antisimetriza 
la coordenada del electrón dispersado con la coordenada del electrón del 
blanco $N$-ésimo. Los coeficientes $a_{ijk}$ y $b_{jk}$ se determinan 
diagonalizando el Hamiltoniano $H^{N+1}$ del sistema de electrones 
$(N+1)$ en la región interna,
\begin{equation}
\langle\Psi_k\left|H^{N+1}\right|\Psi_k'\rangle=E_k^{N+1}\delta_{kk'}\,.
\label{eq:RM-N+1Hamilt}
\end{equation}
Las funciones de base, que constituyen las funciones de onda del sistema 
de $N$ electrones, se pueden separar de forma usual, en sus partes 
radiales, angulares y de espín.
El radio de la región interna $a$ se elije de manera tal que los 
orbitales radiales $P_{nl}$ están contenidos totalmente; esto es, 
$P_{nl}\simeq 0$ para $r>a$. Para cada valor de momento angular $l$, los 
orbitales de la base del continuo $u_j$ se determinan a partir de las 
soluciones de la ecuación
\begin{equation}
\left[-\frac{1}{2}\frac{d^2}{dr^2}+\frac{l(l+1)}{2r^2}+V(r)
-\frac{k_j^2}{2}\right]u_j(r)=-\sum_i\lambda_{ji}P_{n_il}(r)\,,
\label{eq:RM-difeq-uj}
\end{equation}
imponiendo las condiciones de borde 
\begin{equation}
u_j(0)=0\,,\qquad\frac{a}{u_j(a)}\frac{du_j}{dr}\bigg|_{r=a}=b\,,
\end{equation}
y la condición de ortonormalidad
\begin{equation}
\int_0^au_iu_j\,dr=\delta_{ij}\,.
\end{equation}
El potencial $V(r)$ se define de forma conveniente para representar la
distribución de carga del blanco atómico y los multiplicadores de 
Lagrange $\lambda_{ji}$ se utilizan para forzar la ortonormalidad de las
bases del contínuo con todos los orbitales ligados $P_{n_il}(r)$ con el
mismo momento angular $l$. En general, el valor de la segunda condición 
de borde $b$ es arbitraria y usualmente se asume igual a cero.

Definiendo la variable
\begin{equation}
w_{ik}=\sum_j a_{ijk}u_j\,,
\end{equation}
la función de onda radial del electrón dispersado en el canal $i$ con 
energía $E$ resulta
\begin{equation}
y_i=\sum_kA_{Ek}w_{ik}\,.
\end{equation}
Se puede demostrar~\cite{Burke:75}, que los coeficientes $A_{Ek}$ 
están dados por
\begin{equation}
A_{Ek}=\frac{1}{2a\left(E_k^{N+1}-E\right)}\sum_jw_{jk}(a)
\left(a\frac{dy_j}{dr}-by_j\right)_{r=a}\,.
\end{equation}
Así, la función de onda radial dispersada en el límite entre las 
regiones interna y externa se define como
\begin{equation}
y_i(a)=\sum_j R_{ij}\left(a\frac{dy_j}{dr}-by_j\right)_{r=a}\,,
\end{equation}
donde $R_{ij}$ es un elemento de la matrix $R$ dado por
\begin{equation}
R_{ij} = \frac{1}{2a}\sum_k\frac{w_{ik}(a)w_{jk}(a)}{E_k^{N+1}-E}\,.
\label{eq:RM-elements}
\end{equation}
Las amplitudes de superficie $w_{ik}(a)$ y los polos $E_k^{N+1}$ de la 
matrix $R$ se determinan a partir de los autovalores y autovectores del 
Hamiltoniano del sistema de $(N+1)$ electrones dado por la 
Ec.~(\ref{eq:RM-N+1Hamilt}). Los términos en la sumatoria que define 
$R_{ij}$ representa tanto los procesos colisionales resonantes y no 
resonantes, así como los efectos de interferencia entre ambos.

La mayor fuente de error en este método surge por el número finito de 
términos en la expansión de R-Matrix de la Ec.~(\ref{eq:RM-wavefn}), y 
por lo tanto, en el truncamiento en la sumatoria en la 
Ec.~(\ref{eq:RM-elements}). Los autovalores más altos del Hamiltoniano,
que no se incluyen en la expansión, contribuyen principalmente a los 
elementos de la diagonal de la matriz $R$, donde se agregan 
coherentemente. El método desarrollado por Buttle~\cite{Buttle:67} 
permite completar esta sumatoria de forma aproximada. 

Para poder implementar los cálculos de la matrix $R$ de la región 
interna para determinar las matrices de dispersión y, por lo tanto, las 
secciones eficaces, se resuelven las ecuaciones de acoplamiento en la 
región externa ($r>a$) para cada valor de energía. Sin embargo, dado 
que $P_{nl}(r)\simeq 0$ en esta región, el intercambio electrónico es
despreciable y las ecuaciones se simplifican tal que
\begin{equation}
\left[-\frac{1}{2}\frac{d^2}{dr^2}+\frac{l(l+1)}{2r^2}-\frac{(Z-N)}{r}
-\frac{k_i^2}{2}\right]y_i(r)=-\sum_{\lambda=1}^{\lambda_{\textrm{máx}}}
\sum_{j=1}^{n}\frac{c_{ij}^{\lambda}}{r^{\lambda+1}}\,y_j(r)
\quad i=1,\cdots n\,,
\label{eq:RM-outer}
\end{equation}
donde $n$ es el número total de canales de dispersión, abiertos o 
cerrados, y $\lambda^{\textrm{máx}}$ es el máximo valor en la expansión
multipolar permitido por la relaciones triangulares que resultan de las 
integrales angulares. Los coeficientes $c_{ij}^{\lambda}$ están dados 
por
\begin{equation}
c_{ij}^{\lambda}=\langle\bar{\Phi}_i\left|\sum_k^Nr_k^{\lambda}
P_{\lambda}\left(\cos\theta_{k,N+1}\right)\right|\bar{\Phi}_j\rangle\,,
\end{equation}
donde $\theta_{k,N+1}$ es el ángulo que se forma entre los vectores 
$\hat{\mathbf{r}}_{k}$ y $\hat{\mathbf{r}}_{N+1}$. 

Cuando $r\rightarrow\infty$, la solución a estas ecuaciones acopladas
resulta
\begin{equation}
\begin{array}{ll}
y_{ij}\sim\frac{1}{\sqrt{k_i}}\left(\sin\theta_i\delta_{ij}+\cos\theta_i
K_{ij}\right) \,\,&\,\,\textrm{para }\,k_i^2>0\,,\\
y_{ij}\sim\mathcal{O}\left(r^{-2}\right) \,\,&\,\,\textrm{para }\,k_i^2<0\,,
\end{array}
\end{equation}
donde $i=1,\cdots,n_o$ para los canales abiertos, $i=n_o+1,\cdots,n$ 
para los canales cerrados, $K_{ij}$ son los elementos de la matriz $K$,
y 
\begin{equation}
\theta_i=k_ir-\frac{1}{2}l_i\pi+\frac{(Z-N)}{k_i}\ln 2k_i r +\arg \Gamma
\left(l_i+1-i\frac{(Z-N)}{k_i}\right)\,.
\end{equation}
El segundo índice $j$ se introduce para designar las $n_o$ soluciones 
linealmente independientes. Así, resolviendo las ecuaciones acopladas
aproximadas en la región externa, las matrices $R$ y $K$ se relacionan 
y se determinan las secciones eficaces.






%%%%%%%%%%%%%%%%%%%%%%%%%%%%%%%%%%%%%%%%%%%%%%%%%%%%%%%%%%%%%%%%%%%%%%%%
\chapter{Procesos Gaussianos}
%%%%%%%%%%%%%%%%%%%%%%%%%%%%%%%%%%%%%%%%%%%%%%%%%%%%%%%%%%%%%%%%%%%%%%%%
\label{app:gp}

El método de procesos Gaussianos (GP) es un marco de aprendizaje 
automático supervisado probabilístico que se ha utilizado ampliamente 
para tareas de regresión y clasificación. Un modelo de regresión de 
procesos Gaussianos puede hacer predicciones incorporando conocimientos 
previos (kernels) y proporcionar medidas de incertidumbre sobre las 
predicciones~\cite{Rasmussen:06}. El modelo GP es un método de 
aprendizaje supervisado desarrollado por comunidades de ciencia de datos
y estadística. 

\section*{Formalismo matemático}

Un proceso Gaussiano es un proceso aleatorio en el cual a cualquier 
punto $\mathbf{x}\in \mathbb{R}^d$ se le asigna una variable aleatoria 
$f(\mathbf{x})$, donde la distribución de probabilidad conjunta de un 
número dado de estas variables 
$p(f(\mathbf{x}_1),\cdots,f(\mathbf{x}_N))$ es en sí misma Guassiana,
\begin{equation}
p(\mathbf{f}|\mathbf{X})
=\mathcal{N}(\mathbf{f}|\boldsymbol{\mu},\mathbf{K})\,,
\end{equation}
siendo $\mathbf{f}=(f(\mathbf{x}_1),\cdots,f(\mathbf{x}_N))$, 
$\boldsymbol\mu=(m(\mathbf{x}_1),\cdots,m(\mathbf{x}_N))$ y 
$\mathbf{K}_{ij}=\kappa(\mathbf{x}_i,\mathbf{x}_j)$. El término 
$m(\mathbf{x})$ es la función media y es usual tomar $m(\mathbf{x})=0$.
La función $\kappa$ se define positiva y se denonima \textit{kernel}
o función de covariancia. Un proceso Gaussiano es una distribución de 
funciones cuya forma está definida por $\mathbf{K}$. Una de las 
funciones de covarianza más usuales es la llamada exponencial
cuadrada, que se define como
\begin{equation}
\kappa(\mathbf{x}_{\mathbf{p}},\mathbf{x}_{\mathbf{q}})
=cov(f(\mathbf{x}_{\mathbf{p}}),f(\mathbf{x}_{\mathbf{q}}))
=\sigma_f^2\,e^{-\frac{1}{2l^2}(\mathbf{x}_{\mathbf{p}}
-\mathbf{x}_{\mathbf{q}})^2}\,.
\end{equation}
La covarianza de la salida $f(\mathbf{x})$ se escribe como una función 
de la entrada $\mathbf{x}$. La especificación de la función de 
covarianza da lugar a una distribución de funciones, y en este caso 
particular, $l$ determina la longitud característica de dichas funciones
y $\sigma_f$ la varianza de las mismas. 

El GP previo $p(\mathbf{f}|\mathbf{X})$ se puede actualizar en un GP
posterior $p(\mathbf{f}|\mathbf{X},\mathbf{y})$ después de haber medido
algún dato $\mathbf{y}$. La distribución posterior puede ser utilizada 
para realizar predicciones $\mathbf{f}_*$, dada una nueva entrada 
$\mathbf{X}_*$,
\begin{eqnarray}
p(\mathbf{f}_*|\mathbf{X}_*,\mathbf{X},\mathbf{y})
&=\int p(\mathbf{f}_*|\mathbf{X}_*,\mathbf{f}) p(\mathbf{f}|\mathbf{X},
\mathbf{y})\,d\mathbf{f} \\
&=\mathcal{N}(\mathbf{f}_*|\boldsymbol\mu_*,\boldsymbol\Sigma_*)\,.
\end{eqnarray}
Esta última ecuación expresa la distribución posterior predictiva, que 
es también Gaussiana con media $\boldsymbol\mu_*$ y desviación 
$\boldsymbol\Sigma_*$. Por definición del GP, la distribución conjunta
de los datos observados $\mathbf{y}$ y las predicciones $\mathbf{f}_*$ 
está dada por
\begin{equation}
\left(\begin{array}{c}
\mathbf{y}\\
\mathbf{f}_*
\end{array} 
\right)\sim\mathcal{N}\left(\mathbf{0}
\left(
\begin{array}{cc}
\mathbf{K}_y & \mathbf{K}_* \\
\mathbf{K}_*^T & \mathbf{K}_{**}
\end{array}
\right)
\right)
\end{equation}
con $N$ datos previos y $N_*$ nuevos datos, 
$\mathbf{K}_y=\kappa(\mathbf{X},\mathbf{X})+\sigma_y^2\mathbb{I} = 
\mathbf{K}+\sigma_y^2\mathbb{I}$ es de rango $N\times N$, 
$\mathbf{K}_*=\kappa(\mathbf{X},\mathbf{X}_*$ es de rango $N\times N_*$
y $\mathbf{K}_{**}=\kappa(\mathbf{X}_*,\mathbf{X}_*$ es de rango 
$N_*\times N_*$. El término $\sigma_y^2$ está asociado al dispersión de 
las mediciones.
A partir de un significativo trabajo algebraico~\cite{Murphy:12,
Rasmussen:06,deFreitas:13}, se puede demostrar que
\begin{eqnarray}
\boldsymbol\mu_*&=&\mathbf{K}_*^T\mathbf{K}_*^{-1}\mathbf{y}\, \\
\boldsymbol\Sigma_*&=&\mathbf{K}_{**}-\mathbf{K}_*^T\mathbf{K}_*^{-1}
\mathbf{K}_*\,.
\end{eqnarray}

\section*{Función de adquisición}

A partir de la distribución posterior dada por el GP, se puede construir 
una función de adquisición, que determina el próximo punto a evaluar.
Una de las funciones de adquisición más conocidas es la función de 
\textit{expected improvement} (EI), que está dada por
\begin{equation}
 a_{\mathrm{EI}} = \bigg\{
 \begin{array}{ll}
 (\mu(x)-f(x^*)-\xi)\Phi(\mu,\sigma,\xi) 
 + \sigma(x)\phi(\mu,\sigma,\xi) &\,,\quad\sigma(x)>0\\
 0 &\,,\quad\sigma(x)=0\,,
 \end{array}
\end{equation}
donde $\xi$ es un parámetro y las funciones $\Phi$ y $\phi$ son las 
funciones de distribución acumulativa y de densidad de probabilidad, 
respectivamente, que no se verán en detalle. La esencia de la función 
de adquisición está dada por el parámetro $\xi$, que permite configurar 
la función de adquisición en dos modos: exploración o explotación. 
Cuando $\xi<<1$, $\Phi$ es pequeña y la función de adquisición está 
regida por la $\sigma(x)$: el modelo se configura en exploración y le 
dará peso a las regiones donde hay mucha incerteza. Por el contrario,
cuando $\xi>>1$, la función de adquisición está determinada por $\mu(x)$ 
y se explotarán las regiones cercanas a los máximos encontrados hasta 
encontrar el mejor resultado. 



