\chapter*{}%
\addcontentsline{toc}{chapter}{Abstract}%

\begin{center}
\begin{large}
\textbf{Optimization of atomic and molecular targets \\ in collisional processes}
\end{large}
\end{center}

\vspace{1.5cm}
In this Thesis, the inelastic collisional processes arising from the interaction of ions and electrons with atomic and molecular targets are studied. The systems examined throughout this research work are of interest in several fields, from astrophysical plasma diagnostics, characterization and improvement of material design, to the evaluation of damage caused by ions and radiation. In examining collisional problems, it is of great importance to represent the atomic or molecular targets accurately. Although the theoretical grounds that allow one to determine the dynamics of such systems are given by the solutions of the corresponding Schrödinger equation, it is well known that the exact solution for complex systems cannot be obtained. Therefore, we have developed several techniques, each design to be implemented to solve a particular collisional problem. These procedures include a ``depurated inversion method" (DIM) to obtain effective atomic and molecular potentials, a Bayesian method to optimize the atomic structure of the target in electron impact calculations, the implementation of relativistic methods to compute the stopping power in heavy elements, and a stoichiometric model to attain the ionization of biological molecules. 
All these methods have been developed from first principles, and they are originally implemented in this dissertation. 

\vspace{1cm}
\noindent
Palabras claves: 
Atoms, 
Molecules,
Target optimization,
Collisional processes
