\chapter*{}%
\addcontentsline{toc}{chapter}{Abstract}%

\begin{center}
\begin{large}
\textbf{Optimization of atomic and molecular targets \\
in collisional processes}
\end{large}
\end{center}

\vspace{1cm}
This thesis aims to develop novel methodologies that allow obtaining 
precise 
atomic and molecular structure data of targets for their subsequent use 
in calculating inelastic processes. Light and heavy atoms, as well as 
simple and complex molecules, are studied throughout this work. The 
methodologies formulated include an inversion method to 
obtain effective potentials, which are used to calculate the 
ionization of atoms and simple molecules by the impact of protons and 
photons. A stoichiometric model is presented to describe complex 
molecular systems, which allows calculating the ionization of molecules 
with biological interest due to the impact of multi-charged ions. Based 
on the Bayesian inference, a technique widely used in machine learning, 
a procedure is designed to optimize observables in atomic 
collisions. For calculating energy loss in heavy atoms, a perturbative 
relativistic method is prescribed, along with the optimization of the 
electron 
configurations. In all cases, the proposed techniques allow a precise 
description of the electronic structure of the targets, improving the 
quality and efficiency of collisional calculations significantly.

\vspace{1cm}
\noindent
Key words: 
Atomic and molecular collisions, 
Electronic structures, 
Ionization of atoms and molecules, 
Effective potentials 

