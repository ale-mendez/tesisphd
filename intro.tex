\chapter{Introducción}

%%%%%%%%%%%%%%%%%%%%%%%%%%%%%%%%%%%%%%%%%%%%%%%%%%%%%%%%%%%%%%%%%%%%%
\section{Motivación y generalidades}
%%%%%%%%%%%%%%%%%%%%%%%%%%%%%%%%%%%%%%%%%%%%%%%%%%%%%%%%%%%%%%%%%%%%%

% Tesis de Carlos
El estudio de procesos colisionales ha tenido un rol fundamental en el 
desarrollo de la teoría mecanico-cuántica. Estos fenómenos permiten 
estudiar la estructura electrónica de blancos multielectrónicos, así 
como también examinar la naturaleza de las interacciones entre el 
proyectil y el blanco. Las colisiones de átomos y moléculas con 
proyectiles tales como fotones, iones cargados y electrones se han 
estudiado por décadas de forma teórica y experimental, resultando en
el desarrollo de un gran número de modelos que permiten predecir tales 
fenómenos. Sin embargo, muchas de las limitaciones de estas 
aproximaciones están ligadas a una precisa descripción de la estructura 
electrónica de los blancos antes y después de la colisión. 

% Bautista 2000
Desde hace casi un siglo, se han dedicado esfuerzos significativos para
el desarrollo de métodos téoricos que permitan obtener datos atómicos y 
moleculares precisos. Estos datos son esenciales para el análisis de 
espectros de laboratorio y astrofísicos, así como áreas aplicadas que 
van desde el diseño de materiales espaciales hasta el diseño de 
tratamientos médicos. En general, los datos atómicos y moleculares están 
compuestos por dos subgrupos de datos, correspondientes a (1) la 
estructura del blanco y (2) los procesos de dispersión a los que son 
sujetos. 

El objetivo de este trabajo es ampliar el conocimiento que se tiene 
sobre blancos de interés, desarrollando nuevas técnicas de optimización 
y métodos que permiten describir los procesos de forma precisa. Para 
ello, se han desarollado diferentes técnicas, cada una de ellas 
orientada a un problema colisional diferente. Estos desarrollos incluyen 
un ``método de inversión depurada" (DIM) para la obtención de 
potenciales efectivos que permitan describir blancos atómicos y 
moleculares en procesos de ionización simple. Un modelo estequiométrico 
permite describir el cálculo de ionización en moléculas biológicas. Por 
otro lado, se implementa un método Bayesiano para la optimización de la 
estructura del blanco en cálculos de colisiones electrónicas y, en 
procesos de pérdida de energía debido al impacto de iones, blancos 
pesados se describen mediante métodos relativistas. 

%%%%%%%%%%%%%%%%%%%%%%%%%%%%%%%%%%%%%%%%%%%%%%%%%%%%%%%%%%%%%%%%%%%%%
%\subsection{Estructura del blanco}
%%%%%%%%%%%%%%%%%%%%%%%%%%%%%%%%%%%%%%%%%%%%%%%%%%%%%%%%%%%%%%%%%%%%%

%%%%%%%%%%%%%%%%%%%%%%%%%%%%%%%%%%%%%%%%%%%%%%%%%%%%%%%%%%%%%%%%%%%%%
%\subsection{Procesos colisionales}
%%%%%%%%%%%%%%%%%%%%%%%%%%%%%%%%%%%%%%%%%%%%%%%%%%%%%%%%%%%%%%%%%%%%%

%%%%%%%%%%%%%%%%%%%%%%%%%%%%%%%%%%%%%%%%%%%%%%%%%%%%%%%%%%%%%%%%%%%%%
\section{Descripción del trabajo}
%%%%%%%%%%%%%%%%%%%%%%%%%%%%%%%%%%%%%%%%%%%%%%%%%%%%%%%%%%%%%%%%%%%%%

En el Capítulo~\ref{chap:iondim} se estudia la estructura de átomos y 
moléculas pequeñas a través de la ionización simple. La descripción de 
los blancos se obtiene a partir de potenciales efectivos, que resultan 
de la implementación del método de inversión depurada 
(\acs{dim})~\cite{Mendez:16,Mendez:19dim}. El DIM resuelve el problema 
inverso a partir de funciones de onda y energías conocidas. Sin embargo, 
la inversión directa de las soluciones presenta ciertos defectos, los 
cuales son examinados en detalle~\cite{Mendez:18,Mitnik:19}. El objetivo 
principal de este trabajo consiste en explorar la implementación de los
potenciales DIM en la teoría de colisiones atómicas para describir 
procesos inelásticos tales como la ionización por impacto de fotones e 
iones cargados a partir de la primera aproximación de 
Born~\cite{Mendez:19dim}. 

En el Capítulo~\ref{chap:ionmol} implementamos el potencial DIM para 
describir la estructura de átomos que constituyen moléculas biológicas.
A partir del método de onda continua distorsionada con estado inicial 
eikonal, describimos la ionización simple de estos blancos debido al 
impacto de iones cargados. Mediante la introducción del modelo  
estequiométrico, las predicciones atómicas obtenidas son combinadas para  
obtener secciones eficaces de un número significativo de sistemas 
moleculas~\cite{Mendez:20ionmol}, incluyendo las bases de ADN y 
ARN. También exploramos tres reglas de escala, útiles para predecir a 
primer orden la ionización de sistemas blanco--proyectil 
arbitrarios~\cite{Mendez:20scale}. 

En el Capítulo~\ref{chap:heavy} se examina la estructura de blancos 
relativistas neutrales. La ecuación de Dirac en blancos 
multielectrónicos se resuelve implementando un método perturbativo y la 
optimización de las función de onda del sistema. Se calculan orbitales 
radiales y energías de ligadura de nueve blancos relativistas, 
incluyendo metales de transición y lantánidos~\cite{Mendez:19relat}. 
Luego, estos resultados son implementados en el cálculo de pérdida de 
energía por impacto de iones~\cite{Montanari:20}.

En el Capítulo~\ref{chap:bayeopt} se estudia el proceso de excitación
por impacto de electrones en átomos neutros. La optimización de estos 
blancos constituye uno de los cuellos de botella del cálculo colisional 
mediante métodos no perturbativos. En este Capítulo se explora la 
posibilidad de automatizar el ajuste de ciertos parámetros del problema 
mediante la implementación de técnicas que se usan en el campo del 
aprendizaje automatizado~\cite{Mendez:20baye,Mendez:prep}.

Las conclusiones generales de este trabajo de investigación se presentan 
en el Capítulo~\ref{chap:conclusiones}.
A lo largo de este trabajo se usan unidades atómicas ($\hbar=e=m_e=1$)
a menos que se especifique lo contrario.


Los cálculos presentados en el Capítulo~\ref{chap:bayeopt} se realizaron 
en el cluster Piluso perteneciente al Sistema Nacional de Computación de 
Alto Desempeño, ubicado en Rosario, Santa Fé. Sin estos recursos 
computacionales, este trabajo no hubiese sido posible.

\begin{comment}

Potenciales usados aqui para describir las estructuras atómicas y moleculares

\begin{itemize}
\item Método de inversión depurada
\item Método de potencial paramétrico de Klapish (1971)
\item Potenciales modelo (TFDA, STO + pol pot paramétricos)
\end{itemize}

%De Bautista (2008)
%Various methods are known that allow one to model the structure of atoms 
%and ions, such as Hartree-Fock, multiconfiguration Hartree-Fock,
%superposition of configurations, semi-empirical methods, many-body 
%perturbation methods and central field approximation methods, etc. 
%Methods based on the Hartree-Fock formalism are self-consistent and can 
%yield very accurate results for simple atoms or for a few select levels 
%of more complex systems. Methods based on the central field approximation 
%that allow for configuration interaction have been successful in 
%describing large numbers of levels for more complex systems, such as 
%metals. However, computations of large scale models of 3d transition 
%metals remain quite challenging. In order to obtain accurate (∼10%) 
%radiative rates and cross sections for photon and electron interactions 
%one needs to guarantee eigenenergies that compare well (within ∼5%) with
%experimentally determined energies. However, this is difficult to achieve 
%with ab initio calculations for ions with more than just a few electrons. 
%Moreover, comparisons are regularly done with energies relative to the 
%ground level, which can be misleading if the absolute energy of the 
%ground level is overestimated. 




Idea para intro de Chap4: 

Most of the theoretical calculations of the cross section for the scattering of atomic systems are based on the solution of the nonrelativistic Schroedinger equation. For certain systems, this approximation is insufficient, and it  is necessary to resort to a relativistic formulation. This is usually the case when the scattering system contains heavy atoms, with core  electrons  moving at near-relativistic  speeds,  when spin-orbit effects are significant, or when the incident particle has velocities near the speed of light. When the velocity of the incident electron is  relativistic, it is  possible to use modifications of the Born approximation   (Inokuti, 1971) to treat the   dynamics, and that simplifies the  problem. For much lower energy  incident  electrons, the theorist   must often turn to a Pauli or Dirac  formulation, which can considerably  complicate  the numerics. In  other cases, a  properly modified Schroedinger formulation, using effective core potentials  based on some relativistic approximation will suffice (Cowan  and Griffin, 1976). 
\end{comment}

