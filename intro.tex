\chapter{Introducción}

%%%%%%%%%%%%%%%%%%%%%%%%%%%%%%%%%%%%%%%%%%%%%%%%%%%%%%%%%%%%%%%%%%%%%
\section{Motivación y generalidades}
%%%%%%%%%%%%%%%%%%%%%%%%%%%%%%%%%%%%%%%%%%%%%%%%%%%%%%%%%%%%%%%%%%%%%

% Tesis de Carlos
El estudio de procesos colisionales ha tenido un rol fundamental en el 
desarrollo de la teoría mecanico-cuántica. Estos fenómenos permiten 
estudiar la estructura electrónica de blancos multielectrónicos, así 
como también examinar la naturaleza de las interacciones entre el 
proyectil y el blanco. Las colisiones de átomos y moléculas con 
proyectiles tales como fotones, iones y electrones se han estudiado por 
décadas de forma teórica y experimental, resultando en el desarrollo de 
un gran número de modelos que permiten predecir tales fenómenos. Sin 
embargo, muchas de las limitaciones de estas aproximaciones están 
ligadas a una precisa descripción de la estructura electrónica de los 
blancos antes y después de la colisión. 

En general, los datos atómicos y moleculares están compuestos por dos 
subgrupos de datos, correspondientes a (1) la estructura del blanco y 
(2) los procesos de dispersión a los que son sujetos. Los esfuerzos 
dedicados al desarrollo de métodos téoricos que permitan obtener datos
precisos se debe en gran medida a la necesidad de contar tal información 
para el análisis de espectros de laboratorio y astrofísicos, así como en 
áreas aplicadas que van desde el diseño de materiales espaciales hasta 
el diseño de tratamientos médicos. 

El objetivo de este trabajo es ampliar el conocimiento que se tiene 
sobre blancos de interés, desarrollando nuevas técnicas de optimización 
y métodos que permiten describir los procesos de forma precisa. Para 
ello, se han desarollado diferentes técnicas, cada una de ellas 
orientada a un problema colisional diferente. Estos desarrollos incluyen 
un ``método de inversión depurada" (DIM) para la obtención de 
potenciales efectivos que permitan describir blancos atómicos y 
moleculares, y un ``modelo estequiométrico simple'', que
permite describir el cálculo de ionización en moléculas biológicas. 
En procesos de pérdida de energía debido al impacto de iones en átomos 
pesados, la estructura de los blancos se ajustan mediante métodos 
relativistas. Por otro lado, se implementa un método Bayesiano para la 
optimización de la estructura del blanco en cálculos de colisiones 
electrónicas.

%%%%%%%%%%%%%%%%%%%%%%%%%%%%%%%%%%%%%%%%%%%%%%%%%%%%%%%%%%%%%%%%%%%%%
%\subsection{Estructura del blanco}
%%%%%%%%%%%%%%%%%%%%%%%%%%%%%%%%%%%%%%%%%%%%%%%%%%%%%%%%%%%%%%%%%%%%%

%%%%%%%%%%%%%%%%%%%%%%%%%%%%%%%%%%%%%%%%%%%%%%%%%%%%%%%%%%%%%%%%%%%%%
%\subsection{Procesos colisionales}
%%%%%%%%%%%%%%%%%%%%%%%%%%%%%%%%%%%%%%%%%%%%%%%%%%%%%%%%%%%%%%%%%%%%%

%%%%%%%%%%%%%%%%%%%%%%%%%%%%%%%%%%%%%%%%%%%%%%%%%%%%%%%%%%%%%%%%%%%%%
\section{Descripción del trabajo}
%%%%%%%%%%%%%%%%%%%%%%%%%%%%%%%%%%%%%%%%%%%%%%%%%%%%%%%%%%%%%%%%%%%%%

En el Capítulo~\ref{chap:iondim}, se estudia la estructura de átomos y 
moléculas pequeñas a través de la ionización simple. La descripción de 
los blancos se obtiene mediante potenciales efectivos, que resultan de 
la implementación del método de inversión depurada 
(\acs{dim})~\cite{Mendez:16,Mendez:19dim}. El DIM resuelve el problema 
inverso a partir de funciones de onda y energías conocidas. Sin embargo, 
la inversión directa de las soluciones presenta defectos, los cuales son 
examinados en detalle~\cite{Mendez:18,Mitnik:19}. El uso efectivo de los 
potenciales DIM en la teoría de colisiones se investiga en procesos 
inelásticos tales como la ionización por impacto de fotones e iones 
mediante la primera aproximación de Born~\cite{Mendez:19dim}. 
%
En el Capítulo~\ref{chap:ionmol} se examina la ionización de estructuras 
moleculares complejas de interés biológico, incluyendo las bases de ADN y 
ARN. El DIM se implementa para describir la estructura de átomos que las
constituyen y, empleando un método de onda continua distorsionada, se 
calcula la ionización simple de estos blancos debido al impacto de 
diversos iones. El modelo estequiométrico simple permite 
obtener secciones eficaces de un número significativo de sistemas 
moleculares~\cite{Mendez:20ionmol}. También se presentan tres reglas de 
escala, útiles para predecir a primer orden la ionización de sistemas 
blanco--proyectil arbitrarios~\cite{Mendez:20scale}. 
%
En el Capítulo~\ref{chap:heavy}, se investiga la estructura de blancos 
pesados multielectrónicos. La ecuación de Dirac de estos sistemas se 
resuelve a partir un método perturbativo y la optimización de las 
configuraciones electrónicas definidas. Se calculan orbitales radiales y 
energías de ligadura de nueve blancos, incluyendo metales de transición 
y lantánidos~\cite{Mendez:19relat}. Estos resultados son implementados 
en el cálculo de pérdida de energía por impacto de iones mediante 
diversos modelos de primeros principios~\cite{Montanari:20}, los cuales
se comparan con recientes mediciones experimentales y métodos teóricos 
que constituyen el estado del arte.
%
En el Capítulo~\ref{chap:bayeopt} se estudia el proceso de excitación
por impacto de electrones en átomos neutros. La optimización de estos 
blancos constituye uno de los cuellos de botella del cálculo colisional 
mediante métodos no perturbativos. La posibilidad de automatizar el 
ajuste de ciertos parámetros del problema se explora mediante la 
implementación de técnicas que se usan en el campo del aprendizaje 
automatizado. Se obtienen resultados optimizados de energías, 
\textit{oscillator strengths} y secciones eficaces, las cuales se 
comparan con valores de referencia~\cite{Mendez:20baye,Mendez:prep}.
%
Finalmente, las conclusiones generales se presentan en el 
Capítulo~\ref{chap:conclusiones}. A lo largo de este trabajo se usan 
unidades atómicas ($\hbar=e=m_e=1$) a menos que se especifique lo 
contrario.


