\chapter{Introducción}

%%%%%%%%%%%%%%%%%%%%%%%%%%%%%%%%%%%%%%%%%%%%%%%%%%%%%%%%%%%%%%%%%%%%%
\section{Motivación}
%%%%%%%%%%%%%%%%%%%%%%%%%%%%%%%%%%%%%%%%%%%%%%%%%%%%%%%%%%%%%%%%%%%%%




%%%%%%%%%%%%%%%%%%%%%%%%%%%%%%%%%%%%%%%%%%%%%%%%%%%%%%%%%%%%%%%%%%%%%
\section{Descripción general}
%%%%%%%%%%%%%%%%%%%%%%%%%%%%%%%%%%%%%%%%%%%%%%%%%%%%%%%%%%%%%%%%%%%%%


%De Bautista (2008)
%Various methods are known that allow one to model the structure of atoms 
%and ions, such as Hartree-Fock, multiconfiguration Hartree-Fock,
%superposition of configurations, semi-empirical methods, many-body 
%perturbation methods and central field approximation methods, etc. 
%Methods based on the Hartree-Fock formalism are self-consistent and can 
%yield very accurate results for simple atoms or for a few select levels 
%of more complex systems. Methods based on the central field approximation 
%that allow for configuration interaction have been successful in 
%describing large numbers of levels for more complex systems, such as 
%metals. However, computations of large scale models of 3d transition 
%metals remain quite challenging. In order to obtain accurate (∼10%) 
%radiative rates and cross sections for photon and electron interactions 
%one needs to guarantee eigenenergies that compare well (within ∼5%) with
%experimentally determined energies. However, this is difficult to achieve 
%with ab initio calculations for ions with more than just a few electrons. 
%Moreover, comparisons are regularly done with energies relative to the 
%ground level, which can be misleading if the absolute energy of the 
%ground level is overestimated. 
