\chapter*{Introducción}%
\addcontentsline{toc}{chapter}{Introducción}%

Los datos de estructura atómica y molecular son fundamentales para el 
diagnóstico de objetos astrofísicos, atmósferas y plasmas de fusión, 
mientras que los valores de secciones eficaces o coeficientes de tasas 
de procesos colisionales que involucran electrones, iones y fotones, son 
esenciales debido a su aplicabilidad en diferentes áreas de la ciencia y 
la tecnología. 
La descripción teórica de la dinámica de átomos y moléculas está 
totalmente determinada por la ecuación de Schr\"odinger correspondiente. 
Sin embargo, su resolución exacta en sistemas complejos es una tarea 
imposible de realizar. 

El objetivo de esta Tesis consiste en desarrollar nuevos métodos y 
técnicas de optimización que permitan determinar datos de estructura 
precisos, para aplicarlos luego en el cálculo de procesos inelásticos en 
sistemas multielectrónicos. Los blancos examinados a lo largo de esta 
investigación tienen diversas aplicaciones, desde el diagnóstico de 
plasmas astrofísicos, caracterización y mejoramiento del diseño de 
materiales, hasta la evaluación de daño por iones y radiación. En los 
capítulos que comprenden este trabajo se presentan cuatro metodologías, 
cada una orientada a un sistema colisional distinto. 

Una de las dificultades del cálculo de ionización de sistemas atómicos y
moleculares está dado por la representación del estado continuo del 
electrón luego de la colisión. Para abordar este problema, en el 
presente trabajo, se formula el método de Inversión Depurada 
(\acs{dim})~\cite{Mendez:16,Mendez:18,Mitnik:19}. Este método resuelve 
el problema inverso, que consiste en encontrar el potencial que genera 
funciones de onda y energías dadas. Así, el DIM permite estudiar la 
estructura del blanco mediante el uso de potenciales efectivos. El uso 
efectivo de los potenciales se pone a prueba en el cálculo de procesos 
tales como la ionización por impacto de iones y 
fotones~\cite{Mendez:19dim}. 

A medida que el número de átomos y electrones que componen una molécula 
crece, la representación de estos blancos y su interacción con 
proyectiles incidentes se vuelve más compleja. Por ello, en la segunda 
parte de esta Tesis, se presenta un modelo estequiométrico, que permite 
calcular las secciones eficaces de un número significativo de sistemas 
colisionales, incluyendo las bases de ADN y ARN. En esta técnica 
propuesta, el DIM describe la estructura atómica de estas moléculas 
mientras que la ionización de estos blancos debido al impacto de iones 
se calcula empleando un método de onda continua distorsionada. 
Del análisis de los nuevos resultados teóricos, se enuncian tres reglas 
de escala, las cuales resultan útiles para estimar la ionización de 
sistemas molécula--proyectil arbitrarios~\cite{Mendez:20scale}. 

La tercer parte de este trabajo se aboca al estudio de la estructura de 
lantánidos y metales de transición pesados. Para la inclusión de efectos 
relativistas, se prescribe un método perturbativo, que permite resolver 
la ecuación de Dirac correspondiente, en conjunción a la optimización de 
las configuraciones electrónicas empleadas~\cite{Mendez:19relat}. Las 
energías y funciones de onda relativistas resultantes son utilizadas en 
modelos de pérdida de energía por impacto de iones~\cite{Montanari:20}. 
Los resultados obtenidos permiten explicar datos experimentales 
recientes y reafirman la necesidad de incluir efectos relativistas en 
los cálculos de procesos de pérdida de energía. 

En átomos neutros o de bajo grado de ionización, el cálculo de 
excitación por impacto de electrones mediante métodos no perturbativos 
requiere no sólo la correcta representación de los estados ligados y los 
estados Rydbergs, sino también la inclusión del acoplamiento con los 
estados del continuo. Para ello, se deben incluir una gran cantidad de 
configuraciones electrónicas (ligadas y continuas), lo que resulta en 
centenares de niveles de energía. En general, estos niveles se ajustan 
manualmente, lo que constituye uno de los cuellos de botella del cálculo 
colisional. En el último capítulo de esta Tesis, se diseña un método de 
optimización para ajustar en forma automática la estructura de los 
blancos en este cálculo de dispersión. La metodología presentada se 
basa en técnicas empleadas en el campo del aprendizaje automatizado. La 
implementación de este método permite reproducir valores de 
referencia~\cite{Mendez:20baye,Mendez:prep}, reduciendo 
significativamente los esfuerzos computacionales. 



