\chapter*{Agradecimientos}%
\addcontentsline{toc}{chapter}{Agradecimientos}%

La autora agradece a los múltiples canales de financiamiento recibido 
para realizar este trabajo. Al Consejo Nacional de 
Investigaciones Científicas y Técnicas de la República Argentina, quien 
otorgó la beca de doctorado que le posibilitó dedicarse a la 
investigación precarizada. Al estado argentino, que financia (aunque no 
lo suficiente) la educación pública y gratuita. A la Universidad de 
Buenos Aires, la Facultad de Ciencias Exactas y Naturales, y el 
Departamento de Física, que la recibieron con brazos abiertos en sus 
aulas y laboratorios. Al Instituto de Astronomía y Física del Espacio, 
que le puso a disponibilidad todos sus recursos y le brindó la calidez 
humana de quienes habitan sus pasillos. 
Al Sistema Nacional de Computación de Alto Desempeño, quien 
financió parte de este trabajo otorgando tiempo de cálculo en el cluster 
Piluso. 

Los reconocimientos se extienden al Dr. Darío Mitnik, Director de esta 
tesis, quien dedicó años y cantidades inconmensurables de paciencia y 
esfuerzo en la formación de la autora. Especiales agradecimientos al 
Dr. Jorge Miraglia, por su generosidad intelectual y humana, y a la Dra. 
Claudia Montanari, quien día a día le enseña con el ejemplo a ser mejor 
científica y, por sobre todo, mejor persona. Al les colegas del grupo de 
Dinámica Cuántica en la materia, que le compartieron a la autora 
financiamiento, paltas y chocolates. A los científicos de otras 
Instituciones que compartieron su tiempo y conocimiento, en especial, a 
los Dres. N. Badnell y S. Loch.

En una nota personal, la autora le agradece a su madre, que desde el 
recuerdo le sonrie. A sus hermanas: Roxana, Andrea y Emilse, por ser 
fuente de fortaleza en los momentos más difíciles. A su familia, padre y 
tías. A su compañero, sin él este trabajo no existiría por lo que este 
trabajo es en parte también suyo. A les amigues, por los brebajes, la 
terapia y las risas.

